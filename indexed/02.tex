\documentclass[output=paper]{langsci/langscibook} 
\author{Artemis Alexiadou\affiliation{Humboldt-Universität zu Berlin\\Leibniz-Zentrum Allgemeine Sprachwissenschaft, Berlin}\lastand Elena Anagnostopoulou\affiliation{University of Crete}}
\title{An asymmetry in backward control: subject vs. object control}
% \chapterDOI{} %will be filled in at production

% % \epigram{Change epigram in chapters/01.tex or remove it there }
\abstract{In this paper we discuss an asymmetry in the distribution of backward control in Greek. Greek has been argued to have subject backward control; however, as we will show, the language lacks backward object control. We will account for this asymmetry by appealing to the nature of Backward Agree, which seems to require heads of the same type.}
\maketitle

\begin{document}
\citereset
%%please move the includegraphics inside the {figure} environment
%%\includegraphics[width=\textwidth]{OGSVolumeAug2018AlexiadouAnagnostopoulou-img1}

 
%%please move the includegraphics inside the {figure} environment
%%\includegraphics[width=\textwidth]{OGSVolumeAug2018AlexiadouAnagnostopoulou-img2}

\section{Aims and goals}

In this paper, we discuss \is{control!backward control}backward control configurations, focusing on \ili{Greek}, a language showing a prima facie asymmetry between backward subject control (BSC), which is fully productive, and backward \is{control!object control}object control (BOC), which is severely limited. This is a puzzling state of affairs if \ili{Greek} indeed has \is{control!backward control}backward control understood as movement and spell-out of the lower copy of the chain, as has been argued in the literature. Based on new evidence, we argue that the movement approach to \ili{Greek} BSC is an illusion. The correct analysis involves the formation of a chain between the \is{feature!phi-feature}phi-features of the matrix T, the phi-features of the embedded T and those of the embedded subject, which is possible as long as the embedded subject does not intervene between the matrix and the embedded T. The formation of such chains is possible due to the fact that \ili{Greek} has pronominal agreement, being a \isi{pro-drop} language (\citealt{Alexiadou1998}; \citealt{Barbosa2009}). The formation of comparable chains is severely restricted in BOC configurations, which are only possible if the full embedded subject is either a clitic-doubled \isi{experiencer} bearing dative or \is{case!accusative case}accusative case or an emphatic nominative anaphoric pronoun. We will discuss potential reasons why this should be so from the perspective of current approaches to \isi{Agree}.

The paper is structured as follows. We first briefly summarize the arguments in \citet{Alexiadou2010} that \ili{Greek} has backward subject control (BSC), as well as more recent arguments, recently presented in \citet{Tsakali2017}, that this type of phenomenon does not involve scrambling and indeed instantiates agreement chains between a matrix T and an embedded subject. We then discuss the environments that have been argued to show \is{control!object control}object control in \ili{Greek} and point out that there is an asymmetry between BSC (possible) as opposed to backward \is{control!object control}object control (BOC) (generally impossible) in \ili{Greek}. We attribute the lack of BOC to the general unavailability of chain formation between a lower T and a higher \isi{Voice}\slash \textit{v}APPL head, which can be overridden under certain conditions.

\section{Introduction}%1

As has been discussed in the work of \citeauthor{Polinsky2006} (\citeyear{Polinsky2006}; henceforth ‘P\&P’), the \is{control!movement analysis of control}movement analysis of control, put forth in \citet{Hornstein1999}, coupled with the copy-and-delete theory of movement, predicts that next to canonical\slash forward control patterns, where the lower copy of the moved element is deleted, there should also exist \is{control!backward control}backward control patterns, where the higher copy is deleted. A third possibility, which we do not consider in this section, is resumption, where both copies are pronounced, as depicted in \tabref{tab:alexiadou:1}.
    
\begin{table}
\begin{tabular}{ccl}
\lsptoprule
\multicolumn{2}{c}{Copy pronounced}\\\cmidrule{1-2}
Higher & Lower  &  Structure\\\midrule
\ding{51} & * & Forward Control (FC)\\
* & \ding{51} & Backward Control (BC)\\
\ding{51} & \ding{51} & Resumption\\
\lspbottomrule
\end{tabular}
\caption{Typology of control and raising in P\&P (\citeyear{Polinsky2006})\label{tab:alexiadou:1}}
\end{table}

A lot of evidence has been provided in the literature for BC, which can be observed in several unrelated languages. For instance, BSC can be observed in several \ili{Nakh-Daghestanian} languages, in \ili{Northwest Caucasian}, in \ili{Malagasy}, and in \ili{Korean}; see e.g. \citegen{Fukuda2008} overview. The claim that BC exists in natural language is the strongest argument brought by the movement analysis of control against the PRO-based approach; see e.g. \citet{Landau1999} and subsequent work.

In \citet{Alexiadou2010}, we addressed \citegen{Landau2007} objections to BSC. One of the objections raised in \citet{Landau2007} concerned the rarity of the phenomenon in one of the languages in which BC has been argued to exist, namely \ili{Tsez}: in Tsez, only \emph{two} verbs display BC. In other languages, the numbers hardly exceed five. Most commonly, the BC verbs are aspectuals (\textit{begin}, \textit{continue}, \textit{stop}), which also have a standard raising analysis. On the basis of \ili{Greek} and \ili{Romanian} control constructions, we argued that BC is real in these two languages, as it is exhibited by the same verbs that allow OC (hence the ‘rarity’ objection doesn’t hold for \ili{Greek} and Romanian). 

Recently, a re-evaluation of the empirical picture was put forth in \citet{Tsakali2017} that can be summarized as follows: what has been analyzed as BSC in \ili{Greek}, Romanian and \ili{Spanish} is an illusion. In \ili{Spanish}, it involves complex predicate formation, while in \ili{Greek}\slash Romanian it involves co-reference with an embedded subject. Specifically, BC in \ili{Greek} is a side-effect of the availability of an agreement chain between a null main subject and an overt embedded subject in all types of \is{subjunctive}subjunctives (\textit{na}{}-clauses) and, to a certain extent, in indicatives (\textit{that}{}-clauses). While \is{coreference!backward coreference}backward coreference is allowed in both types of clauses if the order is VSO or VOS, embedded SVO orders, which are available in indicatives, lead to a robust Principle C effect. \citet{Tsakali2017} thus propose that what has been analysed as BC actually reflects $\varphi ${}-agreement between matrix T, embedded T and the overt S(ubject), licit only if the S doesn’t intervene between the two T heads, as in \REF{ex:alexiadou:2a}, as opposed to \REF{ex:alexiadou:2b}:

\ea%2
    \label{ex:alexiadou:2}
    \ea[]{[T$\varphi $\textsubscript{k} [\textsubscript{TP/CP} $T\varphi $\textsubscript{k} DP$\varphi $\textsubscript{k}]]\label{ex:alexiadou:2a}}
    \ex[*]{[T$\varphi $\textsubscript{k} [\textsubscript{TP/CP} DP$\varphi $\textsubscript{k} $T\varphi $\textsubscript{k}]]\label{ex:alexiadou:2b}}
    \z
\z    

In what follows, we summarize both aspects of this discussion. Nevertheless, as we will show in §4, such co-reference is not available in the case of \is{control!object control}object control.

\section{BSC in Greek: An epiphenomenon}%2

In \ili{Greek}, control is instantiated in a subset of \isi{subjunctive} complement clauses, as the language lacks \is{infinitive}infinitives; see e.g. \citet{Varlokosta1994} and references therein. These \isi{subjunctive} complement clauses are introduced by the \isi{subjunctive} marker \textit{na} \REF{ex:alexiadou:3}. The embedded verb, similarly to the matrix verb, shows agreement in number and person with the matrix subject.\footnote{\textit{Na} has been analyzed as a \isi{subjunctive} mood marker (cf. \citealt{Philippaki-Warburton1984}), a \isi{subjunctive} \isi{complementizer} (\citealt{Agouraki1991}; \citealt{Tsoulas1993}) or a device to check EPP \citep{Roussou2009}. Here we side with the first view.} 

\ea%3
 \label{ex:alexiadou:3}
\ili{Greek}\\
\gll o  Petros / ego  kser-i / -o        \textbf{na}    koliba-i / -o\\
  the    Peter.\textsc{nom} / I  know-\textsc{3sg} / -\textsc{1sg}    \textsc{sbjv}  swim-\textsc{3sg} / -\textsc{1sg}\\
\glt  ‘Peter/I knows/know how to swim.’
\z

The literature on \ili{Greek} control recognizes two main types of \isi{subjunctive} complements (but cf. \citealt{Spyropoulos2007} and \citealt{Roussou2009} for refinements): Obligatory Control (OC) ones and non-OC ones (NOC) (or \textit{C(ontrolled)-subjunctives} and \textit{F(ree)-subjunctives} in \citegen{Landau2004} terminology).

\begin{description}
\item[1. OC/C-subjunctives] are found as complements of verbs such as \textit{ksero} ‘know how’, \textit{tolmo} ‘dare’, \textit{herome} ‘be happy’, \textit{ksehno} ‘forget’, \textit{thimame} ‘remember’, \textit{matheno} ‘learn’, \textit{dokimazo} ‘try’, aspectual verbs such as \textit{arhizo} ‘start\slash begin’, \textit{sinehizo} ‘continue’.
\ea%4
    \label{ex:alexiadou:4}
  \ea[*]{\gll o Petros      kseri     na   \textbf{kolimbao}\\
      the Peter.\textsc{nom}   knows  \textsc{sbjv}   swim.\textbf{\textsc{1}}\textsc{sg}\\
     \glt   Lit. ‘Peter knows how I swim.’}
   \ex[*]{\gll o  Petros    kseri    na   \textbf{kolimbai}   i    Maria\\
       the Peter.\textsc{nom}  knows  \textsc{sbjv}   swim.\textbf{\textsc{3}}\textsc{sg} the   Mary.\textsc{nom}\\
     \glt   Lit. ‘Peter knows how Mary swims.’}
\z
\z
\item[2. NOC/F-subjunctives] are found with e.g. volitional\slash future-referring predicates:
\ea%5
    \label{ex:alexiadou:5}
    \ea\gll o    Petros      perimeni  na    \textbf{erthun}\\
       the    Peter.\textsc{nom} expects  \textsc{sbjv}  come.\textsc{3pl}\\
        \glt ‘Peter expects that they come.’
     \ex\gll  o     Petros    elpizi  na    figi     i     \textbf{Maria}\\
          the    Peter.\textsc{nom}   hopes  \textsc{sbjv}  go.\textsc{3sg}  the    Mary.\textsc{nom}\\
         \glt ‘Peter hopes that Mary goes.’
\z
\z
\end{description}

\citet{Alexiadou2010} present evidence that \emph{all} OC verbs in \ili{Greek} allow BC. In fact, the subject DP can appear in a number of positions (here \ili{Greek} differs from Tsez). Preverbal subjects are considered to be in a left-dislocated position, while post-verbal subjects are located within the \textit{v}P; see \citet{Alexiadou1998} for discussion. VSO and VOS orders have different information structure properties; see \citet{Alexiadou1999,Alexiadou2000} for discussion.  Generally, the DP in the \isi{subjunctive} complement agrees with both the low and the matrix verb in person and number:



\ea%6
    \label{ex:alexiadou:6}
    \gll (o \textbf{Janis}) emathe (o \textbf{Janis}) na pezi  (o \textbf{Janis}) kithara (o \textbf{Janis})\\
         the John.\textsc{nom} learned.\textsc{3sg} the John.\textsc{nom} \textsc{sbjv} play.\textsc{3sg} the John.\textsc{nom} guitar the John.\textsc{nom}\\
    \glt    ‘John learned to play the guitar.’
    \z

The pattern in which the DP resides in the complement clause qualifies as a case of BC on the basis of P\&P’s argumentation. First, these constructions are bi-clausal (contra \citealt{Roussou2009}), as can be shown on the basis of evidence from \textit{negation} and \textit{event modification}.

Two separate negations are possible:

\ea%7
    \label{ex:alexiadou:7}
\ea
\gll    \textbf{den}  emathe    na    magirevi   o    Janis\\
    not   learned.\textsc{3sg}  \textsc{sbjv}  cook.\textsc{3sg}    the    John.\textsc{nom}\\
\glt    ‘John didn’t learn to cook.’

\ex
\gll    emathe     na   \textbf{min}   magirevi    o    Janis\\
    learned.\textsc{3sg}  \textsc{sbjv} not  cook.\textsc{3sg} the   John.\textsc{nom}\\
\glt ‘John learned not to cook (i.e. ‘John got into the habit of not cooking’).’

\ex
\gll    \textbf{den}   emathe    na   \textbf{min} magirevi    o    Janis\\
    not  learned.\textsc{3sg}  \textsc{sbjv} not  cook.\textsc{3sg}    the   John.\textsc{nom}\\
\glt    ‘John didn’t learn not to cook (i.e. ‘John still has the habit of cooking’).’
\z
\z

The event of each clause can be modified independently:

\ea%8
    \label{ex:alexiadou:8}
\ea \gll  fetos  tolmise     \textbf{tesseris}  \textbf{fores} na   pirovolisi  o    Janis\\
      this.year dared.\textsc{3sg}  four    times  \textsc{sbjv} shoot.\textsc{3sg}   the  John.\textsc{nom}\\
\glt  ‘This year there were four times that John dared to shoot.’ 
\ex
\gll fetos  tolmise  na    pirovolisi  \textbf{tesseris}  \textbf{fores} o    Janis\\
     this.year  dared.\textsc{3sg}  \textsc{sbjv} shoot.\textsc{3sg}   four  times  the  John.\textsc{nom}\\
\glt ‘This year John dared to shoot four times (in a row).’
\z
\z

The subject is truly embedded, as it precedes both embedded objects and embedded VP-modifiers. Clause-final event adverbials have the potential of modifying either the matrix verb or the embedded one, depending on where they are situated:



\ea%9
    \label{ex:alexiadou:9}
\ea \gll  ksehase  na    ksevgali  o    Janis       to    pukamiso  \textbf{teseris}  \textbf{fores}\\
      forgot   \textsc{sbjv} rinse   the  John.\textsc{nom} the  shirt        four     times\\
  \glt    ‘John forgot to rinse the shirt four times.’ (\textit{four rinsings\slash forgettings})
\ex  \gll  ksehase  \textbf{teseris}  \textbf{fores} na  ksevgali  o    Janis      to  pukamiso\\
      forgot  four   times  \textsc{sbjv}  rinse  the  John.\textsc{nom} the  shirt\\
   \glt   ‘John forgot four times to rinse the shirt.’ (\textit{four forgettings})
   \z   
\z

This difference in interpretation depends on the adjunction site of the adverb. When it modifies the matrix verb, it (right-)adjoins to the matrix \textit{v}P or TP \REF{ex:alexiadou:10a}. When it modifies the embedded verb, it adjoins to the embedded \textit{v}P or TP \REF{ex:alexiadou:10b}:

\ea%10
    \label{ex:alexiadou:10}
    \ea High reading\label{ex:alexiadou:10a}\\
    \begin{forest} for tree={fit=band}
        [TP
            [V-\textit{v}-T [forgot] ]
            [\textit{v}P [ \textit{v}P [ V-\textit{v} [\sout{forgot}] ] [VP [V [\sout{forgot} ] ] [Subjunctive Complement [to rinse John the shirt,roof] ] ] ] [four times] ]
        ]
    \end{forest}\pagebreak
    \ex Low reading\label{ex:alexiadou:10b}\\
    \begin{forest}
        [TP
            [V-\textit{v}-T [forgot] ]
            [\textit{v}P
                [\textit{v}-V [\sout{forgot}] ]
                [VP
                    [V [\sout{forgot}] ]
                    [Subjunctive Complement
                    [~] [MoodP
                        [to] [TP
                                [V-\textit{v}-T [rinse ]]
                                [\textit{v}P
                                    [\textit{v}P
                                        [o Janis-\textsc{nom}]
                                        [\textit{v}P
                                            [V-\textit{v} [\sout{rinse}] ]
                                            [VP [\sout{rinse} the shirt, roof]]
                                        ]
                                    ] 
                                    [four times]
                                ]
                            ]
                        ]
                    ]
                ]
            ]
        ]
    \end{forest}
    \z
\z    

 
%%please move the includegraphics inside the {figure} environment
%%\includegraphics[width=\textwidth]{OGSVolumeAug2018AlexiadouAnagnostopoulou-img3}

 
%%please move the includegraphics inside the {figure} environment
%%\includegraphics[width=\textwidth]{OGSVolumeAug2018AlexiadouAnagnostopoulou-img4}


Evidence from \textit{negative concord} potentially suggests that in BC the subject does \emph{not} belong to the higher clause and surface to the right of the embedded verb as a result of rightward scrambling. Negative \is{quantifier}quantifiers in \ili{Greek}, a negative \isi{concord} language, must be either in the clause containing sentential \isi{negation} \REF{ex:alexiadou:11a} or in the \isi{c-command} domain of a higher sentential negation \REF{ex:alexiadou:11b}. They cannot be licensed by a negation in a lower clause \REF{ex:alexiadou:11c} (see \citealt{Giannakidou1997}):


\ea%11
    \label{ex:alexiadou:11}
    \ea[]{\gll o  Petros     dietakse  \textit{na}   \textbf{min} apolithi     \textbf{kanis}\\
    the  Peter.\textsc{nom} ordered  \textsc{sbjv} not   was.fired  nobody.\textsc{nom}\\
    \glt ‘Peter ordered that nobody was fired.’\label{ex:alexiadou:11a}}
    \ex[]{\gll o  Petros \textbf{den} dietakse  \textit{na} apolithi   \textbf{kanis}\\
    the  Peter.\textsc{nom} not  ordered  \textsc{sbjv}  was.fired  nobody.\textsc{nom}\\
    \glt ‘Peter did not order that anybody was fired.’\label{ex:alexiadou:11b}}
    \ex[*]{\gll\textbf{kanis}  dietakse  \textit{na} \textbf{min} apolithi  o  Petros\\
    nobody.\textsc{nom}  ordered  \textsc{sbjv}  not  fired.\textsc{nact}  the    Peter.\textsc{nom}\\\label{ex:alexiadou:11c}}
\z
\z
    
    The same pattern is found in OC contexts:

\ea%12
    \label{ex:alexiadou:12}
\ea[]{\gll \textbf{kanis}   \textbf{den} tolmise      \textit{na} fai      to    tiri\\
    nobody.\textsc{nom}  not   dared.\textsc{3sg}  \textsc{sbjv}  eat.\textsc{3sg} the   cheese.\textsc{acc}\\
    \glt ‘Nobody dared to eat the cheese.’}
    \ex[]{\gll \textbf{den}  tolmise    \textit{na} fai       \textbf{kanis} to   tiri\\
    not  dared.\textsc{3sg}  \textsc{sbjv}  eat.\textsc{3sg} nobody  the    cheese\\
    \glt ‘Nobody dared to eat the cheese.’}   
    \ex[*]{\gll\textbf{kanis}  tolmise  \textit{na} \textbf{min}   fai  to   tiri\\
      nobody  dared.\textsc{3sg}  \textsc{sbjv}  not  eat.\textsc{3sg}   the  cheese\\\label{ex:alexiadou:12c}}
    \z
\z    

If the subject in BC constructions were part of the main clause, we would expect BC sentences with a low negation to have exactly the same status as (\ref{ex:alexiadou:12}c), which contains a negative matrix subject and an embedded sentential negation. This is not what we find. There is a clear difference in status between \REF{ex:alexiadou:12c} and its BC counterpart:


\begin{exe}
\addtocounter{xnumi}{-1}
\ex\begin{xlista}\addtocounter{xnumii}{3}% To produce "11d".
\ex[\%]{\gll tolmise  \textit{na}   \textbf{min} fai   \textbf{kanis}   to  tiri\\
    dared.\textsc{3sg}  \textsc{sbjv}  not  eat  nobody  the   cheese\\\label{ex:alexiadou:12d}}
    \end{xlista}
\end{exe}


Even though \REF{ex:alexiadou:12d} is not perfect, it is much better than \REF{ex:alexiadou:12c}. \citet{Alexiadou2010} take this to be evidence that the subject in BC resides in the embedded clause. 

Negative \isi{concord} points to the existence of a higher copy in BC. If such a copy wasn’t present, \REF{ex:alexiadou:12d} should be fully acceptable. Further evidence in support of this comes from the observation that in \ili{Greek}, \textit{nominal secondary predicates and predicative modifiers} like ‘alone’ agree in gender and number with the c-commanding DP they modify:\pagebreak

\ea%13
    \label{ex:alexiadou:13}
    \ili{Greek}\\
\ea \gll  o    \textbf{Janis}        efige  \textbf{panikovlitos} / *-i      \\
             the John.{\NOM}  left    panicking.\textsc{ms} / *-\textsc{fem}\\ 
    \glt        lit. ‘John left in panic.’
\ex \gll o    \textbf{Janis}         irthe   \textbf{monos tu} / *moni tis\\
               the John.{\NOM}  came   alone-\textsc{ms} / *alone-\textsc{fem} her\\
    \glt       ‘John came alone.’  
    \z
\z

In BC constructions, such modifiers can be licensed in the matrix clause, while the DP they modify resides in the embedded clause; see \citet[103--104, examples (36--38)]{Alexiadou2010}. Hence, a silent copy must be present in the higher clause. 

On the basis of these and similar arguments, \citet{Alexiadou2010} thus conclude that \ili{Greek} has BC. Unlike \ili{Tsez}, BC in \ili{Greek} is optional (FC is also permitted). Crucially, all OC verbs in \ili{Greek} and Romanian allow BC, providing a stronger argument for BC.

\citet{Tsakali2017} re-evaluate the empirical picture, using extensive questionnaires. They focus on the following configurations with OC\slash NOC verbs favoring co-reference and NOC verbs that do not favor coreference:


\ea%14
    \label{ex:alexiadou:14}
    \ea V \textit{na} V Subj Obj
    \ex V \textit{na} V Obj Subj
    \z
\z

Their results suggest the following:

\begin{enumerate}
\item OC verbs show obligatory co-reference which can be analyzed as BC.
\item There is no clear contrast between OC and NOC verbs as far as \isi{Principle C} effects are concerned (contra \citealt{Alexiadou2010}). A significant number of speakers allow co-reference with NOC verbs. 
\end{enumerate}

Note that, along with examples like \REF{ex:alexiadou:6} where the embedded subject is nominative, native speakers were also asked to evaluate examples like \REF{ex:alexiadou:15} below involving BC between an embedded dative\slash genitive or accusative \isi{experiencer} and a matrix null (nominative) subject.\pagebreak

\ea%15
    \label{ex:alexiadou:15}
OC verb (verb of knowing)\\
\ea \gll emathe {siga siga} na   tis aresun i operes otan   gnorise   to Jiani\\
learned.\textsc{3sg} gradually \textsc{sbjv} \textsc{cl.dat\slash gen} like.\textsc{3pl} the opera.\textsc{nom.pl} when met.\textsc{3sg} the Jiani.\textsc{acc}\\
\glt ‘She learned gradually to like opera, when she met John.’

Try\slash manage verbs (strongly favoring coreference)\\

\ex \gll  prospathi   na   min   tin   stenahori   i ikonomiki krisi \\
    try.\textsc{3sg} \textsc{sbjv}  \textsc{neg}  \textsc{cl.acc} feel.sad.\textsc{3sg}  the financial crisis.\textsc{nom}\\
    \glt ‘She tries not to feel sad about the financial crisis.’

\ex\gll  katafere   na   min   tin   apasholi    i ikonomiki krisi\\
    manage.\textsc{3sg} \textsc{sbjv}  \textsc{neg}  \textsc{cl.acc} worry.\textsc{3sg} the financial crisis.\textsc{nom}\\
    \glt ‘She managed not to feel anxious about the financial crisis.’

Future referring verb NOC (not favoring coreference)  \\

\ex \gll apofasise   na   min   tin   katavali   i asthenia\\
         decided.\textsc{3sg} \textsc{sbjv}  \textsc{neg} \textsc{cl.acc}   put.down.\textsc{3sg} the illness.\textsc{nom}\\
    \glt ‘She decided not to become depressed by the illness.’

\ex \gll iposhethike   na min   tin   stenahori   pia i   siberifora    tu   jiu    tis \\
         promised.\textsc{3sg} \textsc{sbjv}  \textsc{neg}   \textsc{cl.acc}  feel.sad.\textsc{3sg} anymore the   behavior.\textsc{nom}  the    son.\textsc{gen}   \textsc{cl.poss}\\
    \glt ‘She promised not to feel sad about her son’s behavior.’
\z
\z

The majority of the speakers these authors asked accept examples of the type in \REF{ex:alexiadou:15}, and the rate of ungrammaticality ranges from 1.9--11.1\,\%.  

\begin{enumerate}\setcounter{enumi}{2}
\item The comparison between VSO and VOS order in \textit{na}{}-clauses shows that the preference for the disjoint reading is stronger in VSO orders than in VOS orders, but co-reference is still possible for many speakers, who do not have a significant contrast between VOS and VSO. 
\end{enumerate}

Importantly, \citet{Tsakali2017} show that the \ili{Greek} pattern cannot be analyzed as involving restructuring implemented in terms of remnant movement, as proposed for \ili{Spanish} by \citet{Ordóñez2009} and \citet{Herbeck2013}, and suggested by an anonymous reviewer. Specifically, Ordóñez presents several arguments against a BC analysis for \ili{Spanish}. First of all, he points out that similar patterns are found in structures that are standardly considered not to involve control. This is the case, for instance, in causative and perception verb constructions, where the subject may appear overtly in the post-infinitival position:

\ea%16
    \label{ex:alexiadou:16}
    \gll Ayer     nos   hizo   leer   Juan   el libro.  \\
           yesterday   to.us   make   to.read Juan   the book  \\
    \glt   ‘Yesterday Juan made us read the book.’  
    \z

           

Second, it is not the case that only main subjects are permitted after the \isi{infinitive}, as assumed by the \is{control!backward control}backward control analysis; the object of a main verb may also be inserted in this post-infinitival position with \is{control!object control}object control verbs. This is shown by the orders \textsc{v do inf xp} and \textsc{v inf do xp} in (\ref{ex:alexiadou:17}a--b). Examples (\ref{ex:alexiadou:17}b) and (\ref{ex:alexiadou:17}c) show that main object controllers, just like main subject controllers, can be embedded and appear after the infinitival verb:

\ea%17
    \label{ex:alexiadou:17}
    \ea[]{
    \gll     Obligaron   a Bush   a firmar   los acuerdos de paz.  \\
             obliged.\textsc{3pl}  to Bush   to sign   the agreements of peace  \\
    \glt     ‘They obliged Bush to sign the peace agreement.’}
    \ex[]{
    \gll Obligaron   a firmar   a Bush   los acuerdos de paz.\\
           obliged.\textsc{3pl}   to sign     to Bush   the agreements of peace\\
    \glt   ‘They obliged Bush to sign the peace agreement.’}
    \ex[?]{
    \gll Obligó   a firmar   el Congreso   a Bush   los acuerdos de paz.\\
             obliged.\textsc{3sg}   to sign   the Congress to Bush   the agreements of peace\\
    \glt     ‘The Congress obliged Bush to sign the peace agreement.’}
    \z
\z    

Ordóñez proposes a remnant movement analysis of BC (and restructuring constructions) in the spirit of \citegen{Hinterhölzl2006} and \citegen{Koopman2000} analyses of verbal complexes:

\ea%18
    \label{ex:alexiadou:18}
    \begin{xlista}
    \setcounter{xnumii}{0}
    \ex \gll\relax [\textsubscript{VP} Juan querer [\textsubscript{CP} PRO [\textsubscript{VP} comprar el libro]]]\\
                    ~ Juan to.want ~ PRO ~ to.buy the book\\
    \end{xlista}
    \textsf{\bfseries Step 1}: Movement of the verb \textit{to want} above VP:
    \begin{xlista}
    \setcounter{xnumii}{1}
    \ex \gll\relax [\textsubscript{TP} querer\textsubscript{} Juan V\textsubscript{i} [\textsubscript{TP} PRO [\textsubscript{VP} comprar el libro]]]\\
      ~ to.want Juan ~ ~ ~ ~ to.buy the book\\
    \end{xlista}
    \textsf{\bfseries Step 2}: Movement of the TP above \textit{to want}:
    \begin{xlista}
    \setcounter{xnumii}{2}
    \ex \gll\relax [[\textsubscript{TP} PRO [\textsubscript{VP} comprar el libro]] [\textsubscript{TP} querer\textsubscript{i} [\textsubscript{VP} Juan V\textsubscript{i} ...\\
       ~ ~ ~ to.buy the book ~ to.want ~ Juan\\
    \end{xlista}
    \textsf{\bfseries Step 3}: Scrambling of the object out of TP + movement of the main subject \textit{Juan} to its licensing position above the scrambled object:
    \begin{xlista}
    \setcounter{xnumii}{3}
    \ex \gll\relax  [ Juan\textsubscript{1} el libro\textsubscript{2} [[\textsubscript{TP} PRO [\textsubscript{VP} comprar \textit{t}\textsubscript{2}]] [\textsubscript{TP} querer\textsubscript{i} [\textsubscript{VP} \textit{t}\textsubscript{1} ...\\
                    {} Juan the book ~ ~ ~ to.buy ~ ~ to.want\\
    \end{xlista}
    \textsf{\bfseries Step 4}: Movement of the VP containing \textit{to buy} above the licensing position of subject and object:
    \begin{xlista}
    \setcounter{xnumii}{4}
    \ex \gll\relax [[\textsubscript{VP} comprar \textit{t}\textsubscript{2}] [ Juan el libro [[\textsubscript{TP} PRO] [\textsubscript{TP} querer\textsubscript{i} [\textsubscript{VP} \textit{t} ...\\
                   ~ to.buy ~ ~ Juan the book ~ ~ ~ to.want\\
    \end{xlista}
    \textsf{\bfseries Step 5}: Movement of TP+\textit{querer} to SpecCP and final Spell-Out:
    \begin{xlista}
    \setcounter{xnumii}{5}
    \ex\relax [\textsubscript{CP} [\textsubscript{TP} querer\textsubscript{i} ... [\textsubscript{VP} \textit{t} ...]] [[\textsubscript{VP} comprar \textit{t}\textsubscript{i}] [ Juan el libro [[\textsubscript{TP} PRO \textit{t}\textsubscript{i} ...
    \end{xlista}
    \z

Crucially for \citet{Ordóñez2009}, object scrambling (step 3) is a local movement and cannot cross a finite clause boundary. This explains why there are no comparable verbal complexes formed with finite clauses:

\ea%19
    \label{ex:alexiadou:19}
    \judgewidth{?*}
    \ea[*?]{
    \gll Ayer   les   hizo\textsubscript{i}  [que comprasen Juan\textsubscript{i} el libro].\\
                           yesterday to.them made          that buy.\textsc{3pl} Juan the book\\}
    \ex[]{
    \gll Ayer     les   hizo\textsubscript{i} comprar   Juan\textsubscript{i}   el libro.\\
       yesterday   to.them made   buy.\textsc{inf}   Juan   the book\\}
    \z
\z
    
Further evidence for the scrambling analysis in \ili{Spanish} is provided by the following contrast. In examples involving infinitival \textit{wh}-islands, as discussed by \citet{Torrego1996}, BC and FC behave differently. While the upper copy is available, the lower one is ungrammatical. According to Ordóñez, the ungrammaticality of (\ref{ex:alexiadou:20}a) can be explained, if scrambling out of non-tensed CPs is blocked by filled SpecCPs.

\ea%20
    \label{ex:alexiadou:20}
    \ea Backward control\\
    \gll*? No sabe   si   contestar Juan las cartas. \\
         ~    not   know whether to.answer Juan the letters\\
    \ex Forward control\\Juan no sabe si contestar Juan las cartas.
    \z
\z
           

\citet{Tsakali2017} show that the \ili{Greek} facts are very different: specifically, there is no blocking of VSO orders and BC in OC constructions involving a filled SpecCP; cf. \REF{ex:alexiadou:21}:

\ea%21
    \label{ex:alexiadou:21}
    \gll de kseri          pos na apandisi     o Janis   ta gramata    \\
          not know{}.\textsc{3sg} how \textsc{sbjv} answer   the.\textsc{nom} John{}.\textsc{nom}   the letters{}.\textsc{acc}\\
    \glt  ‘John does not know how to answer the letters.’
    \z

Moreover, embedding of the main object controller is not possible; i.e. here we have an asymmetry between subjects and objects:

\ea%22
    \label{ex:alexiadou:22}
    \ea[]{\gll anagasan      ton Bush na     ipograpsi   ti {sinthiki irinis}\\
        obliged.\textsc{3pl}   the.\textsc{acc} Bush \textsc{sbjv}    sign.\textsc{3sg}   the {peace agreement.\textsc{acc}}\\
        \glt ‘They obliged Bush to sign the peace agreement.’}
    \ex[*]{\gll anagasan   na   ipograpsi   ton Bush   ti {sinthiki irinis}\\
    obliged.\textsc{3pl}   \textsc{sbjv}  sign.\textsc{3sg}    the.\textsc{acc} Bush   the {peace agreement.\textsc{acc}}\\}
    \z
\z   


Furthermore, in \ili{Spanish}, no argument may intervene between finite verbs and \is{infinitive}infinitives with a postverbal subject. This is not the case in \ili{Greek}, where no locality effect is caused by an IO intervener in the matrix clause:

\begin{exe}%23
    \judgewidth{*?}
    \ex[*?]{
    \label{ex:alexiadou:23}
    \gll  les prometió     a los familiares            [darles el jurado la libertad a los prisioneros] \\
           to.them promised to the family.members to.give the jury the liberty      to the prisoners\\}
\end{exe}

  

\ea%24
    \label{ex:alexiadou:24}
    \gll iposhethikan     tis Marias  na  dosun       i dikastes     amnistia   sto filakismeno      andra     tis\\
         promised.\textsc{3pl}    the.\textsc{gen} Maria.\textsc{gen} \textsc{sbjv} give.\textsc{3pl} the judges.\textsc{nom}   amnesty.\textsc{acc} to.the imprisoned husband hers\\
    \glt   ‘The judges promised Mary to give amnesty to her imprisoned husband.’
    \z

As \ili{Greek} lacks clitic climbing, there is no evidence for restructuring (see \citealt{Terzi1992} and others). Moreover, BC is found with all control verbs, not just with a small class (the restructuring class in \ili{Spanish}).

Finally, \citet{Tsakali2017} show that the obviation of \isi{Principle C} effects in embedded VSO constructions is also found \emph{with finite clauses}, as shown in \REF{ex:alexiadou:25b}. Crucially, there is a robust Principle C effect in embedded \textit{that}{}-SVO sequences illustrated in \REF{ex:alexiadou:25a}, indicating that \ili{Greek} does have Principle C effects caused by a matrix null subject when the embedded subject precedes the inflected verb.

\ea%25
    \label{ex:alexiadou:25}
    \ea\label{ex:alexiadou:25a}
    \gll \textit{pro}\textsubscript{*j/k}  emathe   oti   o Petros\textsubscript{j}   kerdise   to lahio \\
         ~ learned{}.\textsc{3sg}   that the.\textsc{nom} Peter{}.\textsc{nom}   won{}.\textsc{3sg} the lottery{}.\textsc{acc} \\
    \glt ‘He/she learned that Peter won the lottery.’
    \ex\label{ex:alexiadou:25b}
        \gll \textit{pro}\textsubscript{j/k}  emathe   oti   kerdise (o Petros\textsubscript{j}) to lahio (o Petros\textsubscript{j})          \\
        ~     learned{}.\textsc{3sg}    that     won{}.\textsc{3sg} (the.\textsc{nom} Peter{}.\textsc{nom}) the lottery{}.\textsc{acc}  (the.\textsc{nom} Peter{}.\textsc{nom})        \\
        \glt ‘He/she learned that Peter won the lottery.’
    \z
\z

We can thus conclude that \ili{Greek} BC configurations do not involve complex predicate formation. While there is evidence for verb clustering in \ili{Spanish}, there is no such evidence in \ili{Greek}. Moreover, in \ili{Greek}, backward co-reference is even allowed within finite clauses unless the subject is in preverbal position.

\citet{Tsakali2017} show that a backward dependency can productively be established in \ili{Greek} provided that the embedded DP subject remains \textit{in situ}. They propose that what has been analysed as BC should not be analysed in terms of movement, because on a movement analysis it would be hard to explain the emergence of a Principle C effect when the subject occurs preverbally.\footnote{One could attempt to save the movement analysis by appealing to improper movement. Under the hypothesis that SVO orders in \ili{Greek} involve Clitic Left Dislocation (CLLD; \citealt{Alexiadou1998}), one could account for the lack of BC in such configurations by analyzing the preverbal position as an A’-position. Such configurations would thus involve an improper A-A’-A movement chain. However, such an analysis would be strongly undermined by the fact that the subject in SVO orders does have A-properties and that CLLD in general has mixed A\slash A’-properties akin to medium-distance scrambling (see \citealt{Miyagawa2017} for relevant discussion).}  For this reason, they propose that \ili{Greek} BC actually reflects $\varphi ${}-agreement between matrix T, embedded T and the overt S(ubject), which can also take place across embedded indicative CPs and is licit only if the S doesn’t intervene between the two T heads, as in (\ref{ex:alexiadou:2}a), repeated below:

\begin{exe}%2
%     \label{ex:alexiadou:2}
\exi{(1)}    
    \ea[]{[   T$\varphi $\textsubscript{k} [\textsubscript{TP/CP}    $T\varphi $\textsubscript{k}    DP$\varphi $\textsubscript{k}  ]]}
    \ex[*]{[ T$\varphi $\textsubscript{k} [\textsubscript{TP/CP}  DP$\varphi $\textsubscript{k}   $T\varphi $\textsubscript{k}]]}
    \z
\end{exe}


\citet{Tsakali2017} relate the availability of \is{agreement!long-distance agreement}long-distance agreement chains as in (\ref{ex:alexiadou:2}a) to the pro-drop status of the language. Their analysis assumes a version of \REF{ex:alexiadou:30}: see \citet{Rizzi1982}, \citet{Alexiadou1998}, \citet{Holmberg2005}, \citet{Barbosa2009}.\footnote{This is called Hypothesis A in \citet{Holmberg2005} and \citet{Barbosa2009}. Holmberg rejects it while Barbosa argues for a version of it, implemented in terms of \citegen{Pesetsky2007} modification of \citegen{Chomsky2001Beyond} theory of \isi{Agree}.} The crucial intuition is that Agr in null subject languages is pronominal and can thus enter \is{agreement!long-distance agreement}long-distance agreement relationships, like pronouns.

\ea%30
    \label{ex:alexiadou:30}
    The set of \is{feature!phi-feature}phi-features in T (Agr) is pronominal in null subject languages (NSLs); Agr is a referential, definite pronoun, albeit a pronoun phonologically expressed as an affix. As such, Agr is also assigned a subject \isi{theta-role}, by virtue of heading a chain whose foot is in \textit{v}P, receiving the relevant \isi{theta-role}.
    \z

          

In order to make \REF{ex:alexiadou:30} compatible with the theory of \isi{Agree}, \citet{Barbosa2009} proposes that the \is{feature!phi-feature}phi-features of T in consistent null subject languages (NSLs) are valued and can therefore value the phi-features\is{feature!phi-feature} of \textit{v}P-internal \textit{pro} in pro-drop configurations. She furthermore proposes that they are uninterpretable\is{feature!uninterpretable feature}, in order to account for the \isi{Agree} relationship they establish with overt or covert subjects which have \is{feature!interpretable feature}interpretable features. If she is correct, then we must assume that they are not deleted until they form a chain with the higher agreement in \is{agreement!long-distance agreement}long-distance agreement chains, which means that \ili{Greek} has phase-suspension in the relevant configurations (see \citealt{Alexiadou2014} for phase-suspension in long-distance \isi{Agree} configurations arising in raising \is{subjunctive}subjunctives); i.e. there is obligatory phase suspension in OC \is{subjunctive}subjunctives and optional phase suspension in NOC \is{subjunctive}subjunctives with BC, and even in indicatives.  

Alternatively, we can maintain that the \is{feature!phi-feature}phi-features on T in \ili{Greek} are pronominal, and this permits them to enter \is{agreement!long-distance agreement}long-distance agreement relationships, even across finite clauses, like pronouns do. Being pronominal, they can either be taken to be interpretable and unvalued (receiving a value either from a null Topic, as argued for in \citet{Frascarelli2007}, or by entering a chain with a higher DP, depending on context), or valued, as Barbosa proposes, but also interpretable.\footnote{Either way, depending on what the facts in other NSLs turn out to be, we might need to parametrize these hypotheses. Specifically, it is well-known that \ili{Romance} \is{subjunctive}subjunctives show obviation, and this seems to correlate with the fact that they have \is{infinitive}infinitives. Thus, obviation in those contexts can be accounted for by appealing to global competition between \is{infinitive}infinitives and \is{subjunctive}subjunctives. But what has not been investigated so far, to our knowledge, is how finite clauses behave. If they consistently show Principle C effects with embedded VSO and VOS orders, then this would indicate that either the \is{feature!phi-feature}phi-features of T are \is{feature!uninterpretable feature}uninterpretable and thus they disappear after local \isi{Agree} with the \textit{v}P-internal subject (as proposed by \citealt{Barbosa2009}), or that phase-hood cannot be suspended in \ili{Romance} indicatives.} 

Turning to the \isi{Agree} relationships established in BSC configurations, \REF{ex:alexiadou:30} holds in the embedded clause of the non-Principle C VSO\slash VOS cases investigated by \citet{Tsakali2017}, as in \REF{ex:alexiadou:31}:

\ea%31
    \label{ex:alexiadou:31}
     \textsubscript{} [\textsubscript{TP/CP}    $T\varphi $\textsubscript{k}    DP$\varphi $\textsubscript{k}  ] 
\z
       

A further \isi{Agree} relationship is established between matrix T and embedded CP; i.e. in the phase-hood version of BSC (see above), C is not an intervener for \isi{Agree}. Following \citet{Rackowski2005}, \citet{Tsakali2017} assume that \is{phase!Phase Impenetrability Condition}PIC\slash \isi{intervention effects} are obviated if a higher head first agrees with \textit{the entire phase} and then continues on to agree with an element \textit{inside} the phase; see also \citet{Halpert2016}. 

\ea%32
    \label{ex:alexiadou:32}
    [   T$\varphi $\textsubscript{k} [\textsubscript{TP/CP}    $T\varphi $\textsubscript{k}    DP$\varphi $\textsubscript{k}  ]]      
\z
    
Matrix T (and the \textit{v}P-internal pro-subject associated with it) agrees with the CP and then with embedded T which agrees with the \textit{v}P-internal subject. Note here that in \ili{Zulu}, as argued in \citet{Halpert2016}, the EPP forces raising of the embedded subject out of the \textit{v}P. DP-raising does not have to take place in \ili{Greek}\slash Romanian, as V-movement satisfies the EPP (\citeauthor{Alexiadou1998}\linebreak\citeyear{Alexiadou1998}), but when the subject occurs pre-verbally a \isi{Principle C} effect arises. \citet{Tsakali2017} suggest that the embedded subject DP is an intervener blocking \isi{Agree} between matrix and embedded T; i.e. \isi{Agree} between heads can happen as long as no DP intervenes between them. When matrix pronominal agreement directly \is{c-command}c-commands a DP with which it shares no thematic index, it gives rise to a standard Principle C effect. This effect does not arise in embedded VSO\slash VOS orders because matrix T forms a chain with embedded T and embedded T shares the same thematic index with the subject DP.\footnote{Note that this analysis is compatible both with analyses taking full DP-subjects to optionally raise to SpecTP in \ili{Greek} (e.g. \citealt{Spyropoulos2009}) and with analyses taking the pre-verbal subject to reside in a CLLD position (\citealt{Alexiadou1998}; \citealt{Barbosa2009} and others). In the latter approach, we can even sharpen the explanation for the Principle C effect, attributing it to the nature of CLLDed elements as topic shifters (cf. \citealt{Frascarelli2007}).}   

On the basis of this discussion, we can submit the following conclusions: what \citet{Alexiadou2010} called BC in \is{subjunctive}subjunctives actually involves the formation of agreement chains. BC (broadly\slash roughly understood as backward co-reference) involves agreement chains rather than actual movement because there is no obvious way of accounting for the asymmetry between embedded SVO vs. VSO orders (evidenced in finite clauses due to the option of SVO orders, which are unavailable in \is{subjunctive}subjunctives for independent reasons having to do with the phonological clitic-like status of \textit{na}) with respect to Principle C effects in a DP-move\-ment approach. When the word order in the embedded clause is SVO, we get a clear Principle C violation, as expected.

In this light, let us now see what happens in \is{control!object control}object control configurations. The question here is the following: if the availability of ‘BC’ in \ili{Greek} is related to the availability of agreement chains of the type described above, are such agreement chains possible in \is{control!object control}object control configurations?

\section{No object BC in Greek} 

\subsection{Introduction}
Similarly to BSC, it has been argued that \is{control!object control}object control can also be subdivided into forward and backward \is{control!object control}object control (BOC):

\ea%33
    \label{ex:alexiadou:33}
    \ea Forward \is{control!object control}object control\\
    \gll I persuaded   Kim\textsubscript{i}     [{△\textsubscript{i}} to smile]  \\
         {}   {}       \textit{controller}         \textit{controllee}\\
    \ex Backward \is{control!object control}object control\\
    \gll I persuaded  △\textsubscript{i}    [Kim\textsubscript{i}    to smile]  \\
        {}    {}       \textit{controllee}  \textit{controller}\\
        \z
\z

BOC is attested in e.g. \ili{Malagasy} (\citealt{Potsdam2006Backward}; \citeyear{Potsdam2009}), \ili{Korean} \citep{Monahan2003}, and \il{Arabic, Omani}Omani Arabic \citep{Al-Balushi2008}. We illustrate the phenomenon with a \ili{Korean} example in \REF{ex:alexiadou:34}. (\ref{ex:alexiadou:34}a) shows that Korean \is{control!object control}object control predicates permit an accusative-nominative alternation. While the accusative is a constituent of the matrix clause, binding a null element in the embedded clause, (\ref{ex:alexiadou:34}b), the nominative resides in the embedded clause and is coindexed with a null element in the matrix, (\ref{ex:alexiadou:34}c):
 
\ea%34
    \label{ex:alexiadou:34} \il{Korean}
    \ea\gll Cheolsu-neun  Yeonghi-leul/ka     kake-e    ka-tolok   seolteukha-eoss-ta\\
            Cheolsu-\textsc{top}     Yeonghi-\textsc{acc/nom}   store-to  go-\textsc{comp}   persuade-\textsc{past-decl}\\
    \glt    ‘Cheolsu persuaded Yeonghi to go to the store.’
    \ex
    \gll Cheolsu-neun Yeonghi-leul\textsubscript{i} [${\bigtriangleup}$\textsubscript{i} kake-e     ka-tolok]  seolteukha-eoss-ta\\
             Cheolsu-\textsc{top}    Yeonghi-\textsc{acc}    {}      store-to   go-\textsc{comp}   persuade-\textsc{past-decl}\\
    \glt     ‘Cheolsu persuaded Yeonghi to go to the store.’     
    \ex
    \gll Cheolsu-neun ${\bigtriangleup}$\textsubscript{i} [Yeonghi-ka\textsubscript{i}    kake-e    ka-tolok]  seolteukha-eoss-ta\\
             Cheolsu-\textsc{top}  {}       Yeonghi-\textsc{nom}    store-to  go-\textsc{comp}      persuade-\textsc{past-decl}\\
    \glt     ‘Cheolsu persuaded Yeonghi to go to the store.’ 
    \z
\z \il{Korean}

Before we turn to the question of whether BOC can be evidenced in \ili{Greek}, we should offer a brief description of the predicates that have been analyzed as \is{control!object control}object control predicates in \ili{Greek}. This is a controversial issue, as these structures are in principle also amenable to an \is{Exceptional Case Marking}ECM analysis; it thus has to be shown that the DP is generated in the object position of the matrix predicate. \citet{Alexiadou1997} addressed this, and we briefly summarize their argumentation here; see also \citet{Kotzoglou2002} and \citet{Kotzoglou2007}. 

\subsection{Object control in Greek}% 3.2 
Constructions that could be analyzed as \is{Exceptional Case Marking}ECM in \ili{Greek} involve perception and causative verbs (cf. \citealt{Burzio1986} for \ili{Italian}):


\ea%35
    \label{ex:alexiadou:35}
    \ea \gll ida          ton  Petro         na    milai     me   tin Ilektra\\
             saw{}.\textsc{1sg}  the   Peter{}.\textsc{acc}  \textsc{sbjv} talk.\textsc{3sg} with  the Ilektra\\
    \glt     ‘I saw Peter talking with Ilektra.’
    \ex
    \gll evala      ton Petro       na     katharisi   to   domatio tu\\ 
             put{}.\textsc{1sg}  the Peter{}.\textsc{acc} \textsc{sbjv}   clean.\textsc{3sg}  the  room     his\\
    \glt     ‘I made Peter clean his room.’
    \z
\z    

\citet{Iatridou1993} treats cases like (\ref{ex:alexiadou:35}a) as instances of \is{control!object control}object control. In fact, Burzio argues against an \is{Exceptional Case Marking}ECM analysis for (\ref{ex:alexiadou:35}a--b) and his arguments also hold for \ili{Greek} (cf. \citealt[287--290]{Burzio1986}). As \citet{Alexiadou1997} point out, unlike tensed\slash infinitival pairs like \textit{I believe that Eric delivered the speech\slash I believe Eric to have delivered the speech}, which are closely synonymous, pairs like \REF{ex:alexiadou:36} below are not synonymous:

\ea%36
    \label{ex:alexiadou:36}
    \ea
    \gll ida          oti   o    Petros         telioni   ti      diatrivi       tu\\
             saw{}.\textsc{1sg}  that  the Peter{}.\textsc{nom}   finishes  the  dissertation his\\
    \glt     ‘I saw that Peter is finishing his dissertation.’
    \ex
    \gll ida          ton  Petro      na     telioni   ti    diatrivi        tu\\
             saw{}.\textsc{1sg}  the  Peter{}.\textsc{acc} \textsc{sbjv} finishes the  dissertation his\\
    \glt     ‘I saw Peter finishing his dissertation.’
    \z
\z

In (\ref{ex:alexiadou:36}b) the phrase corresponding to \textit{Petros} is the object of direct perception, while this is not true of sentences like (\ref{ex:alexiadou:36}a). A related point has to do with the non-synonymy of active and passive forms. While S complements maintain rough synonymy under passivization, as with \textit{I believe Eric to have delivered the speech vs. I believe the speech to have been delivered by Eric}, the cases under discussion are not synonymous, as is evident from the semantic anomaly of the verb \textit{ida} in (\ref{ex:alexiadou:37}b) below:

\ea%37
    \label{ex:alexiadou:37}
    \ea[]{
    \gll ida/akusa                to   Petro        na    ekfoni        to   logo\\
         saw{}.\textsc{1sg}/heard{}.\textsc{1sg}  the  Peter{}.\textsc{acc}  \textsc{sbjv} deliver.\textsc{3sg} the  speech\\
    \glt ‘I saw/heard Peter delivering the speech.’}
    \ex[\#]{
    \gll ida/akusa              to  logo     na     ekfonite       apo  ton Petro\\
         saw{}.\textsc{1sg}/heard{}.\textsc{1sg}  the speech \textsc{sbjv} {be delivered} by   the Peter\\
    \glt ‘I saw/heard the speech being delivered by Peter.’}    
    \z
\z
    
Another standard test for distinguishing ‘\_ NP S' from ‘\_ S' complements involves the relative scope of \is{quantifier}quantifiers. By this test, the structures in question also qualify as \is{Exceptional Case Marking}non-ECM:\footnote{\citet{Alexiadou2016} point out, however, that in the context of perception verbs, the subject of the embedded clause is assigned accusative in the matrix clause, but is licensed by the negation in the subordinate clause. This is compatible with an \is{Exceptional Case Marking}ECM analysis, suggesting that perception verbs behave like \is{Exceptional Case Marking!quasi-ECM}quasi-ECM predicates in \citegen{Kotzoglou2007} terminology.

\ea \gll Bika  mesa    ke me     ekpliksi idha       kanenan         na min dulevi monos tu. Oli ixan             xoristi      se omades.\\
        entered{}.\textsc{1sg} in   and with   surprise saw{}.\textsc{1sg} nobody{}.\textsc{acc}  \textsc{sbjv}  \textsc{neg} work{}.\textsc{3sg} alone his.\textsc{nom}    all had separated into teams\\
    \glt ‘I entered and to my surprise I saw nobody working on his own. They had all separated into teams.’\\
\z
        }

\ea%38
    \label{ex:alexiadou:38}
    \ea They expected one customs official to check all passing cars.\\
    \begin{xlisti}
    \ex They expected that there would be one customs official who would         check all passing cars.
    \ex They expected that, for each passing car, there would be some           customs official or other who would check it.  
    \end{xlisti}
    \ex 
    \gll ida          enan teloniako            na       elenhi   kathe  aftokinito\\
        saw{}.\textsc{1sg}   one   {customs official}  \textsc{sbjv} control  every  car\\
    \glt ‘I saw a customs official controlling every car.’
    \begin{xlisti}
    \ex[]{I saw one customs official who checked every passing car.}
    \ex[*]{I saw that for each passing car there was one customs official who would check it.}
    \end{xlisti}
    \z
\z

Under the assumption that \isi{quantifier} scope is clause-bounded, the difference between (\ref{ex:alexiadou:38}a) and (\ref{ex:alexiadou:38}b) follows if (\ref{ex:alexiadou:38}b) has the two \is{quantifier}quantifiers in different\linebreak clauses. 

  A further argument against the \is{Exceptional Case Marking}ECM analysis comes from \isi{Clitic Left Dislocation} (CLLD). CLLD of CP clauses in \ili{Greek} involves a clitic which is third person singular neuter:

\ea%39
    \label{ex:alexiadou:39}
    \ea
    \gll oti   irthe   o     Petros      den  to       perimena\\
          that  came the  Peter.\textsc{nom}  \textsc{neg}  \textsc{cl.acc} expected{}.\textsc{1sg}\\
    \glt ‘That Peter came, I didn't expect it.’
    \ex
    \gll na    erthi         o   Petros       den   to      vlepo\\
         \textsc{sbjv} come{}.\textsc{3sg} the Peter.\textsc{nom}  \textsc{neg}  \textsc{cl.acc} see{}.\textsc{1sg}\\
    \glt Lit. ‘I do not see it that Peter will come.’
    \z
\z    
 
If perception verbs took an S complement, then we would expect the same clitic to appear in CLLD. However, this is not what we find:

\ea%40
    \label{ex:alexiadou:40}
    \ea[]{
    \gll\relax [ton logo]\textsubscript{i}     na       ekfonite        den  ton\textsubscript{i} akusa\\
             the  speech    \textsc{sbjv} be.delivered  \textsc{neg}  him heard{}.\textsc{1sg}\\
    \glt     ‘The speech being delivered, I did not hear it.’}
    \ex[*]{
    \gll\relax [ton logo    na     ekfonite]\textsubscript{ i}     den  to\textsubscript{i}  akusa\\
              the speech  \textsc{sbjv} be.delivered   \textsc{neg}  it   heard{}.\textsc{1sg}\\}
    \ex[]{
    \gll\relax [ton Petro]\textsubscript{ i}    na    tiganizi psaria  den   ton\textsubscript{i}  ida \\
            the Peter\textsc{{}-acc}  \textsc{sbjv} fry        fish     \textsc{neg}  him  saw{}.\textsc{1sg}\\
    \glt    ‘Peter frying fish, I did not see him.’}
    \ex[*]{
    \gll\relax [ton Petro  na    tiganizi psaria]\textsubscript{i} den  to\textsubscript{i}  ida\\
               the  Peter  \textsc{sbjv} fry         fish     \textsc{neg}  it  saw{}.\textsc{1sg}\\}
    \z
\z  

These examples are grammatical only with a resumptive clitic, which agrees in features with the DP, not with the whole clause.{} 

On the basis of these examples, then, we can conclude that perception verbs are \is{control!object control}object control predicates in \ili{Greek} (but see footnote 6 for a complication). Other \is{control!object control}object control predicates include \textit{pitho} ‘persuade’, \textit{diatazo} ‘order’, \textit{parakalo} ‘beg’, and \textit{voitho}, ‘help’, which all behave similarly to perception verbs; see \REF{ex:alexiadou:41}, which tests CLLD, and \citet{Kotzoglou2002} for discussion:

\ea[*]{%41
    \label{ex:alexiadou:41}
    \gll\relax [ton Jani  na    aposiri      ti    minisi]\textsubscript{i}        to\textsubscript{ i}  episa\\
             the John  \textbf{\textsc{sbjv}} withdraw  the prosecution it  persuaded{}.\textbf{\textsc{1sg}}\\}
    \z

Before we proceed to the behavior of these predicates in terms of BC, we note that \citet{Kotzoglou2007} discuss so-called \is{Exceptional Case Marking!quasi-ECM}quasi-ECM predicates such as \textit{perimeno} ‘expect’ and \textit{thelo} ‘want’. Applying several of the tests for \is{control!object control}object control, as in \REF{ex:alexiadou:42} (their 27b), involving CP doubling, they conclude that these predicates also involve a matrix DP; i.e. they can be subsumed as a case of \is{control!object control}object control. 

\ea[*]{%42
    \label{ex:alexiadou:42}
    \gll to\textsubscript{i} perimena\textsubscript{} [ton  Jani          na    aghapisi  ti       Maria]\textsubscript{ i}\\
         it expected{}.\textsc{1sg}    the  John\textsc{{}.}\textbf{\textsc{acc}} \textsc{sbjv} love{}.\textsc{3sg}    the Maria\textsc{{}.acc}\\
    \glt ‘I expected John to love Maria.’}
    \z

The authors do, however, notice some important differences between \is{Exceptional Case Marking!quasi-ECM}quasi-ECM verbs and \is{control!object control}object control verbs. First, as they state (\citealt{Kotzoglou2007}: 129), “there is a crucial difference in the thematic information that is realized in the \ili{Greek} examples. Object control verbs cannot select a clause as their single argument, while this was shown to be possible in the \is{Exceptional Case Marking!quasi-ECM}quasi-ECM examples.” Moreover, \is{control!object control}object control verbs “always realize the subject matter role as a clause. They thus lack the PP alternate that is attested with verbs of the ‘\is{Exceptional Case Marking!quasi-ECM}quasi-ECM’ type.” A second difference involves \textit{wh}-extraction, which is banned in \ili{Greek} ‘\is{Exceptional Case Marking!quasi-ECM}quasi-ECM’ domains, but is licit out of the \is{control!object control}object control clause; see \REF{ex:alexiadou:43} (their 42):

\ea%43
    \judgewidth{??}
    \label{ex:alexiadou:43}
    \ea[??]{
    \gll pjon    itheles         ton  prothipurgho           na    entiposiasi? \\
         who\textsc{{}.acc} wanted{}.\textsc{2sg} the  prime.minister\textsc{{}.acc} \textsc{sbjv} impress{}.\textsc{3sg} \\
    \glt ‘Who did you want the prime minister to impress?’}
    \ex[]{
    \gll pjon        epises               ton  prothipurgho           na    entiposiasi? \\
         who\textsc{{}.acc}  persuaded{}.\textsc{2sg}  the  prime.minister\textsc{{}.acc} \textsc{sbjv} impress{}.\textsc{3sg} \\
    \glt ‘Who did you persuade the prime minister to impress?’}
    \z
\z

This, in combination with the observation made in \citet{Kotzoglou2007} that the accusative object of \is{Exceptional Case Marking!quasi-ECM}quasi-ECM verbs licenses nominative secondary predicates in the embedded clause, as in \REF{ex:alexiadou:46}, leads us to suggest that \is{Exceptional Case Marking!quasi-ECM}quasi-ECM configurations actually involve movement of the embedded DP to the CP level, where it is assigned accusative by the matrix predicate. This is an instance of an edge-effect in \citegen{Baker2015} terminology:

\ea%46
    \label{ex:alexiadou:46}
    \gll perimena        to Jani           na    ine arostos/*arosto \\
         expected{}.\textsc{1sg} the John\textsc{{}.acc} \textsc{sbjv} be sick\textsc{{}.nom}/*\textsc{{}.acc} \\
    \glt ‘I expected John to be sick.’ 
    \z


In \REF{ex:alexiadou:46}, the DP is first assigned nominative in the lower clause, and then accusative, after movement, at the CP level. This means that accusative, which we treat following \citet{Marantz1991} and \citet{Baker2015} as dependent case, can be assigned on top of a case assigned lower, inside the embedded clause. As Baker notes, there is cross-linguistic variation as to whether multiple realization is possible. 

Note that from the perspective of the ‘control as movement’ theory, the derivation of \REF{ex:alexiadou:46} is similar, if not identical, to that of control predicates. In both cases, the DP raises from the embedded clause to the matrix clause, where it is assigned dependent accusative. The difference between the two might presumably be related to the fact that in \REF{ex:alexiadou:46} the DP raises to SpecCP, where it is frozen, while in the \is{control!object control}object control cases, it raises higher, to the matrix \textit{v}P, in order to be receive a thematic role. However, on the basis of our argumentation in §3 regarding \citegen{Tsakali2017} results, it is crucial that there is movement in so-called \is{Exceptional Case Marking!quasi-ECM}quasi-ECM environments, but not in control configurations. 

\subsection{Greek lacks BOC}% 3.3 

Interestingly, none of the \is{control!object control}object control verbs in \ili{Greek} allows BOC. The movement analysis of control would predict that the lower copy is spelled out as nominative; i.e. that it bears the case of the embedded clause. However, the examples in (\ref{ex:alexiadou:47}b) and ({ex:alexiadou:48}b--c) are all ungrammatical:



\ea%47
    \label{ex:alexiadou:47}
    \ea[]{
    \gll i     Maria  epise         to   Jani          na     hamogelasi\\
         the Mary  persuaded  the John.\textsc{acc}  \textsc{sbjv} smile\textsc{{}.3sg}\\
    \glt ‘Mary persuaded John to smile.’}
    \ex[*]{
    \gll i   Maria (ton)      epise        na     homogelasi   o    Janis \\
         the Mary  (\textsc{cl.acc}) persuaded \textsc{sbjv} smile\textsc{{}.3sg}      the John\textsc{{}.nom} \\}
    \z
\z    


\ea%48
    \label{ex:alexiadou:48}
    \ea[]{
    \gll i     Maria         voithise to    Jani   na   simazepsi   to   domatio tu\\
         the Mary\textsc{{}.nom}  helped   the  John \textsc{sbjv} tidy.up\textsc{{}.3sg}  the room      his\\
    \glt ‘Mary helped John to tidy up his room.’}
    \ex[]{
    \gll i      Maria        voithise  na    simazepsi  o    Janis        to  domatio tu\\
         the Mary\textsc{{}.nom}  helped    \textsc{sbjv} tidy.up\textsc{{}.3sg} the John\textsc{{}.nom} the room     his\\
    \glt [\textit{good but not on the reading where she helped John}]}
    \ex[*]{
    \gll I Maria (ton) voithise na simazepsi  o    Janis        to  domatio tu\\
         the Mary\textsc{{}.nom} (\textsc{cl.acc}) helped \textsc{sbjv} tidy.up\textsc{{}.3sg} the John\textsc{{}.nom} the room     his\\   }
    \z
\z
% %     [F0B9?] % This sign was printed, maybe an artefact?
 
On the \is{control!backward control}backward control analysis, this asymmetry is puzzling and unexpected. If, however, control does not involve movement, as \citet{Tsakali2017} argue, then the observed asymmetry boils down to configurations that enable co-ref\-er\-ence; i.e. the formation of \is{agreement!long-distance agreement}long-distance agreement chains of the type we described in §3. 

For \ili{Greek}, the above behavior seems to suggest that the distribution of BC patterns is related to the presence of \textit{pro}. \ili{Greek} has subject \textit{pro} and allows BSC. By contrast, \ili{Greek} lacks object \textit{pro} (\citealt{Giannakidou1997}) and disallows BOC. While this would be in agreement with our conclusions in §3, \citet{Potsdam2006Backward,Potsdam2009} argues that this does not hold across languages, as Malagasy lacks object \textit{pro} but allows BOC. One of the arguments Potsdam brings against the \textit{pro} analysis in Malagasy involves variable binding. As he points out, the \textit{pro} analysis would predict that a bound variable interpretation for the controller-controllee relation should be impossible, as there is no \isi{c-command}. However, the example in \REF{ex:alexiadou:49}, involving a distributed universal \isi{quantifier}, shows that variable binding is possible in \is{control!backward control}backward control. Thus, it seems that the controller and controllee must be in a \isi{c-command} relationship to obtain the right configuration for binding. 

\ea%49
    \label{ex:alexiadou:49}
    \gll boky inona avy no nanontania- nao hovidian’ ny mpianatra tsirairay?\\
         book what each \textsc{foc}  ask.\textsc{ct}  you buy.\textsc{tt} the student each\\
    \glt ‘For each \textit{x}, \textit{x} a student, which book did you ask \textit{x} to buy?’ (\citealt{Potsdam2006Backward}: ex. (17a))
    \z

We can thus maintain that Malagasy has BOC control, and that the availability of object \textit{pro} does not correlate with the availability of BOC in true BC-as-movement languages. But, crucially, \ili{Greek} was argued in §3 not to be such a language.

The only cases of BOC that seem possible in \ili{Greek} involve a Gen\slash Dat or Acc object realized as a clitic and a Gen\slash Dat or Acc \isi{experiencer} in the embedded clause, a pattern that seems similar to that of resumption; see \tabref{tab:alexiadou:1}. Note that (\ref{ex:alexiadou:47}b--\ref{ex:alexiadou:48}c) remain ungrammatical in spite of the presence of a clitic in the matrix clause:\largerpage

\ea%50
    \label{ex:alexiadou:50}
    \ea
    \gll o    Janis    \textbf{tu}  epevale / ton katafere   na   \textbf{tu} aresi  \textbf{tu}  \textbf{Kosta}          i opera.\\
                   the John.\textsc{nom}   \textsc{cl.gen}  imposed / \textsc{cl.acc} managed  \textsc{sbjv} \textsc{cl.gen}  like   the Kostas.\textsc{gen} the opera\\
    \glt           ‘John imposed on Kostas to like the opera\slash convinced Kostas to like the opera.’
    \ex
    \gll o    Janis    \textbf{tu}  epevale / ton katafere   na   \textbf{ton} efxaristi  \textbf{ton}  \textbf{Kosta}         i opera.\\
           the John.\textsc{nom} \textsc{cl.gen}  imposed / \textsc{cl.acc} managed  \textsc{sbjv} \textsc{cl.acc}  please the Kostas.\textsc{gen} the opera\\
    \glt   ‘John imposed on Kostas to like the opera\slash convinced Kostas to like the opera.’
    \z
\z

Let us consider now the configuration for OC in comparison to our analysis of BSC: in the case of forward control, an \isi{Agree} relationship must be established between matrix \isi{Voice} and matrix DP and subsequently the \is{feature!phi-feature}phi-features of T in the embedded CP.    

\ea%51
    \label{ex:alexiadou:51}
    [\textsubscript{CP} [\textsubscript{VoiceP} [ DP\textsubscript{$\varphi $}\textsubscript{k} [\textsubscript{TP/CP}    $T\varphi $\textsubscript{k}    ]]]]
    \z

If the \is{feature!phi-feature}phi-features of embedded T are unvalued, we can follow \citet{Grano2016}, building on \citet{Kratzer2009}, and \citet{Landau2015}, who propose two variants for analyzing such configurations, (\ref{ex:alexiadou:52}a--b):


\ea%52
    \label{ex:alexiadou:52}
    \ea
    \begin{xlisti}
    \ex An unvalued pronoun can be valued via feature transmission.
    \ex Transmission of phi-features piggybacks on predication.
    \ex A complement clause can be turned into a predicate via Fin.
    \ex Transmission proceeds from antecedent to Fin and from Fin to [Spec,FinP].
    \end{xlisti}
    \ex
    \begin{xlisti}
    \ex An unvalued pronoun can be valued via feature transmission.
    \ex Transmission of \is{feature!phi-feature}phi-features piggybacks on binding.
    \ex Binding is mediated by verbal functional heads.
    \ex C and \textit{v} intervene for each other in the way they transmit features.
    \end{xlisti}
    \z
\z

On the latter approach, a matrix binder transmits features onto embedded C, and embedded C binds and values an unvalued pronoun in its \isi{c-command} domain.

In forward \is{control!object control}object control configurations, we usually have a genitive or an accusative in the matrix clause that controls the nominative subject of the embedded verb. As we see in \REF{ex:alexiadou:53}, the DP \textit{John} bears accusative, assigned by the matrix predicate. The presence of a nominative modifier in the embedded clause suggests that it has been assigned nominative in that context. Thus, it bears two cases, but only one is realized.

\ea%53
    \label{ex:alexiadou:53}
    \gll vlepo to    Jani        na   pezi        basket  {monos tu}.\\
         see     the John.\textsc{acc} \textsc{sbjv} play.\textsc{3sg} basket  alone.\textsc{nom}\\
    \glt ‘I see John playing basketball alone.’
    \z

This is a so-called multiple-case-marked A-chain similar to the kind discussed for Niuean in \citet[67]{Bejar1999}.

For backward \is{control!object control}object control, what we would need first, similarly to what we outlined for the BSC cases, is for the \isi{Agree} relation to hold within the embedded clause:

\ea%54
    \label{ex:alexiadou:54}
    \textsubscript{} [\textsubscript{TP/CP}    $T\varphi $\textsubscript{k}    DP$\varphi $\textsubscript{k}]  
\z
 
While in the case of subject co-reference the \isi{Agree} chain ultimately holds between two T heads, the matrix and the embedded one, in the case of \is{control!object control}object control the embedded T head must enter \isi{Agree} with the matrix \isi{Voice} head, and this configuration seems generally illegitimate (cf. \citealt{Kayne1989}). We believe that part of the reason for this is the different requirements that T and \isi{Voice} impose. T has been argued to have pronominal \is{feature!phi-feature}phi-features while \isi{Voice} doesn’t: \ili{Greek} is not a rich object agreement, object-drop language, which can be taken to mean that the \is{feature!phi-feature}phi-features of embedded T are not allowed to enter \is{agreement!long-distance agreement}long-distance agreement with the \is{feature!phi-feature}phi-features of the matrix \isi{Voice}. 

But we have seen that this is exceptionally possible if the embedded clause has a dative or accusative \is{clitic!clitic doubling}clitic doubling the \isi{experiencer} and the matrix \isi{Voice} hosts a dative or accusative clitic; i.e. in cases of ‘resumption’ crucially involving an \isi{experiencer} in the downstairs clause. This leads us to formulate the hypothesis in \REF{ex:alexiadou:55} as a condition for BC:\footnote{An anonymous reviewer suggests two alternative hypotheses to us, (i) and (ii).

\ea \glt In a chain with multiple case positions, realize the copy with the more marked case (ACC\slash GEN > NOM).\\
\z
\ea \glt In a chain with multiple case positions, realize the higher copy. If both positions are assigned the same case, the lower copy can be realized.\\
\z    

The second hypothesis would capture the fact that BSC is possible when the lower clause contains an \isi{experiencer} and the higher clause a null \textit{pro} bearing nominative, as was seen in the examples in \REF{ex:alexiadou:16}, but it would have to be reformulated in terms of agreement chains if control does not involve movement, as we suggest in §3. (i) can be reformulated as suggesting that only a dependent case in the sense of \citet{Marantz1991} and \citet{Baker2015} must be realized (see \citealt{Anagnostopoulou2017} for arguments that \ili{Greek} GEN is a dependent case).}

\ea%55
    \label{ex:alexiadou:55}
    Backward \isi{Agree} applies to heads of the same type. 
\z

In the BOC cases at hand, the relationship is between a clitic in the embedded clause and a clitic in the matrix clause. Note that when the downstairs \isi{experiencer} surfaces as a nominative DP, backward co-reference seems to us to be degraded:\footnote{Because these facts have not been investigated before, we are relying on our own intuitions. They need to be checked with a large number of speakers via extensive questionnaires, just as \citet{Tsakali2017} did with the BSC constructions. The same applies to the data discussed immediately below.} 

\ea[\#]{%56
    \label{ex:alexiadou:56}
    \gll o    Janis    \textbf{tu}  epevale / ton katafere   na   \textbf{efxaristiete}  \textbf{o}  \textbf{Kostas}   me tin     opera.\\
         the John{}.\textsc{nom}  \textsc{cl.gen}  imposed / \textsc{cl.acc} managed \textsc{sbjv} please.\textsc{nact}   the Kostas{}.\textsc{nom} with the opera\\
    \glt ‘John imposed on Kostas to like the opera\slash convinced Kostas to like the opera.’}
    \z

Moreover, note that if the clitic-doubled argument in the embedded clause is not an \isi{experiencer}, \is{coreference!backward coreference}backward coreference is not possible (this is indicated by \# in the passive (\ref{ex:alexiadou:57}a), featuring a clitic-doubled goal, which is well-formed in the non-coreference reading, and by ?? in (\ref{ex:alexiadou:57}b), featuring an affected argument combined with an \isi{unaccusative}, which seems to us to admit the coreference reading but to be degraded compared to the \isi{experiencer} cases mentioned above):

\ea%57
\judgewidth{??}
    \label{ex:alexiadou:57}
    \ea[\#]{\gll o    Janis    \textbf{tu}  epevale / ton katafere       na   \textbf{tu} dothi \textbf{tu}  \textbf{Kosta}         to danio.\\
                   the John{}.\textsc{nom} \textsc{cl.gen} imposed / \textsc{cl.acc} managed \textsc{sbjv} \textsc{cl.gen} give.\textsc{nact} the Kostas{}.\textsc{gen} the loan\\
    \glt           ‘John imposed on him for a loan to be given to Kostas.’}
    \ex[??]{
     \gll o    Janis    \textbf{tu}  epevale / ton katafere   na min  \textbf{tu} pesi  \textbf{tu}  \textbf{Kosta}     to vazo. \\
          the John{}.\textsc{nom}   \textsc{cl.gen}  imposed / \textsc{cl.acc} managed  \textsc{sbjv} \textsc{neg} \textsc{cl.gen}  fall   the Kostas{}.\textsc{gen} the vase\\
     \glt ‘John imposed on Kostas not to drop the vase.’   }
    \z
\z    

This seems to suggest that \is{coreference!backward coreference}backward coreference of this type is not only subject to the condition in \REF{ex:alexiadou:55}, but requires, in addition, that the embedded clitic-doubled argument encode point of view. Perhaps this is so because only \is{experiencer}experiencers qualify as subjects at some level of representation, which means that they relate to T (\citealt{Anagnostopoulou1999} for \ili{Greek}; \citealt{Landau2010Locative}). 

\section{Conclusion}% 4. 

In this paper, we have discussed an asymmetry in the distribution of \is{control!backward control}backward control in \ili{Greek}. While the language has been argued to have BSC, it lacks BOC. As we pointed out, \citet{Tsakali2017} have recently argued that BSC in \ili{Greek} is a side effect of the availability of an agreement chain between a null main subject and an overt embedded subject in all types of \is{subjunctive}subjunctives (\textit{na}{}-clauses), and to a certain extent in indicatives (\textit{that}{}-clauses). If this is the correct analysis for BSC, the question still remains whether \ili{Greek} has BOC. We showed in this paper that BOC configurations are severely limited. We related this limitation to the nature of Backward \isi{Agree}, which seems to require heads of the same type. In BOC configurations, the \is{feature!phi-feature}phi-features of embedded T are not allowed to enter \is{agreement!long-distance agreement}long-distance agreement with the \is{feature!phi-feature}phi-features of the matrix \isi{Voice}. Backward co-reference is only possible in case of resumption with a dative\slash genitive clitic in the matrix clause and a clitic-doubled \isi{experiencer} in the embedded clause, and crucially depends on the \isi{experiencer} status of the embedded argument.

\section*{Acknowledgements}

We are indebted to the editors of this volume and an anonymous reviewer for very insightful comments that helped us restructure this paper. AL 554\slash 10-1 is hereby acknowledged.

% \section{ References}
% 
% 
% Agouraki, Yoryia. 1991 A Modern \ili{Greek} \isi{complementizer} and its significance for Universal Grammar. \textit{UCL Working Papers in Linguistics} 3. 1–24.
% 
% 
% 
% Al-Balushi, Rashid. 2008. Control in Omani Arabic. Ms. (Toronto: University of Toronto.)
% 
% 
% Alexiadou, Artemis. 1999. On the properties of some \ili{Greek} word order patterns. In Alexiadou, Artemis \& Horrocks, Geoffrey \& Stavrou, Melita (eds.), \textit{Studies in Greek} syntax, 45–65. Dordrecht: Kluwer.
% 
% Alexiadou, Artemis. 2000. Some remarks on word order and information structure in \ili{Romance} and \ili{Greek}.~\textit{ZAS Papers in Linguistics} 20. 119–136.
% 
% 
% Alexiadou, Artemis \& Anagnostopoulou, Elena. 1997. Notes on ECM, control and raising. \textit{ZAS Papers in Linguistics} 8. 17–27.
% 
% 
% 
% Alexiadou, Artemis \& Anagnostopoulou, Elena. 1998. Parametrizing Agr: Word order, V- movement and EPP-checking. \textit{Natural Language \& Linguistic Theory} 16. 491–539.
% 
% 
% 
% Alexiadou, Artemis \& Anagnostopoulou, Elena. 2016. Rethinking the nature of \is{case!nominative case}nominative case. Ms. (Berlin: Humboldt-Universität zu Berlin \& Rethymnon: University of Crete.)
% 
% 
% 
% Alexiadou, Artemis \& Anagnostopoulou, Elena \& Iordachioaia, Gianina \& Marchis, Mihaela. 2010.\href{http://ifla.uni-stuttgart.de/files/BC paper.pdf}{ No objection to \is{control!backward control}backward control.} In Hornstein, Norbert \& Polinsky, Maria (eds.), \textit{Movement theory of control}, 89–118. Amsterdam: John Benjamins.
% 
% 
% 
% Alexiadou, Artemis \& Anagnostopoulou, Elena \& Wurmbrand, Susi. 2014. Movement vs. long-distance \isi{Agree} in raising: Disappearing phases and feature valuation.~\textit{Proceedings of NELS~}43. 1–12.
% 
% 
% 
% Anagnostopoulou, Elena. 1999. On experiencers. In Alexiadou, Artemis \& Horrocks, Geoffrey \& Stavrou, Melita (eds.), \textit{Studies in \ili{Greek} syntax}, 67–93. Dordrecht: Kluwer.~
% 
% 
% 
% Anagnostopoulou, Elena. 2003. \textit{The syntax of ditransitives: Evidence from clitics}. Berlin: Mouton de Gruyter.
% 
% 
% 
% Anagnostopoulou, Elena \& Sevdali, Christina. 2017. Two modes of dative and \is{case!genitive case}genitive case assignment: Evidence from two stages of \ili{Greek}. Ms. (Rethymnon: University of Crete \& Jordanstown: University of Ulster.)
% 
% 
% 
% Baker, Mark. 2015. \textit{Case: Its principles and its parameters}. Cambridge: Cambridge University Press. 
% 
% 
% Barbosa, Pilar. 2009. Two kinds of subject \textit{pro}. \textit{Studia Linguistica} 63. 2–58.
% 
% 
% Bejar, Susana \& Massam, Diane. 1999. Multiple case checking. \textit{Syntax} 2. 66–79.
% 
% 
% 
% Burzio, Luigi. 1986. \textit{\ili{Italian} syntax}. Dordrecht: Foris.
% 
% 
% 
% Chomsky, Noam. 2001. Beyond explanatory adequacy. In Belletti, Adriana (ed.), \textit{Structures and beyond}, 104–131. Oxford: Oxford University Press.
% 
% 
% Frascarelli, Mara. 2007. Subjects, topics and the interpretation of referential \textit{pro}. \textit{Natural Language \& Linguistic Theory} 25. 691–734.
% 
% 
% Fukuda, Shinichiro. 2008. Backward control. \textit{Language and Linguistics Compass} 2. 168–195.
% 
% 
% 
% Giannakidou, Anastasia. 1997. \textit{The landscape of polarity items}. Groningen: University of Groningen. (Doctoral dissertation.)
% 
% 
% Giannakidou, Anastasia \& Merchant, Jason. 1997. On the interpretation of null \isi{indefinite} objects in \ili{Greek}. \textit{Studies in \ili{Greek} Linguistics} 17. 290–303.
% 
% Grano, Thomas \& Lasnik, Howard. 2016. How to neutralize a finite clause boundary: Phase theory and the grammar of bound pronouns. Ms. (Bloomington, IN: Indiana University \& College Park, MD: University of Maryland.)
% 
% Halpert, Claire. 2016. Raising parameters. \textit{Proceedings of WCCFL 33}. 186–195.
% 
% 
% Herbeck, Peter. 2013. On \is{control!backward control}backward control and clitic climbing: On the deficiency of non-finite domains in \ili{Spanish} and Catalan. \textit{Proceedings of ConSOLE XXI, 2013}. 123–145.
% 
% 
% Hinterhölzl, Roland. 2006. \textit{Scrambling, remnant movement, and restructuring in West Germanic}. Oxford: Oxford University~Press.~
% \textstyleFootnoteSymbol{}
% 
% Holmberg, Anders. 2005. Is there a little pro? Evidence from Finnish. \textit{Linguistic Inquiry} 36. 533–564.
% 
% 
% 
% 
% Hornstein, Norbert. 1999. Movement and control. \textit{Linguistic Inquiry} 30. 69–96.
% 
% 
% 
% Iatridou, Sabine. 1993. On Nominative Case assignment and a few related things. \textit{MIT Working Papers in Linguistics} 19. 175–198.
% 
% 
% Koopman, Hilda \& Szabolcsi, Anna. 2000. \textit{Verbal complexes}. Cambridge, MA: MIT Press.~
% 
% 
% Kotzoglou, George. 2002. \ili{Greek} ‘ECM’ and how to control it. \textit{Reading Working Papers in Linguistics} 6. 39–56. 
% 
% 
% Kotzoglou, George \& Papangeli, Dimitra. 2007. Not really ECM, not exactly control: The ‘\isi{quasi-ECM}’ construction in \ili{Greek}. In Davies, William D. \& Dubinsky, Stanley (eds.), \textit{New horizons in the analysis of control and raising}, 111–132. Dordrecht: Springer. 
% 
% Kratzer, Angelika. 2009. Making a pronoun: Fake indexicals as windows into the properties of pronouns. \textit{Linguistic Inquiry} 40. 187–237.
% 
% 
% Landau, Idan. 1999. \textit{Elements of control}. Cambridge, MA: MIT. (Doctoral dissertation.)
% 
% 
% 
% Landau, Idan. 2004. The scale of finiteness and the calculus of control. \textit{Natural Language \& Linguistic Theory} 22. 811–877.
% 
% 
% Landau, Idan. 2007. Movement-resistant aspects of control. In Davies, William D. \& Dubinsky, Stanley (eds.), \textit{New horizons in the analysis of control and raising}, 293–325. Dordrecht: Springer.
% 
% Landau, Idan. 2010. \textit{The \isi{locative} syntax of experiencers}. Cambridge, MA: MIT Press.
% 
% 
% Landau, Idan. 2015. \textit{A two-tiered theory of control}. Cambridge, MA: MIT Press.
% 
% 
% Marantz, Alec. 1991. Case and licensing. (Paper presented at the 8th Eastern States  Conference on Linguistics, University of Maryland, Baltimore.)
% 
% Miyagawa, Shigeru. 2017. \textit{Agreement beyond phi}. Cambridge, MA: MIT Press.
% 
% 
% Monahan, Philip J. 2003. Backward \is{control!object control}object control in Korean. \textit{Proceedings of WCCFL 22}. 356–369.
% 
% 
% Ordóñez, Francisco. 2009. Verbal complex formation and overt subjects in infinitivals in \ili{Spanish}. Ms. (Stony Brook, NY: Stony Brook University.)
% 
% 
% Philippaki-Warburton, Irene \& Veloudis, Jannis. 1984. The \isi{subjunctive} in complement clauses. \textit{Studies in \ili{Greek} Linguistics} 5. 87–104.
% 
% 
% Pesetsky, David \& Torrego, Esther. 2007. The syntax of valuation and the interpretability of features. In Karimi, Simin \& Samiian, Vida \& Wilkins, Wendy K. (eds.), \textit{Phrasal and clausal architecture: Syntactic derivation and interpretation}, 262–294. Amsterdam: John Benjamins.
% 
% 
% Polinsky, Maria \& Potsdam, Eric. 2002. Backward control. \textit{Linguistic Inquiry} 33. 245–282.
% 
% 
% Polinsky, Maria \& Potsdam, Eric. 2006. \href{http://users.clas.ufl.edu/potsdam/papers/Syntax9.pdf}{Expanding the scope of control and raising}. \textit{Syntax} 9. 171–192.
% 
% Potsdam, Eric. 2006. Backward \is{control!object control}object control in Malagasy: Against an empty category analysis. \textit{Proceedings of WCCFL} \textit{25}. 328–336. 
% 
% 
% Potsdam, Eric. 2009. Malagasy backward \is{control!object control}object control. \textit{Language} 85. 754–784.
% 
% 
% Rackowski, Andrea \& Richards, Norvin. 2005. Phase edge and extraction: A Tagalog case study. \textit{Linguistic Inquiry} 36. 565–599.
% 
% Rizzi, Luigi. 1982. \textit{Issues in \ili{Italian} syntax}. Dordrecht: Foris.
% 
% Roussou, Anna. 2009. In the mood for control. \textit{Lingua} 119. 1811–1836.
% 
% Spyropoulos, Vassilios. 2007. Finiteness and control in \ili{Greek}. In Davies, William D. \& Dubinsky, Stanley (eds.), \textit{New horizons in the analysis of control and raising}, 159–183. Springer.
% 
% 
% Spyropoulos, Vassilios \& Revithiadou, Anthi. 2009. Subject chains in \ili{Greek} and PF processing. In Halpert, Claire \& Hartman, Jeremy \& Hill, David (eds.), \textit{Proceedings of the MIT Workshop in \ili{Greek} Syntax and Semantics}, 293–309. Cambridge, MA: MITWPL.
% 
% 
% 
% Torrego, Esther. 1996. On \isi{quantifier} float in control clauses. \textit{Linguistic Inquiry} 27. 111–126.
% 
% 
% Tsakali, Vina \& Anagnostopoulou, Elena \& Alexiadou, Artemis. 2017. A new pattern of CP transparency: Implications for the analysis of Backward Control. (Paper presented at GLOW 40, Leiden.)
% 
% 
% Tsoulas, George. 1993. Remarks on the structure and the interpretation of \textit{na}{}-clauses. \textit{Studies in \ili{Greek} Linguistics} 14. 191–206.
% 
% 
% 
% Varlokosta, Spyridoula. 1994. \textit{Issues on Modern \ili{Greek} sentential complementation}. College Park, MD: University of Maryland. (Doctoral dissertation.)
% 
% 
% 
% \begin{verbatim}%%move bib entries to  localbibliography.bib
% \end{verbatim} 
% 
% Rackowski, Andrea \& Richards, Norvin. 2005. Phase edge and extraction: A Tagalog case study. \textit{Linguistic Inquiry} 36. 565–599.
% 
% Rizzi, Luigi. 1982. \textit{Issues in \ili{Italian} syntax}. Dordrecht: Foris.
% 
% Roussou, Anna. 2009. In the mood for control. \textit{Lingua} 119. 1811–1836.
% 
% Spyropoulos, Vassilios. 2007. Finiteness and control in \ili{Greek}. In Davies, William D. \& Dubinsky, Stanley (eds.), \textit{New horizons in the analysis of control and raising}, 159–183. Springer.
% 
% 
% Spyropoulos, Vassilios \& Revithiadou, Anthi. 2009. Subject chains in \ili{Greek} and PF processing. In Halpert, Claire \& Hartman, Jeremy \& Hill, David (eds.), \textit{Proceedings of the MIT Workshop in \ili{Greek} Syntax and Semantics}, 293–309. Cambridge, MA: MITWPL.
% 
% 
% 
% Torrego, Esther. 1996. On \isi{quantifier} float in control clauses. \textit{Linguistic Inquiry} 27. 111–126.
% 
% 
% Tsakali, Vina \& Anagnostopoulou, Elena \& Alexiadou, Artemis. 2017. A new pattern of CP transparency: Implications for the analysis of Backward Control. (Paper presented at GLOW 40, Leiden.)
% 
% 
% Tsoulas, George. 1993. Remarks on the structure and the interpretation of \textit{na}{}-clauses. \textit{Studies in \ili{Greek} Linguistics} 14. 191–206.
% 
% 
% 
% Varlokosta, Spyridoula. 1994. \textit{Issues on Modern \ili{Greek} sentential complementation}. College Park, MD: University of Maryland. (Doctoral dissertation.)
% 
% 
% 
% \begin{verbatim}%%move bib entries to  localbibliography.bib
% \end{verbatim} 
% ll-formed in the non-coreference reading, and by ?? in (57b), featuring an affected argument combined with an \isi{unaccusative}, which seems to us to admit the coreference reading but to be degraded compared to the \isi{experiencer} cases mentioned above):
% 
% \ea%57
%     \label{ex:alexiadou:57}
%     \gll\\
%         \\
%     \glt
%     \z
% 
%           a.  \#O    Janis    \textbf{tu}  epevale/ton katafere       na   \textbf{tu} dothi \textbf{tu  Kosta}         to danio.
% 
% the John{}-\textsc{nom} \textsc{cl-gen} imposed/\textsc{cl-acc} managed \textsc{sbjv cl-gen} give-\textsc{nact} the Kostas{}-\textsc{gen} the loan
% 
%     ‘John imposed on him for a loan to be given to Kostas.’
% 
%   b.  ??O    Janis    \textbf{tu}  epevale/ton katafere   na min  \textbf{tu} pesi  \textbf{tu  Kosta}     to vazo.
% 
% the John{}-\textsc{nom}   \textsc{cl-gen}  imposed/\textsc{cl-acc} managed  \textsc{sbjv neg cl-gen}  fall   the Kostas{}-\textsc{gen} the vase’
% 
%     ‘John imposed on Kostas not to drop the vase.’
% 
% This seems to suggest that \is{coreference!backward coreference}backward coreference of this type is not only subject to the condition in (55), but requires, in addition, that the embedded clitic-doubled argument encode point of view. Perhaps this is because only experiencers qualify as subjects at some level of representation, which means that they relate to T (\citealt{Anagnostopoulou1999} for \ili{Greek}, \citealt{Landau2010Locative}). 
% 
% \section{ 4. Conclusion}
% 
% In this paper, we have discussed an asymmetry in the distribution of \is{control!backward control}backward control in \ili{Greek}. While the language has been argued to have BSC, it lacks BOC. As we pointed out, recently TAA (2017) argued that BSC in \ili{Greek} is a side effect of the availability of an agreement chain between a null main subject and an overt embedded subject in all types of subjunctives (\textit{na}{}-clauses), and to a certain extent in indicatives (\textit{that}{}-clauses). If this is the correct analysis for BSC, the question still remains whether \ili{Greek} has BOC. We showed in this paper that BOC configurations are severely limited. We related this limitation to the nature of Backward \isi{Agree}, which seems to require heads of the same type. In BOC configurations, the phi-features of embedded T are not allowed to enter \is{agreement!long-distance agreement}long-distance agreement with the phi-features of the matrix \isi{Voice}. Backward co-reference is only possible in case of resumption with a dative/genitive clitic in the matrix clause and a clitic-doubled \isi{experiencer} in the embedded clause, and crucially depends on the \isi{experiencer} status of the embedded argument.
% 
% \section{ Acknowledgements}
% 
% We are indebted to the editors of this volume and an anonymous reviewer for very insightful comments that helped us restructure this paper. AL 554/10-1 is hereby acknowledged.
% 
% \section{ References}
% 
% 
% Agouraki, Yoryia. 1991 A Modern \ili{Greek} \isi{complementizer} and its significance for Universal Grammar. \textit{UCL Working Papers in Linguistics} 3. 1–24.
% 
% 
% 
% Al-Balushi, Rashid. 2008. Control in Omani Arabic. Ms. (Toronto: University of Toronto.)
% 
% 
% Alexiadou, Artemis. 1999. On the properties of some \ili{Greek} word order patterns. In Alexiadou, Artemis \& Horrocks, Geoffrey \& Stavrou, Melita (eds.), \textit{Studies in Greek} syntax, 45–65. Dordrecht: Kluwer.
% 
% Alexiadou, Artemis. 2000. Some remarks on word order and information structure in \ili{Romance} and \ili{Greek}.~\textit{ZAS Papers in Linguistics} 20. 119–136.
% 
% 
% Alexiadou, Artemis \& Anagnostopoulou, Elena. 1997. Notes on ECM, control and raising. \textit{ZAS Papers in Linguistics} 8. 17–27.
% 
% 
% 
% Alexiadou, Artemis \& Anagnostopoulou, Elena. 1998. Parametrizing Agr: Word order, V- movement and EPP-checking. \textit{Natural Language \& Linguistic Theory} 16. 491–539.
% 
% 
% 
% Alexiadou, Artemis \& Anagnostopoulou, Elena. 2016. Rethinking the nature of \is{case!nominative case}nominative case. Ms. (Berlin: Humboldt-Universität zu Berlin \& Rethymnon: University of Crete.)
% 
% 
% 
% Alexiadou, Artemis \& Anagnostopoulou, Elena \& Iordachioaia, Gianina \& Marchis, Mihaela. 2010.\href{http://ifla.uni-stuttgart.de/files/BC paper.pdf}{ No objection to \is{control!backward control}backward control.} In Hornstein, Norbert \& Polinsky, Maria (eds.), \textit{Movement theory of control}, 89–118. Amsterdam: John Benjamins.
% 
% 
% 
% Alexiadou, Artemis \& Anagnostopoulou, Elena \& Wurmbrand, Susi. 2014. Movement vs. long-distance \isi{Agree} in raising: Disappearing phases and feature valuation.~\textit{Proceedings of NELS~}43. 1–12.
% 
% 
% 
% Anagnostopoulou, Elena. 1999. On experiencers. In Alexiadou, Artemis \& Horrocks, Geoffrey \& Stavrou, Melita (eds.), \textit{Studies in \ili{Greek} syntax}, 67–93. Dordrecht: Kluwer.~
% 
% 
% 
% Anagnostopoulou, Elena. 2003. \textit{The syntax of ditransitives: Evidence from clitics}. Berlin: Mouton de Gruyter.
% 
% 
% 
% Anagnostopoulou, Elena \& Sevdali, Christina. 2017. Two modes of dative and \is{case!genitive case}genitive case assignment: Evidence from two stages of \ili{Greek}. Ms. (Rethymnon: University of Crete \& Jordanstown: University of Ulster.)
% 
% 
% 
% Baker, Mark. 2015. \textit{Case: Its principles and its parameters}. Cambridge: Cambridge University Press. 
% 
% 
% Barbosa, Pilar. 2009. Two kinds of subject \textit{pro}. \textit{Studia Linguistica} 63. 2–58.
% 
% 
% Bejar, Susana \& Massam, Diane. 1999. Multiple case checking. \textit{Syntax} 2. 66–79.
% 
% 
% 
% Burzio, Luigi. 1986. \textit{\ili{Italian} syntax}. Dordrecht: Foris.
% 
% 
% 
% Chomsky, Noam. 2001. Beyond explanatory adequacy. In Belletti, Adriana (ed.), \textit{Structures and beyond}, 104–131. Oxford: Oxford University Press.
% 
% 
% Frascarelli, Mara. 2007. Subjects, topics and the interpretation of referential \textit{pro}. \textit{Natural Language \& Linguistic Theory} 25. 691–734.
% 
% 
% Fukuda, Shinichiro. 2008. Backward control. \textit{Language and Linguistics Compass} 2. 168–195.
% 
% 
% 
% Giannakidou, Anastasia. 1997. \textit{The landscape of polarity items}. Groningen: University of Groningen. (Doctoral dissertation.)
% 
% 
% Giannakidou, Anastasia \& Merchant, Jason. 1997. On the interpretation of null \isi{indefinite} objects in \ili{Greek}. \textit{Studies in \ili{Greek} Linguistics} 17. 290–303.
% 
% Grano, Thomas \& Lasnik, Howard. 2016. How to neutralize a finite clause boundary: Phase theory and the grammar of bound pronouns. Ms. (Bloomington, IN: Indiana University \& College Park, MD: University of Maryland.)
% 
% Halpert, Claire. 2016. Raising parameters. \textit{Proceedings of WCCFL 33}. 186–195.
% 
% 
% Herbeck, Peter. 2013. On \is{control!backward control}backward control and clitic climbing: On the deficiency of non-finite domains in \ili{Spanish} and Catalan. \textit{Proceedings of ConSOLE XXI, 2013}. 123–145.
% 
% 
% Hinterhölzl, Roland. 2006. \textit{Scrambling, remnant movement, and restructuring in West Germanic}. Oxford: Oxford University~Press.~
% \textstyleFootnoteSymbol{}
% 
% Holmberg, Anders. 2005. Is there a little pro? Evidence from Finnish. \textit{Linguistic Inquiry} 36. 533–564.
% 
% 
% 
% 
% Hornstein, Norbert. 1999. Movement and control. \textit{Linguistic Inquiry} 30. 69–96.
% 
% 
% 
% Iatridou, Sabine. 1993. On Nominative Case assignment and a few related things. \textit{MIT Working Papers in Linguistics} 19. 175–198.
% 
% 
% Koopman, Hilda \& Szabolcsi, Anna. 2000. \textit{Verbal complexes}. Cambridge, MA: MIT Press.~
% 
% 
% Kotzoglou, George. 2002. \ili{Greek} ‘ECM’ and how to control it. \textit{Reading Working Papers in Linguistics} 6. 39–56. 
% 
% 
% Kotzoglou, George \& Papangeli, Dimitra. 2007. Not really ECM, not exactly control: The ‘\isi{quasi-ECM}’ construction in \ili{Greek}. In Davies, William D. \& Dubinsky, Stanley (eds.), \textit{New horizons in the analysis of control and raising}, 111–132. Dordrecht: Springer. 
% 
% Kratzer, Angelika. 2009. Making a pronoun: Fake indexicals as windows into the properties of pronouns. \textit{Linguistic Inquiry} 40. 187–237.
% 
% 
% Landau, Idan. 1999. \textit{Elements of control}. Cambridge, MA: MIT. (Doctoral dissertation.)
% 
% 
% 
% Landau, Idan. 2004. The scale of finiteness and the calculus of control. \textit{Natural Language \& Linguistic Theory} 22. 811–877.
% 
% 
% Landau, Idan. 2007. Movement-resistant aspects of control. In Davies, William D. \& Dubinsky, Stanley (eds.), \textit{New horizons in the analysis of control and raising}, 293–325. Dordrecht: Springer.
% 
% Landau, Idan. 2010. \textit{The \isi{locative} syntax of experiencers}. Cambridge, MA: MIT Press.
% 
% 
% Landau, Idan. 2015. \textit{A two-tiered theory of control}. Cambridge, MA: MIT Press.
% 
% 
% Marantz, Alec. 1991. Case and licensing. (Paper presented at the 8th Eastern States  Conference on Linguistics, University of Maryland, Baltimore.)
% 
% Miyagawa, Shigeru. 2017. \textit{Agreement beyond phi}. Cambridge, MA: MIT Press.
% 
% 
% Monahan, Philip J. 2003. Backward \is{control!object control}object control in Korean. \textit{Proceedings of WCCFL 22}. 356–369.
% 
% 
% Ordóñez, Francisco. 2009. Verbal complex formation and overt subjects in infinitivals in \ili{Spanish}. Ms. (Stony Brook, NY: Stony Brook University.)
% 
% 
% Philippaki-Warburton, Irene \& Veloudis, Jannis. 1984. The \isi{subjunctive} in complement clauses. \textit{Studies in \ili{Greek} Linguistics} 5. 87–104.
% 
% 
% Pesetsky, David \& Torrego, Esther. 2007. The syntax of valuation and the interpretability of features. In Karimi, Simin \& Samiian, Vida \& Wilkins, Wendy K. (eds.), \textit{Phrasal and clausal architecture: Syntactic derivation and interpretation}, 262–294. Amsterdam: John Benjamins.
% 
% 
% Polinsky, Maria \& Potsdam, Eric. 2002. Backward control. \textit{Linguistic Inquiry} 33. 245–282.
% 
% 
% Polinsky, Maria \& Potsdam, Eric. 2006. \href{http://users.clas.ufl.edu/potsdam/papers/Syntax9.pdf}{Expanding the scope of control and raising}. \textit{Syntax} 9. 171–192.
% 
% Potsdam, Eric. 2006. Backward \is{control!object control}object control in Malagasy: Against an empty category analysis. \textit{Proceedings of WCCFL} \textit{25}. 328–336. 
% 
% 
% Potsdam, Eric. 2009. Malagasy backward \is{control!object control}object control. \textit{Language} 85. 754–784.
% 
% 
% Rackowski, Andrea \& Richards, Norvin. 2005. Phase edge and extraction: A Tagalog case study. \textit{Linguistic Inquiry} 36. 565–599.
% 
% Rizzi, Luigi. 1982. \textit{Issues in \ili{Italian} syntax}. Dordrecht: Foris.
% 
% Roussou, Anna. 2009. In the mood for control. \textit{Lingua} 119. 1811–1836.
% 
% Spyropoulos, Vassilios. 2007. Finiteness and control in \ili{Greek}. In Davies, William D. \& Dubinsky, Stanley (eds.), \textit{New horizons in the analysis of control and raising}, 159–183. Springer.
% 
% 
% Spyropoulos, Vassilios \& Revithiadou, Anthi. 2009. Subject chains in \ili{Greek} and PF processing. In Halpert, Claire \& Hartman, Jeremy \& Hill, David (eds.), \textit{Proceedings of the MIT Workshop in \ili{Greek} Syntax and Semantics}, 293–309. Cambridge, MA: MITWPL.
% 
% 
% 
% Torrego, Esther. 1996. On \isi{quantifier} float in control clauses. \textit{Linguistic Inquiry} 27. 111–126.
% 
% 
% Tsakali, Vina \& Anagnostopoulou, Elena \& Alexiadou, Artemis. 2017. A new pattern of CP transparency: Implications for the analysis of Backward Control. (Paper presented at GLOW 40, Leiden.)
% 
% 
% Tsoulas, George. 1993. Remarks on the structure and the interpretation of \textit{na}{}-clauses. \textit{Studies in \ili{Greek} Linguistics} 14. 191–206.
% 
% 
% 
% Varlokosta, Spyridoula. 1994. \textit{Issues on Modern \ili{Greek} sentential complementation}. College Park, MD: University of Maryland. (Doctoral dissertation.)
% 
% 
% 
% \begin{verbatim}%%move bib entries to  localbibliography.bib
% \end{verbatim} 
% 
{\sloppy
\printbibliography[heading=subbibliography,notkeyword=this]
}
\end{document}
