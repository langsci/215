\documentclass[output=paper]{langsci/langscibook} 
\author{Artemis Alexiadou\affiliation{Humboldt-Universität zu Berlin}\lastand Elena Anagnostopoulou\affiliation{Leibniz-ZAS, Berlin \& University of Crete}}
\title{An asymmetry in backward control: subject vs. object control}
% \chapterDOI{} %will be filled in at production

% % \epigram{Change epigram in chapters/01.tex or remove it there }
\abstract{In this paper we discuss an asymmetry in the distribution of backward control in Greek. Greek has been argued to have subject backward control, however, as we will show, the language lacks backward object control. We will account for this asymmetry by appealing to case conflicts of the type discussed in the context of free relatives.}
\maketitle

\begin{document}

%%please move the includegraphics inside the {figure} environment
%%\includegraphics[width=\textwidth]{OGSVolumeAug2018AlexiadouAnagnostopoulou-img1}

 
%%please move the includegraphics inside the {figure} environment
%%\includegraphics[width=\textwidth]{OGSVolumeAug2018AlexiadouAnagnostopoulou-img2}

\section{Aims and goals}

In this paper, we discuss backward control configurations, focusing on Greek, a language showing an asymmetry between backward subject control (BSC), which is fully productive, and backward object control (BOC), which is severely limited. This is a puzzling state of affairs if Greek indeed has backward control understood as movement and spell-out of the lower copy of the chain, as has been argued in the literature. Based on new evidence, we argue that the movement approach to Greek BSC is an illusion. The correct analysis involves the formation of a chain between the phi-features of the matrix T, the phi-features of the embedded T and those of the embedded subject, which is possible as long as the embedded subject does not intervene between the matrix and the embedded T. The formation of such chains is possible due to the fact that Greek has pronominal agreement, being a pro-drop language (\citealt{Alexiadou1998}, \citealt{Barbosa2009}). The formation of comparable chains is severely restricted in BOC configurations, which are only possible if the full embedded subject is either a clitic-doubled experiencer bearing dative or accusative case or an emphatic nominative anaphoric pronoun. We will discuss potential reasons why this should be so from the perspective of current approaches to Agree. 

The paper is structured as follows. We first briefly summarize the arguments in Alexiadou, Anagnostopoulou, \citet{Iordachioaia2010} that Greek has backward subject control (BSC), as well as more recent arguments, recently presented in Tsakali, \citet{Anagnostopoulou2017}, that this type of phenomenon does not involve scrambling and indeed instantiates agreement chains between a matrix T and an embedded subject. We then discuss the environments that have been argued to show object control in Greek and point out that there is an asymmetry between BSC (possible) as opposed to backward object control (BOC) (generally impossible) in Greek. We attribute the lack of BOC to the general unavailability of chain formation between a lower T and a higher Voice/vAPPL head, which can be overridden under certain conditions.

\section{Introduction}%1

As has been discussed in the work of \citeauthor{Polinsky2006} (\citeyear{Polinsky2006}; henceforth ‘P\& P’), the movement analysis of control, put forth in \citet{Hornstein1999}, coupled with the copy-and-delete theory of movement, predicts that next to canonical/forward control patterns, where the lower copy of the moved element is deleted, there should also exist backward control patterns, where the higher copy is deleted. A third possibility, which we do not consider in this section, is resumption, where both copies are pronounced, as depicted in \tabref{tab:alexiadou:1}.
    
\begin{table}
\begin{tabular}{lll}
\lsptoprule
Higher copy pronounced & Lower copy pronounced &  Structure\\\midrule
\ding{51} & * & Forward Control (FC)\\
* & \ding{51} & Backward Control (BC)\\
\ding{51} & \ding{51} & Resumption\\
\lspbottomrule
\end{tabular}
\caption{Typology of control and raising in P\& P (\citeyear{Polinsky2006})\label{tab:alexiadou:1}}
\end{table}

A lot of evidence has been provided in the literature for BC, which can be observed in several unrelated languages. For instance, BSC can be observed in several Nahk-Dagestanian languages, in Northwest Caucasian, in Malagasy, and in Korean; see e.g. \citegen{Fukuda2008} overview. The claim that BC exists in natural language is the strongest argument brought by the movement analysis of control against the PRO-based approach; see e.g. \citet{Landau1999} and subsequent work.

In Alexiadou, Anagnostopoulou, Iordachioaia and Marchis (2010; henceforth ‘AAIM’), we addressed \citegen{Landau2007} objections to BSC. One of the objections raised in \citet{Landau2007} concerned the rarity of the phenomenon in one of the languages in which BC has been argued to exist, namely Tsez: in Tsez, only \textbf{two} verbs display BC. In other languages, the numbers hardly exceed five. Most commonly, the BC verbs are aspectuals (\textit{begin}, \textit{continue}, \textit{stop}), which also have a standard raising analysis. On the basis of Greek and Romanian control constructions, we argued that BC is real in these two languages, as it is exhibited by the same verbs that allow OC (hence the ‘rarity’ objection doesn’t hold for Greek and Romanian). 

Recently, a re-evaluation of the empirical picture was put forth in Tsakali, Anagnostopoulou \& Alexiadou (2017; henceforth ‘TAA’) that can be summarized as follows: what has been analyzed as BSC in Greek, Romanian and Spanish is an illusion. In Spanish, it involves complex predicate formation, while in Greek/Romanian it involves co-reference with an embedded subject. Specifically, BC in Greek is a side-effect of the availability of an agreement chain between a null main subject and an overt embedded subject in all types of subjunctives (\textit{na}{}-clauses) and, to a certain extent, in indicatives (\textit{that}{}-clauses). While backward coreference is allowed in both types of clauses if the order is VSO or VOS, embedded SVO orders, which are available in indicatives, lead to a robust Principle C effect. TAA thus propose that what has been analysed as BC actually reflects $\varphi ${}-agreement between matrix T, embedded T and the overt S(ubject), licit only if the S doesn’t intervene between the two T heads, as in \REF{ex:alexiadou:2a}:

\ea%2
    \label{ex:alexiadou:2}
    \ea[]{[T$\varphi $\textsubscript{k} [\textsubscript{TP/CP} $T\varphi $\textsubscript{k} DP$\varphi $\textsubscript{k}]]\label{ex:alexiadou:2a}}
    \ex[*]{[T$\varphi $\textsubscript{k} [\textsubscript{TP/CP} DP$\varphi $\textsubscript{k} $T\varphi $\textsubscript{k}]]\label{ex:alexiadou:2b}}
    \z
\z    

In what follows, we summarize both aspects of this discussion. Nevertheless, as we will show in §3, such co-reference is not available in the case of object control.

\section{BSC in Greek: An epiphenomenon}%2

In Greek, control is instantiated in a subset of subjunctive complement clauses, as the language lacks infinitives; see e.g. \citet{Varlokosta1994} and references therein. These subjunctive complement clauses are introduced by the subjunctive marker \textit{na} \REF{ex:alexiadou:3}. The embedded verb, similarly to the matrix verb, shows agreement in number and person with the matrix subject.\footnote{\textit{Na} has been analyzed as a subjunctive mood marker (cf. Philippaki-\citealt{Warburton1984}), a subjunctive complementizer (\citealt{Tsoulas1993}, \citealt{Aggouraki1991}) or a device to check EPP \citep{Roussou2009}. Here we side with the first view.} 

\ea%3
 \label{ex:alexiadou:3}
\langinfo{Greek}{}{}\\
\gll O  Petros/ego  kser-i/-o        \textbf{na}    koliba-i/-o\\
  the    Peter.\textsc{nom}/I  know-\textsc{3sg}/-\textsc{1sg}    \textsc{sbjv}  swim-\textsc{3sg}/-\textsc{1sg}\\
\glt  ‘Peter/I knows/know how to swim.’
\z

The literature on Greek control recognizes two main types of subjunctive complements (but cf. \citealt{Spyropoulos2007a} and \citealt{Roussou2009} for refinements): Obligatory Control (OC) ones and non-OC ones (NOC) (or \textit{C(ontrolled)-subjunctives} and \textit{F(ree)-subjunctives} in \citegen{Landau2004} terminology).


1. \textbf{OC/C-subjunctives} are found as complements of verbs such as \textit{ksero} ‘know how’, \textit{tolmo} ‘dare’, \textit{herome} ‘be happy’, \textit{ksehno} ‘forget’, \textit{thimame} ‘remember’, \textit{matheno} ‘learn’, \textit{dokimazo} ‘try’, aspectual verbs such as \textit{arhizo} ‘start/begin’, \textit{sinehizo} ‘continue’.



\ea%4
    \label{ex:alexiadou:4}
  \ea[*]{\gll o Petros      kseri     na   \textbf{kolimbao}\\
      the Peter.\textsc{nom}   knows  \textsc{sbjv}   swim.\textbf{\textsc{1}}\textsc{sg}\\
     \glt   Lit. ‘Peter knows how I swim.’}
   \ex[*]{\gll o  Petros    kseri    na   \textbf{kolimbai}   I    Maria\\
       the Peter.\textsc{nom}  knows  \textsc{sbjv}   swim.\textbf{\textsc{3}}\textsc{sg} the   Mary.\textsc{nom}\\
     \glt   Lit. ‘Peter knows how Mary swims.’}
\z
\z



2. \textbf{NOC/F-subjunctives} are found with e.g. volitional/future-referring predicates:



\ea%5
    \label{ex:alexiadou:5}
    \ea\gll O    Petros      perimeni  na    \textbf{erthun}\\
       the    Peter.\textsc{nom} expects  \textsc{sbjv}  come.\textsc{3pl}\\
        \glt ‘Peter expects that they come.’
     \ex\gll  o     Petros    elpizi  na    figi     i     \textbf{Maria}\\
          the    Peter.\textsc{nom}   hopes  \textsc{sbjv}  go.\textsc{3sg}  the    Mary.\textsc{nom}\\
         \glt ‘Peter hopes that Mary goes.’
\z
\z

\citet{AAIM2010} present evidence that \textbf{all} OC verbs in Greek allow BC. In fact, the subject DP can appear in a number of positions (here Greek differs from Tsez). Preverbal subjects are considered to be in a left-dislocated position, while post-verbal subjects are located within the vP; see \citet{Alexiadou1998} for discussion. VSO and VOS orders have different information structure properties; see \citet{Alexiadou1999,Alexiadou2000} for discussion.  Generally, the DP in the subjunctive complement agrees with both the low and the matrix verb in person and number:



\ea%6
    \label{ex:alexiadou:6}
    \gll (O \textbf{Janis}) emathe (o \textbf{Janis}) na pezi  (o \textbf{Janis}) kithara (o \textbf{Janis})\\
         the John.\textsc{nom} learned.\textsc{3sg} the John.\textsc{nom sbjv} play.\textsc{3sg} the John.\textsc{nom} guitar the John.\textsc{nom}\\
    \glt    ‘John learned to play the guitar.’
    \z

The pattern in which the DP resides in the complement clause qualifies as a case of BC on the basis of P\& P’s argumentation. First, these constructions are bi-clausal (contra \citealt{Roussou2009}), as can be shown on the basis of evidence from \textit{negation} and \textit{event modification}.

-\textit{Two separate negations} are possible:



\ea%7
    \label{ex:alexiadou:7}
\ea
\gll    \textbf{Den}  emathe    na    magirevi   o    Janis\\
    not   learned.\textsc{3sg  sbjv}  cook.\textsc{3sg}    the    John.\textsc{nom}\\
\glt    ‘John didn’t learn to cook.’

\ex
\gll    Emathe     na   \textbf{min}   magirevi    o    Janis\\
    learned.\textsc{3sg}  \textsc{sbjv} not  cook.\textsc{3sg} the   John.\textsc{nom}\\
\glt ‘John learned not to cook (i.e. ‘John got into the habit of not cooking’).’

\ex
\gll    \textbf{Den}   emathe    na   \textbf{min} magirevi    o    Janis\\
    not  learned.\textsc{3sg}  \textsc{sbjv} not  cook.\textsc{3sg}    the   John.\textsc{nom}\\
\glt    ‘John didn’t learn not to cook (i.e. ‘John still has the habit of cooking’).’
\z
\z


{}- The \textit{event of each clause can be modified independently}:

\ea%8
    \label{ex:alexiadou:8}
\ea \gll  Fetos  tolmise     \textbf{tesseris}  \textbf{fores} na   pirovolisi  o    Janis\\
      this.year dared.\textsc{3sg}  four    times  \textsc{sbjv} shoot.\textsc{3sg}   the  John.\textsc{nom}\\
\glt  ‘This year there were four times that John dared to shoot.’ 
\ex
\gll Fetos  tolmise  na    pirovolisi  \textbf{tesseris}  \textbf{fores} o    Janis\\
     this.year  dared.\textsc{3sg}  \textsc{sbjv} shoot.\textsc{3sg}   four  times  the  John.\textsc{nom}\\
\glt ‘This year John dared to shoot four times (in a row).’
\z
\z

\textbf{The subject is truly embedded, as it precedes both} \textit{embedded objects} and \textit{embedded VP-modifiers}. Clause-final event adverbials have the potential of modifying either the matrix verb or the embedded one, depending on where they are situated:



\ea%9
    \label{ex:alexiadou:9}
\ea \gll  ksehase  na    ksevgali  o    Janis       to    pukamiso  \textbf{teseris}  \textbf{fores}\\
      forgot   \textsc{sbjv} rinse   the  John.\textsc{nom} the  shirt        four     times\\
  \glt    ‘John forgot to rinse the shirt four times.’ (\textit{four rinsings/forgettings})
\ex  \gll  ksehase  \textbf{teseris}  \textbf{fores} na  ksevgali  o    Janis      to  pukamiso\\
      forgot  four   times  \textsc{sbjv}  rinse  the  John.\textsc{nom} the  shirt\\
   \glt   ‘John forgot four times to rinse the shirt.’ (\textit{four forgettings})
   \z   
\z

This difference in interpretation depends on the adjunction site of the adverb. When it modifies the matrix verb, it (right-)adjoins to the matrix vP or TP (10a). When it modifies the embedded verb, it adjoins to the embedded vP or TP (10b):

\ea%10
    \label{ex:alexiadou:10}
    \ea High reading\\
    \begin{forest} for tree={fit=band}
        [TP
            [V-\textit{v}-T [forgot] ]
            [\textit{v}P [ \textit{v}P [ V-\textit{v} [\st{forgot}] ] [VP [V [\st{forgot} ] ] [Subjunctive Complement [to rinse John the shirt,roof] ] ] ] [four times] ]
        ]
    \end{forest}    
    \ex Low reading\\
    \begin{forest}
        [TP
            [V-\textit{v}-T [forgot] ]
            [\textit{v}P
                [\textit{v}-V [\st{forgot}] ]
                [VP
                    [V [\st{forgot}] ]
                    [Subjunctive Complement
                    [~] [MoodP
                        [to] [TP
                                [V-\textit{v}-T [rinse ]]
                                [\textit{v}P
                                    [\textit{v}P
                                        [o Janis-NOM]
                                        [\textit{v}P
                                            [V-\textit{v} [\st{rinse}] ]
                                            [VP [\st{rinse} the shirt, roof]]
                                        ]
                                    ] 
                                    [four times]
                                ]
                            ]
                        ]
                    ]
                ]
            ]
        ]
    \end{forest}
    \z
\z    

 
%%please move the includegraphics inside the {figure} environment
%%\includegraphics[width=\textwidth]{OGSVolumeAug2018AlexiadouAnagnostopoulou-img3}

 
%%please move the includegraphics inside the {figure} environment
%%\includegraphics[width=\textwidth]{OGSVolumeAug2018AlexiadouAnagnostopoulou-img4}


Evidence from \textit{negative concord} potentially suggests that in BC the subject does \textbf{not} belong to the higher clause and surface to the right of the embedded verb as a result of rightward scrambling. Negative quantifiers in Greek, a negative concord language, must be either in the clause containing sentential negation (11a) or in the c-command domain of a higher sentential negation (11b). They cannot be licensed by a negation in a lower clause (11c) (see \citealt{Giannakidou1997}):


\ea%11
    \label{ex:alexiadou:11}
    \ea[]{\gll O  Petros     dietakse  \textit{na}   \textbf{min} apolithi     \textbf{kanis}\\
    the  Peter.\textsc{nom} ordered  \textsc{sbjv} not   was.fired  nobody.\textsc{nom}\\
    \glt ‘Peter ordered that nobody was fired.’\label{ex:alexiadou:11a}}
    \ex[]{\gll O  Petros \textbf{den} dietakse  \textit{na} apolithi   \textbf{kanis}\\
    the  Peter.\textsc{nom} not  ordered  \textsc{sbjv}  was.fired  nobody.\textsc{nom}\\
    \glt ‘Peter did not order that anybody was fired.’\label{ex:alexiadou:11b}}
    \ex[*]{\gll\textbf{Kanis}  dietakse  \textit{na} \textbf{min} apolithi  o  Petros\\
    nobody.\textsc{nom}  ordered  \textsc{sbjv}  not  fired.\textsc{nact}  the    Peter.\textsc{nom}\\\label{ex:alexiadou:11c}}
\z
\z
    
    The same pattern is found in OC contexts:

\ea%12
    \label{ex:alexiadou:12}
\ea[]{\gll \textbf{Kanis}   \textbf{den} tolmise      \textit{na} fai      to    tiri\\
    nobody.\textsc{nom}  not   dared.\textsc{3sg  sbjv}  eat.\textsc{3sg} the   cheese.\textsc{acc}\\
    \glt ‘Nobody dared to eat the cheese.’}
    \ex[]{\gll \textbf{Den}  tolmise    \textit{na} fai       \textbf{kanis} to   tiri\\
    not  dared.\textsc{3sg  sbjv}  eat.\textsc{3sg} nobody  the    cheese\\
    \glt ‘Nobody dared to eat the cheese.’}   
    \ex[*]{\gll\textbf{Kanis}  tolmise  \textit{na} \textbf{min}   fai  to   tiri\\
      nobody  dared.\textsc{3sg  sbjv}  not  eat.\textsc{3sg}   the  cheese\\}
    \z
\z    

If the subject in BC constructions were part of the main clause, we would expect BC sentences with a low negation to have exactly the same status as (12c), which contains a negative matrix subject and an embedded sentential negation. This is not what we find. There is a clear difference in status between (12c) and its BC counterpart:


\begin{exe}
\addtocounter{xnumi}{-1}
\ex\begin{xlista}\addtocounter{xnumii}{3}% To produce "11d".
\ex[\%]{\gll Tolmise  \textit{na}   \textbf{min} fai   \textbf{kanis}   to  tiri\\
    dared.\textsc{3sg}  \textsc{sbjv}  not  eat  nobody  the   cheese\\\label{ex:alexiadou:12d}}
    \end{xlista}
\end{exe}


Even though \REF{ex:alexiadou:12d} is not perfect, it is much better than (12c). \citet{AAIM2010} take this to be evidence that the subject in BC resides in the embedded clause. 

Negative concord points to the existence of a higher copy in BC. If such a copy wasn’t present, (12d) should be fully acceptable. Further evidence in support of this comes from the observation that in Greek, \textit{nominal secondary predicates and predicative modifiers} like ‘alone’ agree in gender and number with the c-commanding DP they modify:

\ea%13
    \label{ex:alexiadou:13}
    \langinfo{Greek}{}{}
\ea \gll  O    \textbf{Janis}        efige  \textbf{panikovlitos}/*-i      \\
             the John-NOM  left    panicking-MS/*-FEM\\ 
    \glt        lit. ‘John left in panic.’
\ex \gll O    \textbf{Janis}         irthe   \textbf{monos tu}/*moni tis\\
               the John-NOM  came   alone-MS/*alone-FEM\\
    \glt       ‘John came alone.’  
    \z
\z


In BC constructions, such modifiers can be licensed in the matrix clause, while the DP they modify resides in the embedded clause; see AAIM 2010: 103-104, examples (36)-(38). Hence, a silent copy must be present in the higher clause. 

On the basis of these and similar arguments, \citet{AAIM2010} thus conclude that Greek has BC. Unlike Tsez, BC in Greek is optional (FC is also permitted). Crucially, all OC verbs in Greek and Romanian allow BC, providing a stronger argument for BC.

\citet{TAA2017} re-evaluate the empirical picture, using extensive questionnaires, by focusing on the following configurations with OC/NOC verbs favoring co-reference and NOC verbs that do not favor coreference:


\ea%14
    \label{ex:alexiadou:14}
    \ea V \textit{na} V Subj Obj
    \ex V \textit{na} V Obj Subj
    \z
\z

Their results suggest the following:

\begin{enumerate}
\item OC verbs show obligatory co-reference which can be analyzed as BC.
\item There is no clear contrast between OC and NOC verbs as far as Principle C effects are concerned (contra AAIM). A significant number of speakers allow co-reference with NOC verbs. 
\end{enumerate}

Note that, as well as examples like (6) where the embedded subject is nominative, native speakers were also asked to evaluate examples like (15) below involving BC between an embedded dative/genitive or accusative experiencer and a matrix null (nominative) subject. 

\ea%15
    \label{ex:alexiadou:15}
OC verb (verb of knowing)\\
\ea \gll Emathe siga {siga na}   tis aresun i operes otan   gnorise   to Jiani\\
learned-\textsc{3sg} gradually subj \textsc{cl-dat/gen} like-\textsc{3pl} the opera-\textsc{nom-pl} when met-\textsc{3sg} the Jiani-\textsc{acc}\\
\glt ‘She learned gradually to like opera, when she met John.’

Try/manage verbs (strongly favoring coreference)\\

\ex \gll  Prospathi   na   min   tin   stenahori   i ikonomiki krisi \\
    try-\textsc{3sg sbjv}  \textsc{neg}  \textsc{cl-acc} make-\textsc{3sg} sad  the financial crisis-\textsc{nom}\\
    \glt ‘She tries not to feel sad about the financial crisis.’

\ex\gll  Katafere   na   min   tin   apasholi    i ikonomiki krisi\\
    manage-\textsc{3sg} \textsc{sbjv}  \textsc{neg}  \textsc{cl-acc} worry-\textsc{3sg} the financial crisis-\textsc{nom}\\
    \glt ‘She managed not to feel anxious about the financial crisis.’

Future referring verb NOC (not favoring coreference)  \\

\ex \gll Apofasise   na   min   tin   katavali   i asthenia\\
         decided-\textsc{3sg} \textsc{sbjv}  \textsc{neg} \textsc{cl-acc}   put-\textsc{3sg} down the illness-\textsc{nom}\\
    \glt ‘She decided not to become depressed by the illness.’

\ex \gll {Iposhethike   na} min   tin   stenahori   pia i   siberifora    tu   jiu    tis \\
         promised-\textsc{3sg} \textsc{sbjv}  \textsc{neg}   \textsc{cl-acc}  {feel-\textsc{3sg} sad anymore} the   behavior-\textsc{nom}  the    son-\textsc{gen}   \textsc{cl-poss}\\
    \glt ‘She promised not to feel sad about her son’s behavior.’
\z
\z

The majority of the speakers these authors asked accept examples of the type in (15), and the rate of ungrammaticality ranges from 1.9--11.1\%.  

\begin{enumerate}\setcounter{enumi}{2}
\item The comparison between VSO and VOS order in \textit{na}{}-clauses shows that the preference for the disjoint reading is stronger in VSO orders than in VOS orders, but co-reference is still possible for many speakers, who do not have a significant contrast between VOS and VSO. 
\end{enumerate}

Importantly, TAA show that the Greek pattern cannot be analyzed as involving restructuring implemented in terms of remnant movement, as proposed for Spanish by \citet{Ordóñez2009} and \citet{Herbeck2013}, and suggested by an anonymous reviewer. Specifically, Ordóñez presents several arguments against a BC analysis for Spanish. First of all, he points out that similar patterns are found in structures that are standardly considered not to involve control. This is the case, for instance, in causative and perception verb constructions, where the subject may appear overtly in the post-infinitival position:

\ea%16
    \label{ex:alexiadou:16}
    \gll Ayer     nos   hizo   leer   Juan   el libro.  \\
           yesterday   to us   make   to read Juan   the book  \\
    \glt   ‘Yesterday Juan made us read the book.’  
    \z

           

Second, it is not the case that only main subjects are permitted after the infinitive, as assumed by the backward control analysis; the object of a main verb may also be inserted in this post-infinitival position with object control verbs. This is shown by the orders \textsc{v do inf xp} and \textsc{v inf do xp} in (17a--b). Examples (17b) and (17c) show that main object controllers, just like main subject controllers, can be embedded and appear after the infinitival verb:

\ea%17
    \label{ex:alexiadou:17}
    \ea[]{
    \gll     Obligaron   a Bush   a firmar   los acuerdos de paz  \\
             obliged-\textsc{3pl}  to Bush   to sign   the peace agreements  \\
    \glt     ‘They obliged Bush to sign the peace agreement.’}
    \ex[]{
    \gll Obligaron   a firmar   a Bush   los acuerdos de paz\\
           obliged-\textsc{3pl}   to sign     to Bush   the peace agreements\\
    \glt   ‘They obliged Bush to sign the peace agreement.’}
    \ex[?]{
    \gll Obligó   a firmar   el Congreso   a Bush   los acuerdos de pa\\
             obliged   to sign   the Congress to Bush   the peace agreement\\
    \glt     ‘The Congress obliged Bush to sign the peace agreement.’}
    \z
\z    

Ordóñez proposes a remnant movement analysis of BC (and restructuring constructions) in the spirit of \citegen{Hinterhölzl2005} and \citegen{Koopman2000} analyses of verbal complexes:

\ea%18
    \label{ex:alexiadou:18}
    \begin{xlista}
    \setcounter{xnumii}{0}
    \ex \gll\relax [\textsubscript{VP} Juan [querer [\textsubscript{CP} PRO [\textsubscript{VP} comprar el libro]]\\
                     Juan to want PRO to buy the book\\
    \end{xlista}
    \textbf{Step 1}: Movement of the verb \textit{to want} above VP:
    \begin{xlista}
    \setcounter{xnumii}{1}
    \ex \gll\relax [\textsubscript{TP} querer\textsubscript{} Juan V\textsubscript{i} [\textsubscript{TP} PRO[\textsubscript{VP} comprar el libro]]]\\
      to want Juan to buy the book\\
    \end{xlista}
    \textbf{Step 2}: Movement of the TP above \textit{wanted}:
    \begin{xlista}
    \setcounter{xnumii}{2}
    \ex \gll\relax [\textsubscript{TP} PRO [\textsubscript{VP} comprar el libro]] [\textsubscript{TP} querer\textsubscript{i} [\textsubscript{VP} Juan V\textsubscript{i}\\
       to buy the book wanted Juan\\
    \end{xlista}
    \textbf{Step 3}: Scrambling of the object out of TP + movement of the main subject \textit{Juan} to its licensing position above the scrambled object:
    \begin{xlista}
    \setcounter{xnumii}{3}
    \ex \gll\relax  [Juan\textsubscript{1} el libro\textsubscript{2} [\textsubscript{TP} PRO [\textsubscript{VP} comprar \textit{t}\textsubscript{2}]] [\textsubscript{TP} querer\textsubscript{i} [\textsubscript{VP} \textit{t}\textsubscript{1}\\
                    Juan the book to buy to want\\
    \end{xlista}
    \textbf{Step 4}: Movement of the VP containing \textit{to buy} above the licensing position of subject and object:
    \begin{xlista}
    \setcounter{xnumii}{4}
    \ex \gll\relax [[\textsubscript{VP} comprar \textit{t}]] Juan el libro [\textsubscript{TP} PRO] [\textsubscript{TP} querer\textsubscript{i} [\textsubscript{VP} \textit{t}\\
                   to buy Juan the book wanted\\
    \end{xlista}
    Step 5: Movement of TP+\textit{querer} to SpecCP and final Spell Out:
    \begin{xlista}
    \setcounter{xnumii}{5}
    \ex\relax [\textsubscript{CP} [\textsubscript{TP} querer\textsubscript{i} [\textsubscript{VP} \textit{t} [\textsubscript{VP} comprar \textit{t}\textsubscript{i]}] Juan el libro [\textsubscript{CP} PRO \textit{t}\textsubscript{i}
    \end{xlista}
    \z

Crucially for \citet{Ordóñez2009}, object scrambling (step 3) is a local movement and cannot cross a finite clause boundary. This explains why there are no comparable verbal complexes formed with finite clauses:

\ea%19
    \label{ex:alexiadou:19}
    \judgewidth{?*}
    \ea[*?]{
    \gll Ayer   les   hizo\textsubscript{i}  [que comprasen Juan\textsubscript{i} el libro]\\
                           yesterday to them made          that buy-\textsc{3pl} Juan the book\\}
    \ex[]{
    \gll Ayer     les   hizo\textsubscript{i} comprar   Juan\textsubscript{i}   el libro\\
       yesterday   to them made   to buy-\textsc{inf}   Juan   the book\\}
    \z
\z
    
Further evidence for the scrambling analysis in Spanish is provided by the following contrast. In examples involving infinitival wh-islands, as discussed by \citet{Torrego1996}, BC and FC behave differently. While the upper copy is available, the lower one is ungrammatical. According to Ordóñez, the ungrammaticality of (20a) can be explained, if scrambling out of non-tensed CPs is blocked by filled CPs.

\ea%20
    \label{ex:alexiadou:20}
    \ea Backward control\\
    \gll *? No sabe   si   contestar Juan las cartas. \\
         ~    not   know whether {to answer} Juan the letters\\
    \ex Forward control\\Juan no sabe si contestar Juan las cartas.
    \z
\z
           

\citet{TAA2017} show that the Greek facts are very different: specifically, there is no blocking of VSO orders and BC in OC constructions involving a filled SpecCP; cf. (20):

\ea%21
    \label{ex:alexiadou:21}
    \gll de kseri          pos na apandisi     o Janis   ta gramata    \\
          not know{}-\textsc{3sg} how \textsc{sbjv} answer   John{}-\textsc{nom}   the letters{}-\textsc{acc}\\
    \glt  ‘John does not know how to answer the letters.’
    \z

Moreover, embedding of the main object controller is not possible; i.e. here we have an asymmetry between subjects and objects:

\ea%22
    \label{ex:alexiadou:22}
    \ea[]{\gll anagasan      ton Bush na     ipograpsi   ti sinthiki irinis\\
        obliged-\textsc{3pl}   Bush-\textsc{acc} \textsc{sbjv}    sign-\textsc{3sg}   the peace agreement-\textsc{acc}\\
        \glt ‘They obliged Bush to sign the peace agreement.’}
    \ex[*]{\gll anagasan   na   ipograpsi   ton Bush   ti sinthiki irinis\\
    obliged-\textsc{3pl}   \textsc{sbjv}  sign-\textsc{3sg}    Bush-\textsc{acc}   the peace agreement-\textsc{acc}\\}
    \z
\z   


Furthermore, in Spanish, no argument may intervene between finite verbs and infinitives with a postverbal subject. This is not the case in Greek, where no locality effect is caused by an IO intervener in the matrix clause:

\begin{exe}%23
    \judgewidth{*?}
    \ex[*?]{
    \label{ex:alexiadou:23}
    \gll  les prometió     a los familiares            [darles el jurado la libertad a los prisioneros] \\
           to them-promised to the family members to give the jury liberty      to the prisoners\\}
\end{exe}

  

\ea%24
    \label{ex:alexiadou:24}
    \gll iposhethikan     tis Marias  na  dosun       i dikastes     amnistia   sto filakismeno      andra     tis\\
         promised-\textsc{3pl}    Maria-\textsc{gen} \textsc{sbjv} give-\textsc{3pl} the judges-\textsc{nom}   amnesty-\textsc{acc} to the imprisoned husband hers\\
    \glt   ‘The judges promised Mary to give amnesty to her imprisoned husband.’
    \z

As Greek lacks clitic climbing, there is no evidence for restructuring (see \citealt{Terzi1992} and others). Moreover, BC is found with all control verbs, not just with a small class (the restructuring class in Spanish).

Finally, TAA show that the obviation of Principle C effects in embedded VSO constructions is also found \textbf{with finite clauses}, as shown in (25b). Crucially, there is a robust Principle C effect in embedded \textit{that}{}-SVO sequences illustrated in (25a), indicating that Greek does have Principle C effects caused by a matrix null subject when the embedded subject precedes the inflected verb.

\ea%25
    \label{ex:alexiadou:25}
    \ea
    \gll pro\textsubscript{*j/k}  emathe   oti   o Petros\textsubscript{j}   kerdise   to lahio \\
         learned{}-\textsc{3sg}   that Peter{}-\textsc{nom}   won{}-\textsc{3sg} the lottery{}-\textsc{acc} \\
    \glt ‘He/she learned that Peter won the lottery.’
    \ex
        \gll pro\textsubscript{j/k}  emathe   oti   kerdise (o Petros\textsubscript{j} ) to lahio (o Petros\textsubscript{j} )          \\
             learned{}-\textsc{3sg}    that     won{}-\textsc{3sg} (Peter{}-\textsc{nom}) the lottery{}-\textsc{acc}  (Peter{}-\textsc{nom})        \\
        \glt ‘He/she learned that Peter won the lottery.’
    \z
\z

We can thus conclude that Greek BC configurations do not involve complex predicate formation. While there is evidence for verb clustering in Spanish, there is no such evidence in Greek. Moreover, in Greek, backward co-reference is even allowed within finite clauses unless the subject is in preverbal position.

TAA thus conclude that a backward dependency can productively be established in Greek provided that the embedded DP subject remains \textit{in situ}. They propose that what has been analysed as BC should not be analysed in terms of movement, because on a movement analysis it would be hard to explain the emergence of a Principle C effect when the subject occurs preverbally.\footnote{One could attempt to save the movement analysis by appealing to improper movement. Under the hypothesis that SVO orders in Greek involve Clitic Left Dislocation (CLLD; \citealt{Alexiadou1998}), one could account for the lack of BC in such configurations by analyzing the preverbal position as an A’-position. Such configurations would thus involve an improper A-A’-A movement chain. However, such an analysis would be strongly undermined by the fact that the subject in SVO orders does have A-properties and that CLLD in general has mixed A/A’-properties akin to medium-distance scrambling (see \citealt{Miyagawa2017} for relevant discussion).}  For this reason, they propose that Greek BC actually reflects $\varphi ${}-agreement between matrix T, embedded T and the overt S(ubject), which can also take place across embedded indicative CPs and is licit only if the S doesn’t intervene between the two T heads, as in (2a), repeated below:

\begin{exe}%2
%     \label{ex:alexiadou:2}
\exi{(2)}    
    \ea[]{[   T$\varphi $\textsubscript{k} [\textsubscript{TP/CP}    $T\varphi $\textsubscript{k}    DP$\varphi $\textsubscript{k}  ]]}
    \ex[*]{[ T$\varphi $\textsubscript{k} [\textsubscript{TP/CP}  DP$\varphi $\textsubscript{k}   $T\varphi $\textsubscript{k}]]}
    \z
\end{exe}


TAA relate the availability of long-distance agreement chains as in (2a) to the pro-drop status of the language. Their analysis assumes a version of (30): see \citet{Rizzi1982}, \citet{Alexiadou1998}, \citet{Holmberg2005}, \citet{Barbosa2009}.\footnote{This is called Hypothesis A in \citet{Holmberg2005} and \citet{Barbosa2009}. Holmberg rejects it while Barbosa argues for a version of it, implemented in terms of \citegen{Pesetsky2007} modification of \citegen{Chomsky2001} theory of Agree.} The crucial intuition is that Agr in null subject languages is pronominal and can thus enter long-distance agreement relationships, like pronouns.
\todo[inline]{The manuscript jumped from ex no 25 to no 30}

\ea%30
    \label{ex:alexiadou:30}
    The set of phi-features in T (Agr) is pronominal in null subject languages (NSLs); Agr is a referential, definite pronoun, albeit a pronoun phonologically expressed as an affix. As such, Agr is also assigned a subject theta-role, by virtue of heading a chain whose foot is in vP, receiving the relevant theta-role.
    \z

          

In order to make (30) compatible with the theory of Agree, \citet{Barbosa2009} proposes that the phi-features of T in consistent null subject languages (NSLs) are valued and can therefore value the phi-features of \textit{v}P-internal pro in pro-drop configurations. She furthermore proposes that they are uninterpretable, in order to account for the Agree relationship they establish with overt or covert subjects which have interpretable features. If she is correct, then we must assume that they are not deleted until they form a chain with the higher agreement in long-distance agreement chains, which means that Greek has phase-suspension in the relevant configurations (see Alexiadou, \citealt{Anagnostopoulou2014} for phase-suspension in long-distance Agree configurations arising in raising subjunctives); i.e. there is obligatory phase suspension in OC subjunctives and optional phase suspension in NOC subjunctives with BC, and even in indicatives.  

Alternatively, we can maintain that the phi-features on T in Greek are pronominal, and this permits them to enter long-distance agreement relationships, even across finite clauses, like pronouns do. Being pronominal can either be taken to mean that they are interpretable and unvalued (receiving a value either from a null Topic, as argued for in \citealt{Frascarelli2007}, or by entering a chain with a higher DP, depending on context), or they are valued, as Barbosa proposes, but also interpretable.\footnote{Either way, depending on what the facts in other NSLs turn out to be, we might need to parametrize these hypotheses. Specifically, it is well-known that Romance subjunctives show obviation, and this seems to correlate with the fact that they have infinitives. Thus, obviation in those contexts can be accounted for by appealing to global competition between infinitives and subjunctives. But what has not been investigated so far, to our knowledge, is how finite clauses behave. If they consistently show Principle C effects with embedded VSO and VOS orders, then this would indicate that either the phi-features of T are uninterpretable and thus they disappear after local Agree with the vP-internal subject (as proposed by \citealt{Barbosa2009}), or that phase-hood cannot be suspended in Romance indicatives.} 

Turning to the Agree relationships established in BSC configurations, (30) holds in the embedded clause of the non-Principle C VSO/VOS cases investigated by TAA, as in (31):

\ea%31
    \label{ex:alexiadou:31}
     \textsubscript{} [\textsubscript{TP/CP}    $T\varphi $\textsubscript{k}    DP$\varphi $\textsubscript{k}  ] 
\z
       

A further Agree relationship is established between matrix T and embedded CP; i.e. in the phase-hood version of BSC (see above), C is not an intervener for Agree. Following \citet{Rackowski2005}, TAA assume that PIC/intervention effects are obviated if a higher head first agrees with \textit{the entire phase} and then continues on to agree with an element \textit{inside} the phase; see also \citet{Halpert2016}. 

\ea%32
    \label{ex:alexiadou:32}
    [   T$\varphi $\textsubscript{k} [\textsubscript{TP/CP}    $T\varphi $\textsubscript{k}    DP$\varphi $\textsubscript{k}  ]]      
\z
    
Matrix T (and the \textit{v}P-internal pro-subject associated with it) agrees with the CP and then with embedded T which agrees with the \textit{v}P-internal subject. Note here that in Zulu, as argued in \citet{Halpert2016}, the EPP forces raising of the embedded subject out of the \textit{v}P. DP-raising does not have to take place in Greek\slash Romanian, as V-movement satisfies the EPP (\citealt{Alexiadou1998}), but when the subject occurs pre-verbally a Principle C effect arises. TAA suggest that the embedded subject DP is an intervener blocking Agree between matrix and embedded T; i.e. Agree between heads can happen as long as no DP intervenes between them. When matrix pronominal agreement directly c-commands a DP with which it shares no thematic index, it gives rise to a standard Principle C effect. This effect does not arise in embedded VSO/VOS orders because matrix T forms a chain with embedded T and embedded T shares the same thematic index with the subject DP.\footnote{Note that this analysis is compatible both with analyses taking full DP-subjects to optionally raise to SpecTP in Greek (e.g. \citealt{Spyropoulos2009}) and with analyses taking the pre-verbal subject to reside in a CLLD position (\citealt{Alexiadou1998}, \citealt{Barbosa2009} and others). In the latter approach, we can even sharpen the explanation for the Principle C effect, attributing it to the nature of CLLDed elements as topic shifters (cf. \citealt{Frascarelli2007}).}   

On the basis of this discussion, we can submit the following conclusions: what AAIM called BC in subjunctives actually involves the formation of agreement chains. BC (broadly/roughly understood as backward co-reference) involves agreement chains rather than actual movement because there is no obvious way of accounting for the asymmetry between embedded SVO vs. VSO orders (evidenced in finite clauses due to the option of SVO orders, which are unavailable in subjunctives for independent reasons having to do with the phonological clitic-like status of \textit{na}) with respect to Principle C effects in a DP-movement approach. When the word order in the embedded clause is SVO, we get a clear Principle C violation, as expected.

In this light, let us now see what happens in object control configurations. The question here is the following: if the availability of ‘BC’ in Greek is related to the availability of agreement chains of the type described above, are such agreement chains possible in object control configurations?

\section{No object BC in Greek} 

\subsection{Introduction}
Similarly to BSC, it has been argued that object control can also be subdivided into forward and backward object control (BOC):

\ea%33
    \label{ex:alexiadou:33}
    \ea Forward object control\\
    \gll I persuaded   Kim\textsubscript{i}     [{△\textsubscript{i}} to smile]  \\
         {}   {}       \textit{controller}         \textit{controllee}\\
    \ex Backward object control\\
    \gll I persuaded  △\textsubscript{i}    [Kim\textsubscript{i}    to smile]  \\
        {}    {}       \textit{controllee}  \textit{controller}\\
        \z
\z

BOC is attested in e.g. Malagasy (\citealt{Potsdam2006}, 2009), Korean \citep{Monahan2003}, and Omani Arabic (Al-\citealt{Balushi2008}). We illustrate the phenomenon with a Korean example in (34). (34a) shows that Korean object control predicates permit an accusative-nominative alternation. While the accusative is a constituent of the matrix clause, binding a null element in the embedded clause, (34b), the nominative resides in the embedded clause and is coindexed with a null element in the matrix, (34c):
 
\ea%34
    \label{ex:alexiadou:34}
    \ea\gll Cheolsu-neun  Yeonghi-leul/ka     kake-e    ka-tolok   seolteukha-eoss-ta\\
            Cheolsu-\textsc{top}     Yeonghi-\textsc{acc/nom}   store-to  go-\textsc{comp}   persuade-\textsc{past-decl}\\
    \glt    ‘Cheolsu persuaded Yeonghi to go to the store.’
    \ex
    \gll Cheolsu-neun Yeonghi-leul\textsubscript{i} [${\bigtriangleup}$\textsubscript{i} kake-e     ka-tolok]  seolteukha-eoss-ta\\
             Cheolsu-\textsc{top}    Yeonghi-\textsc{acc}    {}      store-to   go-\textsc{comp}   persuade-\textsc{past-decl}\\
    \glt     ‘Cheolsu persuaded Yeonghi to go to the store.’     
    \ex
    \gll Cheolsu-neun ${\bigtriangleup}$\textsubscript{i} [Yeonghi-ka\textsubscript{i}    kake-e    ka-tolok]  seolteukha-eoss-ta\\
             Cheolsu-\textsc{top}  {}       Yeonghi-\textsc{nom}    store-to  go-\textsc{comp}      persuade-\textsc{past-decl}\\
    \glt     ‘Cheolsu persuaded Yeonghi to go to the store.’ 
    \z
\z

Before we turn to the question of whether BOC can be evidenced in Greek, we should offer a brief description of the predicates that have been analyzed as object control predicates in Greek. This is a controversial issue, as these structures are in principle also amenable to an ECM analysis; it thus has to be shown that the DP is generated in the object position of the matrix predicate. \citet{Alexiadou1997} addressed this, and we briefly summarize their argumentation here; see also \citet{Kotzoglou2002} and \citet{Kotzoglou2007}. 

\subsection{Object control in Greek}% 3.2 
Constructions that could be analyzed as ECM in Greek involve perception and causative verbs (cf. \citealt{Burzio1986} for Italian):


\ea%35
    \label{ex:alexiadou:35}
    \ea \gll ida          ton  Petro         na    milai     me   tin Ilektra\\
             saw{}-\textsc{1sg}  the   Peter{}-\textsc{acc}  \textsc{sbjv} talk-\textsc{3sg} with  the Ilektra\\
    \glt     ‘I saw Peter talking with Ilektra.’
    \ex
    \gll evala      ton Petro       na     katharisi   to   domatio tu\\ 
             put{}-\textsc{1sg}  the Peter{}-\textsc{acc} \textsc{sbjv}   clean-\textsc{3sg}  the  room     his\\
    \glt     ‘I made Peter clean his room.’
    \z
\z    

\citet{Iatridou1993} treats cases like (35a) as instances of object control. In fact, Burzio argues against an ECM analysis for (35a-b) and his arguments also hold for Greek (cf. \citealt{Burzio1986}: 287-290). As \citet{Alexiadou1997} point out, unlike tensed/infinitival pairs like \textit{I believe that Eric delivered the speech/I believe Eric to have delivered the speech}, which are closely synonymous, pairs like (36) below are not synonymous:

\ea%36
    \label{ex:alexiadou:36}
    \ea
    \gll Ida          oti   o    Petros         telioni   ti      diatrivi       tu\\
             saw{}-\textsc{1sg}  that  the Peter{}-\textsc{nom}   finishes  the  dissertation his\\
    \glt     ‘I saw that Peter is finishing his dissertation.’
    \ex
    \gll ida          ton  Petro      na     telioni   ti    diatrivi        tu\\
             saw{}-\textsc{1sg}  the  Peter{}-\textsc{acc} \textsc{sbjv} finishes the  dissertation his\\
    \glt     ‘I saw Peter finishing his dissertation.’
    \z
\z

In (36b) the phrase corresponding to \textit{Petros} is the object of direct perception, while this is not true of sentences like (36a). A related point has to do with the non-synonymy of active and passive forms. While S complements maintain rough synonymy under passivization, as with \textit{I believe Eric to have delivered the speech vs. I believe the speech to have been delivered by Eric}, the cases under discussion are not synonymous, as is evident from the semantic anomaly of the verb \textit{ida} in (37b) below:

\ea%37
    \label{ex:alexiadou:37}
    \ea[]{
    \gll ida/akusa                to   Petro        na    ekfoni        to   logo\\
         saw{}-\textsc{1sg}/heard{}-\textsc{1sg}  the  Peter{}-\textsc{acc}  \textsc{sbjv} deliver-\textsc{3sg} the  speech\\
    \glt ‘I saw/heard Peter delivering the speech.’}
    \ex[\#]{
    \gll ida/akusa              to  logo     na     ekfonite       apo  ton Petro\\
         saw{}-\textsc{1sg}/heard{}-\textsc{1sg}  the speech \textsc{sbjv} {be delivered} by   the Peter\\
    \glt ‘I saw/heard the speech being delivered by Peter.’}    
    \z
\z
    
Another standard test for distinguishing \_NP S from \_S complements involves the relative scope of quantifiers. By this test, the structures in question also qualify as non-ECM:\footnote{\citet{Alexiadou2016} point out, however, that in the context of perception verbs, the subject of the embedded clause is assigned accusative in the matrix clause, but is licensed by the negation in the subordinate clause. This is compatible with an ECM analysis, suggesting that perception verbs behave like quasi-ECM predicates in \citegen{Kotzoglou2007} terminology. (i)  Bika         mesa    ke me     ekpliksi idha       kanenan         na min dulevi monos tu.  alone his-\textsc{nom}  entered{}-\textsc{1sg} in   and with   surprise saw{}-\textsc{1sg} nobody{}-\textsc{acc  sbjv  neg} work{}-\textsc{3sg}   Oli ixan xoristi      se omades.  all had separated into teams  ‘I entered and to my surprise I saw nobody working on his own. They had all separated into teams.’}

\ea%38
    \label{ex:alexiadou:38}
    \ea They expected one customs official to check all passing cars.\\
    \begin{xlisti}
    \ex They expected that there would be one customs official who would         check all passing cars.
    \ex They expected that, for each passing car, there would be some           customs official or other who would check it.  
    \end{xlisti}
    \ex 
    \gll ida          enan teloniako            na       elenhi   kathe  aftokinito\\
        saw{}-\textsc{1sg}   one   customs official  \textsc{sbjv} control  every  car\\
    \glt ‘I saw a customs official controlling every car.’
    \begin{xlisti}
    \ex[]{I saw one customs official who checked every passing car.}
    \ex[*]{I saw that for each passing car there was one customs official who would check it.}
    \end{xlisti}
    \z
\z

Under the assumption that quantifier scope is clause-bounded, the difference between (38a) and (38b) follows if (38b) has the two quantifiers in different clauses. 

  A further argument against the ECM analysis comes from Clitic Left Dislocation (CLLD). CLLD of CP clauses in Greek involves a clitic which is third person singular neuter:

\ea%39
    \label{ex:alexiadou:39}
    \ea
    \gll oti   irthe   o     Petros      den  to       perimena\\
          that  came the  Peter-\textsc{nom  neg}  \textsc{cl-acc} expected{}-\textsc{1sg}\\
    \glt ‘That Peter came, I didn't expect it.’
    \ex
    \gll na    erthi         o   Petros       den   to      vlepo\\
         \textsc{sbjv} come{}-\textsc{3sg} the Peter-\textsc{nom  neg}  \textsc{cl-acc} see{}-\textsc{1sg}\\
    \glt Lit. ‘I do not see it that Peter will come.’
    \z
\z    
 
If perception verbs took an S complement, then we would expect the same clitic to appear in CLLD. However, this is not what we find:

\ea%40
    \label{ex:alexiadou:40}
    \ea[]{
    \gll\relax [ton logo]\textsubscript{i}     na       ekfonite        den  ton\textsubscript{i} akusa\\
             the  speech    \textsc{sbjv} be delivered  \textsc{neg}  him heard{}-\textsc{1sg}\\
    \glt     ‘The speech being delivered, I did not hear it.’}
    \ex[*]{
    \gll\relax [ton logo    na     ekfonite]\textsubscript{ i}     den  to\textsubscript{i}  akusa\\
              the speech  \textsc{sbjv} be delivered   \textsc{neg}  it   heard{}-\textsc{1sg}\\}
    \ex[]{
    \gll\relax [ton Petro]\textsubscript{ i}    na    tiganizi psaria  den   ton\textsubscript{i}  ida \\
            the Peter\textsc{{}-acc}  \textsc{sbjv} fry        fish     \textsc{neg}  him  saw{}-\textsc{1sg}\\
    \glt    ‘Peter frying fish, I did not see him.’}
    \ex[*]{
    \gll\relax [ton Petro  na    tiganizi psaria]\textsubscript{i} den  to\textsubscript{i}  ida\\
               the  Peter  \textsc{sbjv} fry         fish     \textsc{neg}  it  saw{}-\textsc{1sg}\\}
    \z
\z  

These examples are grammatical only with a resumptive clitic, which agrees in features with the DP, not with the whole clause.{} 

On the basis of these examples, then, we can conclude that perception verbs are object control predicates in Greek (but see footnote 7 for a complication). Other object control predicates include \textit{pitho} ‘persuade’, \textit{diatazo} ‘order’, \textit{parakalo} ‘beg’, and \textit{voitho}, ‘help’, which all behave similarly to perception verbs; see (41), which tests CLLD, and \citet{Kotzoglou2002} for discussion:

\ea[*]{%41
    \label{ex:alexiadou:41}
    \gll\relax [ton Jani  na    aposiri      ti    minisi]\textsubscript{i}        to\textsubscript{ i}  episa\\
             the John  \textbf{\textsc{sbjv}} withdraw  the prosecution it  persuaded{}-\textbf{\textsc{1sg}}\\}
    \z

Before we proceed to the behavior of these predicates in terms of BC, we note that Kotzoglou and \citet{Papangeli2007} discuss so-called quasi-ECM predicates such as \textit{perimeno} ‘expect’ and \textit{thelo} ‘want’. Applying several of the tests for object control, as in (42) (their (27b)), involving CP doubling, they conclude that these predicates also involve a matrix DP; i.e. they can be subsumed as a case of object control. 

\ea[*]{%42
    \label{ex:alexiadou:42}
    \gll to\textsubscript{i} perimena\textsubscript{} [ton  Jani          na    aghapisi  ti       Maria]\textsubscript{ i}\\
         it expected{}-\textsc{1sg}    the  John\textsc{{}-}\textbf{\textsc{acc}} \textsc{sbjv} love{}-\textsc{3sg}    the Maria\textsc{{}-acc}\\
    \glt ‘I expected John to love Maria.’}
    \z

The authors do, however, notice some important differences between quasi-ECM verbs and object control verbs. First, as they state (\citealt{Kotzoglou2007}: 129), “there is a crucial difference in the thematic information that is realized in the Greek examples. Object control verbs cannot select a clause as their single argument, while this was shown to be possible in the quasi-ECM examples.” Moreover, object control verbs “always realize the subject matter role as a clause. They thus lack the PP alternate that is attested with verbs of the ‘quasi-ECM’ type.” A second difference involves wh-extraction, which is banned in Greek ‘quasi-ECM’ domains, but is licit out of the object control clause; see (43) (their (42)):

\ea%43
    \judgewidth{??}
    \label{ex:alexiadou:43}
    \ea[??]{
    \gll pjon    itheles         ton  prothipurgho           na    entiposiasi? \\
         who\textsc{{}-acc} wanted{}-\textsc{2sg} the  prime-minister\textsc{{}-acc} \textsc{sbjv} impress{}-\textsc{3sg} \\
    \glt ‘Who did you want the prime minister to impress?’}
    \ex[]{
    \gll pjon        epises               ton  prothipurgho           na    entiposiasi? \\
         who\textsc{{}-acc}  persuaded{}-\textsc{2sg}  the  prime-minister\textsc{{}-acc} \textsc{sbjv} impress{}-\textsc{3sg}\\
    \glt ‘Who did you persuade the prime minister to impress?’}
    \z
\z

This, in combination with the observation made in \citet{Kotzoglou2007} that the accusative object of quasi-ECM verbs licenses nominative secondary predicates in the embedded clause, as in (46), leads us to suggest that quasi-ECM configurations actually involve movement of the embedded DP to the CP level, where it is assigned accusative by the matrix predicate. This is an instance of an edge-effect in \citegen{Baker2015} terminology:

\ea%46
    \label{ex:alexiadou:46}
    \gll perimena        to Jani           na    ine arostos/\textsuperscript{*}arosto \\
         expected{}-\textsc{1sg} the John\textsc{{}-acc} \textsc{sbjv} be sick\textsc{{}-nom}/*\textsc{{}-acc} \\
    \glt ‘I expected John to be sick.’ 
    \z


In (46), the DP is first assigned nominative in the lower clause, and then accusative, after movement, at the CP level. This means that accusative, which we treat following \citet{Marantz1991} and \citet{Baker2015} as dependent case, can be assigned on top of a case assigned lower, inside the embedded clause. As Baker notes, there is cross-linguistic variation as to whether multiple realization is possible. 

Note that from the perspective of the ‘control as movement’ theory, the derivation of (46) is similar, if not identical, to that of control predicates. In both cases, the DP raises from the embedded clause to the matrix clause, where it is assigned dependent accusative. The difference between the two might presumably be related to the fact that in (46) the DP raises to SpecCP, where it is frozen, while in the object control cases, it raises higher, to the matrix \textit{v}P, in order to be receive a thematic role. However, on the basis of our argumentation in §2 regarding TAA’s results, it is crucial that there is movement in so-called quasi-ECM environments, but not in control configurations. 

\subsection{Greek lacks BOC}% 3.3 

Interestingly, none of the object control verbs in Greek allows BOC. The movement analysis of control would predict that the lower copy is spelled out as nominative; i.e. that it bears the case of the embedded clause. However, the examples in (47b) and (48b-c) are all ungrammatical:



\ea%47
    \label{ex:alexiadou:47}
    \ea[]{
    \gll I     Maria  epise         to   Jani          na     hamogelasi\\
         the Mary  persuaded  the John-\textsc{acc}  \textsc{sbjv} smile\textsc{{}-3sg}\\
    \glt ‘Mary persuaded John to smile.’}
    \ex[*]{
    \gll I   Maria (ton)      epise        na     homogelasi   o    Janis \\
         the Mary  (\textsc{cl-acc}) persuaded \textsc{sbjv} smile\textsc{{}-3sg}      the John\textsc{{}-nom} \\}
    \z
\z    


\ea%48
    \label{ex:alexiadou:48}
    \ea[]{
    \gll I     Maria         voithise to    Jani   na   simazepsi   to   domatio tu\\
         the Mary\textsc{{}-nom}  helped   the  John \textsc{sbjv} tidy.up\textsc{{}-3sg}  the room      his\\
    \glt ‘Mary helped John to tidy up his room.’}
    \ex[]{
    \gll I      Maria        voithise  na    simazepsi  o    Janis        to  domatio tu\\
         the Mary\textsc{{}-nom}  helped    \textsc{sbjv} tidy.up\textsc{{}-3sg} the John\textsc{{}-nom} the room     his\\
    \glt [\textit{good but not on the reading where she helped John}]}
    \ex[*]{
    \gll I Maria (ton) voithise na simazepsi  o    Janis        to  domatio tu\\
         the Mary\textsc{{}-nom} (\textsc{cl-acc}) helped \textsc{sbjv} tidy.up\textsc{{}-3sg} the John\textsc{{}-nom} the room     his\\   }
    \z
\z
% %     [F0B9?] % This sign was printed, maybe an artefact?
 
On the backward control analysis, this asymmetry is puzzling and unexpected. If, however, control does not involve movement, as TAA argue, then the observed asymmetry boils down to configurations that enable co-reference; i.e. the formation of long-distance agreement chains of the type we described in §2. 

At first sight, the above behavior seems to suggest that the distribution of BC patterns is related to the presence of \textit{pro}. Greek has subject \textit{pro} and allows BSC. By contrast, Greek lacks object \textit{pro} (\citealt{Giannakidou1997}) and disallows BOC. While this would be in agreement with our conclusions in §2, \citet{Potsdam2006,Potsdam2009} argues that this does not hold across languages, as Malagasy lacks object \textit{pro} but allows BOC. One of the arguments Potsdam brings against the \textit{pro} analysis in Malagasy involves variable binding. As he points out, the \textit{pro} analysis would predict that a bound variable interpretation for the controller-controllee relation should be impossible, as there is no c-command. However, the example in (49), involving a distributed universal quantifier, shows that variable binding is possible in backward control. Thus, it seems that the controller and controllee must be in a c-command relationship to obtain the right configuration for binding. 

\ea%49
    \label{ex:alexiadou:49}
    \gll boky inona avy no nanontania- nao hovidian’ ny mpianatra tsirairay ?\\
         book what each \textsc{foc}  ask.\textsc{ct}  you buy.\textsc{tt} the student each\\
    \glt ‘For each \textit{x}, \textit{x} a student, which book did you ask \textit{x} to buy?’ (\citealt{Potsdam2006}: ex. (17a))
    \z

We can thus maintain that Malagasy has BOC control, and that the availability of object \textit{pro} does not correlate with the availability of BOC in true BC-as-movement languages. But, crucially, Greek was argued in §2 not to be such a language.

The only cases of BOC that seem possible in Greek involve a Gen/Dat or Acc object realized as a clitic and a Gen/Dat or Acc experiencer in the embedded clause, a pattern that seems similar to that of resumption; see \tabref{tab:alexiadou:1}. Note that (47b)-(48c) remain ungrammatical in spite of the presence of a clitic in the matrix clause:

\ea%50
    \label{ex:alexiadou:50}
    \ea
    \gll O    Janis    \textbf{tu}  epevale/ton katafere   na   \textbf{tu} aresi  \textbf{tu  Kosta}          i opera.\\
                   the John-\textsc{nom}   \textsc{cl-gen}  imposed/ cl-acc managed  \textsc{sbjv} \textsc{cl-gen}  like   the Kostas-\textsc{gen} the opera\\
    \glt           ‘John imposed on Kostas to like the opera/convinced Kostas to like the opera.’
    \ex
    \gll O    Janis    \textbf{tu}  epevale/ton katafere   na   \textbf{ton} efxaristi  \textbf{ton  Kosta}         i opera.\\
           the John-\textsc{nom} \textsc{cl-gen}  imposed/\textsc{cl-acc} managed  \textsc{sbjv} \textsc{cl-acc}  please the Kostas-\textsc{gen} the opera\\
    \glt   ‘John imposed on Kostas to like the opera/convinced Kostas to like the opera.’
    \z
\z

Let us consider now the configuration for OC in comparison to our analysis of BSC: in the case of forward control, an Agree relationship must be established between matrix Voice and matrix DP and subsequently the phi-features of T in the embedded CP.    

\ea%51
    \label{ex:alexiadou:51}
    [\textsubscript{CP} [\textsubscript{VoiceP} [ DP\textsubscript{$\varphi $}\textsubscript{k} [\textsubscript{TP/CP}    $T\varphi $\textsubscript{k}    ]]]]
    \z

If the phi-features of embedded T are unvalued, we can follow \citet{Grano2016}, building on \citet{Kratzer2009}, and \citet{Landau2015}, who propose two variants for analyzing such configurations, (52a)-(52b):


\ea%52
    \label{ex:alexiadou:52}
    \ea
    \begin{xlisti}
    \ex An unvalued pronoun can be valued via feature transmission.
    \ex Transmission of phi-features piggybacks on predication.
    \ex A complement clause can be turned into a predicate via Fin.
    \ex Transmission proceeds from antecedent to Fin and from Fin to [Spec,FinP].
    \end{xlisti}
    \ex
    \begin{xlisti}
    \ex An unvalued pronoun can be valued via feature transmission.
    \ex Transmission of phi-features piggybacks on binding.
    \ex Binding is mediated by verbal functional heads.
    \ex C and \textit{v} intervene for each other in the way they transmit features.
    \end{xlisti}
    \z
\z

On the latter approach, a matrix binder transmits features onto embedded C, and embedded C binds and values an unvalued pronoun in its c-command domain.

In forward object control configurations, we usually have a genitive or an accusative in the matrix clause that controls the nominative subject of the embedded verb. As we see in (53), the DP \textit{John} bears accusative, assigned by the matrix predicate. The presence of a nominative modifier in the embedded clause suggests that it has been assigned nominative in that context. Thus, it bears two cases, but only one is realized.

\ea%53
    \label{ex:alexiadou:53}
    \gll vlepo to    Jani        na   pezi        basket  monos tu.\\
         see     the John-\textsc{acc} \textsc{sbjv} play-\textsc{3sg} basket  alone-\textsc{nom}\\
    \glt ‘I see John playing basketball alone.’
    \z

This is a so-called multiple-case-marked A-chain similar to the kind discussed for Niuean in Bejar \& \citet[67]{Massam1999}.

For backward object control, what we would need first, similarly to what we outlined for the BSC cases, is for the Agree relation to hold within the embedded clause:

\ea%54
    \label{ex:alexiadou:54}
    \textsubscript{} [\textsubscript{TP/CP}    $T\varphi $\textsubscript{k}    DP$\varphi $\textsubscript{k}  ]  
\z
 
While in the case of subject co-reference the Agree chain ultimately holds between two T heads, the matrix and the embedded one, in the case of object control the embedded T head must enter Agree with the matrix Voice head, and this configuration seems generally illegitimate (cf. \citealt{Kayne1989}). We believe that part of the reason for this is the different requirements that T and Voice impose. T has been argued to have pronominal phi-features while Voice doesn’t: Greek is not a rich object agreement, object-drop language, which can be taken to mean that the phi-features of embedded T are not allowed to enter long-distance agreement with the phi-features of the matrix Voice. 

But we have seen that this is exceptionally possible if the embedded clause has a dative or accusative clitic doubling the experiencer and the matrix Voice hosts a dative or accusative clitic; i.e. in cases of ‘resumption’ crucially involving an experiencer in the downstairs clause. This leads us to formulate the hypothesis in (55) as a condition for BC:\footnote{An anonymous reviewer suggests two alternative hypotheses to us, (i) and (ii). (i)  In a chain with multiple case positions, realize the copy with the more marked case (ACC/GEN >NOM).(ii)  In a chain with multiple case positions, realize the higher copy. If both positions are assigned the same   case, the lower copy can be realized.The second hypothesis would capture the fact that BSC is possible when the lower clause contains an experiencer and the higher clause a null \textit{pro} bearing nominative, as was seen in the examples in (15), but it would have to be reformulated in terms of agreement chains if control does not involve movement, as we suggest in §2. (i) can be reformulated as suggesting that only a dependent case in the sense of \citet{Marantz1991} and \citet{Baker2015} must be realized (see \citealt{Anagnostopoulou2017} for arguments that Greek GEN is a dependent case).}

\ea%55
    \label{ex:alexiadou:55}
    Backward Agree applies to heads of the same type. 
\z

In the BOC cases at hand, the relationship is between a clitic in the embedded clause and a clitic in the matrix clause. Note that when the downstairs experiencer surfaces as a nominative DP, backward co-reference seems to us to be degraded:\footnote{Because these facts have not been investigated before, we are relying on our own intuitions. They need to be checked with a large number of speakers via extensive questionnaires, just as TAA did with the BSC constructions. The same applies to the data discussed immediately below.} 

\ea[\#]{%56
    \label{ex:alexiadou:56}
    \gll O    Janis    \textbf{tu}  epevale/ton katafere   na   \textbf{efxaristiete}  \textbf{o  Kostas}   me tin     opera.\\
         the John{}-\textsc{nom  cl-gen}  imposed/\textsc{cl-acc} managed \textsc{sbjv} please-\textsc{nact}   the Kostas{}-\textsc{nom} with the opera\\
    \glt ‘John imposed on Kostas to like the opera/convinced Kostas to like the opera.’}
    \z

Moreover, note that if the clitic-doubled argument in the embedded clause is not an experiencer, backward coreference is not possible (this is indicated by \# in the passive (57a), featuring a clitic-doubled goal, which is well-formed in the non-coreference reading, and by ?? in (57b), featuring an affected argument combined with an unaccusative, which seems to us to admit the coreference reading but to be degraded compared to the experiencer cases mentioned above):

\ea%57
\judgewidth{??}
    \label{ex:alexiadou:57}
    \ea[\#]{\gll O    Janis    \textbf{tu}  epevale/ton katafere       na   \textbf{tu} dothi \textbf{tu  Kosta}         to danio.\\
                   the John{}-\textsc{nom} \textsc{cl-gen} imposed/\textsc{cl-acc} managed \textsc{sbjv cl-gen} give-\textsc{nact} the Kostas{}-\textsc{gen} the loan\\
    \glt           ‘John imposed on him for a loan to be given to Kostas.’}
    \ex[??]{
     \gll O    Janis    \textbf{tu}  epevale/ton katafere   na min  \textbf{tu} pesi  \textbf{tu  Kosta}     to vazo. \\
          the John{}-\textsc{nom}   \textsc{cl-gen}  imposed/\textsc{cl-acc} managed  \textsc{sbjv neg cl-gen}  fall   the Kostas{}-\textsc{gen} the vase’ \\
     \glt ‘John imposed on Kostas not to drop the vase.’   }
    \z
\z    

This seems to suggest that backward coreference of this type is not only subject to the condition in (55), but requires, in addition, that the embedded clitic-doubled argument encode point of view. Perhaps this is because only experiencers qualify as subjects at some level of representation, which means that they relate to T (\citealt{Anagnostopoulou1999} for Greek, \citealt{Landau2010}). 

\section{Conclusion}% 4. 

In this paper, we have discussed an asymmetry in the distribution of backward control in Greek. While the language has been argued to have BSC, it lacks BOC. As we pointed out, recently \citet{TAA2017} argued that BSC in Greek is a side effect of the availability of an agreement chain between a null main subject and an overt embedded subject in all types of subjunctives (\textit{na}{}-clauses), and to a certain extent in indicatives (\textit{that}{}-clauses). If this is the correct analysis for BSC, the question still remains whether Greek has BOC. We showed in this paper that BOC configurations are severely limited. We related this limitation to the nature of Backward Agree, which seems to require heads of the same type. In BOC configurations, the phi-features of embedded T are not allowed to enter long-distance agreement with the phi-features of the matrix Voice. Backward co-reference is only possible in case of resumption with a dative/genitive clitic in the matrix clause and a clitic-doubled experiencer in the embedded clause, and crucially depends on the experiencer status of the embedded argument.

\section*{Acknowledgements}

We are indebted to the editors of this volume and an anonymous reviewer for very insightful comments that helped us restructure this paper. AL 554/10-1 is hereby acknowledged.

% \section{ References}
% 
% 
% Agouraki, Yoryia. 1991 A Modern Greek complementizer and its significance for Universal Grammar. \textit{UCL Working Papers in Linguistics} 3. 1–24.
% 
% 
% 
% Al-Balushi, Rashid. 2008. Control in Omani Arabic. Ms. (Toronto: University of Toronto.)
% 
% 
% Alexiadou, Artemis. 1999. On the properties of some Greek word order patterns. In Alexiadou, Artemis \& Horrocks, Geoffrey \& Stavrou, Melita (eds.), \textit{Studies in Greek} syntax, 45–65. Dordrecht: Kluwer.
% 
% Alexiadou, Artemis. 2000. Some remarks on word order and information structure in Romance and Greek.~\textit{ZAS Papers in Linguistics} 20. 119–136.
% 
% 
% Alexiadou, Artemis \& Anagnostopoulou, Elena. 1997. Notes on ECM, control and raising. \textit{ZAS Papers in Linguistics} 8. 17–27.
% 
% 
% 
% Alexiadou, Artemis \& Anagnostopoulou, Elena. 1998. Parametrizing Agr: Word order, V- movement and EPP-checking. \textit{Natural Language \& Linguistic Theory} 16. 491–539.
% 
% 
% 
% Alexiadou, Artemis \& Anagnostopoulou, Elena. 2016. Rethinking the nature of nominative case. Ms. (Berlin: Humboldt-Universität zu Berlin \& Rethymnon: University of Crete.)
% 
% 
% 
% Alexiadou, Artemis \& Anagnostopoulou, Elena \& Iordachioaia, Gianina \& Marchis, Mihaela. 2010.\href{http://ifla.uni-stuttgart.de/files/BC paper.pdf}{ No objection to backward control.} In Hornstein, Norbert \& Polinsky, Maria (eds.), \textit{Movement theory of control}, 89–118. Amsterdam: John Benjamins.
% 
% 
% 
% Alexiadou, Artemis \& Anagnostopoulou, Elena \& Wurmbrand, Susi. 2014. Movement vs. long-distance Agree in raising: Disappearing phases and feature valuation.~\textit{Proceedings of NELS~}43. 1–12.
% 
% 
% 
% Anagnostopoulou, Elena. 1999. On experiencers. In Alexiadou, Artemis \& Horrocks, Geoffrey \& Stavrou, Melita (eds.), \textit{Studies in Greek syntax}, 67–93. Dordrecht: Kluwer.~
% 
% 
% 
% Anagnostopoulou, Elena. 2003. \textit{The syntax of ditransitives: Evidence from clitics}. Berlin: Mouton de Gruyter.
% 
% 
% 
% Anagnostopoulou, Elena \& Sevdali, Christina. 2017. Two modes of dative and genitive case assignment: Evidence from two stages of Greek. Ms. (Rethymnon: University of Crete \& Jordanstown: University of Ulster.)
% 
% 
% 
% Baker, Mark. 2015. \textit{Case: Its principles and its parameters}. Cambridge: Cambridge University Press. 
% 
% 
% Barbosa, Pilar. 2009. Two kinds of subject \textit{pro}. \textit{Studia Linguistica} 63. 2–58.
% 
% 
% Bejar, Susana \& Massam, Diane. 1999. Multiple case checking. \textit{Syntax} 2. 66–79.
% 
% 
% 
% Burzio, Luigi. 1986. \textit{Italian syntax}. Dordrecht: Foris.
% 
% 
% 
% Chomsky, Noam. 2001. Beyond explanatory adequacy. In Belletti, Adriana (ed.), \textit{Structures and beyond}, 104–131. Oxford: Oxford University Press.
% 
% 
% Frascarelli, Mara. 2007. Subjects, topics and the interpretation of referential \textit{pro}. \textit{Natural Language \& Linguistic Theory} 25. 691–734.
% 
% 
% Fukuda, Shinichiro. 2008. Backward control. \textit{Language and Linguistics Compass} 2. 168–195.
% 
% 
% 
% Giannakidou, Anastasia. 1997. \textit{The landscape of polarity items}. Groningen: University of Groningen. (Doctoral dissertation.)
% 
% 
% Giannakidou, Anastasia \& Merchant, Jason. 1997. On the interpretation of null indefinite objects in Greek. \textit{Studies in Greek Linguistics} 17. 290–303.
% 
% Grano, Thomas \& Lasnik, Howard. 2016. How to neutralize a finite clause boundary: Phase theory and the grammar of bound pronouns. Ms. (Bloomington, IN: Indiana University \& College Park, MD: University of Maryland.)
% 
% Halpert, Claire. 2016. Raising parameters. \textit{Proceedings of WCCFL 33}. 186–195.
% 
% 
% Herbeck, Peter. 2013. On backward control and clitic climbing: On the deficiency of non-finite domains in Spanish and Catalan. \textit{Proceedings of ConSOLE XXI, 2013}. 123–145.
% 
% 
% Hinterhölzl, Roland. 2006. \textit{Scrambling, remnant movement, and restructuring in West Germanic}. Oxford: Oxford University~Press.~
% \textstyleFootnoteSymbol{}
% 
% Holmberg, Anders. 2005. Is there a little pro? Evidence from Finnish. \textit{Linguistic Inquiry} 36. 533–564.
% 
% 
% 
% 
% Hornstein, Norbert. 1999. Movement and control. \textit{Linguistic Inquiry} 30. 69–96.
% 
% 
% 
% Iatridou, Sabine. 1993. On Nominative Case assignment and a few related things. \textit{MIT Working Papers in Linguistics} 19. 175–198.
% 
% 
% Koopman, Hilda \& Szabolcsi, Anna. 2000. \textit{Verbal complexes}. Cambridge, MA: MIT Press.~
% 
% 
% Kotzoglou, George. 2002. Greek ‘ECM’ and how to control it. \textit{Reading Working Papers in Linguistics} 6. 39–56. 
% 
% 
% Kotzoglou, George \& Papangeli, Dimitra. 2007. Not really ECM, not exactly control: The ‘quasi-ECM’ construction in Greek. In Davies, William D. \& Dubinsky, Stanley (eds.), \textit{New horizons in the analysis of control and raising}, 111–132. Dordrecht: Springer. 
% 
% Kratzer, Angelika. 2009. Making a pronoun: Fake indexicals as windows into the properties of pronouns. \textit{Linguistic Inquiry} 40. 187–237.
% 
% 
% Landau, Idan. 1999. \textit{Elements of control}. Cambridge, MA: MIT. (Doctoral dissertation.)
% 
% 
% 
% Landau, Idan. 2004. The scale of finiteness and the calculus of control. \textit{Natural Language \& Linguistic Theory} 22. 811–877.
% 
% 
% Landau, Idan. 2007. Movement-resistant aspects of control. In Davies, William D. \& Dubinsky, Stanley (eds.), \textit{New horizons in the analysis of control and raising}, 293–325. Dordrecht: Springer.
% 
% Landau, Idan. 2010. \textit{The locative syntax of experiencers}. Cambridge, MA: MIT Press.
% 
% 
% Landau, Idan. 2015. \textit{A two-tiered theory of control}. Cambridge, MA: MIT Press.
% 
% 
% Marantz, Alec. 1991. Case and licensing. (Paper presented at the 8th Eastern States  Conference on Linguistics, University of Maryland, Baltimore.)
% 
% Miyagawa, Shigeru. 2017. \textit{Agreement beyond phi}. Cambridge, MA: MIT Press.
% 
% 
% Monahan, Philip J. 2003. Backward object control in Korean. \textit{Proceedings of WCCFL 22}. 356–369.
% 
% 
% Ordóñez, Francisco. 2009. Verbal complex formation and overt subjects in infinitivals in Spanish. Ms. (Stony Brook, NY: Stony Brook University.)
% 
% 
% Philippaki-Warburton, Irene \& Veloudis, Jannis. 1984. The subjunctive in complement clauses. \textit{Studies in Greek Linguistics} 5. 87–104.
% 
% 
% Pesetsky, David \& Torrego, Esther. 2007. The syntax of valuation and the interpretability of features. In Karimi, Simin \& Samiian, Vida \& Wilkins, Wendy K. (eds.), \textit{Phrasal and clausal architecture: Syntactic derivation and interpretation}, 262–294. Amsterdam: John Benjamins.
% 
% 
% Polinsky, Maria \& Potsdam, Eric. 2002. Backward control. \textit{Linguistic Inquiry} 33. 245–282.
% 
% 
% Polinsky, Maria \& Potsdam, Eric. 2006. \href{http://users.clas.ufl.edu/potsdam/papers/Syntax9.pdf}{Expanding the scope of control and raising}. \textit{Syntax} 9. 171–192.
% 
% Potsdam, Eric. 2006. Backward object control in Malagasy: Against an empty category analysis. \textit{Proceedings of WCCFL} \textit{25}. 328–336. 
% 
% 
% Potsdam, Eric. 2009. Malagasy backward object control. \textit{Language} 85. 754–784.
% 
% 
% Rackowski, Andrea \& Richards, Norvin. 2005. Phase edge and extraction: A Tagalog case study. \textit{Linguistic Inquiry} 36. 565–599.
% 
% Rizzi, Luigi. 1982. \textit{Issues in Italian syntax}. Dordrecht: Foris.
% 
% Roussou, Anna. 2009. In the mood for control. \textit{Lingua} 119. 1811–1836.
% 
% Spyropoulos, Vassilios. 2007. Finiteness and control in Greek. In Davies, William D. \& Dubinsky, Stanley (eds.), \textit{New horizons in the analysis of control and raising}, 159–183. Springer.
% 
% 
% Spyropoulos, Vassilios \& Revithiadou, Anthi. 2009. Subject chains in Greek and PF processing. In Halpert, Claire \& Hartman, Jeremy \& Hill, David (eds.), \textit{Proceedings of the MIT Workshop in Greek Syntax and Semantics}, 293–309. Cambridge, MA: MITWPL.
% 
% 
% 
% Torrego, Esther. 1996. On quantifier float in control clauses. \textit{Linguistic Inquiry} 27. 111–126.
% 
% 
% Tsakali, Vina \& Anagnostopoulou, Elena \& Alexiadou, Artemis. 2017. A new pattern of CP transparency: Implications for the analysis of Backward Control. (Paper presented at GLOW 40, Leiden.)
% 
% 
% Tsoulas, George. 1993. Remarks on the structure and the interpretation of \textit{na}{}-clauses. \textit{Studies in Greek Linguistics} 14. 191–206.
% 
% 
% 
% Varlokosta, Spyridoula. 1994. \textit{Issues on Modern Greek sentential complementation}. College Park, MD: University of Maryland. (Doctoral dissertation.)
% 
% 
% 
% \begin{verbatim}%%move bib entries to  localbibliography.bib
% \end{verbatim} 
% 
% Rackowski, Andrea \& Richards, Norvin. 2005. Phase edge and extraction: A Tagalog case study. \textit{Linguistic Inquiry} 36. 565–599.
% 
% Rizzi, Luigi. 1982. \textit{Issues in Italian syntax}. Dordrecht: Foris.
% 
% Roussou, Anna. 2009. In the mood for control. \textit{Lingua} 119. 1811–1836.
% 
% Spyropoulos, Vassilios. 2007. Finiteness and control in Greek. In Davies, William D. \& Dubinsky, Stanley (eds.), \textit{New horizons in the analysis of control and raising}, 159–183. Springer.
% 
% 
% Spyropoulos, Vassilios \& Revithiadou, Anthi. 2009. Subject chains in Greek and PF processing. In Halpert, Claire \& Hartman, Jeremy \& Hill, David (eds.), \textit{Proceedings of the MIT Workshop in Greek Syntax and Semantics}, 293–309. Cambridge, MA: MITWPL.
% 
% 
% 
% Torrego, Esther. 1996. On quantifier float in control clauses. \textit{Linguistic Inquiry} 27. 111–126.
% 
% 
% Tsakali, Vina \& Anagnostopoulou, Elena \& Alexiadou, Artemis. 2017. A new pattern of CP transparency: Implications for the analysis of Backward Control. (Paper presented at GLOW 40, Leiden.)
% 
% 
% Tsoulas, George. 1993. Remarks on the structure and the interpretation of \textit{na}{}-clauses. \textit{Studies in Greek Linguistics} 14. 191–206.
% 
% 
% 
% Varlokosta, Spyridoula. 1994. \textit{Issues on Modern Greek sentential complementation}. College Park, MD: University of Maryland. (Doctoral dissertation.)
% 
% 
% 
% \begin{verbatim}%%move bib entries to  localbibliography.bib
% \end{verbatim} 
% ll-formed in the non-coreference reading, and by ?? in (57b), featuring an affected argument combined with an unaccusative, which seems to us to admit the coreference reading but to be degraded compared to the experiencer cases mentioned above):
% 
% \ea%57
%     \label{ex:alexiadou:57}
%     \gll\\
%         \\
%     \glt
%     \z
% 
%           a.  \#O    Janis    \textbf{tu}  epevale/ton katafere       na   \textbf{tu} dothi \textbf{tu  Kosta}         to danio.
% 
% the John{}-\textsc{nom} \textsc{cl-gen} imposed/\textsc{cl-acc} managed \textsc{sbjv cl-gen} give-\textsc{nact} the Kostas{}-\textsc{gen} the loan
% 
%     ‘John imposed on him for a loan to be given to Kostas.’
% 
%   b.  ??O    Janis    \textbf{tu}  epevale/ton katafere   na min  \textbf{tu} pesi  \textbf{tu  Kosta}     to vazo.
% 
% the John{}-\textsc{nom}   \textsc{cl-gen}  imposed/\textsc{cl-acc} managed  \textsc{sbjv neg cl-gen}  fall   the Kostas{}-\textsc{gen} the vase’
% 
%     ‘John imposed on Kostas not to drop the vase.’
% 
% This seems to suggest that backward coreference of this type is not only subject to the condition in (55), but requires, in addition, that the embedded clitic-doubled argument encode point of view. Perhaps this is because only experiencers qualify as subjects at some level of representation, which means that they relate to T (\citealt{Anagnostopoulou1999} for Greek, \citealt{Landau2010}). 
% 
% \section{ 4. Conclusion}
% 
% In this paper, we have discussed an asymmetry in the distribution of backward control in Greek. While the language has been argued to have BSC, it lacks BOC. As we pointed out, recently TAA (2017) argued that BSC in Greek is a side effect of the availability of an agreement chain between a null main subject and an overt embedded subject in all types of subjunctives (\textit{na}{}-clauses), and to a certain extent in indicatives (\textit{that}{}-clauses). If this is the correct analysis for BSC, the question still remains whether Greek has BOC. We showed in this paper that BOC configurations are severely limited. We related this limitation to the nature of Backward Agree, which seems to require heads of the same type. In BOC configurations, the phi-features of embedded T are not allowed to enter long-distance agreement with the phi-features of the matrix Voice. Backward co-reference is only possible in case of resumption with a dative/genitive clitic in the matrix clause and a clitic-doubled experiencer in the embedded clause, and crucially depends on the experiencer status of the embedded argument.
% 
% \section{ Acknowledgements}
% 
% We are indebted to the editors of this volume and an anonymous reviewer for very insightful comments that helped us restructure this paper. AL 554/10-1 is hereby acknowledged.
% 
% \section{ References}
% 
% 
% Agouraki, Yoryia. 1991 A Modern Greek complementizer and its significance for Universal Grammar. \textit{UCL Working Papers in Linguistics} 3. 1–24.
% 
% 
% 
% Al-Balushi, Rashid. 2008. Control in Omani Arabic. Ms. (Toronto: University of Toronto.)
% 
% 
% Alexiadou, Artemis. 1999. On the properties of some Greek word order patterns. In Alexiadou, Artemis \& Horrocks, Geoffrey \& Stavrou, Melita (eds.), \textit{Studies in Greek} syntax, 45–65. Dordrecht: Kluwer.
% 
% Alexiadou, Artemis. 2000. Some remarks on word order and information structure in Romance and Greek.~\textit{ZAS Papers in Linguistics} 20. 119–136.
% 
% 
% Alexiadou, Artemis \& Anagnostopoulou, Elena. 1997. Notes on ECM, control and raising. \textit{ZAS Papers in Linguistics} 8. 17–27.
% 
% 
% 
% Alexiadou, Artemis \& Anagnostopoulou, Elena. 1998. Parametrizing Agr: Word order, V- movement and EPP-checking. \textit{Natural Language \& Linguistic Theory} 16. 491–539.
% 
% 
% 
% Alexiadou, Artemis \& Anagnostopoulou, Elena. 2016. Rethinking the nature of nominative case. Ms. (Berlin: Humboldt-Universität zu Berlin \& Rethymnon: University of Crete.)
% 
% 
% 
% Alexiadou, Artemis \& Anagnostopoulou, Elena \& Iordachioaia, Gianina \& Marchis, Mihaela. 2010.\href{http://ifla.uni-stuttgart.de/files/BC paper.pdf}{ No objection to backward control.} In Hornstein, Norbert \& Polinsky, Maria (eds.), \textit{Movement theory of control}, 89–118. Amsterdam: John Benjamins.
% 
% 
% 
% Alexiadou, Artemis \& Anagnostopoulou, Elena \& Wurmbrand, Susi. 2014. Movement vs. long-distance Agree in raising: Disappearing phases and feature valuation.~\textit{Proceedings of NELS~}43. 1–12.
% 
% 
% 
% Anagnostopoulou, Elena. 1999. On experiencers. In Alexiadou, Artemis \& Horrocks, Geoffrey \& Stavrou, Melita (eds.), \textit{Studies in Greek syntax}, 67–93. Dordrecht: Kluwer.~
% 
% 
% 
% Anagnostopoulou, Elena. 2003. \textit{The syntax of ditransitives: Evidence from clitics}. Berlin: Mouton de Gruyter.
% 
% 
% 
% Anagnostopoulou, Elena \& Sevdali, Christina. 2017. Two modes of dative and genitive case assignment: Evidence from two stages of Greek. Ms. (Rethymnon: University of Crete \& Jordanstown: University of Ulster.)
% 
% 
% 
% Baker, Mark. 2015. \textit{Case: Its principles and its parameters}. Cambridge: Cambridge University Press. 
% 
% 
% Barbosa, Pilar. 2009. Two kinds of subject \textit{pro}. \textit{Studia Linguistica} 63. 2–58.
% 
% 
% Bejar, Susana \& Massam, Diane. 1999. Multiple case checking. \textit{Syntax} 2. 66–79.
% 
% 
% 
% Burzio, Luigi. 1986. \textit{Italian syntax}. Dordrecht: Foris.
% 
% 
% 
% Chomsky, Noam. 2001. Beyond explanatory adequacy. In Belletti, Adriana (ed.), \textit{Structures and beyond}, 104–131. Oxford: Oxford University Press.
% 
% 
% Frascarelli, Mara. 2007. Subjects, topics and the interpretation of referential \textit{pro}. \textit{Natural Language \& Linguistic Theory} 25. 691–734.
% 
% 
% Fukuda, Shinichiro. 2008. Backward control. \textit{Language and Linguistics Compass} 2. 168–195.
% 
% 
% 
% Giannakidou, Anastasia. 1997. \textit{The landscape of polarity items}. Groningen: University of Groningen. (Doctoral dissertation.)
% 
% 
% Giannakidou, Anastasia \& Merchant, Jason. 1997. On the interpretation of null indefinite objects in Greek. \textit{Studies in Greek Linguistics} 17. 290–303.
% 
% Grano, Thomas \& Lasnik, Howard. 2016. How to neutralize a finite clause boundary: Phase theory and the grammar of bound pronouns. Ms. (Bloomington, IN: Indiana University \& College Park, MD: University of Maryland.)
% 
% Halpert, Claire. 2016. Raising parameters. \textit{Proceedings of WCCFL 33}. 186–195.
% 
% 
% Herbeck, Peter. 2013. On backward control and clitic climbing: On the deficiency of non-finite domains in Spanish and Catalan. \textit{Proceedings of ConSOLE XXI, 2013}. 123–145.
% 
% 
% Hinterhölzl, Roland. 2006. \textit{Scrambling, remnant movement, and restructuring in West Germanic}. Oxford: Oxford University~Press.~
% \textstyleFootnoteSymbol{}
% 
% Holmberg, Anders. 2005. Is there a little pro? Evidence from Finnish. \textit{Linguistic Inquiry} 36. 533–564.
% 
% 
% 
% 
% Hornstein, Norbert. 1999. Movement and control. \textit{Linguistic Inquiry} 30. 69–96.
% 
% 
% 
% Iatridou, Sabine. 1993. On Nominative Case assignment and a few related things. \textit{MIT Working Papers in Linguistics} 19. 175–198.
% 
% 
% Koopman, Hilda \& Szabolcsi, Anna. 2000. \textit{Verbal complexes}. Cambridge, MA: MIT Press.~
% 
% 
% Kotzoglou, George. 2002. Greek ‘ECM’ and how to control it. \textit{Reading Working Papers in Linguistics} 6. 39–56. 
% 
% 
% Kotzoglou, George \& Papangeli, Dimitra. 2007. Not really ECM, not exactly control: The ‘quasi-ECM’ construction in Greek. In Davies, William D. \& Dubinsky, Stanley (eds.), \textit{New horizons in the analysis of control and raising}, 111–132. Dordrecht: Springer. 
% 
% Kratzer, Angelika. 2009. Making a pronoun: Fake indexicals as windows into the properties of pronouns. \textit{Linguistic Inquiry} 40. 187–237.
% 
% 
% Landau, Idan. 1999. \textit{Elements of control}. Cambridge, MA: MIT. (Doctoral dissertation.)
% 
% 
% 
% Landau, Idan. 2004. The scale of finiteness and the calculus of control. \textit{Natural Language \& Linguistic Theory} 22. 811–877.
% 
% 
% Landau, Idan. 2007. Movement-resistant aspects of control. In Davies, William D. \& Dubinsky, Stanley (eds.), \textit{New horizons in the analysis of control and raising}, 293–325. Dordrecht: Springer.
% 
% Landau, Idan. 2010. \textit{The locative syntax of experiencers}. Cambridge, MA: MIT Press.
% 
% 
% Landau, Idan. 2015. \textit{A two-tiered theory of control}. Cambridge, MA: MIT Press.
% 
% 
% Marantz, Alec. 1991. Case and licensing. (Paper presented at the 8th Eastern States  Conference on Linguistics, University of Maryland, Baltimore.)
% 
% Miyagawa, Shigeru. 2017. \textit{Agreement beyond phi}. Cambridge, MA: MIT Press.
% 
% 
% Monahan, Philip J. 2003. Backward object control in Korean. \textit{Proceedings of WCCFL 22}. 356–369.
% 
% 
% Ordóñez, Francisco. 2009. Verbal complex formation and overt subjects in infinitivals in Spanish. Ms. (Stony Brook, NY: Stony Brook University.)
% 
% 
% Philippaki-Warburton, Irene \& Veloudis, Jannis. 1984. The subjunctive in complement clauses. \textit{Studies in Greek Linguistics} 5. 87–104.
% 
% 
% Pesetsky, David \& Torrego, Esther. 2007. The syntax of valuation and the interpretability of features. In Karimi, Simin \& Samiian, Vida \& Wilkins, Wendy K. (eds.), \textit{Phrasal and clausal architecture: Syntactic derivation and interpretation}, 262–294. Amsterdam: John Benjamins.
% 
% 
% Polinsky, Maria \& Potsdam, Eric. 2002. Backward control. \textit{Linguistic Inquiry} 33. 245–282.
% 
% 
% Polinsky, Maria \& Potsdam, Eric. 2006. \href{http://users.clas.ufl.edu/potsdam/papers/Syntax9.pdf}{Expanding the scope of control and raising}. \textit{Syntax} 9. 171–192.
% 
% Potsdam, Eric. 2006. Backward object control in Malagasy: Against an empty category analysis. \textit{Proceedings of WCCFL} \textit{25}. 328–336. 
% 
% 
% Potsdam, Eric. 2009. Malagasy backward object control. \textit{Language} 85. 754–784.
% 
% 
% Rackowski, Andrea \& Richards, Norvin. 2005. Phase edge and extraction: A Tagalog case study. \textit{Linguistic Inquiry} 36. 565–599.
% 
% Rizzi, Luigi. 1982. \textit{Issues in Italian syntax}. Dordrecht: Foris.
% 
% Roussou, Anna. 2009. In the mood for control. \textit{Lingua} 119. 1811–1836.
% 
% Spyropoulos, Vassilios. 2007. Finiteness and control in Greek. In Davies, William D. \& Dubinsky, Stanley (eds.), \textit{New horizons in the analysis of control and raising}, 159–183. Springer.
% 
% 
% Spyropoulos, Vassilios \& Revithiadou, Anthi. 2009. Subject chains in Greek and PF processing. In Halpert, Claire \& Hartman, Jeremy \& Hill, David (eds.), \textit{Proceedings of the MIT Workshop in Greek Syntax and Semantics}, 293–309. Cambridge, MA: MITWPL.
% 
% 
% 
% Torrego, Esther. 1996. On quantifier float in control clauses. \textit{Linguistic Inquiry} 27. 111–126.
% 
% 
% Tsakali, Vina \& Anagnostopoulou, Elena \& Alexiadou, Artemis. 2017. A new pattern of CP transparency: Implications for the analysis of Backward Control. (Paper presented at GLOW 40, Leiden.)
% 
% 
% Tsoulas, George. 1993. Remarks on the structure and the interpretation of \textit{na}{}-clauses. \textit{Studies in Greek Linguistics} 14. 191–206.
% 
% 
% 
% Varlokosta, Spyridoula. 1994. \textit{Issues on Modern Greek sentential complementation}. College Park, MD: University of Maryland. (Doctoral dissertation.)
% 
% 
% 
% \begin{verbatim}%%move bib entries to  localbibliography.bib
% \end{verbatim} 
% 
{\sloppy
\printbibliography[heading=subbibliography,notkeyword=this]
}
\end{document}
