\documentclass[output=paper]{langsci/langscibook} 
\author{Guido Mensching\affiliation{Georg-August-Universität Göttingen}}
\title{Extraction from DP in French: A minimalist approach}

% \chapterDOI{} %will be filled in at production

% % \epigram{Change epigram in chapters/03.tex or remove it there }
\abstract{This article is about the extraction of French PP complements of nouns headed by \textit{de}, mostly in \textit{wh} and relative clause contexts. After a review of the literature on extraction in French, it addresses the issue of the constraints on extraction in cases with multiple arguments, eventually following \citet{Kolliakou1999} in assuming that there can only be one argument of a noun, whereas other expressions are adjuncts. I then explain the relevant extractions within the Minimalist Program: on the assumption that DPs are phases, an extracted item must first move to the phase edge, as assumed in previous accounts. The exact extraction mechanism is then modeled by assuming a phi-probe plus an unvalued operator feature on the D head. The fact that only complements introduced by the preposition \textit{de} can be extracted from the DP is explained by considering \textit{de} as a post-syntactic marking for genitive case, which is assigned by the phi-probe.}
\maketitle

\begin{document}

\section{Introduction}% 1. 

\subsection{Aims and organization of the article}% 1.1 

This article readdresses the extraction of elements from a DP, which has been a topic for the last fifty years or so within the context of long-distance dependencies and DP islands (cf. \citealt{Ross1967}; \citealt{Sportiche1981}; \citealt{Huang1982}; \citealt{Obenauer1985Connectedness,Obenauer1985identification,Obenauer1994}; \citealt{Chomsky1986Barriers}; \citealt{Cinque1990}; \citealt{Szabolcsi2006}, among many others). I concentrate on French, a language for which the phenomena at issue have been intensely discussed within generative grammar, in particular in the 1980s and 1990s (e.g. by \citealt{Tellier1990}; \citealt{Sportiche1981}; \citealt{Obenauer1994}; \citealt{Pollock1989}; \citealt{Valois1991}; \citealt{Godard1992}). Before going into the reasons that motivate my reopening this debate, let me illustrate the structures that I am interested in.

\ea%1
    \label{ex:mensch:1}
    \ea[*]{
    \gll Qui  connais-tu [\textsubscript{DP}  l’  homme  qui     a   vu \sout{qui}] ?\\
         who  know-you   {} the  man    who    has  seen  whom\\
    \glt ‘Who do you know the man who has seen \_\_?’ (cf. \citealt{Sportiche1981}: 222) }
    \ex[*]{
    \gll \relax [\textsubscript{PP} De  qui]  est-ce que [\textsubscript{DP}  la  secrétaire  \soutp{[\textsubscript{PP}}{3}  \soutp{de}{2}  \soute{qui]}]      t’a  téléphoné? \\
       {}  of  whom   is-this that {}  the  secretary  {}  of  whom  you.has called \\
    \glt ‘Of whom has the secretary \_\_ phoned?’ (cf. \citealt{Tellier1990}: 306–307) }
    \z
\z    



\ea%2
    \label{ex:mensch:2}
    \ea[*]{
    \gll \relax[\textsubscript{PP} Sur qui]  as-tu  lu  [\textsubscript{DP}  le  livre \soutp{[\textsubscript{PP}}{3}  \soutp{sur}{2} \soute{qui]}] ?\\
       {}  on  whom  have-you  read {} the  book {}  on whom\\
    \glt ‘On whom have you read the book \_\_’?   }
    \ex[*]{
    \gll \relax[\textsubscript{PP}  A qui]    avez-vous vu [\textsubscript{DP}  une  amie \soutp{[\textsubscript{PP}}{3}  \soutp{a}{3} \soute{qui]}] ?\\
        {} to whom  have-you seen {}  a  friend {}   to whom\\
    \glt ‘Of whom have you seen a friend \_\_? (similar to \citealt{Grosu1974}: 312, Footnote 3)  }
    \ex[]{
    \gll \relax[\textsubscript{PP}  De  qui]     avez-vous vu  [\textsubscript{DP} une  photo \soutp{[\textsubscript{PP}}{3}  \soutp{de}{2}  \soute{qui]}] ?\\
        {} of  whom  have-you seen {} a  photo  {}  of  whom\\
    \glt ‘Of whom have you seen a photo \_\_?’  }
    \ex[]{
    \gll \relax[\textsubscript{PP}  De quel livre]  connais-tu [\textsubscript{DP}  la  fin   \soutp{[\textsubscript{PP}}{3} \soutp{de}{2} \soutp{quel}{5} \soute{livre]}] ?\\
        {} of which book  know-you  {}  the  end {} of which book\\
    \glt ‘Of which book do you know the end \_\_?’ (cf. \citealt{Sportiche1981}: 224) }
    \z
\z

The examples in (1a–b) illustrate a complex DP island and a subject island, respectively. These structures involve the extraction of a constituent (in this case a \textit{wh} element) from a deeply embedded syntactic region (usually a clause) within a DP, or from a DP that is a subject. Extractions from complex DP islands and subject islands are usually considered as ungrammatical in all languages and are not the focus of this article.\footnote{The subject condition does not hold for all subjects, but mainly for subjects of transitive and unergative verbs; cf. e.g. \citet[153–154]{Chomsky2008} (see \citealt{Broekhuis2005}: 64–65 for discussion); for French cf. \citet[90]{Tellier1991}. For other exceptions, see \citet{Truswell2005} and the references mentioned there, in particular with respect to “possessor extraction”, to which the French cases mentioned by \citet{Tellier1990} for the relative element \textit{dont} (also cf. \citealt{Heck2008,Heck2009}) can be argued to belong. \citet{Stepanov2007} claims that subject islands are not universal, in contrast to adjunct islands.} Instead, I will be mostly concerned with cases like those in (2), i.e. the extraction of a PP from a complement or adjunct DP. Without considering the status of the PP for now, the data in (2a–d) suggest that, in French, extraction of a PP that contains a \textit{wh} element is grammatical when the PP is headed by the preposition \textit{de} and ungrammatical with other prepositions. These facts also apply to relative clauses:\footnote{And, in addition, to focusing via fronting (if available) and clefting, see (23c) and Footnote 17 in §4.}

\ea%3
    \label{ex:mensch:3}
    \ea[]{
    \gll le  linguiste [\textsubscript{PP} duquel/dont]\footnotemark{}  tu  as  lu  le  livre [\textsubscript{DP} \soute{[\textsubscript{PP}}        \soute{duquel/dont]}]\\
             the  linguist {} of.which  you  have  read  the  book      {} {}   of.which\\
    \glt     ‘the linguist of (= by) whom you have read the book \_\_’}
    \ex[*]{
    \gll  le  linguiste [\textsubscript{PP}  sur lequel / sur qui]  tu  as  lu  le  livre  [\textsubscript{DP} \soutp{[\textsubscript{PP}}{2}  \soutp{sur}{2} \soutp{lequel}{2} \soutp{/}{2} \soutp{sur}{2} \soute{qui]}]\\
          the  linguist {}  on which / on whom  you  have  read  the  book {} {} on which / on whom\\
    \glt ‘the linguist on whom you have read the book \_\_’ }
    \z
\z
\footnotetext{French has a relative complementizer \textit{que}, which cannot be used after prepositions and is thus irrelevant here. After prepositions, we find the relative pronoun \textit{lequel} (fem.: \textit{laquelle}, plur.: \textit{lesquels/lesquelles}), which combines with the preposition \textit{de} in the masculine singular and in the plural forms (\textit{duquel}, \textit{desquel(le)s}). The element \textit{dont} is invariable and equivalent to \textit{duquel}, \textit{de laquelle}, \textit{desquel(le)s}, thus representing a kind of relative pro-PP. In addition, for persons, \textit{de} \textit{qui} ‘of whom’ can be used. I will assume here that relative pronouns move to [spec,CP], pied-piping the preposition (i.e. the whole PP moves). I will not consider \citegen{Kayne1994} raising-analysis of relative clauses, nor the idea that \textit{dont} might better be analyzed as a complementizer. However, my approach presented in §4 can easily be made compatible with these theories.}

As we shall see, this first rough approximation needs some refinement, and, in addition, problems arise when the DP contains more than one PP headed by \textit{de}, as shown in (4a,b) from Milner (\citeyear{Milner1978,Milner1982}; quoted in \citealt{Sag1994}). Although the relative element \textit{dont} is generally exempt from the subject condition mentioned above,\footnote{See Footnote 1.} as shown by the grammaticality of (4a), the example in (4b) is ungrammatical:

\ea%4
    \label{ex:mensch:4}
    \ea[]{
    \gll M.  X  [\textsubscript{PP} dont]\textsubscript{} [\textsubscript{DP}    la  maison  [\textsubscript{PP}  de  {Le Corbusier}] \soutp{[\textsubscript{PP}}{1} \soute{dont]}]  n’  est  guère  confortable.\\
         Mr.  X {}   of.whom {}  the  house     {} of  {Le C.}   {}    of.whom   \textsc{neg}  is  hardly  comfortable\\
    \glt ‘Mr. X, whose house of (= by) Le Corbusier \_\_ is hardly comfortable.’}
    \ex[*]{
    \gll Le Corbusier [\textsubscript{PP} dont] [\textsubscript{DP}  la  maison \soutp{[\textsubscript{PP}}{2}  \soute{dont]} [\textsubscript{PP}  de  M.  X]] n’  est  guère  confortable.\\
         Le C.  {}    of.whom {}  the  house {}   of.whom {}  of  Mr. X \textsc{neg}  is  hardly  comfortable\\
    \glt ‘Le Corbusier, by whom the house \_\_ of Mr. X. is hardly comfortable.’ }
    \z
\z

Similar facts can also be observed when extraction takes place from a direct object. There have been several proposals in the literature mostly assuming a thematic hierarchy (such as, e.g., \citealt{Pollock1989}; \citealt{Godard1992}), but this problem has never been fully resolved. The goals of this article are (i) to readdress the question of which constituent can be extracted in cases like (4) by adapting a very promising approach by \citet{Kolliakou1999}, which was formulated within the HPSG framework and has never been considered in the minimalist literature; (ii) to explain the extraction mechanism within a minimalist probe-goal approach (following \citealt{Chomsky2000} et seq.). Both goals are connected in the following sense: Kolliakou’s approach assumes that, when there is more than one PP headed by \textit{de} in a DP, only one is an argument and the other one is an adjunct (in particular, a property-denoting expression, see \citealt{Chierchia1982,Chierchia1985}), which cannot be extracted. But since there is no general ban in UG against the extraction of adjuncts, a minimalist analysis must be able to predict this property of extractions from DP. The approach I suggest at the end of the article builds on the old idea that cyclic movement must use [spec,DP] as an ‘escape hatch’ (cf. among others \citealt{Gavruseva2000}, following older ideas that go back to \citealt{Cinque1980}). In the Minimalist Program, this means that the DP is a phase (see, e.g. \citealt{Heck2008,Heck2009}, among others), and consequently, extractions must pass through its phase edge. In the constructions at issue, [Spec,DP] acts as a kind of filter that admits only argumental DPs. In a framework such as \citeauthor{Chomsky2000} (\citeyear{Chomsky2000} et seq.), argumental DPs must be identified by an unvalued case feature. In my approach, this case feature is checked and valued as [genitive] by the D head, which leads me to adopt a view that treats ‘genitive’ \textit{de} in French as a kind of case marker rather than a preposition.

This article is organized as follows. In the rest of the introduction (§1.2), I explain the framework I adopt (in particular concerning A’-movement in a probe-and-goal-based approach). In §2, I present some of the basic data at stake and summarize the discussions that took place within the GB framework. I then turn to \citegen{Kolliakou1999} explanation of the data shown in (4) and finally develop a tree structure that is compatible with Kolliakou’s view. §3 summarizes two articles (\citealt{Gutiérrez-Bravo2001}; \citealt{Cinque2014}) that analyze data from Spanish and Italian, respectively, using phase-based approaches, which are not, however, formulated according to the feature-checking system of \citeauthor{Chomsky2000} (\citeyear{Chomsky2000} et seq.). Nevertheless, both solutions offer some important insights, namely that the mechanism for extraction of material from within a DP is related to the assignment of genitive case in Romance, and that movement to the phase edge of DP at least partially involves properties of A-movement. In §4 I develop my own analysis, arguing that the data at issue can be explained straightforwardly by applying Chomsky’s (2000 et seq.) probe-goal mechanism, and, ultimately, by the feature composition of French D heads. In particular, my proposal amounts to saying that D heads contain two phi probes, one that is responsible for agreement between D and the N head, and another one that takes the complement of N as a goal, valuing its unvalued case feature as [genitive]. This second probe has an optional unvalued operator feature that comes with an [EPP]-feature, which ultimately licenses the extraction. The article ends in §5 with some conclusions. Note that most of the ingredients of my own approach can be found elsewhere, but, as far as I can see, this is the first time that they have been coherently put together using the machinery assumed in a modern minimalist framework.

\subsection{Theoretical framework}% 1.2 

For the minimalist analysis, I assume phase theory and the probe-goal approach of \citeauthor{Chomsky2000} (\citeyear{Chomsky2000} et seq.). According to phase theory, syntactic structure is built up in a step-wise fashion, where some categories (such as v and C, but crucially not T) are so-called phase heads. Every time such a phase head has projected its full structure (vP, CP), the phase domain (which is the whole complement of the relevant phase head) is sent to Spell-Out and is therefore not available for further syntactic operations.\footnote{Ultimately, this follows from the \textit{Phase Impenetrability Condition} (PIC). For the cases at issue here, it is irrelevant whether we adopt the version of \citet[108]{Chomsky2000} or \citet[13–14]{Chomsky2001Derivation}.}

Movement of elements related to the case-agreement system is implemented by unvalued phi-features on a functional head. Such features are called probes, which search the tree downward (under c-command) for a matching goal (valued phi-features). A valid goal is identified by an unvalued case feature. Matching of features triggers the operation Agree, which basically consists of three steps: (i) the probe’s unvalued phi-features receive the values of the goal; (ii) the goal’s unvalued case feature is valued according to the nature of the head that bears the probe (e.g., [Nom] in the case of T, [Acc] in the case of v, and – as I will argue – [Gen] in the case of D); (iii) the goal is licensed for movement, which takes place if the category that bears the probe has an [EPP]-feature (essentially an instruction to project a specifier) that is not checked otherwise (e.g. by an expletive). In this article, I will not consider further elaborations of the probe-goal framework such as \citet{Pesetsky2007} or \citet{Zeijlstra2012}, although my solution can be easily implemented in these and other frameworks.

The probe-goal approach has also been extended to A’-movement. I will here use a system adapted from \citet[419ff.]{Radford2004}, who assumes that the target category of A’-movement bears a probe consisting of an uninterpretable operator feature (uOp) and an [EPP] feature, while the item undergoing movement has an interpretable operator feature, with values such as [\textit{wh}], [rel(ative)] or [focus]. \citet{Chomsky2007,Chomsky2008} assumes that, instead of [EPP]-features, phase heads can optionally have other movement-inducing features, so-called edge features (EFs), which do not depend on a probe-goal relationship. In particular, a phase head can have an EF when it can trigger a movement step that causes some effect, e.g. a necessary intermediate movement step in order for the derivation to converge (cf. \citealt[149]{Chomsky2008}, \citealt{Mueller2010deriving}). The idea of EFs has been criticized in the literature, among other reasons because, since optional EFs (or P(eripheral)-features in earlier minimalist work) are held to be a universal property, it is difficult in such a framework to model cross-linguistic variation (cf. \citealt{Ceplova2001}; \citealt{Boeckx2007}; \citealt{Boeckx2011}, among others). Thus, “[d]ifferent domains count as opaque in different languages; it makes sense to look for features that vary cross-linguistically and that may induce islandhood” \citep[4]{Boeckx2011}. In the cases to be discussed in the present article, the variation at issue is even intra-linguistic, i.e. within the same language. If the (un)grammaticality of the French cases presented in §1.1 is due to the phase property of DP, as I assume, in a case such as (2a), movement of [\textsubscript{PP} \textit{sur qui}] to the DP-phase edge would be needed to make the derivation converge, and thus an EF could be freely generated on the D° head. However, the structure is ungrammatical, which calls the EF approach into question. I therefore assume a probe-goal approach for A’-movement as sketched above, in the sense that the D head contains an [EPP]-feature bound to a probe that is sensitive to particular kinds of features.

\section{Basic data and state of the art}% 2. 

As I have already mentioned in §1.1, a PP can be extracted from a complement DP in \textit{wh} and relative constructions when the PP is headed by \textit{de}, as is illustrated again in (5) and (6):

\ea%5
    \label{ex:mensch:5}
    \ea
    \gll \relax [\textsubscript{PP}  De  qui]  avez-vous vu [\textsubscript{DP}  une  photo \soutp{[\textsubscript{PP}}{2}  \soutp{de}{2}  \soute{qui]}] ?\\
      {}   of  whom  have-you seen {}  a  photo {}   of  whom \\
    \glt ‘Of whom have you seen a photo \_\_?’ (repeated from (2c))
    \ex  
    \gll \relax[\textsubscript{PP}  De  quel  livre]  connais-tu [\textsubscript{DP}  la  fin \soutp{[\textsubscript{PP}}{2}  \soutp{de}{2} \soutp{quel}{5} \soute{livre]}] ?\\
    {}     of  which  book  know-you {}  the   end {}  of which book\\
    \glt ‘Of which book do you know the end \_\_ ?’ (cf. \citealt{Sportiche1981}: 224)
    \ex  
    \gll \relax [\textsubscript{PP}  De quel linguiste]   avez-vous  rencontré [\textsubscript{DP}  les parents \soutp{[\textsubscript{PP}}{2}  \soute{de} \soutp{quel}{8}  \soute{linguiste]}] ?\\
      {}   of which linguist  have-you  met {}  the parents {}  of which linguist\\
    \glt ‘Of which linguist have you met the parents \_\_ ?’
    \z
\z


\ea%6
    \label{ex:mensch:6}
    \ea
    \gll la  maison  [\textsubscript{PP} dont / de laquelle]  vous avez  vu  [\textsubscript{DP}  une      photo \soutp{[\textsubscript{PP}}{2} \soutp{dont}{8} \soutp{/}{2} \soutp{de}{2} \soute{laquelle]}]\\
         the  house  {}  of.which /  of which     you have  seen {}   a      photo {}  of.which / of which\\
    \glt ‘the house of which you have seen a photo \_\_’  (\citealt{Grosu1974}: 312, Footnote 3)
    \ex  
    \gll un   linguiste [\textsubscript{PP}  de qui / dont]    vous  avez  rencontré [\textsubscript{DP}  les parents \soutp{[\textsubscript{PP}}{2} \soutp{de}{2} \soutp{qui}{6} \soutp{/}{2} \soute{dont]}]\\
         a  linguist {}  of whom / of.which  you  have  met {}    the     parents {}  of whom / of.which\\
    \glt ‘a linguist of whom/of which you have met the parents \_\_’  \citep[90]{Tellier1991}
    \z
\z

However, as observed by \citet[225]{Sportiche1981}, extraction is barred when the preposition \textit{de} indicates source/origin (also cf. \citealt{Tellier1991}: 90):

\ea%7
    \label{ex:mensch:7}
    \ea[*]{
    \gll \relax [\textsubscript{PP}  De  quel  pays]    avez-vous  rencontré [\textsubscript{DP} les arrivants  \soutp{[\textsubscript{PP}}{2} \soutp{de}{5} \soutp{quel}{4}  \soute{pays]}] ?\\
        {} from  which  country  have-you  met {}    the arrivals {} from     which  country\\
    \glt ‘From which country have you met the arrivals \_\_’?  }
    \ex[*]{
    \gll Cette  prison, [\textsubscript{PP}  de laquelle] [\textsubscript{DP}  le transfert \soutp{[\textsubscript{PP}}{2} \soutp{de}{2}  \soute{laquelle]}  de      l’  accusé    au   tribunal]…\\
         this  jail  {}  of which  {}  the transfer {}  of  which    of  the   accused   to.the  court\\
    \glt ‘This jail, from which the transfer \_\_ of the defendant to the court ...’ \citep[225]{Sportiche1981}  }
    \z
\z

Taking this together with the observation made in §1.1, according to which PPs headed by prepositions other than \textit{de} cannot be extracted either, \citet{Sportiche1981} arrives at the descriptive generalization in (8) for extractable constituents:

\begin{exe}
\ex%8
\label{ex:mensch:8}
Provisional descriptive generalization (I)\\
\noindent “Class 1: genitive PP’s [sic] introduced by the preposition ‘de’.

Class 2: PP’s introduced by other prepositions (including the one homophonous to ‘de’ indicating the source).

[...] the second class of PP’s is, in general, not wh-extractable. [...] However, PP’s in the first class sometimes are; PP’s in this class introduce either the object of the head noun, its subject or its possessor (if possible).” \citep[225]{Sportiche1981}
\end{exe}

Note by the way that Sportiche calls the extractable PPs ‘genitive PPs’, which he further divides into those representing “the object of the head noun, its subject or its possessor”, corresponding to the traditional division into objective, subjective and possessive genitives. In the literature published after \citet{Sportiche1981}, we observe two tendencies. First, phrases introduced by \textit{de} are considered to be arguments, whereas phrases introduced by other prepositions are taken to be adjuncts, to which those headed by \textit{de} indicating source/origin can also be argued to belong (cf. \citealt{Cinque1990}; \citealt{Moritz1994}; \citealt[586]{Alexiadou2007Noun}). Second, the terms subject, object and possessor were replaced by the theta-roles \AGENT, \THEME and \POSSESSOR (e.g. \citealt{Pollock1989}; \citealt{Valois1991}; \citealt{Godard1992}). Crucially, theta-roles were argued to be responsible for determining which constituent can be extracted in the case of multiple PPs headed by \textit{de}:

\ea%9
    \label{ex:mensch:9}
    \ea[*]{
    \gll La  jeune  femme   [\textsubscript{PP} dont] [\textsubscript{DP}  le  portrait [\textsubscript{PP} de Corot] \soutp{[\textsubscript{PP}}{2} \soute{dont]}]   se  trouve  à  la  Fondation Barnes …\\
         the  young  woman {} of.which {} the portrait {}  of C. {}  of.which      \textsc{refl}  finds  at  the  Foundation Barnes\\
    \glt ‘The young woman, the portrait of whom by Corot \_\_ is located in the Barnes Foundation ...’  }
    \ex[]{
    \gll Corot [\textsubscript{PP} dont]  [\textsubscript{DP}  le portrait  \soutp{[\textsubscript{PP}}{2} \soute{dont]} [\textsubscript{PP}  de cette  jeune        femme]]  se  trouve  à  la  Fondation Barnes …  \\
         Corot {} of.which {}  the portrait {}  of.which    {}    of this  young        woman  \textsc{refl}  finds  at  the  Foundation Barnes\\
    \glt ‘Corot, by whom the portrait \_\_ of this young woman is located in the Barnes Foundation ...’ (examples from \citealt{Godard1992}: 268–269, following \citealt{Ruwet1972}; also cf. \citealt{Sag1994}).  }
    \z
\z    



\ea%10
    (repeated from (4), from \citealt{Milner1978,Milner1982}; quoted in \citealt{Sag1994})\label{ex:mensch:10}\\
    \ea[*]{
    \gll Le Corbusier [\textsubscript{PP} dont] [\textsubscript{DP}  la  maison \soutp{[\textsubscript{PP}}{2}  \soute{dont]} [\textsubscript{PP}  de  M.  X]]     n’  est  guère  confortable ...\\
             Le C.        {}         of.which {} the  house  {}  of.which {}  of   Mr. X     \textsc{neg}  is  hardly  comfortable\\
    \glt     ‘Le Corbusier, by whom the house \_\_ of Mr. X. is hardly comfortable ...’  }
    \ex[]{
    \gll     M.  X  [\textsubscript{PP} dont]\textsubscript{} [\textsubscript{DP}    la  maison  [\textsubscript{PP}  de  Le Corbusier] \soutp{[\textsubscript{PP}}{2} \soute{dont]}]   n’  est  guère  confortable ...\\
             Mr.  X {}  of.which {} the  house  {}    of  Le C. {}     of.which   \textsc{neg}  is  hardly  comfortable\\
    \glt     ‘Mr. X, whose house of (=by) Le Corbusier \_\_ is hardly comfortable ...’  }
    \z
\z

The examples in (9) are about a portrait featuring a young lady (\THEME) painted by Corot (\AGENT). The distinction between (9a) and (9b) is supposed to show that, when there is an \AGENT and a \THEME in the same DP, the \AGENT can be extracted and the \THEME cannot. Similar facts seem to apply to (10), where the presence of a \POSSESSOR seems to block the extraction of an \AGENT. This was assumed to follow from the following thematic hierarchy: \POSSESSOR > \AGENT > \THEME (cf. \citealt{Pollock1989}; \citealt{Godard1992}). Let us assume this for now, so that the descriptive generalization in (8) can be replaced by the one in (11):

\begin{exe}
\ex%11
    \label{ex:mensch:11}
Provisional descriptive generalization (II):\\
          Argument PPs of nouns can be extracted if they
          
            \begin{itemize}
                \item are introduced by \textit{de} and
                \item bear the theta-role \AGENT, \THEME or \POSSESSOR.
            \end{itemize}
If the noun has more than one complement, only the highest in the hierarchy  \POSSESSOR > \AGENT > \THEME can be extracted. Adjunct PPs cannot be extracted.
\end{exe}

However, \citet[160]{Pollock1989} mentions some exceptions, in which the \THEME is extractable even when an \AGENT is expressed:

\ea%12
    \label{ex:mensch:12}
    \ea
    \gll La symphonie  [\textsubscript{PP} dont]  j’  aime  [\textsubscript{DP} l’  interprétation      [\textsubscript{PP} de Karajan]  \soutp{[\textsubscript{PP}}{2} \soute{dont]}] ...\\
         the symphony {} of.which  I   love   {}     the interpretation     {}       of  K. {}     of.which\\
    \glt ‘The symphony, of which (\THEME) I love the interpretation \_\_ by K. (\AGENT) ...’
    \ex  
    \gll L’  histoire [\textsubscript{PP}  dont]     je   n’  ai  jamais  pu  avoir [\textsubscript{DP} la    version   [\textsubscript{PP}  de Marie] \soutp{[\textsubscript{PP}}{2}  \soute{dont]}] ...\\
         the  story  {}   of.which     I   \textsc{neg}  have  never  could  have {}  the      version {}    of M.  {}  of.which \\
    \glt ‘The story, of which (\THEME) I could never have Mary’s (\AGENT) version \_\_ ...’
    \ex  
    \gll Les  événements [\textsubscript{PP}  dont]  j’ ai  apprécié [\textsubscript{DP} le  {compte rendu}     [\textsubscript{PP}  du Monde] \soutp{[\textsubscript{PP}}{2}  \soute{dont]}]...\\
         the  events {}    of.which  I have  appreciated {}  the  report {}  of L.M.  {}  of.which\\
    \glt ‘The events of which (\THEME) I appreciated the report \_\_ by Le Monde (\AGENT) ...’
    \z
\z

As \citet[268, Footnote 31]{Godard1992} observes (for 12c), “the complement \textit{du Monde} is a modifier rather than an argument; it is interpreted as a location, equivalent to the RC ‘which appeared in Le Monde’”. In a similar vein, \citet[86–87, Footnote 2]{Milner1982} remarks that, in an expression such as \textit{La symphonie de Beethoven de Karajan} (lit. ‘The symphony of Beethoven of Karajan’), it is not obvious that \textit{Beethoven} is an authentic \AGENT, whereas one might consider \textit{symphonie de Beethoven} as a kind of compound noun, of which \textit{Karajan} would be the only complement.

Such considerations led \citet{Kolliakou1999} to the conclusion that, in all the examples at stake – i.e. such as those in (9), (10), and (12) – there is only one argument PP in the DP.\footnote{Also cf. \citet[93]{Cinque2014}, who arrives at similar conclusions for Italian, cf. Footnote 11.} Her basis is Chierchia's (\citeyear{Chierchia1982,Chierchia1985}) distinction between IDPs (\textit{individual-denoting phrases}), i.e. phrases denoting individuals that refer to an entity in discourse, and PDPs (\textit{property-denoting phrases}), i.e. phrases denoting properties that determine a type of entity. When there is more than one PP headed by \textit{de} in a DP, there can only be one IDP, whereas the other one is necessarily a PDP. Consider the examples in (13) from \citet[736]{Kolliakou1999}:

\ea%13
    \label{ex:mensch:13}
    \ea
    \gll En ce moment,  \textit{une}  \textit{attaque}  \textit{de}  \textit{partisans}  serait  fatale. (PDP)\\
         in this  moment  an  attack  of  partisans  would.be  fatal\\
    \glt ‘At this moment a partisan attack would be fatal.’
    \ex  
    \gll \textit{L’} \textit{attaque}  \textit{des}  \textit{partisans}  a  commencé  à {7 heures}. (IDP)\\
         the attack  of.the  partisans  has  begun    at {7 o’clock}\\
    \glt ‘The attack of the partisans began at 7 o’clock.’
    \ex  
    \gll \textit{L’}  \textit{attaque}  \textit{de(s)}   \textit{partisans}  ce  matin    n’ était    pas       une  attaque  de  partisans.\\
         the  attack    of(.the) partisans  this  morning  \textsc{neg} was  not    an  attack    of  partisans\\
    \glt ‘The attack of (the) partisans this morning wasn’t a partisan attack.’
    \z
\z

Actually, the French expression \textit{une attaque de partisans} can be translated into English either as ‘an attack by partisans’ or as ‘a partisan attack’, where, in the latter case, \textit{partisans} cannot be interpreted as an \AGENT, but designates a property; thus, a \textit{partisan attack} is an attack that is typical for partisans. As Kolliakou observes, such PDPs are adjuncts, whereas only IDPs can act as arguments. In French, there are no compound expressions of the English type, but the PDP can sometimes be substituted for by a corresponding adjective; e.g. in (12a) \textit{l’interprétation de Karajan} could be paraphrased by \textit{interprétation} \textit{karajanienne}, a test which shows that \textit{de Karajan} is a PDP. That fact that PDPs behave like post-nominal adjectives confirms the idea that such expressions are adjuncts. Importantly, Kolliakou also observes that, in cases without extraction, the PDP is closer to the head noun than the argument (IDP). This is illustrated in (14), from \citet[714]{Kolliakou1999}, which also shows the interaction of syntax with extra-linguistic factors, in this case world knowledge:

\ea%14
    \label{ex:mensch:14}
    \ea[]{
    \gll la   maison  [\textsubscript{PP}   de  Le Corbusier]  [\textsubscript{PP}  de  Monsieur X] \\
         the  house  {}    of  L. C.    {}    of  Mr. X\\}
    \ex[\#]{
    \gll la  maison [\textsubscript{PP}  de Monsieur X] [\textsubscript{PP}  de Le Corbusier]    \\
         the  house  {}  of Mr. X {}   of L. C.\\ }
    \glt (cf. \citealt{Kolliakou1999}: 730)
    \z
\z    

(14b) is not accepted because, syntactically, \textit{de Monsieur X} is closer to the head noun than \textit{de Le Corbusier}, which means that it must be interpreted as a PDP-adjunct and not as a possessor argument. The unacceptability then results from the fact that, without a very specific context, there is no such thing as a typical “Mr. X house”. Just like some putative agents or possessors, apparent themes can also qualify as PDPs. Thus, the ungrammaticality of (9a) also falls into place if \textit{le portrait de la jeune femme de Corot} is interpreted as something like ‘the young woman-portrait of/by Corot’.\footnote{The status of an expression as either an IDP or a PDP often depends on world knowledge. For further illustration, see \citegen{Kolliakou1999} remarks concerning (i) and (ii):

\ea \gll le  portrait  [ d’  Aristote]  [ de  Rembrandt]\\
        the  portrait {} of  Aristotle {}  of  Rembrandt\\
    \z
    
\ea \gll le  portrait  [ de  Rembrandt]  [ d’  Aristote]\\
        the  portrait {} of  Rembrandt {} of  Aristotle\\
    \z
    
    \noindent“[\textit{D}]\textit{’Aristote} in [(i)] is different from \textit{d’Aristote} in [(ii)]: \textit{portrait} \textit{d’Aristot}\textit{e}\textbf{ }in [(i)] can identify a typical portrait representing Aristotle (or even a typical Aristotle portrait depicting someone else); on the other hand, \textit{d’Aristo}\textit{te} in [(ii)] refers to an individual named ‘Aristotle’, and who in principle can be associated in one out of many ways with the portrait (painter, owner, etc.) – provided we leave aside the historical/‘meta-linguistic’ information that biases our interpretation” \citep[748]{Kolliakou1999}.}

On these grounds, let us reject the descriptive generalization in (11) and, instead, adopt (15):

\begin{exe}
\ex%15
    \label{ex:mensch:15}
Descriptive generalization (III)

\begin{itemize}
\item A noun can select only one PP argument headed by the preposition \textit{de}, usually expressing \AGENT, \THEME or \POSSESSOR. This argument can be extracted in \textit{wh} and relative constructions.
\item Other PPs headed by \textit{de} (including those indicating \SOURCE and \linebreak PDPs), as well as PPs headed by other prepositions, are adjuncts. Adjuncts cannot be extracted in \textit{wh} and relative constructions.
\end{itemize}
\end{exe}


Kolliakou herself uses an HPSG account to derive the structures at issue in this section. Note that her theories about arguments and PDP-adjuncts in the DP can easily be expressed in a minimalist framework. I cannot discuss here the numerous proposals for the internal structure of DPs and the position of post-verbal adjectives in Romance. A quite widespread approach is to assume one or more functional projections between DP and NP (cf. the discussion in \citealt{Alexiadou2007Noun}), represented as FP in the simplified version in (16) representing (14a):

\ea%16
\label{ex:mensch:16}
\begin{forest}
[DP
    [D\\la]
    [FP, s sep+=5mm% to give the arrow some space
        [F\\maison,name=maison]
        [NP
            [PP[de Le Corbusier\\{\textit{(adjunct)}},roof]]
            [NP
                [N\\\sout{maison},name=maison2]
                [PP
                    [de Monsieur X, roof,name=monsieur]
                ]
            ]
        ]
    ]
]
\draw[-{Triangle[]}] (maison2.250) -- ++(0,-\baselineskip) -| (maison);
\draw[dashed,-{Triangle[]}] (maison2.290) -- ++(0,-3\baselineskip) -| (monsieur.south) node [near start,above] {Θ: \textsc{poss}};
\end{forest}
\z


Here, the noun \textit{maison} has one argument, to which the theta-role \POSSESSOR is assigned. The other PP, the PDP de \textit{Le Corbusier} is an adjunct, left-adjoined to NP. The head noun is raised to a functional projection (maybe NumP). It follows naturally from this analysis that the PDP adjunct is closer to the head noun, in conformity with what Kolliakou observes.

However, all this does not explain why the adjunct cannot be extracted and how the extraction of the complement can be modeled within the Minimalist Program. I will solve these problems within the account that I will develop in §4, but let us first look at some more recent work in which the DP is considered as a phase.

\section{Phase-based approaches}% 3. 

The idea that cyclic movement uses [spec,DP] (or [spec,NP] in former frameworks) as an “escape hatch” can already be found in \citet{Cinque1980} and has been elaborated on, e.g., by \citet{Stowell1989,Szabolcsi1983,Giorgi1991}, and \citet{Gavruseva2000}. In more recent work, the DP has been considered to be a phase (cf. e.g. \citealt{Svenonius2004}; \citealt{Chomsky2008}; \citealt{Heck2008,Heck2009}). For data concerning extractions from DP that are similar to those considered here, originating from Spanish and Italian, respectively, I briefly summarize \citet{Gutiérrez-Bravo2001} and \citet{Cinque2014}.

\citet{Gutiérrez-Bravo2001} is a surprisingly early article on phase theory, which, however, still follows \citet{Chomsky1995} with respect to (strong and weak) features, checking theory, and agreement. The Spanish data that this article aims to explain are similar to the French data in §2:

\ea%17
Spanish (\citealt{Gutiérrez-Bravo2001}: 111)\label{ex:mensch:17}\\
    \ea[]{
    \gll \relax [\textsubscript{PP} De  quién]  perdiste  [\textsubscript{DP}  la  traducción [\textsubscript{PP}  de  La    Odisea] \soutp{[\textsubscript{PP}}{2}  \soutp{de}{2}  \soute{quién]}] ?\\
     {}    of  whom  lost.\textsc{2sg} {}   the  translation {}    of  the        Odyssey {}  of  whom\\
    \glt ‘Of whom did you lose the translation of The Odyssey \_\_?’}
    \ex[*]{
    \gll \relax [ De qué] perdiste [\textsubscript{DP}  la  traducción [\textsubscript{PP}  de Juan] \soutp{[\textsubscript{PP}}{2}  \soutp{de}{2} \soute{qué]}]?\\
       {}  of what   lost.\textsc{2sg} {}   the   translation {}   of J. {}    of what\\
    \glt ‘Of what did you lose Juan’s translation \_\_?’}
    \z
\z

As shown in (18) below, Gutiérrez-Bravo assumes an AgrGen[itive] projection situated lower than D. In order to attract the \textit{wh} constituent to the phase edge, the D head has a strong [\textit{wh}]-feature. The covert AgrGen head has a genitive feature, which is adjoined to D. This feature will then attract the PP that also bears a [GEN]-feature to [Spec,DP] (recall that, in Chomsky’s 1995 framework, features are checked via specifier-head agreement). The attracted PP must also bear a [\textit{wh}]-feature, which checks the [\textit{wh}]-feature of D. Within the NP, a PP that encodes the \AGENT is merged in [Spec,NP], whereas the PP bearing the \THEME theta-role is the complement of N. The ungrammatical case (17b) is then explained by the Minimal Link Condition, which forces the [GEN] feature in D to attract the closest constituent that also bears a [GEN] feature. In (17b), this would be the PP \textit{de Juan}, which does not, however, bear a [\textit{wh}]-feature. For (17a), the derivation would converge in the following way:

\begin{exe}\nopagebreak\ex%18
\label{ex:mensch:18}%
\resizebox{\linewidth}{!}{\begin{forest}
[DP
    [PP\textsubscript{j}[de quién\\{[wh]}\\{[\textsc{gen}]},name=quienwhgen,roof]]
    [D'
        [D
            [$\emptyset$\textsubscript{i}\\{[\textsc{gen}]},name=emptysetgen]
            [D\\la\\{[wh]}]
        ]
        [AgrGenP
            [AgrGen\\t\textsubscript{i},name=ti]
            [FP
                [F\\traducción\textsubscript{k}] [NP
                    [PP\\t\textsubscript{j},name=tj]
                    [N'
                        [N\\t\textsubscript{k}]
                        [PP
                            [de \textit{La Odisea},roof]
                        ]
                    ]
                ]
            ]
        ]
    ]
]
\draw[-{Triangle[]}] (tj.south) -- ++(0,-\baselineskip) -| (quienwhgen.south);
\draw[-{Triangle[]}] (ti.south) -- ++(0,-\baselineskip) -| (emptysetgen.south);
\end{forest}}%
\end{exe}

Apart from the fact that this account uses an early minimalist framework, it contains some weak points. For example, the adjunction of the [GEN]-feature is not motivated, and questions arise concerning how the second genitive feature (on the N complement in (18)) is checked, and why a PP can bear a case feature in the first place. However, Gutiérrez-Bravo’s approach contains an interesting point: the incorporation of the case feature into D creates a complex D head that has both A and A’ properties. In other words, the extraction from DP “is conditioned by the possibility of the extracted constituent to check the Case feature of the adjoined functional head” (\citealt{Gutiérrez-Bravo2001}: 116).

In a similar vein, \citet[23]{Cinque2014} assumes, for parallel Italian data, that “DPs are phases (which forces movement to the highest specifier of DP, before extraction takes place)”. He argues that this specifier is an A- (rather than an A’-) position. Cinque furthermore makes use of the notion “subject of DP” (cf. \citealt{Cinque1980}), which can be identified, among others, by means of the following test: “the subject is the only argument of the noun which can be expressed by a possessive adjective” (\citealt{Cinque2014}: 95, Footnote 1).\footnote{This also applies to French. A second test for Italian is mentioned by Cinque (ibid.): “the subject is the only argument of the noun which cannot be expressed by a 1st and 2nd pers. sing. pronoun preceded by \textit{di}”. French is even stricter here, since it includes the third person, too. Thus: \textit{la maison *de moi / *de toi / *de lui / *d’elle / *d’eux}. I will not examine this property here, but agree with \citegen[49]{Cinque2014} idea, which roughly amounts to saying that these pronouns are incompatible with the genitive case because their forms are fixed for oblique case.}

His derivation of a DP containing a “subject” is very complex and can only be sketched here. It starts off as shown in (19) (representing the DP \textit{l’opinione di Gianni} (‘Gianni’s opinion’, lit. ‘the opinion of Gianni’), where the “subject” moves from its “thematic (Merge) position to a licensing position (Spec AgrSP or NominativeP [...])” (2014: 92):

\ea%19
    \label{ex:mensch:19}
    \begin{forest}
    [AgrSP
        [NP\textsubscript{i}[Gianni,roof]]
        [AgrS'
            [AgrS°] [XP
                [t\textsubscript{i}] [X'
                    [X°]
                    [NP [opinione,roof]]
                ]
            ]
        ]
    ]
    \end{forest}
\z

The rest of the derivation yields the structure sketched in (20).\footnote{The three dots represent the “criterial subject position” of the DP (SubjP; cf. \citealt{Rizzi2007Properties}; \citealt{Rizzi2007Strategies}), which remains empty (containing \textit{pro}) in the case at issue and is irrelevant for our discussion.} A GP (genitive phrase) headed by the element \textit{di} is merged above AgrSP, whereas the XP remnant containing the head noun is moved to a higher specifier (cf. \citealt{Kayne1999,Kayne2004}, among others).

\ea%20
    \label{ex:mensch:20}
\begin{forest}
[DP
    [D°\\l']
    [\ldots
        [\ldots]
        [ZP
            [XP\textsubscript{j}[{[t\textsubscript{i}[\textsubscript{NP} opinione]]},roof]]
            [Z'
                [Z°] [GP
                    [G\\di]
                    [AgrSP
                        [NP\textsubscript{i} [Gianni\textsubscript{i}]]
                        [AgrS' [AgrS°] [t\textsubscript{j}]]
                    ]
                ]
            ]
        ]
    ]
]
\end{forest}
\z
 
It is precisely the “subject” of a DP that can be extracted, for example if we use the \textit{wh} element \textit{chi} ‘who’ instead of \textit{Gianni} in (19) and (20); technically, the whole GP must be extracted in this case. \citet[94--95]{Cinque2014} explains the impossibility of extracting other arguments when the “subject” position is filled (cf. our French cases in §2) in terms of relativized minimality (\citealt{Rizzi1990,Rizzi2001,Rizzi2004}): the external specifier of DP is an A-position in Italian and other Romance languages, essentially because “if it were an A’-position, we would expect any argument or adjunct to be able to move into it” (2014: 91).\footnote{For further evidence with respect to the properties of [spec,DP] as an A-position, see \citet[87--91]{Cinque2014}.} Although this is not very clear in the article, what is meant here is that a non-“subject” argument would have to move out of the NP in the situation in (19), thus crossing the subject in Spec-AgrSP, an A-position. Since the ultimate goal of the constituent is [spec,DP], which is an A-position, too, this movement is barred by relativized minimality.

Note that Cinque’s idea concerning the A-status of [spec,DP] is similar to Gutiérrez-Bravo’s assumption. Although Gutiérrez-Bravo assumes that [spec,DP] has a mixed A and A’ status, the primary trigger of movement is the case feature, while the operator feature (e.g. [\textit{wh}]) is checked as a “free rider”. With respect to the issue of why a PP can have case, the advantage of Cinque’s approach is that the relevant types of N arguments are not PPs but DPs, and the element \textit{di} or French \textit{de} is a genitive head.\footnote{Note by the way that Cinque arrives at conclusions very similar to those of \citet{Kolliakou1999} sketched in §2: “The ungrammaticality (or marginal status) of two \textit{di}{}-phrases with derived nominals based on transitive verbs (*[\textit{la distruzione} [\textit{del ponte}] [\textit{dei nemici}]] ‘the destruction of the bridge of the enemies’/[*\textit{la distruzione} [\textit{dei nemici}] [\textit{del ponte}]] ‘the destruction of the enemies of the bridge’, as opposed to [\textit{la distruzione} [\textit{del ponte}] [\textit{da parte dei nemici}]] ‘the destruction of the bridge by the enemies’) [...] may suggest that, in the Italian DP, only one \textit{di} is available to license genitive Case [...]. Where two \textit{di}-forms appear to be (marginally) possible ([\textit{l’organizzazione} [\textit{della mostra}] [\textit{di Gianni}]] ‘the organization of the exhibition of G.’), the subject \textit{di Gianni} might in fact be a reduced relative clause ([\textit{l’organizzazione} [\textit{della mostra}] [(\textit{che era}) \textit{di Gianni}]] ‘the organization of the exhibition which was by Gianni’.” For French, similar speaker judgments apply; i.e. \textit{Le portrait d’Aristote de Rembrandt} is more marginal than \textit{Le portrait d’Aristote par Rembrandt}.} Within the Minimalist Program, it remains to be seen, however, how the A or A’ status of a projection should be encoded, an issue to which I will return in §4.

Cinque’s article contains some other interesting aspects, e.g. his criticism of those explanations that involve a thematic hierarchy (cf. §2):

\begin{quote}\relax [...] this is true only inasmuch as thematic roles enter into the determination of what eventually counts as the syntactic subject. When divorced from the notion of subject the thematic hierarchy fails to predict what can be extracted and what cannot. Not all Agents/Experiencers can extract in the absence of Possessors (e.g. those introduced by a \textit{by} phrase). Not all Themes can extract in the absence of Agents/Experiencers and Possessors (e.g. the Theme of Ns like \textit{desiderio} ‘desire’; cf. Cinque 1980, p. 64; Longobardi 1991, p. 66; Kolliakou 1999, sect. 2.3). Ultimately, only what qualifies by the two diagnostics above as the \textit{syntactic} subject of the DP can extract.\\\hbox{}\hfill{\citep[95–96, Footnote 1]{Cinque2014}}\end{quote}

Note, however, that Cinque’s argumentation pushes the question back one step, because what is going to be realizable as a subject of DP may still depend on some kind of hierarchy.\footnote{Thanks to an anonymous reviewer for pointing this out.}

Summarizing, both \citet{Gutiérrez-Bravo2001} and \citet{Cinque2014} consider DPs to be phases.\footnote{For further arguments for the status of DP as a phase in connection with \textit{pied-piping} in French, see \citet{Heck2008}.} Thus, a constituent can only be extracted from the DP if it first moves to its (external) specifier. Both accounts agree on the fact that this movement step has properties of A-movement, related to case properties (the extracted constituent bears genitive case). We can thus observe a kind of circle in the discussion of Romance extraction phenomena: while at the beginning of the 1980s the constituents extracted from the DP were considered to be genitives, the discussion in the course of the 1980s and 1990s turned on theta-roles rather than case, an initiative with doubtful success. More modern (minimalist) approaches based on phase theory have returned to considering the relevant constituents (headed by elements such as French \textit{de} or Italian \textit{di}) as exponents of genitive case. This will be important to keep in mind for what follows.

\section{A phase- and probe-based minimalist analysis} %

\subsection{Basic outline}%4.1

In this section, I develop a phase-based account that is in conformity with the minimalist probe-and-goal framework (\citealt{Chomsky2000} et seq.). As we will see, some of the insights of the previous solutions sketched in §3 independently follow from the application of this framework. Let us begin with the illustration in (21), based on (16):

\ea%21
    \label{ex:mensch:21}
    \begin{forest}
    [DP, s sep+=1cm
        [{[spec]},name=spec]
        [D',name=Dprime
            [D\\la,name=dla]
            [FP,name=FP
                [F\\maison,name=maison2] [NP
                    [PP [de Le Corbusier\\{*}de qui\\\textit{(adjunct)},roof,name=Corbusier]]
                    [NP, s sep+=1cm
                        [N\\\sout{maison},name=maison]
                        [PP[de Monsieur X\\de qui,name=dequi,roof]]
                    ]
                ]
            ]
        ]
    ]
    \tikzset{decoration=crosses}
    \draw[-{Triangle[]}] (Corbusier.west) -| (spec.300) ;
    \draw[decorate] (spec.300) ++(0,-1ex) |- (Corbusier.west);
    \draw[-{Triangle[]}] (maison) |- (dequi.170) node[near end,above] {Θ: \textsc{poss}};
    \draw[-{Triangle[]}] (dequi.195) -| (spec.240);
    \draw [dotted,thick] let \p1=(Dprime), \p2=(FP), \p3=(maison2), \p4=(dla) in (\x2,\y1) to [looseness=0.5,bend right] (\x4,\y3) to  ++(0,-7\baselineskip);
    \end{forest}
\z

The dotted curved line represents the phase boundary; i.e. the part lower than the D head is not accessible for further computation. The element that we want to extract must be raised to [spec,DP] as indicated by the arrows. The problem is now why the extraction of the adjunct is barred while that of the complement is not, even though the adjunct is closer to the D head.\footnote{Note that this problem could easily be resolved by assuming the right-ascending theory of adjuncts (see, e.g., \citealt{Andrews1983}) together with relativized minimality. In this article, I prefer to follow \citegen{Kayne1994} antisymmetric approach, according to which adjuncts always left-adjoin. In addition, the problem persists independently of the approach chosen, because adjuncts cannot be extracted even if there is no argument in the structure (see below). In \citegen{Cinque2014} framework, this would follow because [spec,DP] is an A-position, but note that, according to the Minimalist Program as assumed here, the notion “A-position” is not a primitive of syntax and must be expressed through features, as will be done in what follows. As for relativized minimality, note that it is not easily compatible with a minimalist, derivational approach; see \citet{Boeckx2008,Boeckx2009} for discussion.}

  As mentioned in §1.2, and as became obvious, as I hope, in §2–3, we cannot just assume an optional edge feature (EF) on D, which could just attract any constituent needed for further computation – in (21), either the \textit{wh}{}-marked complement or the \textit{wh}{}-marked adjunct. If we cannot use an EF, we need to assume an [EPP]-feature that is connected to a probe. One necessary condition for the probe (following the framework adopted in §1.2) is that it contains an unvalued operator feature (uOp). Let us provisionally assume (22), a probe that is sensitive to interrogative, relative and focalized elements, which I argue bears [vOp] (see §1.2 above),\footnote{In addition, the extractable item must have another unvalued feature [uF] (not corresponding to the case feature discussed in §4.2), which is valued by the probe of the final landing site. Thus, for the cases at issue, C° will have a probe that not only contains [uOp] but is also able to value [uF] on D. This is to ensure that the \textit{wh} or relative phrase cannot remain in any of the intermediate positions ([spec,DP] or [spec,vP]), which would lead to ungrammaticality. I will not investigate the nature of this feature in the present article.} thus licensing the structures in (23):\footnote{Movement to the phase edge of v is not represented in these simplified structures.} 

\ea%22
    \label{ex:mensch:22}
   Features of D (provisional formalization I)\\
   \begin{forest}
   [{[}uOp{]},baseline
    [{[}EPP{]}]
   ]
   \end{forest}with $\text{uOp} ≙ \text{Op} = \text{X, X} \in \{\textit{wh}, \text{rel}, \text{Focus}\}$
\z

           

\ea%23
    \label{ex:mensch:23}
    \ea
    \gll \relax [\textsubscript{PP}  De qui]  as-tu  vu [\textsubscript{DP} \soutp{[\textsubscript{PP}}{2}  \soutp{de}{2} \soute{qui]} [\textsubscript{DP}   la  photo \soutp{[\textsubscript{PP}}{2} \soutp{de}{2} \soute{qui]}] ?\\
       {}  of whom  have-you  seen  {} {}  of whom {}  the  photo   {}     of whom\\
    \glt ‘Of whom have you seen the photo \_\_ ?’
    \ex  
    \gll le  prof [\textsubscript{PP}  dont]    j’  ai  vu [\textsubscript{DP} \soutp{[\textsubscript{PP}}{2}  \soute{dont]} [\textsubscript{DP}  la   photo  \soutp{[\textsubscript{PP}}{2} \soute{dont]}] \\
         the  professor {}  of.which  I  have  seen {} {}   of.which {}  the      photo  {}  of.which\\
    \glt ‘the professor of whom I have seen the photo \_\_’
    \ex  
    \gll (C’est) [\textsubscript{PP} DE JEAN]   (que)  j’ ai  vu   [\textsubscript{DP} \soutp{[\textsubscript{PP}}{2} \soutp{de}{2} \soute{Jean]}  la  photo \soutp{[\textsubscript{PP}}{2} \soutp{de}{2} \soute{Jean]}].\footnotemark\\
         it.is {}  of  Jean    that  I have seen {} {}  of J.    the  photo {}     of J.\\
    \glt ‘It is JEAN that I have seen the photo of \_\_.’
    \z
\z
\footnotetext{The Standard French focusing strategy is to use a cleft sentence. The focus-fronting option indicated by the brackets is available in other varieties of French.}

However, the provisional formalization in (22) clearly overgenerates: a D head with this feature composition could also attract the adjunct \textit{de qui} in the upper PP in (21) so as to yield the ungrammatical (24a) vs. the grammatical (24b). In fact, according to the descriptive generalization in (15), adjuncts can never be attracted, witness the PP indicating \SOURCE in (24c) and an adjunct with another preposition as in (24d):

\ea%24
    \label{ex:mensch:24}
    \ea[*]{
    \gll \relax[\textsubscript{PP}  De  qui]  as-tu  vu  [\textsubscript{DP} \soutp{[\textsubscript{PP}}{2}  \soutp{de}{2} \soute{qui]} [\textsubscript{DP}   la  maison \soutp{[\textsubscript{PP}}{2} \soutp{de}{2}  \soute{qui]} [\textsubscript{PP}  de M. X]]] ?\\
      {}     of  whom  have-you  seen {} {}  of whom {}  the  house {}   of  whom {}   of Mr. X\\
    \glt     ‘Of whom have you seen the house \_\_ of Mr. X?’  }
    \ex[]{
    \gll \relax[\textsubscript{PP}  De  qui]  as-tu  vu [\textsubscript{DP} \soutp{[\textsubscript{PP}}{2}  \soutp{de}{2} \soute{qui]} [\textsubscript{DP}  la   maison     [\textsubscript{PP}  de Le Corbusier] \soutp{[\textsubscript{PP}}{2}  \soutp{de}{2}  \soute{qui]}]]?\\
        {} of  whom  have-you  seen  {} {}  of whom {} the  house  {}      of Le C. {}     of  whom\\
    \glt ‘Of whom have you seen the house of Le Corbusier \_\_?’ }
    \ex[*]{
    \gll \relax[\textsubscript{PP}  D’  où]  aimes-tu [\textsubscript{DP} \soutp{[\textsubscript{PP}}{2}  \soutp{d’}{10}  \soute{où]} [\textsubscript{DP} les  bananes \soutp{[\textsubscript{PP}}{2}   \soutp{d’}{10}   \soute{où]}]] ?\\
         {} from  where  like-you  {} {}  from  where {}  the  bananas {} from where\\
    \glt ‘From where do you like the bananas \_\_?’ }
    \ex[*]{
    \gll \relax[\textsubscript{PP}  Sur  qui]  as-tu  lu [\textsubscript{DP} \soutp{[\textsubscript{PP}}{2}  \soutp{sur}{2}  \soute{qui]} [\textsubscript{DP} un  livre \soutp{[\textsubscript{PP}}{2} \soutp{sur}{2}  \soute{qui]}]] ?\\
         {} on  whom  have-you  read {} {}  on  whom {}  a  book {}  on  whom\\
    \glt ‘On whom have you read a book \_\_’?  }
    \z
\z
 
For this reason, the feature set in (22) is not enough. Since we are looking for features that can act as a probe for detecting only arguments, an obvious solution is phi-features. Recall that, according to \citet{Chomsky2000}, subjects of sentences are the goals of a phi-probe in T, whereas direct objects are the goals of a phi-probe in v. If we can generalize from this, arguments are typically the goals of a phi probe. Let us therefore modify (22) by assuming that the D head has a complex probe consisting of the unvalued operator features plus unvalued phi-features. The refined version of (22) is given in (25):

\ea%25
    \label{ex:mensch:25}
    Features of D (provisional formalization II)\\
    \begin{tabbing} [uOp]\hspace{1em}\= Lorem Ipsum\kill
        [uOp] \> with $\text{uOp} ≙ \text{Op} = \text{X}, \text{X} \in \{\textit{wh}, \text{rel}, \text{Focus}\}$\\
        \begin{forest}
            [{[}u$\varphi ${]},baseline    
            [{[}EPP{]}]
        ]\end{forest} \> with u$\varphi ≙ \text{person} = \text{X}, \text{number} = \text{Y}, \text{gender} = \text{Z}$
    \end{tabbing}
\z

Although this cannot be the final version, as we will see in §4.2–4.3, this formalization has the advantage of coinciding with \citegen{Gutiérrez-Bravo2001} hypothesis that [spec,DP] is a hybrid A/A’-position. This is so because the [EPP]-feature is linked to a complex probe that contains an operator feature (held responsible for A’-movement) and phi-features (related to A-movement).\footnote{Note that the lexical entry of a D head in (25), as well as the further elaboration in what follows (in particular on genitives with \textit{de}), is language-specific. Hence, other languages may show a behavior different from French. See, e.g., English \textit{Who have you seen a picture of} or even \textit{?What did you read books about?}}

\subsection{The argumental status of extractable PPs and the case problem}% 4.2 

Unfortunately (25) does not yet bring the desired result, since, as in (24a), the adjunct \textit{de qui} must also be argued to contain phi-features. An even clearer example is the relative pronoun \textit{lequel}, which is inflected for gender and number, but is still ungrammatical when it forms part of an adjunct PP extracted from a DP, while it is grammatical if it is part of an argumental PP. In order to further refine the formalization in (25), some more considerations, in particular concerning the status of the preposition \textit{de}, are in order, along the following lines: crucially for my analysis, in the case of subjects and direct objects, it is the unvalued case feature that makes the goal visible to the probe \citep[123]{Chomsky2000}.\footnote{As an anonymous reviewer points out, late-adjunction approaches would also predict that adjuncts cannot be found by the probe, because when Agree takes place the adjunct is simply not yet present.} This leads me to assume that those expressions in French containing the element \textit{de} that can be extracted from a DP are themselves DPs (containing an unvalued case feature), and not PPs, so the element \textit{de} is not a preposition but a kind of genitive case marker. Note that the idea that the expressions under discussion here are genitives coincides with the conclusions reached by \citet{Gutiérrez-Bravo2001} and \citet{Cinque2014} for independent reasons and in other frameworks (see §3). However, it is a natural outcome of my attempt to apply a minimalist approach of the kind proposed in \citeauthor{Chomsky2000} (\citeyear{Chomsky2000} et seq.).

To treat arguments of N that contain \textit{de} or analogous elements in other Romance languages (such as Italian \textit{di}) as genitives is an old idea formulated as early as \citet{Benveniste1966}, and later implemented in various ways in generative grammar. We have already seen one possibility in §3: the G[enitive]P assumed by \citet{Cinque2014}. This is not, however, appropriate in the framework adopted here, in which case features are valued later by a higher probe. A more neutral label is K[ase]P (cf. \citealt{Bittner1996}; \citealt{Neeleman1999}, among many others), although such a label should in fact be avoided, as it does not have semantic content (cf. \citealt{Chomsky1995}). In my view, it would be preferable to assume that the element \textit{de} is inserted post-syntactically, just like synthetic case morphology (see, e.g., \citealt{Marchis2018}). I cannot discuss this any further here and remain rather theory-neutral with respect to the status of Romance analytical genitives. I provisionally use the term KP, following \citet[23]{Biggs2014}, in that KP is “primarily employed as a placeholder for (late) morpho-phonological insertion.” I will not be concerned with the internal structure of KPs, on which various views exist, so I will label the whole expression as KP.

Let us then assume that arguments of N that represent \AGENT, \THEME or \POSSESSOR are KPs,\footnote{To these we may possibly add “partitive” complements of nouns designating a quantity (such as \textit{moitié} ‘half’, \textit{plupart} ‘majority’, \textit{litre} ‘liter’, \textit{kilo} ‘kilo’ etc.), cf. \citet[236–237]{Godard1992}; \citet{Doetjes1997}.} where KP has valued phi-features plus an unvalued case feature. I furthermore assume that the D-head assigns genitive case under Agree (cf. \citealt{Radford2004}: 368–369; \citealt{Rappaport2006}, among others), which I formalize with a valued case feature on D.\footnote{Since the whole DP itself has a case feature that is valued by a higher functional head, this is a complex issue (cf. \citealt{Weisser2012}). Also see Footnote 22.} For a case without extraction, such as \textit{le livre de Jean}, lit. ‘the book of Jean’, this can be illustrated as follows:

\ea%26
    \label{ex:mensch:26}
\begin{forest}
[DP, s sep+=1cm
    [D\\article\\\fbox{uφ}\\{[}vCase{]}\\\relax{(\textsc{gen})},name=D]
    [FP
        [F\\head noun]
        [NP, s sep+=1cm
            [N\\\sout{head noun},name=headnoun]
            [KP
                [de \ldots\xspace \ldots\\{[}vφ{]}\\{[}uCase{]},roof,name=dots]
            ]
        ]
    ]
]
\draw[-{Triangle[]},dashed] (headnoun.345) -- ++(1.25cm,0) node[midway,above] {Θ};
\begin{scope}[>=Triangle]
\draw[<->,dashed] (D.east) -| ++(1ex,-1ex) |- (dots.west);
\end{scope}
\end{forest}
\z



The unvalued phi-features of D act as a probe and find the valued phi-features of the KP, triggering Agree and valuing the KP’s [uCase] as [Gen], which is spelled out as \textit{de} on the K head. Intervening adjuncts do not have an unvalued case feature and will not be seen by the probe. The D head has no intrinsic [EPP]-feature in French, so the KP is not attracted to [spec,DP] in (26). A determiner with this feature composition is thus not apt for our extraction cases, which is the desired result, since only a subset of KPs, namely those containing an additional valued operator feature, can be extracted (recall my provisional formalization in (25)). Thus, if something needs to be extracted (ultimately because C has an unvalued operator feature), the determiner must be merged with the special feature composition shown in the (still provisional) formalization in (27) (revised from (25)): 

\begin{exe}\ex%27
    \label{ex:mensch:27}
        \interlinepenalty10000 Features of D (provisional formalization II)\\
\begin{forest}
    [{[}uOp{]}\\{[}u$\varphi ${]}\\{[}vCase{]} (Gen), align=center,base=bottom
                [{[}EPP{]}]
            ]
        ]
    ]
\end{forest}
    \end{exe}



Although this feature composition of the D head responsible for extraction makes the right predictions, it is still incomplete. The reason is that the head noun also has phi-features, and possibly even an unvalued case feature, which must be valued by a head outside the DP (e.g. with [Nom] or [Acc]). This problem will be addressed in the next subsection. 

\subsection{The feature composition of French determiners}% 4.3  

Thus far in §4, I have been concerned with determining the features that must be assumed for the D head in its functions as a genitive assigner and as a phase head that can permit, in some special cases, the extraction of DP-internal material to its specifier in order to license further extraction. But, of course, the D head has another, more obvious property, namely agreement with the head noun. Thus, crucially, in an expression like \textit{la maison de Pierre} (the-f.sg. house-f.sg. of Pierre-m.sg.) the determiner (\textit{la}) agrees in gender with the head noun \textit{maison} and not with the KP \textit{de Pierre}. This seems to cast serious doubt on the probe approach that I have just developed. The solution that I will adopt is that French determiners actually have two phi-sets, corresponding to two probes. Thus, French articles have the following basic feature composition:

\ea%28
    \label{ex:mensch:28}
    \begin{tabular}[t]{rl@{}l}
     & [uφ\textsubscript{1}] & \\
     \ldelim({2}{0mm} & [uφ\textsubscript{2}] & \rdelim){2}{0mm}\\
     & [vCase] (Gen) & \\
    \end{tabular}
\z

This is an instance of multiple probes (cf. \citealt{Chomsky2008}), and, more particularly, of feature-stacking (see \citealt{Manetta2011}: Chapter 2 for discussion and literature). In our case, this means that the two probes are ordered, with [u$\varphi $\textsubscript{1}] having to probe first (finding the head noun as its goal) and [u$\varphi $\textsubscript{2}] second (finding the KP). The bracket around the second probe indicates that it is optional (i.e. it must enter the derivation only if a KP with an unvalued case feature is present). The fact that multiple phi-sets can exist on D-heads can be seen in French possessives, as shown in (29):

\ea%29
    \label{ex:mensch:29}
    \gll   son    livre   \hspace{1.5em} ma  veste  \hspace{1.5em}  tes    livres\\
           his/her.\textsc{m}  book.\textsc{m} {} my.\textsc{f}  jacket.\textsc{f} {} your.\textsc{sg-pl}  books.\textsc{pl}\\
\z

The morphemes \{s-\}, \{m-\}, \{t-\} represent the phi-features (3rd sg., 1st sg., 2nd sg.) of the possessor, whereas the morphemes \{-on\}, \{-a\}, \{-es\} reflect those of the head noun. I actually assume that the complex possessive forms in (29) are the spell-out of D heads with a very similar feature composition to that in (28). Let us assume that the expressions in (29) are DPs that contain a phonologically empty KP (something like a covert pronominal). Since articles combine with referring expressions whereas pronominals do not, the difference between articles and possessives can be modeled by a further feature that I provisionally identify as [+/- ref(erential)]:\footnote{I do not address the issue of external case assignment to the DP. As an anonymous reviewer points out, D heads also need [uCase], which is valued by a probe from outside the DP (e.g. the probe contained in v), and then probably passed to the N head (concord).}

\ea%30
    \label{ex:mensch:30}
    \ea articles\footnote{And, possibly, demonstratives.}\\\relax
        
         \begin{tabular}{rl@{}l}
                          &   [D]                                      & \\\\[-1em]
                          &   [u$\varphi $\textsubscript{1}]   & \\\\[-1em]
         \ldelim({3}{0mm} & [u$\varphi $\textsubscript{2}]  & \rdelim){3}{0mm}\\
                          &  [vCase] (Gen)            & \\
                          &        [+ref]             &  \\     
         \end{tabular}
    \ex    possessives\\
    \begin{tabular}{rl}
         &   [D]\\\\[-1em]
         &  [u$\varphi $\textsubscript{1}]\\\\[-1em]
         &  [u$\varphi $\textsubscript{2}]\\\relax           
         &  [vCase] (Gen)\\\relax           
         &   [\textminus ref]
    \end{tabular}
    \z
\z

We can now return to the problem of extraction and proceed to the final revision of the formalization initiated in (22) and further refined in (25) and (27): the D head needed for the extraction cases at issue here is a variant of (30a), enriched with the [uOp] feature and the [EPP]-feature connected to it (cf. (27) above). We can integrate this as a further optional part of (30a) so as to yield (31):

\begin{exe}\ex%31
    \label{ex:mensch:31}
    \interlinepenalty10000 Features of D (articles; final formalization)\\\\
      \begin{tabular}{@{}r@{\hspace{1em}}l@{\hspace{.5em}}l}
                       & [D]\\\\[-1em]
                       & [u$\varphi $\textsubscript{1}]\\\\[-1em]
                       & \begin{tabular}{@{}l@{}}
                                 [u$\varphi $\textsubscript{2}]         \\\relax
                                 [vCase] (GEN)\\\relax
                                 [+ref]       \end{tabular} & \\\\[-1em]
                 \ldelim({-6}{0mm}       & \begin{tabular}[b]{r@{\hspace{1em}}c@{\hspace{.5em}}l@{}} 
                                \ldelim({3}{0mm} & [uOp] & \rdelim){3}{0mm}\\
                                                 &  {\textbar} & \\
                                                 & [EPP] & \\
                         \end{tabular} & \rdelim){-6}{0mm}\\
      \end{tabular}
\end{exe}

This complex entry reads as follows: minimally, a French article (or a demonstrative) has one unvalued phi-set which probes the head noun and determines the morphology of the determiner. Optionally, there can be an additional probe with a (genitive-)case-assigning property. This option is needed when the head noun has a KP complement and can be enriched by an unvalued operator feature connected to an [EPP]-feature if the KP needs to be extracted.

\section{Conclusions}% 5. 

In this article, I have focused on the extraction of PPs from DPs (in \textit{wh}, relative and focus contexts), which is sometimes permitted but at other times leads to ungrammaticality. The main aim of this article has been to determine the conditions that allow/disallow extraction, on which there has been controversy in the literature, and to develop a phase-based account within Chomsky’s (\citeyear{Chomsky2000} et seq.) probe-and-goal framework. I have concentrated on French, although the data are similar in other Romance languages.

On a purely descriptive level, it appears that a subset of PPs headed by the preposition \textit{de}, in which the PP represents an \AGENT, \THEME or \POSSESSOR, is extractable. Other instances of [\textsubscript{PP} \textit{de} ...] (e.g. PPs indicating source/origin or – following \citealt{Kolliakou1999} – PPs that are property-denoting expressions and have a similar function to adjectives), as well as PPs with prepositions other than \textit{de}, cannot be extracted from the DP, thus yielding a kind of “island effect”. Still at a descriptive level, this can be generalized by assuming that only PPs with \textit{de} that represent an \AGENT, \THEME or \POSSESSOR are complements, while all other PPs are adjuncts, and that adjuncts are not extractable from DPs. A major issue that is discussed in the literature is the presence, in a single structure, of two or more PPs that fit the relevant extraction criteria. I have followed \citegen{Kolliakou1999} argument that, in such cases, only one of them can be an argument, whereas the others are adjuncts (property-denoting expressions).

The relevant subset of PPs introduced by \textit{de} has sometimes been informally classified as “genitives” in the literature, but in the 1980s and 1990s, the discussion mostly turned on theta-roles, trying to predict the extraction of PPs from DPs in terms of a thematic hierarchy. By contrast, two minimalist accounts that I have summarized in §3 (\citealt{Gutiérrez-Bravo2001}; \citealt{Cinque2014}) assume that extractable PPs headed by \textit{de} or equivalent prepositions in other Romance languages represent genitive case in the technical sense. However, these accounts assume agreement phrases, which are not compatible with more recent minimalist literature. Both accounts are nevertheless formulated within phase theory, an idea that logically continues the view formulated by \citet{Cinque1980}, according to which extraction from DP (NP in \citealt{Cinque1980}) to a higher category (usually CP) necessarily passes through [spec, DP] (formerly [spec,NP]). 

In §4, I applied Chomsky’s probe-goal approach in order to explain extraction of a PP to the DP phase edge. Essentially, what we need to assume is that D° can have an unvalued operator feature ([uOp]) bound to an [EPP]-feature. The operator feature alone, however, does not qualify as a probe capable of explaining the data, since it would match any constituent containing [vOp], crucially including adjuncts. There must therefore be an additional feature on the goal that guarantees the visibility of arguments (but not of adjuncts) to the probe. Following the logic of the probe-goal approach, this should be an unvalued case feature, a solution that lends further support to the genitive hypothesis. Since case belongs to the agreement system, the data must be explained by a complex probe on the D head, which contains unvalued phi-features in addition to [uOp]. I then argued that Romance articles can optionally have this complex probe, in addition to their “regular” unvalued phi-set (which regulates agreement between the article and the head noun). Thus, if something needs to be extracted from a DP, the D head has two phi-sets. Note that the extracted constituent must be classified as a DP or a KP rather than a PP.

French possessives, another type of determiner, can also be argued to have two phi-sets (in this case both are morphologically visible). This is encoded in the lexical entries of possessives, which do not have [uOp] and lack an [EPP]-feature. Interestingly, the fact that possessives and articles compete for the D° position then explains the incompatibility of extraction with the presence of possessives, which has been observed in the literature (cf. \citealt{Milner1978,Milner1982}; quoted in \citealt{Sag1994}):

\ea%32
    \label{ex:mensch:32}
    \ea[]{
    \gll la  neuvième,  dont    j’ ai  beaucoup  aimé  l’  interprétation    de  Karajan\\
         the  ninth    of.which  I have  much    loved  the  interpretation    by  K.\\
    \glt ‘the Ninth (symphony), of which I have much loved the interpretation by Karajan’}
    \ex[*]{
    \gll la  neuvième,  dont    j’ ai  beaucoup  aimé  son  interprétation\\
         the  ninth    of.which  I have  much    loved  his  interpretation\\
    \glt ‘the Ninth (symphony), of which I have much loved his interpretation’  }
    \z
\z

The theory sketched here also explains complex DP islands. In an example such as (1a), the special complex probe in the D head needs to identify a constituent with an unvalued case feature as its goal in order to attract it to the DP phase edge. However, the case feature of the putative goal, \textit{qui} in the relative clause, is already valued as accusative in the relative clause.

A final note concerns the assumption that constituents introduced by prepositions other than \textit{de} are adjuncts, which is assumed in most of the literature and which I have adopted as a working hypothesis during the whole article. My final approach makes this assumption unnecessary, since such constituents would be PPs and not KPs.

On a more general theoretical level, the study of the phenomena at issue here shows that not all intermediate movement steps can be explained by \citegen{Chomsky2008} edge features (EFs). EFs are designed to optionally apply whenever needed, but they are not capable of selecting specific goals. Whereas we can assume an EF on v for ensuring cyclic movement from the DP edge to the CP, we cannot do so for the movement to the DP edge itself.

\section*{Acknowledgements}

My research on extraction from DPs in French is funded by the Deutsche Forschungsgemeinschaft (DFG) in a project with Stefan Müller (Hum\-boldt-Uni\-ver\-si\-tät zu Berlin) and was also part of my activities within the Courant Research Centre “Text Structures” at the University of Göttingen.
% 
% \section{ References}
% 
% Alexiadou, Artemis \& Haegeman, Liliane \& Stavrou, Melita. 2007. \textit{Noun phrase in the generative perspective}. Berlin: Mouton de Gruyter.
% 
% Andrews, Avery. 1983. A note on the constituent structure of modifiers. \textit{Linguistic Inquiry} 14. 695–697.
% 
% Benveniste, Émile. 1966. Pour l’analyse des fonctions casuelles: Le génitif latin. In Benveniste, Émile (ed.), \textit{Problèmes de linguistique générale}, 140–148. Paris: Gallimard.
% 
% Biggs, Alison. 2014. \textit{Dissociating case from theta-roles: A comparative investigation.} Cambridge: King’s College. (University of Cambridge dissertation.)
% 
% Bittner, Maria \& Hale, Ken. 1996. The structural determination of case and agreement. \textit{Linguistic Inquiry} 27. 1–68.
% 
% Boeckx, Cedric. 2008. \textit{Aspects of the syntax of agreement}. London: Routledge.
% 
% Boeckx, Cedric. 2009. \textit{Understanding minimalist syntax: Lessons from locality in long-distance dependencies}. Oxford: Blackwell.
% 
% Boeckx, Cedric. 2011. The character of island conditions: Thoughts inspired by contemporary linguistic theory. (Paper presented at the workshop \textit{Islands in contemporary linguistic theory}, Barcelona, Universidad Autónoma, Nov. 16th–18th 2011.)\\
% https://ehutb.ehu.es/uploads/material/Video/1764/Handout-Boeckx.pdf (last consultation January 16, 2017).
% 
% Boeckx, Cedric \& Grohmann, Kleanthes K. (2007): Putting phases in perspective. \textit{Syntax} 10. 204–222.
% 
% Broekhuis, Hans. 2007. Extraction from subjects: Some remarks on Chomsky’s “On phases”. In Broekhuis, Hans et al. (eds.), \textit{Organizing grammar: Linguistic studies in honor of Henk van Riemsdijk}, 59–68. Berlin/New York: Mouton de Gruyter.
% 
% Ceplova, Marketa. 2001. \textit{Minimalist islands}. Ms., MIT.
% 
% Chierchia, Gennaro. 1982. Nominalization and Montague grammar. \textit{Linguistics and Philosophy} 5. 303–354.
% 
% Chierchia, Gennaro. 1985. Formal semantics and the grammar of predication. \textit{Linguistic Inquiry} 16. 417–443.
% 
% Chomsky, Noam. 1986. \textit{Barriers}. Cambridge, MA: MIT Press.
% 
% Chomsky, Noam. 2000. Minimalist inquiries: The framework. In Martin, Roger et al. (eds.), \textit{Step by step: Essays on minimalist syntax in honor of Howard Lasnik}, 89–155. Cambridge, MA: MIT Press.
% 
% Chomsky, Noam. 2001. Derivation by phase. In: Kenstowicz, Michael (ed.), \textit{Ken Hale: A life in language}, 1–52. Cambridge, MA: MIT Press.
% 
% Chomsky, Noam. 2004. Beyond explanatory adequacy. In Belletti, Adriana (ed.), \textit{Structures and beyond: The cartography of syntactic structures}, vol. 3, 104–131. Oxford/New York: Oxford University Press.
% 
% Chomsky, Noam. 2007. Approaching UG from below. In: Sauerland, Uli \& Gärtner, Hans-Martin (eds.), \textit{Interfaces + recursion = language? Chomsky’s minimalism and the view from syntax-semantics}, 1–29. Berlin/New York: Mouton de Gruyter.
% 
% Chomsky, Noam. 2008. On phases. In: Freidin, Robert \& Otero, Carlos P. \& Zubizarreta, Maria Luisa (eds.), \textit{Foundational issues in linguistic theory: Essays in honor of Jean-Roger Vergnaud}, 133–166. Cambridge, MA: MIT Press.
% 
% Cinque, Guglielmo. 1980. On extraction from NP in Italian. \textit{Journal of Italian Linguistics} 5. 47–99.
% 
% Cinque, Guglielmo. 1990. \textit{Types of A’-dependencies.} Cambridge, MA: MIT Press.
% 
% Cinque, Guglielmo. 2014. Extraction from DP in Italian revisited. In Aboh, Enoch O. et al. (eds.), \textit{Locality}, 86–103. New York: Oxford University Press.
% 
% Doetjes, Jenny. 1997. \textit{Quantifiers and selection: On the distribution of quantifying expressions in French, Dutch and English}. Universiteit Leiden: HIL.
% 
% Gavruseva, Elena. 2000. On the syntax of possessor extraction. \textit{Lingua} 110. 743–772.
% 
% Giorgi, Alessandra \& Longobardi, \citealt{Giuseppe1991}. \textit{The syntax of noun phrases: Configurations, parameters, and empty categories}. New York: Cambridge University Press.
% 
% Godard, Danièle. 1992. Extraction out of NP in French. \textit{Natural Language and Linguistic Theory} 10. 233–277.
% 
% Grosu, Alexander. 1974. On the nature of the left branch condition. \textit{Linguistic Inquiry} 5. 308–319.
% 
% Gutiérrez-Bravo, Rodrigo. 2001. Phases, case and accessibility: The case of extraction from DP in Spanish. In McCloskey, James (ed.), \textit{Syntax and semantics at Santa Cruz} 3, 104–118. Santa Cruz, CA: Linguistics Research Center, University of California.
% 
% Heck, Fabian. 2008. \textit{On pied-piping –} wh\textit{{}-movement and beyond}. Berlin/New York: Mouton de Gruyter.
% 
% Heck, Fabian. 2009. On certain properties of pied-piping. \textit{Linguistic Inquiry} 40. 75–111.
% 
% Huang, Cheng-Teh James. 1982. \textit{Logical relations in Chinese and the theory of grammar.} Cambridge, MA (MIT dissertation).
% 
% Kayne, Richard S. 1999. Prepositional complementizers as attractors. \textit{Probus} 11. 39–73.
% 
% Kayne, Richard S. 1994. \textit{The antisymmetry of syntax.} Cambridge, MA: MIT Press.
% 
% Kayne, Richard S. 2004. Prepositions as probes. In Belletti, Adriana (ed.), \textit{Structures and beyond: The cartography of syntactic structures}, vol. 3, 192–212. Oxford: Oxford University Press.
% 
% Kolliakou, Dimitra. 1999. De-phrase extractability and individual/property denotation. \textit{Natural Language and} \textit{Linguistic Theory} 17. 713–781.
% 
% Manetta, Emily. 2011. \textit{Peripheries in Kashmiri and Hindi-Urdu: The syntax of discourse-driven movement}. Amsterdam: John Benjamins.
% 
% Milner, Jean-Claude. 1978. \textit{De la syntaxe à l’interprétation: Quantités, insultes, exclamations.} Paris: Éd. Du Seuil.
% 
% Milner, Jean-Claude. 1982. \textit{Ordre et raisons de langue.} Paris: Éd. Du Seuil.
% 
% Moritz, Luc \& Valois, Daniel. 1994. Pied-piping and specifier-head agreement. \textit{Linguistic Inquiry} 25. 667–707.
% 
% Müller, Gereon. 2010a. On deriving CED effects from the PIC. \textit{Linguistic Inquiry} 41. 35–82.
% 
% Müller, Gereon. 2010b. \textit{Operator islands, maraudage, and the intermediate step corollary}. (Ms., University of Leipzig; http://home.uni-leipzig.de/muellerg/mu241.pdf (19 \citealt{May2016})).
% 
% Neeleman, Ad \& Weerman, Fred. 1999. \textit{Flexible syntax: A theory of case and arguments}. Dordrecht: Kluwer.
% 
% Obenauer, Hans-Georg. 1985a. Connectedness, variables, and stylistic inversion in French. In Guéron, Jacqueline \& Obenauer, Hans-Georg \& Pollock, Jean-Yves (eds.), \textit{Grammatical representation}, 237–257. Dordrecht: Foris.
% 
% Obenauer, Hans-Georg. 1985b. On the identification of empty categories. \textit{The Linguistic Review} 4. 153–202.
% 
% Obenauer, Hans-Georg. 1994. \textit{Aspects de la syntax}e A-barre. Université de Paris VIII. (Thèse de doctorat d’État.)
% 
% Pesetsky, David \& Torrego, Esther. 2007. The syntax of valuation and the interpretability of features. In Karimi, Simin \& Samiian, Vida \& Wilkins, Wendy K., \textit{Phrasal and clausal architecture: Syntactic derivation and interpretation:} \textit{In honor of Joseph E. Emonds}. 262–294. Amsterdam: John Benjamins.
% 
% Pollock, Jean-Yves. 1989. Opacity, genitive subjects and extraction from NP in English and French. \textit{Probus} 1. 151–162.
% 
% Radford, Andrew. 2004. \textit{Minimalist syntax: Exploring the structure of English}. Cambridge: Cambridge University Press.
% 
% Rappaport, Gilbert C. 2006. The syntax of possessors in the nominal phrase: Drawing the lines and deriving the forms. In Kim, Ji-yung \& Lander, Yuri A. \& Partee, Barbara H. (eds.), \textit{Possessives and beyond: semantics and syntax} (=University of Massachusetts Occasional Papers 2), 243–261. Amherst, MA: GLSA Publications.
% 
% Rizzi, Luigi. 1990. \textit{Relativized minimality}. Cambridge, MA: MIT Press.
% 
% Rizzi, Luigi. 2001. Relativized minimality effects. In Baltin, Marc \& Collins, Chris (eds.), \textit{The handbook of contemporary syntactic theory}, 89–110. Oxford: Oxford University Press.
% 
% Rizzi, Luigi. 2004. Locality and left periphery. In Belletti, Adriana (ed.), \textit{Structures and beyond. The cartography of syntactic structures}, vol. 3, 223–251. New York: Oxford University Press.
% 
% Rizzi, Luigi. 2007. On some properties of criterial freezing. \textit{Studies in linguistics} (CISCL Working Papers) 1. 145–158. http://www.ciscl.unisi.it/doc/doc\_pub/STiL-2007-vol1.pdf (19 May, 2016).
% 
% Rizzi, Luigi \& Shlonsky, \citealt{Ur2007}. Strategies of subject extraction. In Sauerland, Uli \& Gärtner, Hans-Martin (eds.), \textit{Interfaces + recursion = language? Chomsky’s minimalism and the view from syntax-semantics}, 115–160. Berlin: Mouton de Gruyter.
% 
% Ross, John Robert. 1967. \textit{Constraints on variables in syntax}. Cambridge, MA. (MIT dissertation.)
% 
% Rouveret, Alain. 1980. Sur la notion de proposition finie: Gouvernement et inversion. \textit{Langages} 60. 75–107.
% 
% Ruwet, Nicolas. 1972. \textit{Théorie syntaxique et syntaxe du français.} Paris: Éd. Du Seuil.
% 
% Sag, Ivan A. \& Godard, Danièle. 1994. Extraction of \textit{de}{}-phrases from the French DP. \textit{Proceedings of the North Eastern Linguistics Society} (NELS) 24, vol. 2. 519–541.
% 
% Sportiche, Dominique. 1981. Bounding nodes in French. \textit{The Linguistic Review} 1. 219–246.
% 
% Stepanov, Arthur. 2007. The end of CED? Minimalism and extraction domains. \href{https://www.researchgate.net/journal/1467-9612_Syntax}{\textit{Syntax}} 10. 80–126.
% 
% Stowell, Tim. 1989. Subjects, specifiers, and X-bar theory. In Baltin, Marc R. \& Kroch, Anthony S. (eds.), \textit{Alternative conceptions of phrase structure}, 232–262. Chicago, Illinois: Chicago University Press.
% 
% Svenonius, Peter. 2004. On the edge. In Adger, David \& De Cat, Cécile \& Tsoulas, George (eds.), \textit{Peripheries: Syntactic edges and their effects}, 261–287. Dordrecht: Kluwer.
% 
% Szabolcsi, Anna. 1983/1984. The possessor that ran away from home. \textit{Linguistic Review} 3. 89–102.
% 
% Szabolcsi, Anna. 2006. Strong and weak islands. In Everaert, Martin \& Van Riemsdijk, Henk (eds.), \textit{The Blackwell companion to syntax}, vol. 4, 479–532. London: Blackwell.
% 
% Tellier, Christine. 1990. Subjacency and subject condition violations in French. \textit{Linguistic Inquiry} 21. 306–311.
% 
% Tellier, Christine. 1991. \textit{Licensing theory and French parasitic gaps}. (Studies in natural language and linguistic theory.) Dordrecht: Kluwer.
% 
% Tellier, Christine. 2001. On some distinctive properties of parasitic gaps in French. In Culicover, Peter W. \& Postal, Paul M. (eds.), \textit{Parasitic gaps}, 341–367. Cambridge, MA: MIT Press.
% 
% Truswell, Robert. 2005. Phase heads, multiple spell-out, and a typology of islands. \textit{Papers in Linguistics from the University of Manchester (PLUM)}. 125–141.
% 
% Valois, Daniel. 1991. \textit{The internal syntax of DP}. Los Angeles, CA. (University of California dissertation.)
% 
% Weisser, Philipp \& Klein, Timo \& Georgi, Doreen \& Assmann, Anke. 2012. \textit{Possessor case in Udmurt.} (Paper presented at the \textit{Syntax-Kolloquium}, University of Leipzig, Institut für Linguistik, Nov. 16th 2012.) http://www.uni-leipzig.de/{\textasciitilde}asw/lab/lab90/LAB90\_02\_udmurt-ankeetal.pdf (19 May, 2016).
% 
% Zeijlstra, Hedde. 2010. \textit{There is only one way to agree}. (Paper presented at the 33rd GLOW Colloquium, Wrocław, Poland.)
% 
% 
% \begin{verbatim}%%move bib entries to  localbibliography.bib
% \end{verbatim} 

\sloppy
\printbibliography[heading=subbibliography,notkeyword=this] 
\end{document}
