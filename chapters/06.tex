\documentclass[output=paper]{langsci/langscibook} 
\author{Paolo Lorusso\affiliation{Università di Firenze\slash CRIL Università del Salento}}
\title{A person split analysis of the progressive forms in Barese}

% \chapterDOI{} %will be filled in at production

% % \epigram{Change epigram in chapters/03.tex or remove it there }
\abstract{This paper explores the distribution of progressive aspect in some varieties of the Barese (dialect of Apulia). In many of these varieties the progressive is expressed through an aspectual inflected construction (in the terms of \citealt{Manzini2005}): it is formed from an inflected stative verb stɛ (=‘to stay’), a connecting element a (=‘to’) and the present indicative of the lexical verb, which agrees in person and number with the matrix verb. The multiple agreement configuration, as in pseudo{}-coordinations (\citealt{Jaeggli1993}) is not interpreted as a coordination of two events occurring at the same time, but as a single complex event, with V1 having scope over V2, an interpretation that is usually realized with a non-finite form of V2 to represent an aspectual semantic value. In the same variety, however, we can find a parallel construction to express the progressive in which the embedded lexical verb is not inflected. The 1\textsuperscript{st} and 2\textsuperscript{nd} persons plural are not found in the aspectual inflected constructions, but allow only the infinitival counterpart. Differences in the pattern of the morphological derivation of the 1\textsuperscript{st} and 2\textsuperscript{nd} persons plural are quite common across Romance languages \citep{Manzini2005,Manzini2011Bio}: I will argue that they in fact involve a more complex referentiality than other persons (as in \citealt{Bobaljik2008Missing}).}
\maketitle

\begin{document}
%%please move the includegraphics inside the {figure} environment
%%\includegraphics[width=\textwidth]{OGSVolumeAug2018Lorusso-img1.tif}

 
%%please move the includegraphics inside the {figure} environment
%%\includegraphics[width=\textwidth]{OGSVolumeAug2018Lorusso-img2.tif}

 
%%please move the includegraphics inside the {figure} environment
%%\includegraphics[width=\textwidth]{OGSVolumeAug2018Lorusso-img3.tif}

 
%%please move the includegraphics inside the {figure} environment
%%\includegraphics[width=\textwidth]{OGSVolumeAug2018Lorusso-img4.tif}

\todo[inline]{Please check this paper for any missing revisions from the original document}
\section{The progressive inflected and non-inflected constructions in Barese}% 1. 

In some varieties of Barese, the progressive is expressed through an aspectual inflected construction (in the terms of \citealt{Manzini2005}): an inflected stative verb \textit{stɛ} (=‘to stay’), a connecting element \textit{a} (=‘to’) and the present indicative of the lexical verb, which agrees in person and number with the matrix verb. This progressive construction contains a multiple agreement configuration involving an inflected auxiliary and a finite complement introduced by \textit{a}, as in finite control constructions of the Balkan type.\footnote{The phenomenon of finite control in the Balkan languages (and in Hebrew and many Southern Italian varieties) involves the appearance of inflected subjunctive complements which exhibit Obligatory Control (\citealt{Landau2004}, among others): finite complements in these languages cover pretty much the entire spectrum of obligatory control or raising predicates (for an overview, see \citealt{Ledgeway2015,Manzini2017}). The verbs embedding \textit{a} complements, such as the ones we are describing, are a much more restricted set than the obligatory control\slash raising verbs in Balkan languages. The Apulian varieties under analysis, for example, include ‘go’ and ‘be\slash stay’ aspectual periphrases; we will concentrate on the ‘stay’ periphrases (for an analysis of the differences in aspectual finite constructions across southern Italian varieties, see \citealt{Manzini2017}).}  The example in (1) shows the progressive aspectual construction in the variety spoken in Conversano (Apulia): 

\ea%1
    \label{ex:lorusso:1}
    \gll Stek    a   fatsə    u    pɛn.     \\
         stay.\textsc{1sg}   to  do.\textsc{1sg} the    bread\\
    \glt ‘I am making the bread.’
    \z


In the same variety, we can find a parallel construction to express the progressive in which the embedded lexical verb is not inflected. In (2), the embedded verb \textit{fɛ} ('to do') is infinitival: 

\ea%2
    \label{ex:lorusso:2}
    \gll Stek     a  fɛ     u    pɛn.      \\
         stay.\textsc{1sg} to   do.\textsc{inf}    the    bread\\
    \glt ‘I am making the bread.’
    \z

In Conversanese, the aspectual inflected construction is not found with the 1\textsuperscript{st} and 2\textsuperscript{nd} persons plural, as shown in (3): only the construction with an embedded infinitival lexical verb is available to express the progressive, as in (4):\footnote{Similar patterns are found in the varieties from the same area (i.e. in the south-east of Bari, in A: Mola di Bari, Rutigliano, Castellana, Turi). Throughout the paper we will refer mainly to the variety of Conversano, but we will also sketch some relevant differences between the variety of Conversano and some other varieties of the same group in §1.}

\ea%3
    \label{ex:lorusso:3}
    \gll Nojə\textsubscript{j}/voʊ\textsubscript{k}   stɛmə\textsubscript{j}/stɛtə\textsubscript{k}   a   *man’dʒɛmə\textsubscript{j}/*man’dʒɛtə\textsubscript{k}.\\
         we/you  stay.\textsc{1pl}/\textsc{2pl} to   eat.\textsc{1pl}/\textsc{2pl} \\
    \z


\ea%4
    \label{ex:lorusso:4}
    \gll Nojə/voʊ   ʹstɛmə/ʹstɛtə   a  manʹdʒɛ.\\
         we/you   stay.\textsc{1pl}/\textsc{2pl} to   eat.\textsc{inf}\\
    \glt ‘We/you are eating.’
    \z


Both types of structure share a locative derivation: the majority of progressive forms crosslinguistically, in fact, are derived from expressions involving locative elements  \citep{BybeePerkinsPagliuca1994,Mateu1999,Laka2006}\todo{BybeeEtAl missing}. The two parallel constructions differ in the aspect of the denoted event. On the one hand, the construction with the embedded inflected verb, (1), denotes an event identification between the auxiliary and the lexical verb and seems to work like a restructuring or serial verb construction. On the other hand, the construction with the embedded infinitive, (2), involves a frequentative reading which is not found with genuine progressive constructions \citep{Chierchia1995} and seems to pattern with aspectual control verbs. The 1\textsuperscript{st} and 2\textsuperscript{nd} persons plural are not found in the aspectual inflected construction in (3), but allow only the infinitival counterpart (4). Differences in the pattern of the morphological derivation of the 1\textsuperscript{st} and 2\textsuperscript{nd} persons plural is quite common across Romance languages \citep{Manzini2005,Manzini2011Bio}: they involve, in fact, a more complex referentiality than other persons \citep{Bobaljik2008Missing}; they are not mere plurals of the discourse participants, but may refer to other referents not directly involved in the discourse (event participants). In a lexical parametrization analysis \citep{Manzini2011Bio}, languages involve a parametric distinction for plural and singular: plural persons do not show a pattern of parametric distinction between discourse (1\textsuperscript{st} and 2\textsuperscript{nd}) and event participants (3\textsuperscript{rd}) found with singular persons. 

In §2, the distribution of the pattern of inflection across the different varieties is described: the insights of previous accounts are also listed. §3 introduces the analysis of the progressive as a locative\slash unaccusative construction (in the terms of \citealt{Mateu1999}). §4 presents a syntactic analysis of the phenomenon. In §5, the aspectual differences between the two progressive patterns are described. §6 is devoted to some reflections on the person split pattern found in the progressive constructions in Conversanese. §7 resumes the insight and the main concerns of the present analysis. 

\section{The distribution of aspectual inflected constructions}% 2. 
\subsection{Introduction}% 2.1. 

Various studies have focused on periphrastic verbal constructions in some Southern Italian varieties that involve two inflected verbs.\footnote{As suggested by an anonymous reviewer, this construction apparently shares the derivation of hyper-raising constructions \citep{Harford1985,Martins2005,Nunes2008,Zeller2006}, but there are few elements that allow us to take them as non- hyper raising constructions. In this paper we are dealing mainly with the auxiliary \textit{stare} (=‘stay’) in the progressive constructions, which is not a raising predicate. Furthermore, in many Southern Italian varieties these constructions are also found with motion verbs (\textit{go}, \textit{come}) or modal auxiliaries (\textit{want}) (\citealt{Manzini2005,Di2015,Manzini2017}; Cardinaletti \& Giusti to appear\todo[inline]{bib refs missing: Cardinaletti \& Giusti, Manzini 2000 below}), but no genuine raising predicate is involved. The subject is base-generated (and case-assigned) under the T of the matrix verb. One more contrast with hyper-raising constructions is that no expletive counterpart of the sentences is available in the languages under analysis (or any version with embedded subjects). These constructions share more similarities with finite control constructions found in Balkan languages \citep{Landau2004,Landau2013,Manzini2000} and in Southern Italian varieties \citep{Manzini2005,Ledgeway2015}.}  The main characteristic of these constructions is that a matrix aspectual auxiliary inflected for number and person selects a lexical verb that is also inflected. The lexical embedded verb may or may not be introduced by a preposition. The auxiliary loses its lexical meaning and the complex VP is interpreted as a single predicate, the embedded lexical verb being the one that gives the referential meaning to the event denoted by the complex VP. For example, in (5) the subject \textit{Ma'ri} is not ‘staying’ and then eating, but just eating:

\ea%5
         Conversano, Apulia\label{ex:lorusso:5}\\
    \gll Ma'ri  ste     a   manʤɜ.\\
         Maria  stay.\textsc{3sg}  to  eat.\textsc{3sg}\\
    \glt ‘Maria is eating.’
    \z          

Similar patterns are found in different Southern Italian varieties. \citet{Ledgeway1997} refers to imperative structures in Neapolitan that involve two inflected verbs as asyndetic constructions. A fully inflected verb is embedded under another fully inflected matrix verb, as in (6). There is no preposition introducing the embedded element. In his terms, these constructions define a family of coordinative constructions grammaticalized into subordination. These imperative constructions are paratactic in the sense that “..they contain as many assertions as there are clauses […]” \citep[231]{Ledgeway1997}; in (6), in fact, there are two assertions (7), whereas the progressive construction in Conversanese contains only one assertion ranging over the entire construction.

\ea%6
    \label{ex:lorusso:6}
    \gll Va     spanne     'epanne  nfuse.\\
         go.\textsc{imp.2sg} hang.out.\textsc{imp.2sg}  the.clothes  wet\\
    \glt ‘Go and hang out the washing.’ \citep[230]{Ledgeway1997}
    \z

\ea%7
    \label{ex:lorusso:7}
    \ea   Va!\\
    \glt  ‘Go!’
    \ex   Spanne 'epanne nfuse!\\
    \glt  ‘Hang out the washing!’ \citep[231]{Ledgeway1997}
    \z
\z

Most Sicilian dialects make use of a construction with a functional verb (usually of motion), followed by the linking element \textit{a} and an inflected inflected verb. \citet{Cardinaletti2001,Cardinaletti2003}\todo{bib ref missing: Cardinaletti 2001} label these structures Inflected Constructions.\footnote{In the terms of \citet{Cruschina2013}, these are Double Inflected Constructions.} They are “..similar to what is generally known as ‘Serial Verb Construction’ in other language families (cf. \citealt{Aikhenvald2006}), in which the two verbs (V1 and V2) share the same inflection for Tense and person […]” \citep[392]{Di2015}. The examples in (8) from the dialect spoken in Delia (Caltanissetta) are considered by \citet{Di2015} as monoclausal constructions with a functional verb, in opposition to their infinitival counterparts (9), which are the only option in standard Italian (10) and are biclausal constructions:

\ea%8
    \label{ex:lorusso:8}
    \gll La  sira    mi    veni    a  ccunta    du   cosi.\\
         the  evening  to.me\textsc{(cl)}  come.\textsc{3sg}  to tell.\textsc{3sg} two  things\\
    \glt ‘He comes to tell me some stories at night.’
    \z


\ea%9
    \label{ex:lorusso:9}
    \gll La  sira    mi    veni     a  ccuntari  du   cosi. \\
         the  evening  to.me\textsc{(cl)}  come.\textsc{3sg}  to  tell.\textsc{inf} two  things\\
    \z



\ea%10
    \label{ex:lorusso:10}
    \gll La  sera    mi    viene     a   raccontare / *racconta  delle  storie.\\
         the  evening  to.me\textsc{(cl)} come.\textsc{3sg}  to  tell.\textsc{inf} / tell.\textsc{3sg}  some  stories\\
    \glt ‘He comes to tell me some stories at night.’ \citep{Di2015}
    \z


In the present analysis, both the inflected and the infinitival constructions will be analyzed, following the intuition of \citet[1:688]{Manzini2005}, as biclausal structures: while the inflected construction involves event identification (§3 and §4), the infinitival counterparts do not. The differences in the aspectual reading (see §4) of the two types of progressive construction in Conversanese will confirm this analysis.

\citet[I:688--689]{Manzini2005} propose an event identification analysis for all the aspectual constructions with finite verbs found in Apulian, Calabrian and Sicilian varieties. These aspectual constructions are found with different matrix verbs: progressive (\textit{stay}) in (11), motion verbs (\textit{go}, \textit{come}) in (12), and modals (\textit{want}, \textit{will}, \textit{must}) in (13).

\ea%11
         Taranto, Apulia\label{ex:lorusso:11}\\
    \gll Stɔk    a  bbeivə.     \\
         stay.\textsc{1sg}  to  drink.\textsc{1sg}    \\
    \glt ‘I am drinking.’
\z


\ea%12
    \label{ex:lorusso:12}
    \ea  Minervino Murge, Apulia\\
    \gll Væ    u    cæmə.\\
         go.\textsc{2sg}  him(\textsc{cl})   call.\textsc{2sg}    \\
    \glt ‘You go to call him.’
    \ex  Modica, Sicily\\
    \gll Vaju    a  mmaɲtʃu.       \\
         go.\textsc{1sg} to  eat.\textsc{1sg}\\
    \glt ‘I go to eat.’     
    \ex  Umbriatico, Calabria\\
    \gll U    vəju    cəmu.\\
         him  go.\textsc{1sg}  call.\textsc{1sg}\\
    \glt ‘I go to call him.’
    \z
\z

\ea%13
    \label{ex:lorusso:13}
    \ea  Brindisi, Apulia\\
    \gll Ti      vɔʄʄu    a  vveʃu.    \\
         you\textsc{.acc(cl)} want.\textsc{1sg}  to  see.\textsc{1sg}\\
    \glt ‘I want to see you.’
    \ex  Mesagne, Apulia\\
    \gll Vɔʄʄu    mmaɲʤu.          \\
         want.\textsc{1sg}   eat.\textsc{1sg}\\
    \glt ‘I want to eat.’
\z
\z

In the present work we will be dealing mainly with the progressive constructions involving the auxiliary \textit{stay}, but the assumptions of the present analysis can also be applied to the other aspectual constructions with inflected verbs.

\subsection{The progressive aspectual consructions with finite verbs in the Apulian varieties}% 2.2. 

In the Southern Apulian variety of Conversano, the present continuous progressive is expressed through an aspectual inflected construction involving the inflected stative verb \textit{stɛ} (=‘to stay’), a connecting element \textit{a} (=‘to’) and the present indicative of the lexical verb, which agrees in person and number with the matrix verb. In \tabref{tab:lorusso:1}, the paradigm of inflection for the present indicative is presented. The same pattern of inflection is not found in past tenses or the imperative. The inflection is also not found on embedded verbs in the case of the 1\textsuperscript{st} and 2\textsuperscript{nd} persons plural.\footnote{Other varieties have the very same paradigm with respect to the lack of an aspectual infinitive construction for the 1st and 2nd persons plural and with past tenses and imperatives: the varieties of Castellana, Turi, Rutigliano, Mola and Poligano. These towns are also in the south-eastern part of Bari.} 

%%please move \begin{table} just above \begin{tabular
\begin{table}
\begin{tabular}{*{4}{l}}
\lsptoprule
Indicative present & Auxiliary \textit{stay} & Prep. & Lexical Verb\\\midrule
\scshape 1sg & stek & a & manʤə\\
\scshape 2sg & ste & a & manʤə\\
\scshape 3sg & ste & a & manʤə\\
\scshape 1pl & stɛm & a & *manʤɛmə\\
\scshape 2pl & stɛt & a & *manʤɛtə\\
\scshape 3pl & stan & a & ˈmanʤənə\\
\lspbottomrule
\end{tabular}
\caption{Progressive for the verb \textit{ma’nʤɛ} (= to eat) in the variety of Conversano\label{tab:lorusso:1}}
\end{table}


In the same area, there are varieties, such as those of Putignano (\tabref{tab:lorusso:2}) and Martina Franca (\tabref{tab:lorusso:3}) \citep[I:689–690]{Manzini2005}, where specialized forms are found in the inflection for the auxiliary \textit{stay} (\textsc{2sg, 3sg, 1pl, 2pl}), which differs from the inflected forms of the lexical verb \textit{stay}. With 1\textsuperscript{st} \textsc{sg} and 3\textsuperscript{rd} \textsc{pl} the inflected forms of the auxiliary coincide with those of the lexical counterpart \textit{stay}. 

\begin{table}
\begin{tabular}{*{4}{l}}
\lsptoprule
Indicative present & Auxiliary \textit{stay} & Prep. & Lexical Verb\\\midrule
\scshape 1sg & \textbf{stok} & a & ffatsə\\
\scshape 2sg & ste & ${\emptyset}$ & ffaʃə\\
\scshape 3sg & ste & ${\emptyset}$ & ffaʃə\\
\scshape 1pl & sta & ${\emptyset}$ & ffaʃeimə\\
\scshape 2pl & sta & ${\emptyset}$ & ffaʃeitə\\
\scshape 3pl & \textbf{ston} & a & ‘ffaʃənə\\
\lspbottomrule
\end{tabular}
\caption{Progressive for the verb \textit{ffɔ} (= to make) in the variety of Putignano\label{tab:lorusso:2}}
\end{table}

\begin{table}
\begin{tabular}{*{4}{l}}
\lsptoprule
Indicative present & Auxiliary \textit{stay} & Prep. & Lexical Verb\\\midrule
\scshape 1sg & \textbf{stɔ} & ${\emptyset}$ & ccɛmə\\
\scshape 2sg & stɛ & ${\emptyset}$ & ccɛmə\\
\scshape 3sg & stɛ & ${\emptyset}$ & ccɛmə\\
\scshape 1pl & stɛ & ${\emptyset}$ & ccamɛ:mə\\
\scshape 2pl & stɛ & ${\emptyset}$ & ccamɛ:tə\\
\scshape 3pl & \textbf{stɔ}\textbf{nə} & a & ‘ccɛmənə\\
\lspbottomrule
\end{tabular}
\caption{Progressive for the verb \textit{ccɛ}\textit{’mɛ} (= to call) in the variety of Martina Franca\label{tab:lorusso:3}}
\end{table}

In both the variety of Putignano and that of Martina Franca (Tables 2 and 3), when the forms of the auxiliary coincide with the forms of the lexical \textit{stay}, the embedded predicate is introduced by the preposition \textit{a} (see 1\textsuperscript{st} \textsc{sg} and 3\textsuperscript{rd} plural for Putignano and 3\textsuperscript{rd} plural for Martina Franca). Along this line of analysis, there is the variety of Mesagne where the auxiliary ‘stay’ shares only its root with the lexical \textit{stay}: a specialized inflection is found in the progressive construction which is different from the lexical use of the verb (\tabref{tab:lorusso:3}), as noted by \citet[I:691]{Manzini2005}. Since the auxiliary has specialized forms, there is no preposition introducing the embedded verbs.

%%please move \begin{table} just above \begin{tabular
\begin{table}
\begin{tabular}{*{4}{l}}
\lsptoprule
Indicative present & Auxiliary \textit{stɛ} & Prep. & Lexical Verb\\\midrule
\scshape 1sg & sta & ${\emptyset}$ & ffatsu\\
\scshape 2sg & sta & ${\emptyset}$ & ffatʃi\\
\scshape 3sg & sta & ${\emptyset}$ & ffatʃi\\
\scshape 1pl & sta & ${\emptyset}$ & ffatʃimu\\
\scshape 2pl & sta & ${\emptyset}$ & ffatʃiti\\
\scshape 3pl & sta & ${\emptyset}$ & ffannu\\
\lspbottomrule
\end{tabular}
\caption{Progressive for the verb \textit{ffari} (= to make) in the variety of Mesagne}
\label{tab:lorusso:4}
\end{table}

Apparently, in all the varieties in which there are specialized forms for the aspectual auxiliary, we do not find any restriction on the inflection of the embedded verb. So while in Conversanese (\tabref{tab:lorusso:1}) there are no specialized forms for the auxiliary, and with 1\textsuperscript{st} pl and 2\textsuperscript{nd} pl we do not find the full inflected embedded verb, in the other varieties, when the aspectual auxiliary has specialized forms, the embedded verb is always inflected. While this generalization seems to hold for the Apulian varieties under analysis (Tables 1–3), it is not attested in all varieties (including those from Sicily, Calabria and Salento) described by \citet{Manzini2017}. Following these authors, we assume that different micro{}-parameters cluster together across varieties, such as the presence\slash absence of the preposition \textit{a} and the inflectional morphology on the specialized forms of the auxiliary. In the majority of varieties, Manzini et al. found that only one verb shows the complete inflectional paradigm, either the auxiliary or the embedded verb; a huge number of dialects have inflections on the embedded verb – with the possibility of partial phi{}-feature inflection on the matrix verb (as in the reduced forms of the specialized auxiliary in the cases of Putignano in \tabref{tab:lorusso:2} and Martina Franca in \tabref{tab:lorusso:3}). Thus, the parametric variation seems to be linked mainly to where the inflection appears: on the auxiliary, on the embedded verb or on both. \footnote{Following the data of \citet{Manzini2017}, we can find only two varieties in which both the matrix auxiliary and the embedded verb show the full inflectional paradigm (with no specialized forms for the auxiliary): the Apulian variety of Torre S. Susanna and the Sicilian variety of Modica. Nevertheless, there is a single example of the matrix verb bearing the full inflectional specifications to the exclusion of the embedded verb, namely Carmiano (Apulia). For a detailed analysis of the micro-parametric variation in aspectual inflected constructions, see \citet{Manzini2017}.} 

In many varieties, \citet{Manzini2017} do not find a 1\textsuperscript{st} and 2\textsuperscript{nd} person plural split for finite\slash non-finite embedding; rather, the splits involve different persons or the number feature alone (singular vs. plural).\footnote{In the variety of Camporeale (Manzini, \citealt{Manzini2017}: 38).} With regard to our data, the 1\textsuperscript{st} and 2\textsuperscript{nd} person plural split found in the distribution of the progressive aspectual inflected construction in Conversanese is linked to a general pattern found across Romance varieties, according to which 1\textsuperscript{st} and 2\textsuperscript{nd} persons plural show different inflectional patterns (\citealt{Manzini2005,Manzini2007,Manzini2011Bio}) because they are more referentially complex; we will return to this topic in §6. 

In sum, this general pattern of aspectual inflection is quite widespread in the Southern varieties. These constructions may vary in the aspectual auxiliary that participates in these derivations (progressive, modal, motion verb) and in the tense (present, past) and mood (imperative, indicative) in which they are found. Furthermore, the appearance of agreement morphology on V1 and V2 is subject to microparametric variation. Within the spectrum of variation, some varieties, such as that of Conversano, show a person split for 1\textsuperscript{st} and 2\textsuperscript{nd} persons plural, for which the inflected construction is not available. However, all these progressive aspectual inflected constructions share locative properties (for example, the second verb introduced by the preposition \textit{a}). In the next section, a crosslinguistic analysis of the locative-like system of the progressive will be presented in order to provide a background for the syntactic proposal in §4.

\section{The progressives as locative unaccusative constructions}% 3.
In the typological literature, progressives have been claimed to involve locative constructions. This fits with a very widespread characteristic of human language: progressive is often realized in syntax in the form of a locative predication. The pervasiveness of this grammatical isomorphism between progressive and spatial location was clearly documented in the typological overview undertaken by Bybee, Perkins \& Pagliuca (1994).\todo{not in bib} The progressive involving a locative construction can be distinguished in terms of how the locative relation is expressed: either by a preposition or an auxiliary.

Languages like Italian and Spanish may encode the progressive through the use of the auxiliary ‘stay’: \textit{stare} (in Italian) in (14) and \textit{estar} in Spanish (15). The same auxiliary is found with locative expressions and with stage-level predicates, as in the Spanish examples (16) and (17).

\ea%14
        Italian\\
        Gianni sta mangiando.\\
    \glt ‘Gianni is eating.’
\z

\ea%15
         Spanish\\
        Juan está estudiando.\\
    \glt ‘Juan is studying.’
\z

\ea%16
         Spanish (locative construction)\\
        Juan está en la habitación.\\
    \glt ‘Juan is in the room.’
\z

\ea%17
    	 Spanish (stage-level predicate)\\
        Juan está cansado.\\
    \glt ‘Juan is tired.’
\z

\citet{Mateu1999}, among others, show that in a wide range of languages progressives are also expressed through the use of locative prepositions. Examples (18-20) show that progressives are expressed through an overt locative preposition in Dutch (18) and French (19), while Middle English expressed the progressive through the preposition \textit{on} (20).

\ea%18
         Dutch (van Gelderen 1993: 180–182)\todo{provide ref}\\
    \gll Ik ben aan het/’t  werken.\\
         I  am  on  the working\\
    \glt ‘I am working.’
\z

\ea%19
        French (Demirdache & Uribe-Etxebarria 1997: 9; 1998: 25)\todo{The reference in the bibliography is from the year 2000, but this example does not appear to be taken from there; there is no 1997 or 1998 reference in the bibliography}\\
    \gll Zazie  est en  train   de  miauler.\\
         Zazie  is  in  along   of  miaowing\\
    \glt ‘Zazie is miaowing.'
\z

\ea%20
        Middle English (Jespersen 1949: 168, apud Bybee et al. 1994: 132)\todo{provide refs}\\
        He is  on  hunting.
\z

In languages like Gungbe, there is a progressive particle \textit{tò} which means literally ‘be at’. The lexical verb, when it immediately follows the progressive particle, similarly to what happens in Conversanese, may undergo a process of reduplication \citep{Aboh2004,Aboh2009}, as in (21), where \textit{ɖa} is the verb and \textit{ɖiɖa} is its reduplicated form.

\ea%21
    \label{ex:lorusso:21}
    \gll ɛtɛ   wɛ   mi   tò       ɖiɖa   na  Aluku \\
          what  \textsc{foc} \textsc{2pl} {\textsc{prog}(=‘be at’)}   cook  to  Aluku \\
    \glt ‘What are you cooking for Aluku?’ \citep{Aboh2004} 
\z   

\citet{Mateu1999}, referring to this general analysis of progressives as locative constructions, further argue that progressives are universally unaccusative. In their proposal, two assumptions are made in order to refer to progressives as unaccusatives: the first is that, since progressives are expressed in the majority of the languages in the world by a locative structure, locatives are unaccusatives, and so progressive represents a process of unaccusativization for the lexical verbs that enter into the progressive derivation. This unaccusativization does not involve a change in the argument structure of the embedded verb. The thematic roles are assigned by the embedded verb that is selected in the locative construction. This kind of change is a type-changing operation (\citealt{deSwart1998}; \citealt{Fernald1999})\todo{bib refs missing}: the event expressed by the embedded verb becomes a state through the locative construction involving the auxiliary and\slash or the locative preposition.\footnote{In this respect, \citet{Manzini2017} do not use the term ‘unaccusativization’ in the same way as \citet{Mateu1999}. The change in the semantics of the embedded verbs is linked to the instantiation of a part\slash whole relation between the event (denoted by the embedded predicate) and the auxiliary: the embedded predicate is the event whose internal aspect represents the \textit{whole}, while the auxiliary represents the time of utterance and it is the part of the event which is stressed by the progressive form (for a discussion of this semantic proposal, see \citealt{Higginbotham2009} and \citealt{Landman1992}\todo[inline]{bib refs missing}). I will be using the term ‘unaccusativization’ just to refer to this event type change, as was also the case in the original framework of \citet{Mateu1999}; see Footnote 7 in this respect.}  The second assumption is strictly linked to the first assumption: the process of unaccusativization is implied by the fact that the subject of a progressive structure enters in a central coincidence relation with the event denoted by the lexical verb (i.e. its lexical aspect or aktionsart). The central coincidence relation is the location within the locative structure: it is one precise moment within the event.\footnote{\citet{Mateu1999} argues that there is a syntactically relevant semantic structure, which can be represented in a tree structure (cf . \citealt{Bouchard1995} for the same proposal). In their lexical-conceptual structure (LCS), the argument structure of the verbs (including locative constructions) can be viewed as a spatial relation in the sense that it purely relates elements to our cognitive space: Figure (i.e. the subject) and Ground (the locative complement), to use \citegen{Talmy1985} terminology. On this approach, the timeframe of an event is also represented through a spatial relation.}  For telic predicates, such as in (22), the event has a natural endpoint in the sense that John ‘finished’ building the house. In the progressive version (23), the subject \textit{John} is centrally located within the temporal contour of the event of building the house, so he is represented in a moment in which the the process of building is not yet complete.\footnote{For an analysis of how languages encode the central coincidence relation or terminal coincidence relation first introduced by \citet{Hale1993}, see \citet{Mateu2004}\todo[inline]{bib ref missing} and \citet{Ramchand2001}.}

\ea%22
    \label{ex:lorusso:22}
    John built the house.\\
\textsc{john built the house}
    \z


\ea%23
    \label{ex:lorusso:23}
    John was building the house.\\
\textsc{john did not build the house}
    \z


In ergative languages like Basque, the single argument (‘subject’) of an intransitive verb behaves like the object of a transitive verb and is marked with the absolutive case, and it differs from the agent (‘subject’) of a transitive verb, which is marked with the ergative case. \citet{Laka2006} argues that progressive structures in Basque are homomorphic with locative\slash unaccusative structures, which results from the fact that the progressive auxiliary \textit{ari} involves a biclausal syntactic structure (26). The main verb \textit{ari} ‘to be engaged’ takes a locative PP (‘in something’) expressed through the locative suffix, as in the intransitive structures in (24, 26): the PP can take a nominal complement (24b) or a VP (26b). 

\ea%24
    \label{ex:lorusso:24}
    \ea
    \gll Emakume-a     danza-n   ari    da.\\
         woman-\textsc{det(abs)} dance-\textsc{loc} engaged  is\\
    \glt ‘The woman is engaged in dance.’ (‘The woman is dancing.’) \citep{Laka2006}
    \ex
\begin{forest}
[IP
        [DP [emakume-a,roof] ]
        [I'
        [VP [PP [DP [dantza,roof]
                    ]
                [P [n]]
                ]
              [V [ari]]
            ]
            [I [da]]
        ]
    ]
\end{forest}
    \z
\z\todo{please review and approve tree}


With transitive verbs, \citet{Laka2006} points out that there is a contrast between canonical transitive sentences, in which the subject receives accusative case (25), and their progressive equivalents, in which the subject and the nominalized clause \textit{ogia jaten} (26) receive absolutive case (zero marked).

\ea%25
    \label{ex:lorusso:25}
    \gll Emakume{}-a-k   ogi-a      jaten  du.\\
         woman-\textsc{det}{}-\textsc{erg} bread-\textsc{det} eating  has\\
    \glt ‘The woman eats (the) bread.’
\z


\ea%26
    \label{ex:lorusso:26}
    \ea
    \gll  Emakume-a     ogi-a      ja-te-n   ari     da.\\
          woman-\textsc{det}(\textsc{abs)} bread-\textsc{det} eat-\textsc{nom}{}-\textsc{loc} engaged   is\\
    \glt ‘The woman is (engaged in) eating the bread.’
    \ex \citep{Laka2006}\\
    \begin{forest}
    [IP
        [DP [emahume-a\textsubscript{i},roof]]
        [I'
            [I [da]]
            [VP 
                [V[ari]]
                [PP
                    [P[-n]]
                    [NP
                        [N[-te]]
                        [VP
                            [DP[PRO\textsubscript{i}]]
                            [V'
                                [V[{ja(n)}]]
                                [DP[ogi-a,roof]]
                            ]
                        ]
                    ]
                ]
            ]
        ]
    ]
    \end{forest}
\z
\z\todo{please review and approve the tree}

These data concerning overt case marking in Basque confirm that progressive structures imply an unaccusativization of the event: when the progressive auxiliary is expressed, the subject is marked with absolutive case, as in all intransitive (unaccusative) structures. Furthermore, the presence of a PP as a complement of the auxiliary supports the crosslinguistic generalization that progressives are unaccusative locative constructions. The next section is devoted to the analysis of the progressive constructions in Conversanese as locative constructions. 

\section{A syntactic analysis of the progressive inflected constructions}% 4. 
\subsection{Introduction}% 4.1. 
The main progressive construction in Conversanese, which we introduced in §1 and §2 and is repeated here in (27), is formed from an inflected stative verb \textit{stɛ} (=‘to stay’), a locative preposition \textit{a} and an inflected lexical verb. It patterns with the unaccusative locative construction (28) formed from a stative auxiliary and a locative phrase. 

\ea%27
    \label{ex:lorusso:27}
    \gll Stek     a  'fatsə    u    pɜn.     \\
         stay.\textsc{1sg} to   do\textsc{.1sg} the    bread\\
    \glt ‘I am making the bread.’
\z


\ea%28
    \label{ex:lorusso:28}
    \gll Stek     a  'kɜsə.     \\
         stay.\textsc{1sg} at   home  \\
    \glt ‘I am at home.’
\z

The main difference between the two sentences is that in (28) the complement of the preposition is an NP: the subject is in a spatial relation with the NP \textit{'k}\textit{ɜsə} (=‘home’). In (27), the subject is centrally located within the timeframe denoted by the telic event of making the bread. The progressive involves a PP that introduces an IP. We propose for (27) the derivation suggested by \citet{Manzini2005}: the aspectual inflected construction involves a connecting preposition which is selected by the aspectual auxiliary (29). 

\ea%29
\label{ex:lorusso:29}
Stek a 'fatsə  u pɜn.\\     
‘I am making the bread.’\\
\begin{forest}
[IP
    [I[St-ek,name=stek]]
    [VP[V[(\sout{stɛ}),name=ste]]
        [PP loc [P loc[a]]
            [IP[I[fatsə,name=fatse]]
                [VP[V[(\sout{\textit{fɛ}}),name=fe]]
                [DP[u pɜn]]
                ]
            ]
        ]
    ]
]
\path (fe.south)  edge[{Circle[]}-{Triangle[]}, bend left] (fatse.south)
      (ste.south) edge[{Circle[]}-{Triangle[]}, bend left] (stek.south);
\end{forest}
\z\todo{ɛ doesn't print in the tree}


The sentence in (29) is a biclausal structure, since both the auxiliary and the embedded verb show overt present indicative morphology. These constructions can be considered biclausal if we follow one of the diagnostics proposed to account for the biclausality of present perfect (for English, \citealt{Chomsky1957,Chomsky1981,Chomsky1995}; for Romance languages, \citealt{Kayne1993,Manzini2005,Manzini2007,Manzini2011Bio}): that is, the optionality of clitic placement in Romance languages \citep{Manzini2011Bio}. The progressive in Conversanese shows long-distance clitic placement (30): the clitic climbs to a proclitic position before the auxiliary, as in the ‘restructuring’ present perfect constructions in the sense of \citet{Rizzi1982}. However, there are also varieties in which the clitic is found not only in a long-distance configuration, but also as a proclitic on the embedded verb, as in the the following examples of the aspectual inflected construction from Minervino Murge (31), Montemilone (32), Mesagne (33) and Alliste (34). The examples from Mesagne (33) show that optionality of clitic placement is found within the same variety (33a vs. 33b). The optionality of clitic placement across and within varieties in Romance shows that the parameter is independent of the monoclausal vs. biclausal status of the construction involved. In this respect, long-distance clitic placement cannot be taken as proof of monoclausality (see \citealt{Manzini2011Bio,Manzini2017} for discussion).

\ea%30
         Conversano, Apulia\label{ex:lorusso:30}\\
    \gll U   stek    a   (*u)  mandʒə \\
         it\textsc{(cl)} stay.\textsc{1sg} at  it\textsc{(cl)} eat.\textsc{1sg} \\
    \glt ‘I am eating it.’
\z

\ea%31
         Minervino Murge, Apulia (\citealt{Manzini2005})\label{ex:lorusso:31}\\
    \gll Væ     u    cæmə. \\
         go.\textsc{2sg} him(\textsc{cl})   call.\textsc{2sg}    \\
    \glt ‘You go to call him.’ 
    \z

\ea%32
         Montemilone, Basilicata (\citealt{Manzini2005})\label{ex:lorusso:32}\\
    \gll Va/vinə   u    camə         \\
         go/come  him   call.\textsc{2sg}\\
    \glt ‘You go to call him.’ 
    \z



\ea%33
    \label{ex:lorusso:33}
    \ea  Mesagne, Apulia (\citealt{Manzini2005})\\
    \gll Vɔʄʄu     lu  veʃu.      \\
         want.\textsc{1sg} it   see.\textsc{1sg}\\
    \glt ‘I want to see it.’ 
    \ex 
    \gll Lu   sta   ffattsu.\\
         it\textsc{(cl)}  stay  do.\textsc{1sg}\\
    \z
\z

\ea%34
         Alliste (\citealt{Manzini2005})\label{ex:lorusso:34}\\
    \gll ʃta     llu     tʃɛrku    \\
         stay\textsc{(aux)} (\textsc{1sg})him/it  search\\
    \glt ‘I am searching for him/it.’ 
\z

As pointed out in \citet{Laka2006} for the Basque progressive auxiliary \textit{ari}, the verb \textit{stɛ} coincides with the lexical verb ‘stay’: the same form of the verb is used for both locative\slash progressive constructions and for sentences involving other PPs, (35). In varieties where the progressive auxiliary differs from the lexical \textit{stay}, such as in Putignano, we have the progressive forms without the connecting preposition, (36), and the lexical \textit{stay} with a preposition, (37).\footnote{This pattern found in the variety of Putignano is quite stable, anyway it is not found in other varieties such as that of Martina Franca, in which both the lexical and the progressive forms of stay coincide. In other varieties, the presence of a specialized progressive form does not always imply the absence of the connecting locative element (see \citealt{Manzini2017}). Further analysis is needed in these varieties to understand the relevance of the pattern found in the variety of Putignano.}

\ea%35
         Conversano, Apulia\label{ex:lorusso:35}\\
    \gll Stɛm     kə  la   makənə.       \\
         stay.\textsc{1pl} with   the  car \\
    \glt ‘We are by car.’
\z

\ea%36
         Putignano, Apulia\label{ex:lorusso:36}\\
    \gll Sta     ffaʃeimə.\\
         stay\textsc{(aux)}   (\textsc{1pl.})do(.\textsc{1pl})        \\
    \glt ‘We are doing.’
    \z



\ea%37
    Putignano, Apulia\label{ex:lorusso:37}\\
    \ea
    \gll Stam    aə  la  ‘mɛkənə.\\
         stay.\textsc{1pl}   with  the  car\\
    \glt ‘We are by car.’
    \ex
    \gll Stam     a   kɛsə.\\
         stay.\textsc{1pl}   at  home \\
    \glt ‘We are at home.’
    \z
\z

These biclausal progressive constructions, as \citet{Manzini2005} suggest, involve event identification between the two inflected verbs, contrary to the asyndetic constructions of the imperative in Neapolitan \citep{Ledgeway1997}, where each verb represents an assertion (see the examples in 6-7). Event Identification is defined by \citet{Kratzer1996} as a recursive operation involving the external argument and the aspectual reading that is applied to the event denoted by the embedded lexical VP.\footnote{In \citet{Kratzer1996}, the lexical root (embedded verb) contains information about the internal argument, but the external argument is introduced by a hierarchically superior functional head \textit{v}. This was initially posited by Kratzer as a mechanism for joining the external argument onto a verb using Voice. Event identifying Voice and the verbal event adds the condition that the verb has an Agent. Event Identification takes one function of type <e,<s,t{>}{>} (a function from individuals to functions from events to truth values) and another function of type <s,t> (a function from events to truth values) and returns a function of type <e,<s,t{>}{>}. In other words, Event Identification combines two predicates of events by abstracting over both of their event arguments. The insight of \citegen{Kratzer1996} Event Identification is that it is a recursive operation that allows an n-clausal syntactic structure to be mapped onto a mono-eventive semantic representation. Although T is usually assumed to close off the event variable introduced by V and \textit{v}, successive event identifications with higher functional heads allow for different aspectual interpretations. In the cases discussed here, the recursive use of Event Identification allows us to add (through a second recursive operation after the introduction of the external argument) further aspectual information about the event denoted by the embedded lexical verb.} It relates the external argument, introduced by a \textit{v} head or by aspectual heads, to the predicate via an identification of the event variable of the embedded predication. The overt effect of Event Identification is the agreement morphology on both the auxiliary and the embedded verb. Roughly, Event Identification allows us to add further aspectual information to the event described by the verb. Only if the two predicates have compatible aktionsarten may event identification take place. With respect to the constructions discussed here, the progressive auxiliary allows for event identification, following \citegen{Vendler1967} classification, with activities and accomplishments, but not with achievements or states.

\ea%38
\label{ex:lorusso:38}
\ea[]{Activity\\
    \gll Stec     a  manʤə.\\
         stay.\textsc{1sg} to   eat.\textsc{1sg}   \\
    \glt ‘I am eating.’}
\ex[]{Accomplishment\\
    \gll Stek     a  fatsə    la  kɜsə.\\
         stay.\textsc{1sg} to   build.\textsc{1sg} the   house\\
    \glt ‘I am   building the house.’}
\ex[]{State\\
    \gll \#Stek    a  satʧə. \\
         stay.\textsc{1sg} at  know.\textsc{1sg} \\
    \glt ‘I am knowing.’}
\ex[\#]{Achievement\\
    \gll Stek     a  canəskə  u  ‘sennəkə.        \\
         stay.\textsc{1sg} at   know.\textsc{1sg} the   mayor \\}
    \z
\z    

The structure in (29) cannot be accounted for in terms of a serial verb construction if we follow \citegen{Baker1989Object} analysis, for which the serial verbs must share the same object. However, as \citet{Cruschina2013} suggests, we can consider these aspectual inflected constructions as serial verb constructions if we adopt a less rigid definition of serial verbs, such as that of \citet[12]{Aikhenvald2006}: “Prototypical serial verb constructions share at least one argument. Serial verb constructions with no shared arguments are comparatively rare, but not non-existent.” The aspectual progressive constructions under discussion share the same subject, which is also marked on the overt morphology of both verbs. 

The presence of the connecting element \textit{a} should also support an analysis of the aspectual inflected constructions as non-serial-verb constructions.\footnote{Two hypotheses are found in the literature regarding the origins of \textit{a}: (i) it comes from the Latin preposition \textit{ad}; and (ii) it derives from the Latin coordinating conjunction \textit{ac} used in spoken and late Latin (cf. \citealt[§§710,761]{Rohlfs1969}). Although in other southern Italian varieties there are cases in which the \textit{a} is used both as a locative preposition and a conjunction, in the present analysis we analyze the \textit{a} as a locative preposition (given the locative nature of the progressive). Further evidence comes from the aspectual non-inflected construction in (39).} Nevertheless, in the varieties of Putignano, Martina Franca and Mesagne, we do not find such a connecting element (see Tables~\ref{tab:lorusso:2}, \ref{tab:lorusso:3}, \ref{tab:lorusso:4}). With regard to such “unstable” connecting elements found with serial verbs, \citet{Aikhenvald2006} admits that serial verb constructions “may include a special marker which distinguishes a SVC from other types of constructions but does not mark any dependency relations between the components” \citep[20]{Aikhenvald2006}. So in the case of the locative progressive inflected structure in (29), we can call it a serial verb construction since the two verbs are inflected and the connecting locative preposition is a special marker of the instantiation of a central coincidence relation (not a dependency relation) between the two verbs: the output is a unique event. In contrast, the progressive locative construction with the embedded uninflected verb has a different structure and distribution: it does not imply event identification and it is not a serial verb construction, since the embedded verb is an infinitival complement which is in a dependency relation with the matrix auxiliary. 

\subsection{The progressive ‘uninflected’ constructions}% 4.2. 

In Conversanese, there is a parallel progressive construction that we introduced in §1 and §2 and is repeated here in (39). It is formed from an inflected stative verb \textit{stɛ} (=‘to stay’), the locative preposition \textit{a} and an uninflected lexical verb (infinitive). It differs from the aspectual inflected construction mainly in its syntactic structure and aspectual entailment. 

\ea%39
    \label{ex:lorusso:39}
    \gll Stek     a  fɛ     u  pɜn.      \\
         stay.\textsc{1sg} to   do.\textsc{inf}  the   bread\\
    \glt ‘I am making the bread.’
\z

Like the aspectual inflected progressive (30), it allows only long-distance clitic placement, (40). But since the embedded verb is an infinitive, it allows enclitics, (41), which are not possible with the finite verbs in the inflected aspectual counterpart. 

\ea%40
    \label{ex:lorusso:40}
    \gll U   stek     a (*u)   man’dʒɛ. \\
         it\textsc{(cl)}  stay.\textsc{1sg} at   it\textsc{(cl)} eat.\textsc{inf} \\
    \glt ‘I am eating it.’ 
\z

\ea%41
    \label{ex:lorusso:41}
    \gll Stek  a   mandʒa-llə.\\
         stay  at  eat.\textsc{inf}\textsc{-acc(cl)} \\
    \glt ‘I am eating it.’
\z

As for the locative structures in (28) and the aspectual inflected constructions in (27), we have a locative construction where the aspectual auxiliary selects a locative PP, but in (39) the PP introduces an infinitive that is a full indefinite CP\textsubscript{I} in the terms of \citet{Manzini2003}: “The domain, labelled C\textsubscript{I} ,to suggest Indefiniteness, is identified with the ‘indefinite’ modality lexicalized by infinitivals” (\citealt{Manzini2003}:97). The infinitival verb raises to a CP\textsubscript{I} position and the accusative enclitic is embedded in a nominal position before the the inflectional domain, as in (42).

\ea%42
    \label{ex:lorusso:42}
Stek a mandʒa-llə.
\glt ‘I am eating it.’\\
\begin{forest}
[IP[X[Stek]]
[VP [V]
    [PP loc
        [P loc[a]]
        [CP\textsubscript{I}
            [CP\textsubscript{I} [mandʒa]]
            [N [N [-llə]] [IP [~~~~~,roof]]]
        ]
    ]
]
]
\end{forest}
\z

The structure in (42) is a locative structure: the subject is located in a position within the indefinite event expressed by the embedded infinitival verb. While in (29) we have been saying that the subject is centrally located within the event denoted by the embedded lexical verb, in (42) the subject is located (not centrally) within the event. In fact, we also find this type of progressive construction with states (43) and achievements (44) that were banned for the aspectual inflected construction. In (43) and (44) the interpretation of the sentence is inchoative: the subject is located in the starting point of the event denoted by the embedded verb.

\ea%43
         State\label{ex:lorusso:43}\\
    \gll Stek     a  saˈpe.     \\
         stay.\textsc{1sg} at  know.\textsc{inf}  \\
    \glt ‘I am realizing it.’ (=‘I am starting to know.’) 
\z


\ea%44
         Achievement\label{ex:lorusso:44}\\
    \gll Stek     a  canəʃə   u  ‘sennəkə.\\
         stay.\textsc{1sg} at   know.\textsc{inf} the   mayor \\
    \glt ‘I am getting in touch with the mayor.’
\z

These constructions do not identify a unique event. Similarly to the asyndetic imperative constructions in Neapolitan \citep{Ledgeway1997} in (6) and (7), these constructions may be decomposed into two subevents: the auxiliary denotes both a truly locative and a progressive periphrasis.\footnote{They do differ from the asyndetic constructions of Ledgeway (1997), since there is a connecting element between the two verbs and they cannot be interpreted as truly paratactic constructions.} Due to the indefiniteness of the infinitival verb in CP\textsubscript{I}, the subject is controlled by the matrix subject.\footnote{For the purposes of the present work, the CPI has to be interpreted merely as tenseless, in the sense that it lacks independent tense specification and thus agrees in tense with the matrix auxiliary. However, for a complete analysis of the CPI, see \citet{Manzini2005,Manzini2007,Manzini2011Bio}.} This is confirmed by the presence of the accusative enclitic, (41-42). No special forms are found for the matrix auxiliary with the uninflected construction (compare the specialized matrix auxiliary for the inflected construction in the varieties of Putignano, Martina Franca and Mesagne) and the connecting element can never be omitted. Nevertheless, the aspectual infinitive constructions with the verb \textit{stay} are still interpreted as progressive constructions: they are the sole progressive forms available for 1\textsuperscript{st} and 2\textsuperscript{nd} persons plural (§5) and they mark an ambiguous progressive form. The next section is devoted to sketching the aspectual differences between the inflected and non-inflected aspectual progressive constructions. 

\section{Aspectual analysis of the inflected and non-inflected progressive constructions}% 5. 
Both inflected and uninflected aspectual progressive constructions are interpreted as truly progressive: in both cases the event entails an ongoing reading (as in \citealt{Arosio2011} among others).\footnote{We refer all over the present paper to the progressive uninflected constructions as opposed to the inflected ones: we want to stress simply on the fact that the embedded predicate is not inflected.} In other words, the event does not have an entailment of termination. So, for example, telic events with a natural endpoint, such as ‘eat the bread’, are interpreted as unfinished both in inflected (45) and non-inflected constructions (46). 

\ea%45
         Inflected construction\label{ex:lorusso:45}\\
    \gll Stek     a  mandʒə   u  paninə. \\
         stay.\textsc{1sg} to  eat.\textsc{1sg} the   sandwich \\
    \glt ‘I am eating the bread.’\\\textsc{i have not eaten the bread}
\z



\ea%46
         Uninflected construction\label{ex:lorusso:46}\\
    \gll Stek     a  man’dʒɛ  u  pɜninə.     \\
         stay.\textsc{1sg} to  eat.\textsc{inf} the   sandwich \\
    \glt ‘I am eating the sandwich.’\\\textsc{i have not eaten the bread}
\z 

They differ from simple present forms, since they are not found with habitual constructions, as shown in (47): in (47a) the temporal modifier ‘every year’ is found with the present tense, while we cannot find this ‘habitual’ temporal modifier with inflected (47b) and uninflected (47c) progressives.

\ea%47
    \label{ex:lorusso:47}
    \ea[]{
    \gll Tottə  i  annə   vek     o   mɛr. \\
         all  the   years  go.\textsc{1sg} to.the  sea\\
    \glt ‘Every year I go to the sea.’  }
    \ex[\#]{
    \gll Tottə   i  annə   stek     a  vekə     o  mɛr. \\
             all    the   years  stay.\textsc{1sg} to   go.\textsc{1sg} to.the  sea\\
    \glt     ‘\#Every year I am going to the sea.’  }
    \ex[\#]{
    \gll Tottə i  annə   stek    a   ʃʃi  o  mɛr.\\
         all   the   years  stay.\textsc{1sg} to  go.\textsc{inf} to.the  sea\\
    \glt ‘\#Every year I am going to the sea.’  }
    \z
\z

A major difference is found between the aspectual interpretations of the two constructions. This is linked to the episodic value of progressives: \citet{Chierchia1995}, among others, suggests that while individual-level predicates express properties of individuals that are permanent or tendentially stable, progressives and stage-level predicates, by contrast, attribute transitional and episodic properties to individuals. Frequentative adverbs roughly indicate the repetition of the same action, and thus are mainly incompatible with progressive episodic operators. We might expect, then, that neither inflected nor uninflected constructions can be found with frequentative adverbs, but this is not the case: uninflected progressives can be found with frequentative adverbs.

In both type of constructions, the morpheme \textit{a} is the only element that can intervene between the two verbs. Adverbs like \textit{sembə} (=‘always’), which encodes frequentative aspectual properties \citep{Cinque1999}, cannot be found between the functional and the lexical verb, but are only allowed after the complex predicate with both type of constructions, (48) and (49). Furthermore, with the ‘uninflected’ construction in (49) we can also find the frequentative adverb between the matrix auxiliary and the locative PP, while it is ruled out in the inflected construction in (48). 

\ea%48
         Inflected embedded verb\label{ex:lorusso:48}\\
    \gll Mariː  stɜ    (*sembə)  a  (*sembə)   mandʒə   (sembə). \\
         Maria  stay.\textsc{3sg} (always)  to   (always)  eat.\textsc{3sg} (always)\\
    \glt ‘Maria is always eating.’
\z


\ea%49
         Inflected embedded verb \label{ex:lorusso:49}\\
    \gll Mariː  stɜ     (sembə)   a  (*sembə)  man’dʒɛ  (sembə). \\
         Maria  stay.\textsc{3sg} (always)  to   (always)  eat.\textsc{inf} (always)\\
    \glt ‘Maria is always eating.’
\z

\citet{Cardinaletti2003}, in their analysis of aspectual inflected constructions with motion verbs in Sicilian, take the different distribution of frequentative adverbs as proof that the inflected version is monoclausal while the uninflected one is biclausal. Our proposal, on the contrary, is that both types of progressives are biclausal. The presence of the frequentative temporal quantifier with the uninflected construction is linked to the indefinite CP\textsubscript{I} selected by the locative preposition. The subject of the embedded verb in CP\textsubscript{I} must receive a variable\slash operator interpretation, since no person and number morphology is found on it as in the control constructions. The subject of the matrix auxiliary is just located within the event denoted by the embedded verb, but it is not in a central coincidence relation with the embedded predicate. The frequentative adverbial modifier can bind the variable introduced by the embedded infinitival verb in (49) and allow a frequentative interpretation of the progressive locative construction.\footnote{Since the embedded verb is tenseless and aspectless, an adverb can work as an operator that binds it, intervening, as a modifier, in the aspectual relation instantiated between the matrix aspectual auxiliary and the embedded verb: the embedded verb, in fact, has no overt morphology marking its inherent aspect, so its aspect can be more easily modified\slash marked by an (extra) adverbial item.} The double inflection of (48), on the other hand, marks the fact that event identification has taken place and the fact that the subject is centrally located within the event denoted by the embedded predicate: no temporal and aspectual binding is possible, since both the auxiliary and the embedded verb show the same inflectional morphology. Nevertheless, besides these minor aspectual differences, both types of constructions still imply a progressive reading: the ‘uninflected’ construction, in fact, is the only progressive form found with the 1\textsuperscript{st} and 2\textsuperscript{nd} persons plural. The next section is devoted to analysing the distribution of the aspectual constructions inflected for person and number.

\section{Person split in the progressive aspectual inflected constructions}% 6. 

The progressive aspectual inflected construction is not found with 1\textsuperscript{st} and 2\textsuperscript{nd} persons plural. As we mentioned in §1, (3-4), repeated here as (50-51), the 1\textsuperscript{st} and 2\textsuperscript{nd} persons plural do not allow the progressive constructions involving the inflected embedded verb (50), but are only found in the construction involving an embedded infinitival verb, (51). 

\ea%50
    \label{ex:lorusso:50}
    \gll Nojə\textsubscript{j}/voʊ\textsubscript{k}   stɛmə\textsubscript{j}/stɛtə\textsubscript{k}   a       *mandʒɛmə\textsubscript{j}/*mandʒɛtə\textsubscript{k}.\\
         we/you  stay\textsc{.1pl/2pl}  to   eat.\textsc{1pl/2pl}\\
    \glt ‘We/you are eating.’
\z


\ea%51
    \label{ex:lorusso:51}
    \gll Nojə/voʊ   stɛmə/stɛtə  a   manʹdʒɛ.\\
         we/you  stay.\textsc{1pl/2pl}  to   eat\textsc{.inf}\\
    \glt ‘We/you are eating.’
\z


Similar data are also found in other varieties. \citet{Cardinaletti2003} found a similar pattern in their analysis of the inflected constructions in the dialect of Marsala. \citet{Manzini2005} mention many other southern varieties (not only in Apulia) in which the aspectual inflected constructions are not found with 1\textsuperscript{st} and 2\textsuperscript{nd} persons plural, while the other persons allow it; (51) and (52) provide examples from the Sicilian varieties of Villadoro e Calascibetta.

\ea%52
         Villadoro\label{ex:lorusso:52}\\
    \gll Jamo/jete   a   mmanŋʤarɪ.\\
         go.\textsc{1pl/2pl}   to  eat \\
    \z


\ea%53
         Calascibetta\label{ex:lorusso:53}\\
    \gll Imu/iti   a  mmaɲdʒarɪ.  \\
         go.\textsc{1pl/2pl} to  eat \\
    \z

Why do the 1\textsuperscript{st} and 2\textsuperscript{nd} persons plural not allow the \textit{a}+inflected form construction? Is it worth talking of a person split? Our answer is that the 1\textsuperscript{st} and 2\textsuperscript{nd} persons plural are referentially more complex than the other singular and plural (3\textsuperscript{rd}) persons. Their complexity is linked to the fact that the 1\textsuperscript{st} and 2\textsuperscript{nd} persons plural are not merely plural versions of the 1\textsuperscript{st} and 2\textsuperscript{nd} persons singular. In this sense we are dealing with a person split different from the one attested for the singular persons in auxiliary selection (\citealt{Manzini2005,Manzini2007,Manzini2011Bio}).

\citet{Bobaljik2008Missing} proposes a two-valued binary feature system [±speaker] and [±hearer] to account for the personal pronominal system across languages.\footnote{With varying choices of feature labels, a similar argument has been presented and defended in one form or another by \citet{Ingram1978,Harley2002Person} and, in particular detail, \citet[Chapter~2]{Noyer1997}.} The two-valued person feature system lacks a feature ‘third person’, which is then analyzed as [\textminus{}speaker, \textminus{}hearer]. For plural persons, \citet{Bobaljik2008Missing} argues, along the lines of \citet{Lyons1968} and \citet{Benveniste1966}, that 1\textsuperscript{st} and 2\textsuperscript{nd} persons plural are not merely plurals of the singular 1\textsuperscript{st} and 2\textsuperscript{nd} persons: “We (‘first person plural’) does not normally stand in the same relationship to I (‘first person singular’) as boys, cows, etc., do to boy, cow, etc. The pronoun we is to be interpreted as ‘I, in addition to one or more other persons’… In other words, we is not ‘the plural of I’: rather, it includes a reference to ‘I’ and is plural” \citet[277]{Lyons1968}. So Bobalijk suggests that “[i]t is indeed meaningful to speak of a first person plural, but it is important to note that plural, for the first person, normally means an associative or group plural, rather than a multiplicity of individuals sharing the property [speaker]” \citep[209]{Bobaljik2008Missing}. The same is also true of the 2\textsuperscript{nd} person plural, which is not merely the plural of singular \textit{you}. So while the 1\textsuperscript{st} person plural is not just a sum of [speaker], but is the sum of speaker plus others, the 2\textsuperscript{nd} person plural is not just a sum of [hearer], but is the sum of hearer plus others. Furthermore, \citet{Bobaljik2008Missing} resumes this discussion by saying that while the 1\textsuperscript{st} person plural is the sum of all persons, (54), the 2\textsuperscript{nd} person plural is the sum of all persons excluding the [speaker].

\ea%54
    \label{ex:lorusso:54}
    ‘we’ is 1st (+ 2nd) (+ 3rd) 
\z

          

\ea%55
    \label{ex:lorusso:55}
    ‘you’ is 2nd (+3rd). (adapted from \citealt{Bobaljik2008Missing}) 
\z

          

Following similar considerations on the person system, \citet{Manzini2007,Manzini2011Bio} use a person split analysis to describe the patterns found in other constructions (i.e. auxiliary selection with present perfect) where the 1\textsuperscript{st} and 2\textsuperscript{nd} persons singular (discourse-anchored pronouns: [+speaker, +hearer]) and the 3\textsuperscript{rd} person singular (event-anchored pronouns: [\textminus{}speaker, \textminus{}hearer]) show different morphosyntactic patterns. For the analysis of plural persons, \citet{Manzini2011Bio} argue that “the 1st person plural does not necessarily denote a plurality of speakers (though it may), or the speaker and hearer only (though again it may); rather its denotation routinely involves one speaker and a certain number of other individuals that are being referred to together with the speaker. The same is true for the 2nd person singular, which does not necessarily (or normally) denote a plurality of hearers but simply refers to the hearer taken together with a certain number of other individuals …Because of this referential structure of the so-called 1st and 2nd plural, it is reasonable to propose that even varieties that activate the person split in the singular may not do so in the plural” \citep[213]{Manzini2011Bio}. In a lexical parametrization approach \citep{Manzini1988,Manzini2011Bio}\todo{bib ref missing: Manzini 1988}, languages involve a parametric distinction for plural on the one hand and the discourse participants and event participants may not apply in the plural. 

With respect to the constructions being discussed here, the person split we found in the aspectual inflected progressive of Conversanese is not directly linked to the split involving discourse vs. event participants, but to the referential complexity of the 1\textsuperscript{st} and 2\textsuperscript{nd} persons plural. More precisely, we have been contending that the progressive aspectual inflected constructions are based on a locative structure where the subject of the matrix subject enters into a central coincidence relation within the event denoted by the embedded predicates (as in \citealt{Mateu1999}; \citealt{Laka2006}). The 1\textsuperscript{st} and 2\textsuperscript{nd} persons plural may not enter into this derivation because the referential complexity of the plurality does not allow the instantiation of a central coincidence relation as tight as the one found in the aspectual inflected constructions with other persons, (29). The main idea is that the central coincidence relation entails a reading for which a referentially unique (easily identifiable) event participant is centrally located within the eventive structure. 1\textsuperscript{st} and 2\textsuperscript{nd} persons plural, however, cannot be centrally located due to their referential complexity, which does not allow the identification of a unique participant or group of participants. That is, only clearly identifiable referents can be centrally located in the aspectual progressive constructions, at least in Conversanese. The microparametric variation in the aspectual inflected constructions (see §2.2) shows that different dimensions may determine the finite\slash non-finite split (person and number features, reduced inflectional paradigms, \textit{a}/bare embedding). Conversanese does not allow finite embedding, which encodes a central coincidence relation, for referentially unclear referents; this is an interpretative requirement which blocks the multiple agreement configurations for 1\textsuperscript{st} and 2\textsuperscript{nd} persons plural.\footnote{While some authors define agreement as a mere computational mechanism at work in syntax that may or may not involve a semantic counterpart (the case of default agreement, as in \citealt{Preminger2014}), others claim that agreement always plays a role in semantic interpretation \citep{Manzini2007,Manzini2011Bio}. On this view, agreement does not involve a feature{}-checking operation, but in the terms of \citet{Manzini2007} it represents the sharing of referentially relevant properties that play a role in semantic interpretation. So, under our proposal double agreement represents a marked aspectual reading at the semantic interface, which is obtained through event identification.} To express the progressive with the 1\textsuperscript{st} and 2\textsuperscript{nd} persons plural, the subject is ‘located’ within the event denoted by the embedded verb, but this locative relation is not a central coincidence relation (§4.1): the different aspectual flavors of the two constructions interact with the referential complexity of the 1\textsuperscript{st} and 2\textsuperscript{nd} plural persons.

\section{Conclusions}% 7. 
In this paper, we have presented a preliminary analysis of the progressive form in a number of Apulian dialects, focusing on the variety of Conversano (Apulia). In Conversanese, two forms of the progressive are available. Both constructions are formed from an inflected stative verb, a connecting preposition and a lexical verb. The two constructions differ in the inflection found on the lexical verb selected by the preposition: one type of construction involves an inflected embedded verb, and we have defined this as the aspectual (progressive) inflected construction (following \citealt{Manzini2005}); the other type of construction involves an uninflected embedded lexical verb, and we have defined this as the aspectual uninflected construction. 

Both types of structure share a locative derivation: the majority of progressive forms crosslinguistically, in fact, are derived from expressions involving stative auxiliaries and\slash or locative prepositions (Bybee, \citealt{Perkins1994}\todo{bib ref missing}, \citealt{Mateu1999}, \citealt{Laka2006}). In (29) and (42) we proposed a biclausal syntactic derivation for both inflected and uninflected progressive constructions. The difference is that, while in the inflected construction the locative preposition selects a full IP, in the uninflected one the locative preposition selects an indefinite CP\textsubscript{I}. The distinction between the structures has been used to account for the different syntactic and aspectual properties of the two progressive constructions. 

On the one hand, the aspectual inflected constructions: 1) denote an event identification between the auxiliary and the lexical verb; 2) seem to work like serial verb constructions; 3) allow long-distance clitic placement; and 4) locate the matrix subject of the inflected progressive centrally within the event denoted by the embedded verb. On the other hand, the aspectual uninflected progressive constructions: 1) may denote a a frequentative aspectual reading; 2) seem to work like control constructions; 3) allow enclitic placement on the embedded infinitival verb; 4) locate the subject in a given position (although not in a central coincidence relation) within the event denoted by the embedded verb. 

The 1\textsuperscript{st} and 2\textsuperscript{nd} persons plural are not found in the aspectual inflected constructions, but are only possible in the infinitival counterpart. Differences in the pattern of the morphological derivation of 1\textsuperscript{st} and 2\textsuperscript{nd} persons plural are quite common (\citealt{Manzini2005,Manzini2011Bio}) across Romance languages: these persons are more complex than other persons \citep{Bobaljik2008Missing} because they involve a complex reference to the discourse participants (as with 1\textsuperscript{st} and 2\textsuperscript{nd} singular), to the plurality of participants and to the event participants. However, further analysis is needed in order to account for the nature of this person split: for present purposes, the complexity of the referentiality seems to pattern with certain aspectual interpretations (such as the inchoative interpretation attributed to (43-44) when the embedded verb is infinitival) linked to the complex referentiality, such as the inclusion\slash exclusion of the subject(s) within the complex locative\slash progressive constructions, which involve an event identification\slash change. In a lexical parametrization analysis (\citealt{Manzini2011Bio}), languages involve a parametric distinction for plural persons: the difference between discourse participants and event participants found in the singular (1\textsuperscript{st} and 2\textsuperscript{nd} singular person vs. 3\textsuperscript{rd} person) may not apply in the plural, but different overlapping referents may influence the status of the plural persons and imply their overt morphological realization as a parametric choice across languages.

% % % \section{ References}
% % % 
% % % Aboh, Enoch Oladé. 2004. \textit{The morphosyntax of complement-head sequences: Clause structure and word order patterns in Kwa.} New York: Oxford University Press.
% % % 
% % % Aboh, Enoch Oladé. 2009. Clause structure and verb series. \textit{Linguistic Inquiry} 40. 1–33.
% % % 
% % % Aikhenvald, Alexandra Y. 2006. Serial verb constructions in typological perspective. In Aikhenvald, Alexandra Y. \& Dixon, R. M. W. (eds.), \textit{Serial verb constructions: A cross-linguistic typology}, 1–68. Oxford: Oxford University Press.
% % % 
% % % Aikhenvald, Alexandra Y. \& Dixon, R. W. (eds.). 2006. \textit{Serial verb constructions: A cross-linguistic typology}. Oxford: Oxford University Press.
% % % 
% % % Arosio, Fabrizio, 2011. Infectum and perfectum: Two faces of tense selection in Romance languages. \textit{Linguistics and Philosophy} 33. 171–214.
% % % 
% % % Baker, Mark. 1989. Object sharing and projection in serial verb constructions. \textit{Linguistic Inquiry} 20. 513–553.
% % % 
% % % Benveniste, Émile. 1966\textit{. Problèmes de linguistique générale}. Paris: Gallimard.
% % % 
% % % Bouchard, Denis. 1995. \textit{The semantics of syntax: A minimalist approach to grammar}. Chicago \& London: University of Chicago Press.
% % % 
% % % Bobaljik, Jonathan. 2008. Missing persons: A case study in morphological universals. \textit{The Linguistic Review} 25. 203–230.
% % % 
% % % Cardinaletti, Anna \& Giusti, Giuliana. 2003. Motion verbs as functional heads. In Tortora, Christina (ed.), \textit{The syntax of Italian dialects}, 31–49. New York: Oxford University Press.
% % % 
% % % Chierchia, Gennaro. 1995. Individual-level predicates as inherent generics. In Carlson, Greg N. \& Pelletier, Francis Jeffry (eds.), \textit{The generic book}, 176–223. Chicago \& London: University of Chicago Press. 
% % % 
% % % Chomsky, Noam. 1957. \textit{Syntactic structures.} The Hague: Mouton.
% % % 
% % % Chomsky, Noam. 1981. \textit{Lectures on government and binding}. Dordrecht: Foris.
% % % 
% % % Chomsky, Noam. 1995. \textit{The Minimalist Program.} Cambridge, MA: MIT Press.
% % % 
% % % \begin{styleSfondomedioiColorexi}
% % % Chomsky, Noam. 2001. Derivation by phase. In Kenstowicz, Michael (ed.), \textit{Ken Hale: A life in language}, 1–52. Cambridge, MA: MIT Press.
% % % \end{styleSfondomedioiColorexi}
% % % 
% % % Cinque, Guglielmo. 1999. \textit{Adverbs and functional heads: A cross-linguistic perspective.} Oxford \& New York: Oxford University Press.
% % % 
% % % Cruschina, Silvio. 2013. Beyond the stem and inflectional morphology: An irregular pattern at the level of periphrasis. In Cruschina, Silvio \& Maiden, Martin \& Smith, John Charles (eds.), \textit{The boundaries of pure morphology}, 262–283. Oxford: Oxford University Press.
% % % 
% % % Di Caro, Vincenzo \& Giusti, Giuliana. 2015. A protocol for the inflected construction in Sicilian dialects. \textit{Annali di Ca’ Foscari. Serie occidentale} 49: 393–421.
% % % 
% % % Demirdache, Hamida \& Uribe-Etxebarria, Myriam. 1997. The primitives of temporal relations. In Martin, Roger \& Michaels, David \& Uriagereka, Juan (eds.), \textit{Step by step: Essays on Minimalist syntax in honor of Howard Lasnik}, 157–186. Cambridge, MA: MIT Press.
% % % 
% % % Hale, Kenneth \& Keyser, Samuel Jay. 1993. On argument structure and the lexical expression of syntactic relations. In Hale, Kenneth \& Keyser, Samuel Jay (eds.), \textit{The view from Building 20: Essays in linguistics in honor of Sylvain Bromberger}, 53–109. Cambridge, MA: MIT Press. 
% % % 
% % % Harford Perez, Carolyn. 1985. \textit{Aspects of complementation in three Bantu languages}. Madison, WI: University of Wisconsin-Madison. (Doctoral dissertation.)
% % % 
% % % Harley, Heidi \& Ritter, Elizabeth. 2002. Person and number in pronouns: A feature geometric analysis. Ms. (Tucson, AZ: University of Arizona \& Calgary: University of Calgary.)
% % % 
% % % Jaeggli, Osvaldo \& Hyams, Nina. 1993. On the independence and interdependence of syntactic and morphological properties: English aspectual \textit{come} and \textit{go}. \textit{Natural Language \&}  \textit{Linguistic Theory} 11. 313–346.
% % % 
% % % Kayne, Richard. 1993. Toward a modular theory of auxiliary selection. \textit{Studia Linguistica} 47. 3–31.
% % % 
% % % Kratzer, Angelika. 1996. Severing the external argument from the verb. In Rooryck, Johan \& Zaring, Laurie (eds.), \textit{Phrase structure and the lexicon} 109–137. Dordrecht: Kluwer. 
% % % 
% % % Ingram, David. 1978. Typology and universals of personal pronouns. In Greenberg, Joseph H. (ed.). \textit{Universals of human language, vol. III: Word structure}, 213–248. Stanford, CA: Stanford University Press.
% % % 
% % % Laka, Itziar. 2006.~Deriving split-ergativity in the progressive: The case of Basque.~In Johns, Alana \& Massam, Diane \& Ndayiragije, Juvénal (eds.). \textit{Ergativity: Emerging issues}, 173–195. Dordrecht \& Berlin: Springer.
% % % 
% % % Landau, Idan. 2004. The scale of finiteness and the calculus of control. \textit{Natural Language \&}  \textit{Linguistic Theory} 22. 811–877.
% % % 
% % % Landau, Idan. 2013. \textit{Control in generative grammar: A research companion}. Cambridge: Cambridge University Press.
% % % 
% % % Ledgeway, Adam. 1997. Asyndetic complementation in Neapolitan dialect. \textit{The Italianist} 17. 231–273.
% % % 
% % % Ledgeway, Adam. 2015. Reconstructing complementiser-drop in the dialects of the Salento: A syntactic or phonological phenomenon. In Biberauer, Theresa \& Walkden, George (eds.), \textit{Syntax over time: Lexical, morphological, and information-structural interactions}, 146–162. Oxford: Oxford University Press.
% % % 
% % % Lyons, John. 1968. \textit{Introduction to theoretical linguistics}. Cambridge: Cambridge University Press.
% % % 
% % % Manzini, M. Rita \& Lorusso, Paolo \& Savoia, Leonardo M. 2017. A/bare finite complements in Southern Italian varieties: Monoclausal or bi-clausal syntax?~\textit{Quaderni di Linguistica e Studi Orientali (QULSO)} 3. 11–59.~Florence: Florence University Press.
% % % 
% % % Manzini, M. Rita \& Savoia, Leonardo M. 2003. The nature of complementizer. \textit{Rivista di Grammatica Generativa} 28. 87–110.
% % % 
% % % Manzini, M. Rita \& Savoia, Leonardo M. 2005. \textit{I dialetti italiani e romance: Morfosintassi generativa}. Alessandria: Edizioni dell‘Orso. 
% % % 
% % % Manzini, M. Rita \& Savoia, Leonardo M. 2007. \textit{A unification of morphology and syntax}. London: Routledge.
% % % 
% % % Manzini, M. Rita \& Savoia, Leonardo M. 2011. (Bio)linguistic diversity: Have/be alternations in the present perfect. In Di Sciullo, Anna \& Boeckx, Cedric (eds.), \textit{The biolinguistic enterprise}, 222–265. Oxford: Oxford University Press.
% % % 
% % % Manzini, M. Rita \& Wexler, Kenneth. 1987. Binding theory, parameters and learnability. \textit{Linguistic Inquiry} 18. 413–444.
% % % 
% % % Martins, Ana Maria \& Nunes, Jairo. 2005. Raising issues in Brazilian and European Portuguese. \textit{Journal of Portuguese Linguistics} 4. 53–77.
% % % 
% % % Mateu, Jaume. 2002. \textit{Argument structure: Relational construal at the syntax semantics interface}. Barcelona: Universitat Autónoma de Barcelona. (Doctoral dissertation.)
% % % 
% % % Mateu, Jaume \& Amadas, Laia. 1999. Extended argument structure: Progressive as unaccusative. \textit{Catalan Working Papers in Linguistics} 7. 159–174.
% % % 
% % % Noyer, Rolf. 1997. \textit{Features, positions and affixes in Autonomous Morphological Structure}. New York: Garland.
% % % 
% % % Nunes, Jairo. 2008. Inherent Case as a licensing condition for A-movement: The case of hyper-raising constructions in Brazilian Portuguese. \textit{Journal of Portuguese Linguistics} 7. 83–108.
% % % 
% % % Preminger, Omer. 2014. \textit{Agreement and its failures}. Cambridge, MA: MIT Press. 
% % % 
% % % Ramchand, Gillian. 2001. Aktionsart, l-syntax and selection: Perspectives on aspect. Ms. (Utrecht: Utrecht Institute of Linguistics.)
% % % 
% % % Rizzi, Luigi. 1982. \textit{Issues in Italian syntax}. Dordrecht: Foris.
% % % 
% % % Rohlfs, Gerhard. 1969. \textit{Grammatica storica della lingua italiana e dei suoi dialetti, vol. 3: Sintassi e formazione delle parole}. Torino: Einaudi.
% % % 
% % % Talmy, Leonard. 1985. Lexicalization patterns: Semantic structures in lexical forms. In Shopen, Timothy (ed.), \textit{Language typology and syntactic description III: Grammatical categories and the lexicon}. Cambridge: Cambridge University Press.
% % % 
% % % Vendler, Zeno. 1967. \textit{Linguistics in philosophy}. Ithaca, NY: Cornell University Press.
% % % 
% % % Zeller, Jochen. 2006. Raising out of finite CP in Nguni: The case of fanele. \textit{South African Linguistics and Applied Language Studies} 24. 255–275.
% % % 
% % % 
% % % \begin{verbatim}%%move bib entries to  localbibliography.bib
% % % \end{verbatim} 

\sloppy
\printbibliography[heading=subbibliography,notkeyword=this] 
\end{document}
