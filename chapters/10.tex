\documentclass[output=paper]{langsci/langscibook} 
\author{Anja Weingart\affiliation{Georg-August-Universität Göttingen}}
\title{Null possessives in European Portuguese}

% \chapterDOI{} %will be filled in at production

% % \epigram{Change epigram in chapters/03.tex or remove it there }
\abstract{The paper investigates the referential properties of so-called null possessives in European Portuguese. The term refers to the phenomena that the possessor argument of relational nouns may be left unrealized. Structural diagnostics like locality of binding, and c-command and the interpretative diagnostics like reading under ellipsis and in only-context, split-antecedents, and binding by a quantifier are discussed. The result of these test is that a syntactic analysis in terms of movement or agreement is not feasible in EP. Further a comparison to the referential properties of simple and complex possessive is presented with the aim of discussing a possible semantic analysis.}
\maketitle

\begin{document}

\section{Introduction}% 1. 

This article investigates the interpretation of null possessor arguments of inherently relational nouns in European Portuguese (EP). Inherently relational nouns (kinship and body part nouns) express a relation between two arguments, the possessum and the possessor. For example, the relational noun \textit{pai} in (1), relates the referent of the DP \textit{o pai} (the possessum) to the referent of the name \textit{o João} (the possessor) through the relation \textit{being-the-father-of}.

\ea%1
         EP\label{ex:wein:1}\\
    \gll o   pai   do   João\\
         the   father of.the   J.\\
    \glt ‘João’s father’
    \z


As shown in \citet{Barker2011} and \citet{Löbner2011}, the possessor argument is lexically determined by the type of noun, and even if it is not syntactically realized, it is present for semantic and/or pragmatic interpretation. In European Portuguese it can be left unrealized, as shown in (2).

\ea%2
         EP\label{ex:wein:2}\\
    \gll O  João\textsubscript{i}   conversou     com [o   pai ]\textsubscript{[i, h=speaker+, *m]}.\\
         the J.     talked   with the father\\
    \glt ‘João talked with his father.’\slash ‘João talked with our father.’
\z

The empty possessor argument in (2) may be interpreted as coreferential with the subject \textit{o João,} but it may not corefer with another person present in discourse (indicated by the index ‘m’). Furthermore, it can be interpreted like the 1\textsuperscript{st} person plural possessive pronoun (as shown by the index ‘h=speaker+’). For this option there are three possible combinations of referents: (i) the speaker and the hearer; (ii) the speaker, the hearer and \textit{o João} and (iii) the speaker and \textit{o João.} The referents will be siblings or belong to a group in which the use of the definite DP \textit{o pai} has a unique referent, independent of the context. In this sense the DP \textit{o pai} is interpreted like a proper name.

Additionally, the possessor argument can be realized as an overt possessive pronoun. European Portuguese has two types of possessive pronouns, a simple possessive pronoun and a complex possessive. The complex possessive is formed with the preposition \textit{de} and a personal pronoun. Both forms are shown in (3).

\ea%3
    EP\label{ex:wein:3}\\
    \ea
    \gll O João\textsubscript{i} conversou com o seu\textsubscript{[?i, m, h]} pai.  \\
         the J. talked with the his father\\
    \ex
    \gll O João\textsubscript{i} conversou com o   pai   dele\textsubscript{[i, m, *h]}.\\
         the J. talked with the father of.him\\
    \z
\z

The simple possessive may be interpreted in three ways: it can refer to the subject \textit{o João} or to another person salient in the context (indicated by the index ‘m’), or it can be used as a polite form addressing the hearer (indicated by the index ‘h’). The interpretation as polite form is in fact the preferred interpretation of (3a), according to native speakers.\footnote{Thanks to Ana Maria Martins (CLUL) and Sandra Pereira (CLUL) for their judgements.} The complex possessive in (3b) may take the subject as antecedent and it may corefer with some other person present in the context. In case the use of the simple possessive creates an ambiguity between the interpretation as politeness form and the anaphoric interpretation, the complex possessive will be used.\footnote{For a detailed discussion of the use and interpretation of simple and complex possessives, see \citet{Castro2005}.}

All three possessives (simple, complex and null) can be used anaphorically and refer to a sentence-internal antecedent. But according to native speakers, the use of the null possessive is preferred over the simple possessive with a sentence-internal antecedent. The null possessive can be used anaphorically or as a kind of indexical referring to the speaker and possibly including the hearer and other persons. In the present article I will focus on the anaphoric use.

Furthermore, inherently relational nouns are not uniform with respect to the interpretation of the null possessor argument in combination with definite and indefinite determiners. In (4b), but not in (4a), the relational noun allows for a null possessor interpretation. (4a) cannot mean that \textit{Maria} talks to one of her mothers.

\ea%4
    EP\label{ex:wein:4}\\
    \ea 
    \gll A Maria conversa com uma mãe.\\
         the M. talks with a mother\\
    \glt ‘Maria talks with a mother.’
    \ex  
    \gll A Maria conversa com uma amiga.\\
         the M. talks with a friend\\
    \glt ‘Maria talks with a friend.’
    \z
\z

(4a) could be uttered in the following context: Maria is a teacher, and at a reunion with the pupils’ parents she is talking to the mother of a pupil. In this context, the DP \textit{uma mãe} is interpreted as a non-relational noun with a meaning such as ‘female parent’ or ‘female legal guardian’. A similar effect has been observed for proper names in \citet{Longobardi1994}. In (4b), the relational interpretation is not affected by the indefinite determiner. In \citet{Löbner2011}, this contrast is related to the interaction between concept type (relational, functional), possessor specification and definite/indefinite determination. Nouns like \textit{father}, \textit{mother} or w\textit{eather} are functional in the sense that they are inherently relational and inherently unique. Functional nouns are affected by indefinite determination in that it neutralizes their inherent uniqueness. Relational nouns are inherently relational, but inherently non-unique and are thus not affected by indefinite determination. But the way the possessor is realized plays an important role.\footnote{\citet{Löbner2011} additionally assumes that the possessor argument is existentially saturated in order to shift a functional to a sortal concept as in (4a).} If the possessor is overt (simple or complex), as in (5a) and (5b), a relational reading is available.\footnote{In EP possessives are prenominal with a definite article and postnominal with indefinite article. Details on the placement are introduced in section 4.1.}

\ea%5
    EP\label{ex:wein:5}\\
    \ea
    \gll O João\textsubscript{i} conversou com uma mãe sua\textsubscript{i} .  \\
         the J.   talked   with a mother his\\
    \ex  
    \gll O João\textsubscript{i} conversou com uma mãe dele\textsubscript{i} \\
         the J. talked    with a mother of.him \\
    \glt ‘João talked with one of his mothers.’ 
    \z
\z

The sentences are perfectly acceptable if \textit{João} grew up in a patchwork family. Thus, the interpretation of a relational noun is affected by determination and by the way a possessor argument is realized.

Null possessives have three properties: (i) they are lexically determined arguments. (ii) they affect the overall reference of the DP together with concept type and determination and (iii) they are interpreted in two ways: anaphorically and indexically.

In previous work, the phenomenon of null possessives has been observed for European Portuguese by \citet[350, Footnote 30]{Mateus2003}, and for Brazilian Portuguese (BP) by \citet{Floripi2003}, \citeauthor{Floripi2009} (\citeyear{Floripi2009}; henceforth F\& N), and \citet{Rodrigues2010}. To my knowledge, the referential properties of null possessors have not been investigated for European Portuguese, at least in published work.\footnote{It was pointed out to me at the workshop on Non-Local Dependencies in the Nominal and Verbal Domain (FCSH – Universidade Nova de Lisboa, 13 \citealt{November2015}) by Ana Maria Martíns and João Costa that Ana Maria Brito had given a talk on null possessives in EP, but an abstract or handout is not available.} The analysis of F\& N and \citet{Rodrigues2010} is taken as a starting point for the investigation of EP null possessives.

\subsection{A movement-based analysis: \citet{Floripi2009} and \citet{Rodrigues2010}}% 1.1. 

F\& N and \citet{Rodrigues2010} present a movement-based analysis of null possessive elements for (Colloquial) Brazilian Portuguese. Their analysis is based on the referential properties of null possessive elements with 3\textsuperscript{rd} person, sentence-internal antecedents. An important difference between EP and (colloquial) BP is that BP has lost the 3\textsuperscript{rd} person simple possessive. The possessive form \textit{seu} in (6a) is 2\textsuperscript{nd} person and exclusively refers to the addressee.

\ea%6
    BP\label{ex:wein:6}\\
    \ea
    \gll O João\textit{\textsubscript{i}} conversou com o   seu\textit{\textsubscript{[*i,j]}} pai.\\
         the J.   talked   with the his   father\\
    \ex  
    \gll O João conversou com o pai.\\
         the J. talked with the father\\
    \ex  
    \gll O João\textsubscript{i} conversou com o pai dele\textsubscript{i}.\\
         the J.   talked with the father of.him\\
    \z
\z

In order to refer to the third person, both the null possessive and the complex possessive are used, as shown in (6b) and (6c). F\& N and \citet{Rodrigues2010} state that (6b) and (6c) do not differ with respect to interpretation or markedness.\footnote{They conclude that the null possessive is not subject to the Avoid Pronoun Principle. The principle is formulated in \citet{Chomsky1981} and just says “Avoid Pronoun”. For example, in structures like (i) the overt and covert pronoun are equally possible, but they differ with respect to their interpretation. The covert pronoun is interpreted as coreferential with \textit{John} and the overt pronoun is interpreted as disjoint, at least if it is unstressed. (i) John would much prefer [his/PRO going to the movie]. \citep[65]{Chomsky1981}\citet[65]{Chomsky1981} also mentions that this principle may be a conversational principle like “Do not say more than is necessary”, either a principle of deletion-up-to-recoverability or a principle of grammar.} But the complex and the null possessive differ with respect to their referential properties: the former shows anaphoric properties and the latter pronominal properties. F\& N use the terms ‘anaphoric’ and ‘pronominal’ in the sense of the classical binding theory of \citet{Chomsky1981,Chomsky1986}. Whether a lexical item qualifies as an anaphor or as a pronominal can be determined by a set of structural and interpretative diagnostics. The structural diagnostics are diagnostics for locality and c-command. An anaphoric element has to be bound by a c-commanding antecedent in a local domain.\footnote{$\alpha $ binds $\beta $ iff (i) $\alpha $ and $\beta $ are co-indexed and (ii) $\alpha $ c-commands $\beta $ (cf. \citealt{Chomsky1981}: 184).} IP/TP and DP have been detected as local domains for binding of anaphoric elements. If the antecedent may be non-local, outside the TP or DP, the nominal element qualifies as pronominal, and it is free to corefer with some salient antecedent. The c-command requirement, explicit in the definition of binding, states that an anaphoric/bound element must be c-commanded by its antecedent. A pronominal element is free in reference and can thus take a non-c-commanding antecedent.

The interpretative diagnostics distinguish a bound or free (co-referential) reading of a nominal element. There are two contexts that help to detect this difference: (i) in VP ellipsis and (ii) in \textit{only}{}-contexts. Anaphors only allow for a bound reading and pronominal elements also permit a coreferential reading. Furthermore, the availability of split antecedents tests whether a pronoun can pick out a referent in discourse.

F\& N show that a null possessive has anaphoric properties in a position from which movement is possible. If it is in a position that disallows movement, the possessive exhibits pronominal properties. F\& N and \citet{Rodrigues2010} follow the approaches of \citet{Hornstein2001,Hornstein2007} and Boeckx, \citet{Hornstein2010}, who derive anaphoric dependencies, including obligatory control, by movement. F\& N and Rodrigues assume that a null possessive contained in an object DP, a position from which a DP may be moved, is a copy of the moved DP. The anaphoric properties are the effect of this movement operation. In case the null possessive is contained in a subject DP (a position from which movement cannot take place), it shows pronominal properties, and F\& N assume a kind of last resort pronominalization, as proposed in \citet{Hornstein2001,Hornstein2007}. The movement analysis of null possessives is illustrated in Figures 1 and 2 for the sentence in (6b).\footnote{This analysis diverges from the following basic Minimalist conceptions of \citet{Chomsky1995,Chomsky2000,Chomsky2001,Chomsky2004}: (i) movement into theta-positions is allowed, (ii) a DP may bear more than one theta-role and (iii) theta-roles may be assigned or discharged after movement.}\textsuperscript{} The relational noun is base-generated together with its possessor argument, as shown in \figref{fig:wein:1}.

 
%%please move the includegraphics inside the {figure} environment
%%\includegraphics[width=\textwidth]{OGSVolumeAug2018Weingart-img1.jpg}
\begin{figure}
\caption{Base-generation of a relational noun and its possessor argument\label{fig:wein:1}}
\begin{forest}
[DP [D°[o]] [NP [NP\\pai\\{[θ-role]}\\{[uCase]}] [DP [o Jo\~ao\\{[θ-role]}\\{[uCase]}]]]]
\end{forest}
\todo[inline]{please check tree}
\end{figure}

After \textit{v}° has merged, the possessor DP \textit{o João} moves to Spec\textit{v}P, the position of subject/agent DPs, as shown in \figref{fig:wein:2}.

\begin{figure}
\caption{Movement of the possessor argument to the subject position\label{fig:wein:2}}
\begin{forest}
[TP
    [T°,name=T] [vP
        [DP\\o Jo\~ao\\{[\textsc{agent}]}\\{[vCase]},name=vcase]
        [v' 
            [v°,name=v] [VP
                [{NODE MISSING}]
            ]
        ]
    ]   
]
\end{forest}
\end{figure}

The DP and its copy fulfil two distinct roles: the DP is the subject of the sentence and the copy is the possessor argument. The movement analysis of anaphoric dependencies is one branch of recent Minimalist approaches to binding theory. The aim of such approaches is to derive the interpretation of nominal elements by means of their lexical properties (features), by principles of the computational system (Narrow Syntax) and from interface conditions. In the following section, the following three interpretative options are briefly introduced: mechanisms of syntactic encoding, semantic binding and coreference.

\subsection{Theoretical background}% 1.2. 

The approaches of \citet{Hornstein2001,Hornstein2007}, Boeckx, \citet{Hornstein2010}, \citet{Zwart2002} and \citet{Kayne2002} aim at deriving the interpretative and structural dependencies of anaphora (condition A/B of classical Binding Theory) by movement of the antecedent. Other approaches derive anaphoric dependencies by the operation Agree. For example, \citet{Hicks2009} assumes that an anaphoric relation is established via (upward) agreement of semantico-syntactic features. The distinction between anaphors and pronominals is encoded by referential features. An anaphor has an unvalued feature that is valued during the derivation via upward Agree with its antecedent. A pronominal enters the derivation with a valued feature, which induces a free variable interpretation. Locality restrictions on anaphoric dependencies are derived by restrictions on the operation Agree and by phases. \citet{Reinhart2006} and \citet{Reuland2011}, based on previous work (e.g., \citealt{Reinhart1983}; \citealt{Reinhart1993}; 1995; \citealt{Reuland2001}) present a predicate-based account of bound anaphora. Anaphoric dependencies between co-arguments, which create a reflexive (syntactic) predicate, are encoded in syntax by formation of a chain. A chain is formed by several agreement steps. Whether a pronoun can be part of a chain depends on its feature composition.

The main focus of all these accounts is on anaphors and pronominals as arguments of verbal predicates. The behaviour of anaphors and pronominals as arguments of a nominal predicate was already puzzling from the point of view of GB binding theory. In nominal contexts, in particular inside so-called picture nouns, anaphors and pronominals ‘misbehave’, in the sense that anaphors can take a non-local antecedent and pronominals can corefer with a local antecedent. This is exemplified in (7) and (8). 

\ea%7
    \citep[166–167]{Chomsky1986}\label{ex:wein:7}
\ea  They\textsubscript{i} heard [stories about each other\textsubscript{i}\slash them\textsubscript{i}].
\ex  They heard [PRO\textsubscript{i} stories about each other\textsubscript{i} ].
\ex  They heard [PRO\textsubscript{k} stories about them\textsubscript{i} ].
    \z
\z

\ea%8
    \label{ex:wein:8}
    \ea They\textsubscript{i} told [stories about each other\textsubscript{i}\slash *them\textsubscript{i}].
    \ex They heard [ PRO\textsubscript{i} stories about each other\textsubscript{i} /them\textsubscript{*i}].
    \z
\z

In order to explain the difference between (7) and (8), \citet{Chomsky1986} assumed the presence of a covert nominal argument, PRO, as subject of the NP. The presence of a nominal subject is taken to be essential for the definition of a binding domain. Furthermore, co-indexing of PRO with the subject of the sentence is taken to be a lexical property of the verb. The verb obligatorily or optionally controls the subject of the NP. In the experimental studies of \citet{Runner2005}, it was shown that the presence or absence of a subject is not decisive for the interpretation of anaphors with picture nouns. In (9) a non-local anaphor is allowed despite the presence of a subject, but the local pronominal is still excluded.

\ea%9
    Runner \& \citet[597]{Kaiser2005}\label{ex:wein:9}\\
    \ea Ebenezer\textsubscript{i} saw Jacob\textsubscript{k}´s picture of himself\textsubscript{i/k}.
    \ex Ebenezer\textsubscript{i} saw Jacob\textsubscript{k}´s picture of him\textsubscript{i/*k/m}.
    \z
\z    

The non-local anaphor in (9a) is labeled an \textit{exempt anaphor} or \textit{logophor}; cf. \citet{Reuland2011}.\footnote{An anaphor is exempt (from Binding Condition A) if c-command by the antecedent is not required and if a reflexive pronoun is not in complementary distribution with a personal pronoun (cf. Büring 2005).}\textsuperscript{,}\footnote{The term ‘logophor’ in its narrow sense is used for pronouns that refer to an individual whose viewpoint, words or thoughts are being reported; cf. \citet{Speas2004}. \citet{Reuland2011} uses the term in a broader sense for (morphological) anaphors that have pronominal-like referential properties in certain environments.}  Inside an NP, anaphors can be used like pronominals. Any approach that assumes a (blind) syntactic encoding of anaphoric dependencies has to explain why anaphors have different interpretative properties inside the nominal domain and why syntactic encoding is blocked. What is important here is that the interpretative dependencies of exempt anaphora can be established by semantic binding and pragmatic coreference. \citet{Reinhart2006} and \citet{Reuland2011} discuss the competition between semantic binding and coreference. Semantic binding is restricted by sentence-internal structural conditions. The definition is given in (10). 

\ea%10
    \label{ex:wein:10}
    A-binding: logical-syntax based definition \citep[171]{Reinhart2006}\\
    $\alpha $ A-binds $\beta $ iff $\alpha $ is the sister of a $\lambda $-predicate whose operator binds $\beta $.
\z

If a pronominal (a semantic variable) is not bound, it remains free and gets assigned a value from discourse. Coreference is taken to be determined by discourse principles. \citet{Reuland2011} argues that the encoding of referential dependencies follows a kind of economy hierarchy: syntactic encoding is more economical than variable binding, and variable binding is more economical than a free, discourse-based interpretation. \citet{Reuland2011}, following the work of \citet{Reinhart1983}, assumes that sentence-internal coreference should be blocked if variable binding is possible and both methods yield an identical interpretation. \citet{Reinhart2006} revisits her older proposal, based on different economy considerations. She proposes a C-I interface condition that restricts sentence-internal coreference as follows.

\ea%11
    Rule I \citep[185]{Reinhart2006}\\
$\alpha $ and $\beta $ can not be covalued in a derivation D, if\\\label{ex:wein:11}
    \ea $\alpha $ is in a configuration to A-bind $\beta $, and
    \ex $\alpha $ cannot A-bind $\beta $ in D, and
    \ex the covaluation interpretation is indistinguishable from what would be obtained if $\alpha $ A-binds $\beta $.
    \z
\z

The definition works as follows. In (12) a covaluation interpretation is allowed, because clause (a) of Rule I does not hold. The possessive pronoun is not in a configuration to bind the DP \textit{Max,} because \textit{Max} is not a variable that can be bound.

\ea%12
    \citep[186]{Reinhart2006}\\\label{ex:wein:12}
    His mother loves Max.\\\relax
    [His mother] ( $\lambda $x (x loves y)\slash his = Max\\
\z

In (13), both bound and covaluation interpretations are allowed. The DP \textit{Max} is in a configuration to bind the possessor. Furthermore, binding is possible between the DP and the possessive. Therefore Rule I does not hold and a coreferent interpretation is not blocked.

\ea%13
    \citep[186]{Reinhart2006}\\\label{ex:wein:13}
    Max loves his mother.\\
    Max ( $\lambda $x (x loves y’s mother)\slash y = Max\\
\z

An import aspect of Reinhart’s proposal is that the interface with the interpretative component operates on PF structures. In her words, an economy principle should state something like “minimize interpretative options of a given PF” \citep[103]{Reinhart2006}. English has just one possessive pronoun, but EP has two ways to realize a possessive, and presumably also a third, covert, possessive, which is restricted to relational nouns. The theories mentioned above offer the following options for anaphora resolution. Anaphors are either bound by syntactic means or they are logophoric. Logophoricity in the sense of \citet{Reuland2011} includes bound and coreferential readings. Pronominals are either bound semantically or they are coreferential.

\subsection{Aims and structure of the article}% 1.3. 

The article aims at investigating the referential properties of EP null possessives. Based on these referential properties, conclusions can be drawn about the nature of null possessives and the way they are interpreted. F\& N propose a syntactic encoding for BP null possessives. In section 2, the diagnostics for the referential properties of null possessives in BP, as presented in F\& N and in \citet{Rodrigues2010}, are compared to those in EP. The individual diagnostics will be discussed, and it will be shown that null possessives in EP are not subject to structural conditions like locality and c-command. Although they do not obey the structural conditions attributed to anaphoric elements, they are (partially) anaphoric from an interpretative point of view. It will be concluded that the referential dependency between a null possessive and its antecedent cannot be derived by means of syntactic operations like movement or Agree.

If EP null possessives are not copies of a moved NP, what kind of element are they and how are their referential properties to be explained? In section 3, additional diagnostics are presented that support the idea that null possessor arguments are realized as possessive \textit{pro}. The idea is based on a comparison of the referential properties of simple and complex possessives with the properties of null possessives. The results of the comparison are somewhat puzzling, because the possessive elements do not fit well into any particular category, nor do the interpretative options account for their behaviour. It will be concluded that null possessives can be classified as possessive logophors.

Given the assumption that null possessive \textit{pro} may exist in the grammar of EP, in section 4 a semantically motivated syntactic account of the restriction to relational nouns is proposed. The main claim of this analysis is that relational and non-relational nouns have a different internal syntax. The former select a possessive as their external argument in Spec\textit{n}P, while the latter combine with a PossP. The conclusion summarizes the findings of the article.

\section{Referential properties of null possessives in Brazilian and European Portuguese}% 2. 

The referential properties of null possessives are determined by a set of structural and interpretative diagnostics. These diagnostics are presented for BP null possessives and discussed for the corresponding data in EP. In order to avoid reference to a specific analysis, the notation \textit{$\emptyset$-poss} will be used to symbolise a null possessive element. No assumptions about its status as a syntactic object or about its position are made in this section.  

In the next subsections the following diagnostics are presented. The diagnostics for locality and c-command are presented in 2.1 and 2.2. The interpretative diagnostics in 2.3 and 2.4 distinguish between a bound and a free reading of a pronominal element contained in a VP-ellipsis site and in a sentence with the exclusive particle \textit{only}. The split-antecedent diagnostic is presented in 2.5.

\subsection{Locality}% 2.1. 

The locality requirement is illustrated by the examples in (14). In (14a) the relational noun containing the null possessive is in object position. In principle, there are two possible antecedents: the subject of the embedded clause, \textit{o André,} and the subject of the matrix clause, \textit{a Marcela.} But only the DP \textit{o André} may be interpreted as the possessor. The DP \textit{a Marcela} is outside the local domain (the embedded TP) and does not qualify as the antecedent for the null possessive. 

\ea%14
    BP (F\& N: 42, 45)\label{ex:wein:14}\\
    \ea
    \gll A Marcela\textsubscript{i} disse que [\textsubscript{TP} o André\textsubscript{k} ligou para o $\emptyset$-poss\textsubscript{[}\textsubscript{*i/k]} amigo].\\
         the M. said that {} the A. called to the $\emptyset$\textsc{.poss} friend\\
    \glt ‘Marcela said that André called his friend.’
    \ex  
    \gll A Marcela\textsubscript{i} acha que [o João\textsubscript{k} disse que [\textsubscript{TP} o $\emptyset$-poss\textsubscript{i/k} irmão vai viajar]].\\
         the M. thinks that {} the J. said that the $\emptyset$\textsc{.poss} brother goes travel\\
    \glt ‘Marcela thinks that João said that his brother is going to travel.’
    \z
\z

In (14b), the relational noun is the subject of the embedded clause and the null possessive may take also a non-local DP as antecedent. F\& N argue that the local dependency in (14a) falls out from movement of the DP \textit{o André}. In (14b) the relational noun is in a position from which movement is not licit in BP. Hence, the null possessive is realized as (last resort) little \textit{pro} allowing for a non-local referential dependency. As mentioned above, such a subject-object asymmetry is not found in European Portuguese, as shown by the examples in (15a) and (15b).

\ea%15
    EP\label{ex:wein:15}\\
    \ea
    \gll A Marcela\textsubscript{i} disse que [\textsubscript{TP} o André\textsubscript{k} ligou para o $\emptyset$-poss\textsubscript{[}\textsubscript{i/k]} amigo].\\
         the M. said that {} the A. called to the $\emptyset$\textsc{.poss} friend\\
    \glt ‘Marcela said that André called his/her friend.’
    \ex  
    \gll A Marcela\textsubscript{i} acha que [o João\textsubscript{k} disse que [\textsubscript{TP} o $\emptyset$-poss\textsubscript{[}\textsubscript{?i/k]} irmão vai viajar]].\\
         the M. thinks that {} the J. said that the $\emptyset$\textsc{.poss} brother goes travel\\
    \glt ‘Marcela thinks that João said that his/her brother is going to travel.’
    \z
\z

In (15a) the null possessive is in object position and it may take both the local and non-local DP as antecedent. The same is true for (15b). Both DPs are possible antecedents for the null possessive.

\subsection{C-command requirement}% 2.2. 

In the sentences in (16) there are again two possible antecedents for the null possessive: the DP \textit{o amigo} and the DP \textit{o João}. In (16a), only the DP \textit{o amigo} is accepted as antecedent. The embedded DP \textit{o João} cannot be the antecedent because it fails to c-command the null possessive.

\ea%16
    BP (F\& N: 42, 50)\label{ex:wein:16}\\
    \ea
    \gll\relax [O amigo [d[o João]\textsubscript{i}]]\textsubscript{k} telefonou para [a $\emptyset$-poss\textsubscript{[}\textsubscript{*i/k]} mãe].\\
         the friend of.the J. called to the $\emptyset$\textsc{.poss} mother\\
    \glt ‘João’s friend called his mother.’
    \ex  
    \gll\relax [O namorado d[a Maria]\textsubscript{i}]\textsubscript{k} saiu quando [um $\emptyset$-poss\textsubscript{[i/}\textsubscript{k]} parente] entrou.\\
         the boyfriend of.the M. left when a $\emptyset$\textsc{.poss} relative entered\\
    \glt ‘Maria’s boyfriend left when a relative of hers/his came in.’
    \z
\z

In case the null possessive is contained in a DP in subject position, as in (16b), it does not need to be c-commanded. Both DPs, \textit{o namorado} and \textit{a Maria}, can function as antecedent of the null possessor. Contrary to BP, the subject-object asymmetry is again not found in EP. The examples in (17a) and (17b) show that a non-c-commanding DP cannot be the antecedent of the null possessive, irrespective of the position of the relational noun.

\ea%17
    EP\label{ex:wein:17}\\
    \ea
    \gll [O amigo d[o João]\textsubscript{i}]\textsubscript{k} telefonou para [a $\emptyset$-poss\textsubscript{[*}\textsubscript{i/k]} mãe].\\
         the friend of.the J. called to the mother\\
    \glt ‘João’s friend called his mother.’
    \ex  
    \gll [A mãe d[a Maria]\textsubscript{i}]\textsubscript{k} saiu quando [um/o $\emptyset$-poss\textsubscript{[*i/k]} amigo] entrou. \\
         the mother of.the M. left when the friend entered \\
    \glt ‘Maria’s mother left when her friend came in.’ 
    \z
\z

In EP, null possessors in both positions obey the c-command requirement, at least with embedded DPs. In section 3.1, the c-command requirement will be discussed in more detail and it will be argued that c-command is not the relevant condition for ruling out co-reference between an embedded DP and a null possessor argument.

\subsection{Sloppy and strict identity under ellipsis}% 2.3. 

It was observed by \citet{Ross1967,Ross1969} that a pronoun inside an elided VP may have two readings. These are exemplified for the sentence in (18). The strict identity reading is shown in (18b) and the sloppy identity reading in (18c).

\ea%18
\citep[207]{Ross1967}\label{ex:wein:18}\\
\ea John scratched his arm and Mary did so, too.
\ex Strict identity\\Mary scratched his (= John’s) arm.
\ex Sloppy identity\\Mary scratched her arm.
\z
\z

Since \citet{Sag1980}, the ambiguous interpretation of the pronoun has been attributed to the possibility of two distinct LF representations. The strict reading is the result of a coreferential or free variable interpretation and the sloppy reading is the result of a bound variable interpretation. It has been observed that reflexive pronouns (in complement position) are interpreted as bound variables in these contexts; cf. \citet{Sag1980} and \citet{Hicks2009}. Thus, the restriction to a sloppy reading is taken to be an anaphoric referential property. This is also true for EP, as shown in (19).

\ea%19
    EP\label{ex:wein:19}\\
    \ea \gll A Maria ama-se a si própria e o Rui também.\\
         the M. loves.\textsc{se.cl} to herself and the R. also does\\
    \glt ‘Maria loves herself and Rui does so, too.’
    \ex Strict identity\\
        *Rui loves Maria.\\
    \ex Sloppy identity\\
        Rui loves himself.
    \z
\z
          
Applying this test to null possessives, F\& N show that null possessives are restricted to a sloppy reading, but only if the relational noun is in object position, as in (20). But both readings are available if the relational noun is in subject position (a position from which movement cannot take place), as in (21). Once again, in EP there is no subject-object asymmetry. The strict reading is not available either in object or in subject position.

\ea%20
         EP\slash BP (F\& N: 44 for BP)\label{ex:wein:20}
    \ea
    \gll O João vai telefonar para a mãe e a Marcela também vai.\\
         the J. goes call to the mother and the M. also goes  \\
    \glt ‘João will call his mother and Marcela will do so, too.’
    \ex Strict identity: *BP\slash *EP\\Marcela will call João’s mother.
    \ex Sloppy identity: BP\slash EP\\Marcela will call her mother.
    \z
\z

\ea%21
         EP\slash BP (F\& N: 51 for BP)\label{ex:wein:21}
    \ea
    \gll A Maria vai recomendar a pessoa que um amigo entrevistou e o João também vai.\\
         the M. goes recommend the person that a friend interviewed and the J. also goes\\
    \glt ‘Maria is going to recommend the person that a friend of hers interviewed and João will do so, too.’
    \ex Strict identity: BP\slash *EP\\João is going to recommend the person that a friend of Maria’s interviewed.\\
    \ex Sloppy identity: BP\slash EP\\João is going to recommend the person that a friend of his interviewed. 
    \z
\z

The diagnostic shows that in EP a null possessive is interpreted as a bound variable in both positions.

\subsection{\textit{Only}-contexts}% 2.4. 

The same opposition between a bound and a free/co-referential reading of a pronoun is found in contexts in which the antecedent is modified by the exclusive particle \textit{only.} The interpretation of pronouns in this context is discussed in \citegen{Horn1969} analysis of \textit{only}. In his account, the terms \textit{presupposition} and \textit{assertion} are terms of pragmatics (cf. \citealt{Pagin2016}). From a semantic perspective, Horn’s assertion corresponds to the notion \textit{entailment}. Irrespective of the perspective, the sentence in (22a) (pragmatically) presupposes (22b) and asserts or entails (22c). The examples are represented in the notation of \citet{Horn1969}. 

\ea%22
    \citet[98–99]{Horn1969}\label{ex:wein:22}\\
    \ea 
    Only Muriel voted for Hubert.\\
    vote (m, h)\\   
    \glt ‘Muriel voted for Hubert.’
    \ex  ¬(∃y) (y ${\neq}$ m ${\wedge}$ vote (y, h))
    \glt ‘Nobody else voted for Hubert.’
    \z
\z

If the sentence contains a pronoun, as in (23a), there are two distinct assertions, depending on the interpretation of the pronoun. The entailment in (23b) contains a pronoun translated into a free/co-referential variable and the entailment in (23c) contains a bound variable.

\ea%23
    \citet[98–99]{Horn1969}\label{ex:wein:23}\\
    \ea Only Muriel voted for her brother.
    \ex  ¬(${\exists}$y) [ y voted for m’s brother] (y ${\neq}$ m)  
    \glt Nobody else voted for Muriel’s brother.
    \ex ¬(${\exists}$y) [ y voted for y’s brother]
    \glt Nobody else voted for his own brother.
    \z
\z

\citet[102]{Horn1969} accepts only the bound reading of (23c), but Boeckx, Hornstein \& \citet[197]{Nunes2010} accept both the bound and co-referential readings of the possessive pronoun. With respect to the interpretation of null possessives in BP, there is once again an asymmetry between object and subject position. And once again there is no such asymmetry in EP. The judgements for null possessives in object position are given in (24) and for null possessives in subject position in (25).

\ea%24
    EP\slash BP (F\& N: 44 for BP)\label{ex:wein:24}\\
    \ea 
    \gll Só o João ligou para a $\emptyset$-poss mãe.\\
         only the J. called to the $\emptyset$\textsc{.poss} mother\\
    \glt ‘Only João called his mother.’
    \ex Bound reading: BP\slash EP\\Nobody else called his own mother.
    
    \ex Co-referent reading: *BP\slash *EP\\Nobody else called João’s mother.
    \z
\z

In (25), the relational noun is in subject position and both readings are available in BP, but not in EP. In EP the empty possessor can only receive a bound interpretation.

\ea%25
    EP\slash BP (F\& N: 52 for BP)\label{ex:wein:25}\\
    \ea 
    \gll Só o João leu o livro que [a $\emptyset$-poss mãe] indicou.\\
         only the J. read the book that the $\emptyset$\textsc{.poss} mother recommended\\
    \glt ‘Only João read the book that his mother recommended.’
    \ex Bound reading: BP\slash EP\\Nobody else read the book his own mother recommended.
    \ex Co-referent reading: BP\slash *EP\\Nobody else read the book João’s mother recommended.
    \z
\z

In EP, there is no difference in the interpretation of the null possessive with respect to its position inside an object or subject DP. In both positions only the bound reading is acceptable.

\subsection{Split antecedents}% 2.5. 

\citet{Rodrigues2010} provides a diagnostic testing for so-called split antecedents. This diagnostic was first introduced by \citet{Lebeaux1985} for locally and non-locally bound reflexives. In (26a) the reflexive is inside a \textit{picture}{}-NP (an exempt position) and may take the subject and object of the main clause as a plural antecedent. In this position the reflexive is free to pick out a plural referent. But if the reflexive is in a local configuration with the subject and the object, as in (26b), split antecedents are not acceptable; it has to be bound by a unique antecedent.

\ea%26
    \citep[346]{Lebeaux1985} [indices by AW]  \label{ex:wein:26}\\
    \ea[]{John\textsubscript{i} told Mary\textsubscript{k} that there were some pictures of themselves\textsubscript{[i+k]} inside.}
    \ex[*]{John\textsubscript{i} told Mary\textsubscript{k} about themselves\textsubscript{[i+k]}.}
    \z
\z

In BP, null possessives cannot take split antecedents, as shown in (27), which corroborates the movement analysis: if the null possessive is the copy of the antecedent, two independent DPs cannot be its antecedent. Also in EP, null possessives do not allow for split antecedents. But as shown above, the structural requirements for a movement analysis are not met.

\ea[*]{%27
    EP\slash BP \citep[130]{Rodrigues2010}\label{ex:wein:27}\\
    \gll A Maria\textsubscript{i} disse que o Paulo\textsubscript{k} encontrou o $\emptyset$-poss\textsubscript{[i+k ]} amigo.\\
         the M. said that the P. met the $\emptyset$\textsc{.poss} friend  \\
    \glt Intended meaning: ‘Maria said that Paulo met their friend.’}
    \z

It is worth mentioning that EP differs from English with respect to this diagnostic. Even if a reflexive pronoun occurs in an exempt position, it may not take split antecedents. Only (personal) pronouns can do so. The EP examples are given in (28a) and (28b), respectively.

\ea%28
    EP\label{ex:wein:28}\\
    \ea[*]{
    \gll O Rui\textsubscript{i} contou à Maria\textsubscript{k} que algumas fotos de si próprios\textsubscript{[i+k]} estão á venda.\\
         the R.   told to.the M. that some photos of themselves are for sale\\}
    \ex[]{
    \gll O Rui\textsubscript{i} contou à Maria\textsubscript{k} que algumas fotos deles\textsubscript{[i+k]} estão á venda.\\
         the R.   told to.the M. that some photos of.them are for sale\\
    \glt ‘Rui told Mary that some photos of themselves are for sale.’}
    \z
\z

The diagnostic has to be evaluated differently for EP. It seems that this result is better related to the feature composition of the nominal elements. In English, the third person reflexive pronoun is composed of \textit{them} + \textit{selves}, and the pronominal part (\textit{them}) overtly realizes a referential plural feature. In EP, the pronominal form \textit{si} does not overtly realize either referential number or gender. Although these features are present as concord features on the intensifing adjective \textit{próprios,} they are not referential in the sense that they restrict the set of possible referents. The pronoun \textit{ele} in (28b) is marked for referential number and gender, just like the English reflexive pronoun, and both are capable of taking split antecedents.

I will return to this diagnostic in section 3.4.4, showing that the (3\textsuperscript{rd} person) simple possessive also disallows split antecedents and has a similar feature composition to the 3\textsuperscript{rd} person reflexive pronoun: it does not overtly realize number and gender features. The diagnostic shows that reflexive pronouns and possessive elements behave alike, not because they belong to the class of anaphoric elements, but because they are defective with respect to the same (referential) features.

\subsection{Preliminary conclusion}% 2.6. 

The interpretative and structural diagnostics have shown that the subject-object asymmetry of BP null possessives is not present in EP. The results are summarized in \tabref{tab:wein:1} below.

\begin{table}
\begin{tabularx}{\textwidth}{lXQl}
\lsptoprule
{Diagnostic} & Position of null possessive & \multicolumn{2}{c}{{Language}}\\\cmidrule(lr){3-4}
&  & BP & EP\\\midrule
Local domain & object & yes & no\\
             & subject & no & no\\
C-command    & object & yes & yes\\
             & subject & no & yes\\
Reading under ellipsis & object & sloppy only & sloppy only\\
                       & subject & sloppy and strict & sloppy only\\
Reading in \textit{only}-contexts & object & bound only & bound only\\
                                    & subject & bound and co-referential & bound only\\
Split antecedents &  & no & no\\
\lspbottomrule
\end{tabularx}
\caption{Summary of the structural and referential properties of null possessives in BP and EP}
\label{tab:wein:1}
\end{table}

The interpretative diagnostics clearly show that null possessives in EP are interpreted as anaphors or bound variables. Given the lack of locality restrictions, the referential dependency between a null possessive and its antecedent cannot be derived by a syntactic operation such as movement as in F\& N and \citet{Rodrigues2010} or by Agree as in \citet{Hicks2009}. As for the structural diagnostics, EP null possessives are non-local but subject to c-command. \citet{Lebeaux1985} has shown that anaphoric elements that allow for a non-local antecedent also do not require a c-commanding antecedent. From this perspective, the results for EP are contradictory; I will return to the c-command requirement in the next section.

\section{Additional diagnostics and comparison with the referential properties of simple and complex possessives}% 3. 

In this section, the referential properties of null possessives are compared with the referential properties of simple and complex possessives. In subsection 3.1, the c-command requirement is discussed in more detail. In subsections 3.2 and 3.3, the diagnostics of quantifier binding and sentence-external antecedents are introduced. In subsection 3.4, the structural and interpretative diagnostics of section 2 are applied to simple and complex possessives. The results are summarized in 3.5.

\subsection{C-command revisited}% 3.1. 

The sentences in (17a) and (17b), repeated here as (29), show that the null possessive needs a c-commanding antecedent.

\ea%29
    EP\label{ex:wein:29}\\
    \ea
    \gll \relax[O amigo d[o João]\textsubscript{i}]\textsubscript{k} telefonou para [a $\emptyset$-poss\textsubscript{k/*i} mãe].\\
         the friend of.the J. called   to the $\emptyset$\textsc{.poss} mother\\
    \glt ‘João’s friend called his mother.’
    \ex  
    \gll \relax [A mãe d[a Maria]\textsubscript{i}]\textsubscript{k} saiu quando [o $\emptyset$-poss\textsubscript{*i/k} amigo] entrou.\\
         the mother of.the M. left when the $\emptyset$\textsc{.poss} friend entered\\
    \glt ‘Maria’s mother left when her friend came in.’
    \z
\z    


The simple possessive also needs a c-commanding antecedent, as shown in (30).

\ea%30
    EP\label{ex:wein:30}\\
    \ea
    \gll \relax[O amigo d[o João]\textsubscript{i}]\textsubscript{k} telefonou para a   sua\textsubscript{[*i/k]} mãe].\\
         the friend of.the J. called   to the his mother\\
    \glt ‘João’s friend called his mother.’
    \ex  
    \gll \relax[A mãe d[a Maria]\textsubscript{i}]\textsubscript{k} saiu quando [o seu\textsubscript{[*i/k]} amigo] entrou.\\
         the mother of.the M. left when the her friend entered\\
    \glt ‘Maria’s mother left when her friend came in.’
    \z
\z

Only the complex possessive allows for both interpretations. The examples are given in (31). In fact, there is even a preference to interpret the embedded DP \textit{o João} as antecedent of the pronoun \textit{ele.}

\ea%31
         EP\label{ex:wein:31}\\
    \ea  
    \gll \relax [O amigo d[o João]\textsubscript{i}]\textsubscript{k} telefonou para [a mãe dele\textsubscript{[i/k]}].\\
         the friend of.the J. called to the mother of.him\\
    \glt ‘João’s friend called his mother.’
    \ex  
    \gll \relax [A mãe d[a Maria]\textsubscript{i}]\textsubscript{k} saiu quando [o amigo dela\textsubscript{[i/k]}] entrou.\\
         the mother of.the M. left when the friend of.her entered\\
    \glt ‘Maria’s mother left when her friend came in.’
    \z
\z

Semantic binding accounts for the reading under c-command. According to Rule I, the covalued interpretation between the embedded DP and the possessives should also be possible. But this option is only allowed for the complex possessive. What blocks the covaluation interpretation with null and simple possessives? Are they obligatorily bound, as indicated by the results of the diagnostics of VP ellipsis and \textit{only}{}-contexts in (20)/(21) and (24)/(25), respectively? If this is true, null possessives should be excluded from contexts that only allow a coreferential interpretation, as in the English example (12) above. This prediction is not borne out, as shown by the examples in (32).

\ea%32
         EP\label{ex:wein:32}\\
    \ea  
    \gll Os $\emptyset$-poss\textsubscript{i} filhos não gostam [do João e da Maria]\textsubscript{i}.\\
         the $\emptyset$\textsc{.poss} children not like of.the J. and of.the M.\\
    \glt ‘Their children don’t like João and Maria.’
    \ex 
    \gll O João\textsubscript{i} adorou o presente que a $\emptyset$-poss\textsubscript{[i/k]} amiga deu à Maria\textsubscript{k}.\\
         the J. adored the gift that the $\emptyset$\textsc{.poss} friend gave to.the M.\\
    \glt ‘João adored the gift that a friend of his/hers gave to Maria.’
    \z
\z    

In both examples, the null possessive is interpreted as referring to a DP that does not c-command it at any stage of the derivation: the conjunct [\textit{o João e a Maria}] in (32a) and the DP \textit{a Maria} in (32b). With simple possessives, a covaluation interpretation is not possible, although this should be allowed according to Rule I. In both sentences of (33), coreference between the simple possessive and the non-c-commanding (sentence-internal) antecedent is not accepted.

\ea%33
         EP\label{ex:wein:33}\\
    \ea  
    \gll \relax [Os seus\textsubscript{?-*i} filhos] não gostam [do João e da Maria]\textsubscript{i}.\\
         the their children not like of.the J. and of.the M.\\
    \glt ‘Their children don’t like João and Maria.’
    \ex  
    \gll O João\textsubscript{i} adorou o presente que a sua\textsubscript{[i/*k]} amiga deu à Maria\textsubscript{k}.\\
         the J. adored the gift that the his friend gave to.the M.\\
    \glt ‘João adored the gift that a friend of his gave to Pedro.’
    \z
\z

Interestingly, the complex possessive is also unacceptable in these contexts, as shown in (34).

\ea%34
         EP\label{ex:wein:34}\\
    \ea  
    \gll \relax [Os filhos deles\textsubscript{??i}] não gostam [do João e da Maria]\textsubscript{i}.\\
         the children of.them not like of.the J. and of.the M.\\
    \glt ‘Their children don’t like João and Maria.’
    \ex  
    \gll O João\textsubscript{i} adorou o presente que a amiga dele\textsubscript{[i/*k]} deu ao Pedro\textsubscript{k}.\\
         the J. adored the gift that the friend of.him gave to.the P.\\
    \glt ‘João adored the gift that a friend of his gave to Pedro.’
    \z
\z

These results are quite puzzling, because covaluation should be permitted in these contexts. Furthermore, if a possible antecedent is embedded in an inanimate DP which is not in competition for interpretation as possessor of a kinship noun, c-command does not play a role, as shown in (35).\footnote{Thanks to an anonymous reviewer for pointing out this configuration.}

\ea%35
         EP\label{ex:wein:35}\\
    \ea            
    \gll A falta do respeito da Maria\textsubscript{i} chateia a mãe\textsubscript{?i}\\
         the lack of.the respect of.the Maria subsets the mother\\
    \glt ‘Maria’s lack of respect upsets the mother.’
    \ex  
    \gll A falta do respeito da Maria\textsubscript{i} chateia a sua\textsubscript{i} mãe.\\
         the J. adored the gift that the his friend gave to.the M.\\
    \glt ‘João adored the gift that a friend of his gave to Pedro.’ 
    \z
\z


Thus, for the null and simple possessive, binding is preferred over coreference in case two antecedents are inside the same DP. If binding is not possible, the null possessive permits a coreferential interpretation. As for the simple possessive, it seems that precedence, which is one way to render an antecedent salient, is necessary for its interpretation. This could account for the difference between (33) and (35a). The same is true for the complex possessive. For covaluation of null possessives, the sentence structure seems not to be relevant.

\subsection{Binding by a quantifier}% 3.2. 

Another diagnostic for referential properties is binding by a quantifier, as mentioned in \citet{Barker2011} and \citet{MateusEtAl2003}. The interpretations of null, simple and complex possessives are given in (36), (37) and (38) respectively:

\ea%36
    EP\label{ex:wein:36}\\
    \ea[]{Todos os pais gostam dos filhos.}
    \ex[]{All x (x = parents) x like children of x.}
    \ex[*]{All x (x = parents) x like children of y.}
    \z
\z    

\ea%37
        EP\label{ex:wein:37}\\
    \ea Todos os pais gostam dos seus filhos.
    \ex All x (x = parents) x like children of x.
    \ex All x (x = parents) x like children of y.
    \z  
\z


\ea%38
    EP\label{ex:wein:38}\\
    \ea[]{Cada menino pensa no pai dele.}
    \ex[*]{Every x (x=kid) x thinks about the father of x.}
    \ex[]{Every x (x=kid) x thinks about the father of y.}
    \z
\z

The null possessive must be bound, the simple possessive allows for both a free and a bound reading, and the complex possessive is restricted to a free reading. In \citet[1112]{Barker2011}, the interaction between null possessor arguments and quantifiers was interpreted as evidence that the possessor argument is grammatically present. Furthermore, this diagnostic corroborates the claim that null possessives are only present with relational nouns, but not with non-relational nouns. This is exemplified in (39).

\ea%39
    EP\label{ex:wein:39}\\
    \ea  Bound/possessive reading\\
    \gll Cada menino pensa que o seu bici é fixe.\\
     every kid thinks that the his bike is cool\\
    \ex  No bound/possessive reading\\
    \gll Cada menino pensa que o bici é fixe.\\
     every kid thinks that the bike is cool\\
    \z
\z

The difference in interpretation between simple and complex possessives in (37) and (38) is similar to what have been called ‘Montalbetti’s facts’. \citet{Montalbetti1984} and Alonso-Ovalle \& D’\citet{Introno2001} observed for Spanish that overt and covert pronouns can be interpreted as a free variable, but only the covert pronoun can be bound by a quantifier. For EP, similar facts have been reported in \citet{Lobo2013}. In the case of EP possessives, it is the simple possessive that shows the properties of \textit{pro} and the complex possessive that shows the properties of overt pronouns. As mentioned in section 2.5, and as will be discussed in more detail in 3.4.4, the simple possessive has only a referential person feature, which may explain this difference. But what is the property that explains the obligatory bound reading of null possessives? In this article I will assume that it is the lack of phonetic content, as it is with argumental subject \textit{pro.} This diagnostic is then taken to support the assumption that the null possessor is present in EP syntax as null possessive \textit{pro.}

\subsection{Sentence-external antecedents}% 3.3. 

Pronominals are able to pick out a referent in the discourse context, a sentence-external antecedent. Anaphors lack this ability. For example, reflexive pronouns, even in exempt positions, cannot take a sentence-external antecedent, as discussed in e.g. \citet{Campos1995}. The following examples from the CRPC corpus show that the null possessive is capable of taking a sentence-external antecedent.\footnote{The Reference Corpus of Contemporary Portuguese (CRPC) can be accessed at http://alfclul.clul.ul.pt/CQPweb/.}

\ea%40
    EP   CRPC [last access 08-04-16]\label{ex:wein:40}\\
    “A questão da luta interna do partido é empolada. Os problemas são discutidos nas reuniões do partido e é a decisão da maioria que temos que respeitar”, refere \textbf{Maria João Barradas}, de 26 anos, membro da JCP. O interesse pelo PCP foi prematuro. \textbf{O pai} foi trabalhador na Lisnave e isso marcou a sua infância e adolescência.\\
    \glt ‘“The issue of party-internal conflicts is complicated. The problems are discussed at the party conferences and it is the decision of the majority that we have to respect,” reports Maria João Barradas, 26 years old, member of JCP. Her interest in the party began early. Her father was a worker at Lisnave and this influenced her childhood and adolescence.’
\z

The fact that null possessives can take a sentence-external antecedent seems to contradict the other diagnostics presented so far, because null possessives should not be capable of taking a sentence-external referent. A first approximation to this puzzling result could be along the following lines. The text passage in (40) is about \textit{Maria João Barradas}, and the interpretation of the DP \textit{o pai} as father of Maria is the only possible interpretation. The context does not allow for any other interpretation; thus, the interpretation of \textit{o pai} could be the result of an existentially saturated possessor argument plus a definite determiner. The DP \textit{o pai} would be interpreted as a kind of \textit{definite associative anaphor}.\footnote{The term definite associative anaphora in the sense of \citet{Hawkins1978} describes the interpretation of the definite DP \textit{the battery} in (i).(i)I found a watch under the tent. It was fine except for the battery.The DP \textit{the battery} is understood as belonging to the previously mentioned watch, even if the battery itself has not been explicitly mentioned before. If the watch is mentioned, all of its parts are also in the common ground and can be referred to by a definite DP; cf. \citet{Heim1991}. Similarly, if a person is mentioned, the parents are also part of the common ground.}\textsuperscript{} 

\subsection{Comparison with the referential properties of simple and complex possessives}% 3.4. 

For the sake of completeness, the diagnostics of section 2 are briefly presented for simple and complex possessives.

\subsubsection{Locality}% 3.4.1. 

With respect to locality, both simple and complex possessives may refer to a local or a non-local antecedent, as shown in (41) and (42).

\ea%41
    \label{ex:wein:41}
    \ea
    \gll Marcela\textsubscript{i} disse que o André\textsubscript{k} ligou para o seu\textsubscript{[i/k]} amigo.\\
         the M. said that the A. called to the his friend\\
    \glt ‘Marcela said that André called his friend.’
    \ex  
    \gll A Marcela\textsubscript{i} disse que a Luisa\textsubscript{k} ligou para o amigo dela\textsubscript{[i/k]}.\\
         the M. said that the L. called to the friend of.her\\
    \glt ‘Marcela said that Luisa called her friend.’
    \z
\z



\ea%42
    \label{ex:wein:42}
    \ea
    \gll A Maria\textsubscript{i} acha que o João\textsubscript{k} disse que o seu\textsubscript{[i/k]} amigo vai viajar.\\
         the M. thinks that the J. said that the his friend goes travel\\
    \glt ‘Marcela said that André called his friend.’
    \ex  
    \gll A Maria\textsubscript{i} acha que a Luisa\textsubscript{k} disse que o amigo dela\textsubscript{[i/k]} vai viajar.\\
         the M. thinks that the L. said that the friend of.her goes travel\\
    \glt ‘Maria thinks that Luisa said that her friend is going to travel.’
    \z
\z

\subsubsection{Ellipsis}% 3.4.2. 

The readings under ellipsis are shown in (43) for simple possessives and in (44) for complex possessives. Under ellipsis, the simple possessive only allows for the sloppy reading:

\ea%43
         EP\label{ex:wein:43}\\
    \ea  
    \gll O João vai telefonar para a sua mãe e a Maria também vai.\\
         the J.   goes call to the his mother and the M. also goes\\
    \glt ‘João will call his mother and Marcela will do so, too.’
    \ex Sloppy reading\\
        Marcela will call her mother.
    \ex Strict reading\\
        *Marcela will call João’s mother.
    \z  
\z

The complex possessive allows both the sloppy and the strict reading:

\ea%44
         EP\label{ex:wein:44}\\
    \ea  
    \gll O João vai telefonar para a mãe   dele e   a Maria também vai.\\
         the J. goes call to the mother of.him and the M. also goes\\
    \glt ‘João will call his mother and Marcela will do so, too.’
    \ex  Sloppy reading\\
         Marcela will call her mother.
    \ex  Strict reading\\
         Marcela will call João’s mother.
    \z
\z



\subsubsection{\textit{Only}-contexts}% 3.4.3. 

The simple possessive does not show anaphoric properties in \textit{only}{}-contexts. Both simple and complex possessives allow for a bound and a coreferential reading, as shown in (45) and (46).

\ea%45
         EP\label{ex:wein:45}\\
    \ea  
    \gll Só o João ligou para a sua mãe.\\
         only the J. called to the his mother\\
    \glt ‘Only João called his mother.’
    \ex  Bound reading\\
         Nobody else called his own mother.
    \ex  Co-referential reading\\
         Nobody else called João’s mother.
    \z
\z


\ea%46
         EP\label{ex:wein:46}\\
    \ea  
    \gll Só o João ligou para a mãe dele.  \\
         only the J. called to the mother\\
    \glt ‘Only João called his mother.’
    \ex  Bound reading\\
         Nobody else called his own mother.
    \ex  Co-referential reading\\
         Nobody else called João’s mother.
    \z
\z



\subsubsection{Split antecedents}% 3.4.4. 

As mentioned in 2.5, the simple possessive disallows split antecedents and only full pronouns can take this kind of antecedent. The relevant examples are given in (47).

\ea%47
         EP\label{ex:wein:47}\\
    \ea[*]{
    \gll A Maria\textsubscript{i} disse que o Paulo\textsubscript{k} encontrou o seu\textsubscript{(i+k)} amigo.\\
         the M. said that the P. met the his friend\\
    \glt ‘Maria said that Paulo met his friend.’}
    \ex[]{
    \gll A Maria\textsubscript{i} disse que o Paulo\textsubscript{k} encontrou o amigo deles\textsubscript{(i+k)}. \\
         the M. said that the P. met the friend of.them \\
    \glt ‘Maria said that Paulo met their friend.’}
    \z
\z

The ability to take split antecedents is better attributed to the morphophonological realization of features than to the labels ‘pronominal’ or ‘anaphoric’. The overt personal pronoun has a full set of referential phi-features (including case assigned by the preposition) that agree with those of its antecedent. The simple possessive has two types of features: referential features agreeing with the antecedent and concord features, like other adjectives, agreeing with the possessum NP. Crucially, possessives in the 3\textsuperscript{rd} person lack overt number and gender marking, as shown in (48).

\ea%48
         EP\label{ex:wein:48}\\
    \ea  
    \gll \relax [A Maria]\textsubscript{i} encontrou o seus\textsubscript{i} amigos.\\
         the M. met the her friends\\
    \ex  
    \gll \relax [A Maria e o Paulo]\textsubscript{k} encontraram o seu\textsubscript{k} amigo.\\
         the M. and the P. met the their friend.\\
    \z
\z

Both the simple possessive and the reflexive pronoun lack overt number and gender marking and both disallow split antecedents. An example showing this for reflexives is given in (28) above. 

\subsection{Summary of referential properties}% 3.5. 

The lack of locality constraints does not affect the application of semantic binding. Semantic binding in the sense of \citet{Reinhart2006} is detectable by diagnostics (v) and (vi) and can account for the interpretation of all possessives with a local and non-local antecedent. As for null possessives, the diagnostics in (iii) and (v-viii) even indicate that binding is the only option. Simple and complex possessives allow for both interpretations, bound and coreferential. What is puzzling is their behaviour in those contexts that should allow a covaluation interpretation. According to Rule I, covaluation should be possible if the possessive precedes its (indended) referent. Given the results of diagnostics~(v-viii), it is surprising that simple and complex possessives disallow covaluation in this context, but null possessives allow for it. As mentioned in the introduction, a null possessive can be related to the speaker (similarly to the 1\textsuperscript{st} person possessive in singular and plural) and to a sentence-internal 3\textsuperscript{rd} person. Null (and simple) possessives appear to have contradictory properties. They have a particular mode of interpretation, something in between a bound variable interpretation and an indexical interpretation, or even an interpretation similar to proper names. It seems that the semantic value of a null possessive is determined by the given state, the kinship relations given by the speaker or the kinship relations that are known by the discourse participants to hold for a 3\textsuperscript{rd} person. In this sense, they can be tentatively classified as possessive logophors reflecting the relations given by the speaker or by the person talked about. \tabref{tab:wein:2} below summarizes the interpretative properties of the three types of possessive.

\begin{table}
\begin{tabularx}{\textwidth}{lQQQQ} 
\lsptoprule
& {Diagnostic} & {Null} & {Simple} & {Complex}\\\midrule
(i) & Local domain & no & no & no\\
(ii) & C-command with embedded NP & yes/no & yes/no & no\\
(iii) & Precedence & no & yes & yes\\
(iv) & Extra-sentential antecedent & yes & yes & yes\\
(v) & Under ellipsis & only sloppy & only sloppy & sloppy and strict\\
(vi) & \textit{Only}{}-contexts & only bound & bound and co-referential & bound and co-referential\\
(vii) & Split antecedents & no & no & yes\\
(viii) & Quantifier binding & only bound & bound and free & only free\\
\lspbottomrule
\end{tabularx}
\caption{Summary of the referential properties of the three types of possessive}
\label{tab:wein:2}
\end{table}

\section{The EP null possessive is \textit{pro}}% 4. 

Given that a possessive \textit{pro} exists in the grammar of EP, a null possessive would consist only of a covert person feature. Admittedly this would make it a very strange element. But as it contributes to determining the referent of the relational noun, its existence would be justified. In 4.1, the syntactic distribution of simple possessives inside the DP is briefly reviewed. In 4.2, an idea is presented concerning how the restriction of a possessive \textit{pro} to relational nouns could be derived.

\subsection{Distribution of possessive elements}% 4.1.

In the surface syntax, the distribution of the simple possessive is not affected by the type of NP (relational and non-relational nouns). Rather, the distribution of simple possessives in EP is affected by definiteness. In EP, the simple possessive occurs prenominally with a definite determiner and postnominally with an indefinite determiner, as shown in (49).

\ea%49
         EP\label{ex:wein:49}\\
    \ea  
    \gll a minha cadeira / amiga\\
         the my chair / friend\\
    \ex  
    \gll uma cadeira / amiga minha\\
         a chair / friend my\\
    \z
\z

\citet{Brito2007}, \citet{Castro2003,Castro2005,Castro2007}, and \citet{Miguel2002a,Miguel2002b,Miguel2004} study the placement of EP possessives and the variation in EP dialects. The pattern presented here corresponds to the pattern classified as the dominant grammar in \citet{Brito2007}. Some varieties permit the indefinite article and a prenominal possessive, but no dialect has a postnominal possessive with a definite article. With respect to grammatical category, \citet{Brito2007} and \citet{Miguel2002a,Miguel2002b} assume that the possessive pronoun is an adjective phrase. In some varieties, the prenominal possessive tends to become a determiner head, as assumed in \citet{Castro2003}.

\subsection{Restriction to relational nouns} % 4.2. 

The syntactic analysis here is inspired by the account of \citet{Partee1997} regarding the interpretation of genitives. She proposes two different structures for non-relational nouns (plain one-place predicates of type <e,t>) and relational nouns (two-place predicates of type <e,<e,t>>). Both types of nouns combine with a possessive pronoun or a genitive PP, but differ with respect to the way they combine with it. Relational nouns lexically determine the type of relation that is established between its arguments. For example, the noun \textit{amigo} establishes the relation of \textit{being-friend-of}. \citet{Partee1997} labels this relation ‘inherent R’ and it is represented for the noun \textit{amigo} as in (50).

\ea%50
    \label{ex:wein:50}
    [amigo (\textit{y}, \textit{x})]
    \z


The variable \textit{y} stands for the referent of the possessive pronoun and the variable \textit{x} for the referent of the DP \textit{o amigo}. The possessive is thus conceived of as an argument. If a plain noun combines with a possessive, the possessive relation is not lexically determined. In the sentences in (51), the relation established between the DP \textit{Rui} and the flower is not necessarily that of possession. It can be any relation given in the utterance context; e.g. \textit{the stone Rui found} or \textit{the stone that is in Rui’s garden}.

\ea%51
         EP  \label{ex:wein:51}\\
    \gll O Rui desenha uma pedra sua.\\
         the R. draws a stone his\\
    \glt ‘Rui is drawing a stone of his.’
    \z


Such a relation is labeled \textit{free R} in \citet{Partee1997}. The relational interpretation of a one-place predicate is due to its combination with a possessive pronoun or genitive PP. The possessor DP is conceived of as a nominal modifier. The representation in (52) shows that the \textit{free R} is added to the DP \textit{pedra} as a conjunct.

\ea%52
    \label{ex:wein:52}
    [pedra (\textit{x}) \& R(\textit{y})(\textit{x})]
\z

The arguments of relational nouns are present in the syntactic and semantic representation of a sentence. But nothing has been said about why the possessives cannot be null (without phonetic content) with non-relational nouns. If this distinction is reflected in syntax, the restriction of null possessives to relational nouns could be derived from the internal syntax of this type of DP.

In what follows, I sketch this idea. With respect to the placement of the possessive within the DP, there are two types of structures that have been discussed in the literature. The structures are represented in \figref{fig:wein:3} as type A (\citealt{Kupisch2011}; \citealt{Alexiadou2005}) and type B (\citealt{Parodi1994}; \citealt{Brito2007}).

\begin{figure}\footnotesize%
\noindent\parbox[t]{.45\textwidth}{Type A\\\citep{Kupisch2011,Alexiadou2005}\\
\begin{forest}
[DP 
    [D°] [FP
        [specFP [poss,tier=word,edge=dashed]] [F'
            [F°] [NumP
                [Num°[noun,tier=word,edge=dashed]] [nP
                    [specnP[poss,tier=word,edge=dashed]] [n'
                        [n° [{\color{gray}{noun}},tier=word,edge={dashed,gray}]] [NP [{\color{gray}{noun}},tier=word,edge={dashed,gray}]]
                    ]
                ]
            ]
        ]
    ]
]
\end{forest}}%
\noindent\parbox[t]{.45\textwidth}{Type B\\\citep{Parodi1994,Brito2007}\\
\begin{forest}
[DP
    [D°] [AgrP
        [specAgrP[poss,tier=word]] [Agr'
            [Agr° [noun,tier=word]] [PossP
                [specPoss[poss,tier=word]] [Poss'
                    [Poss°] [NumP
                        [Num° [\color{gray}{noun},tier=word,edge={dashed,gray}]] [NP [\color{gray}{noun},tier=word,edge={gray,dashed}]]
                    ]
                ]
            ]
        ]
    ]
]
\end{forest}}%
\caption{\label{fig:wein:3}Position of the possessive pronoun in EP}
\end{figure}


In both types of accounts, it is assumed that DPs a have an internal structure analogous to IP/TP, with lexical/thematic layers (NP/\textit{n}P) and functional layers (NumP, FP or NumP, AgrP). The last/highest projection is the determiner phrase. In type A, the possessive is generated as the ‘external argument’ in Spec\textit{n}P. In type B, the possessive is generated as specifier of its own projection, between NumP and AgrP. In both types, the prenominal position is derived by movement of the possessive to a higher functional position, FP and AgrP respectively. All these accounts give a derivational explanation for the pre- and postnominal positions. But I want to focus on the different positions proposed for base-generation of the possessive. In type A, the possessive is generated as the ‘external argument’ of \textit{n}° in the Spec\textit{n}P position. By definition, an XP is a specifier of a head if it satisfies the EPP of that head via internal merge, or if it is semantically selected by the head and merged externally (cf. \citealt{Demonte2005}: 95). In type B, it is the specifier of its own projection. From a derivational perspective, it is Poss° that selects for NumP as its complement. 

I would like to propose an account for the restriction of null possessives to relational nouns along the lines of \citegen{Demonte2005} account of adjectives in Spanish. She elaborates on the idea that non-predicative (prenominal) adjectives are selected by N° to a specifier position and that non-predicative (postnominal) adjectives select for N° “in a certain sense” (cf. \citealt{Demonte2005}: 95). I propose that relational nouns, whose semantics is an inherent R in the sense of \citet{Partee1997}, select a possessive pronoun in the specifier of \textit{n}P. The selected possessive realizes the argument of the relational noun. This idea is shown in the tree structure in \figref{fig:wein:4}.

\begin{figure}
\begin{forest}
[nP
    [specnP\\possessive XP(x),name=poss]
    [n'
        [n°\\father(x)(y),name=father] [NP]
    ]
]
\draw[dashed] (father.south) -- ++(0,-.5\baselineskip) -| (poss.south)  node[near end,left] {selection of argument};
\end{forest}
\caption{\label{fig:wein:4}Selection of possessive argument}
\end{figure}

Non-relational nouns can receive a relational interpretation only when they are combined with an overt possessive pronoun or a genitive DP that induces the free R interpretation. It is the possessive that provides the relation. I propose that the free R is realized in syntax in the form of the possessive phrase that selects for a \textit{n}P/NumP. The possessor is generated in SpecPoss, as shown in the tree structure in \figref{fig:wein:5}.

  
\begin{figure}
\begin{forest}
[PossP
    [specPoss [possessive XP(y),name=poss]] [Poss'
        [Poss°] [nP\slash NumP
            [flor(x),roof,name=flor]
        ]
    ]
]
\draw[dashed] (flor.south) -- ++(0,-.5\baselineskip)  node[near start,below=1.5ex,font=\small] {establishing a free R(elation) between y and x} -| (poss.south);
\end{forest}
\caption{\label{fig:wein:5}Free R established by Poss°}
\end{figure}

Possibly, the free R is associated with Poss° and merger with an NumP or \textit{n}P generates the conjunction structure as presented in (52). The possessor can only be null/phonetically empty in the structure shown in \figref{fig:wein:4}, but not in the structure represented in \figref{fig:wein:5}. The idea is that a null possessive can be licensed in the sense of \citet{Rizzi1986} in the specifier of a relational \textit{n}° because it is not only the selecting head but also the head with which the possessor DP agrees and by which it gets case-marked. In (48), the possessor also agrees with the noun, but it is not its selecting head. Hence, the null possessive cannot be licensed. This analysis treats relational and functional nouns alike. As mentioned in the introduction, there is an interaction between concept types and determination. How this could be derived from the internal syntax of DPs has to be left for future research. But looking at other languages with adjective possessives (allowing the determiner + possessive), like Italian, a similar effect is found. With a relational noun as in (53), both the determiner and the possessive are present. But with functional nouns, either the possessive pronoun or the definite determiner has to be used, as shown in (54a). Co-occurrence of the definite article and the possessive pronoun is not acceptable, as shown in (54b).\footnote{These examples have been pointed out to me by an anonymous reviewer.}

\ea%53
         Italian (Google) \label{ex:wein:53}\\
    \gll Elefantino “salva” dalle acque il suo amico umano.\\
         little elephant saves from.the water the his friend human\\
\z

\ea%54
    \label{ex:wein:54}
    \ea[]{
    \gll Gianni ha accarezzato suo padre / il padre.\\
         G. has caressed his father / the father\\}
    \ex[*]{
    \gll Gianni ha accarezzato il suo padre.\\
         G. has caressed the his father\\}
    \z
\z

Whether this interaction can be accounted for by a syntactic analysis has to be left for future research.

\section{Conclusion}% 5. 
In this article, the referential properties of null possessive elements in EP have been determined by a set of interpretative and structural diagnostics. Null possessives are not subject to structural conditions, but they show a bound variable reading. Due to these properties, it has been concluded that a syntactic analysis in terms of movement (or Agree) is not feasible in EP. In order to shed more light on the phenomenon of null possessives in EP, the referential properties of simple and complex possessive have been taken into account. It has been shown that the (semantic) approach of \citet{Reinhart2006} neither covers the interpretation of null possessives nor the interpretation of simple and complex possessives. Null possessives are interpreted in a different manner. They are classified as possessive logophors, which are not sensitive to discourse principles like salience, but they reflect the given states and their use is closer to that of proper names and indexicals. In section 4, a syntactic explanation for the restriction of null possessives to relational nouns was proposed. What remains open is the role of determination and a more detailed analysis of contexts in order to distinguish between the anaphoric and indexical uses of null possessives.

% \section{ References}
% 
% Alexiadou, Artemis. 2005. Possessors and (in)definiteness. \textit{Lingua} 115. 787–819.
% 
% Alonso-Ovalle, Luis \& D’Introno, Francesco. 2001. Full and null pronouns in Spanish: The Zero Pronoun Hypothesis. In Campos, Héctor \& Herburger, Elena \& Morales-Front, Alfonso \& Walsh, Thomas J. (eds.), \textit{Hispanic linguistics at the turn of the millennium: Papers from the 3rd Hispanic Linguistics Symposium}, 189–210. Somerville, MA: Cascadilla Press.
% 
% Barker, Chris. 2011. Possessives and relational nouns. In von Heusinger, Klaus \& Maienborn, Claudia \& Portner, Paul H. (eds.), \textit{Semantics: An international handbook of natural language meaning}, 1109–1130. Berlin: Mouton De Gruyter.
% 
% Boeckx, Cedric \& Hornstein, Norbert \& Nunes, Jairo. 2010. \textit{Control as movement}. Cambridge: Cambridge University Press.
% 
% Brito, Ana Maria. 2007. European Portuguese possessives and the structure of DP. \textit{Cuadernos de Lingüística} 14. 21–50.
% 
% Büring, Daniel. 2005. \textit{Binding theory}. Cambridge: Cambridge University Press.
% 
% Campos, Héctor. 1995. Reconstruction and picture nouns in Spanish. In Kempchinsky, Paula M. (ed.), \textit{Evolution and revolution in linguistic theory: A Festschrift in honor of Carlos Otero}, 25–50. Washington, D.C.: Georgetown University Press.
% 
% Castro, Ana. 2005. Possessives in European Portuguese. Ms. (Lisbon: Universidade Nova de Lisboa \& Paris: Paris VIII Saint Denis.)
% 
% Castro, Ana. 2007. Sobre os possessivos simples em português. In Lobo, Maria \& Coutinho, Maria Antónia (eds.), \textit{Actas do XXII Encontro da Associação Portuguesa de Linguística}, 223–237. Lisbon: APL/Colibri. 
% 
% Castro, Ana \& Costa, João. 2003. Weak forms as Xº: Prenominal possessives and preverbal adverbs in Portuguese. In Pérez-Leroux, Ana Teresa \& Roberge, Yves (eds.), \textit{Romance linguistics: Theory and acquisition}, 95–110. Amsterdam: John Benjamins.
% 
% Chomsky, Noam. 1981. \textit{Lectures on government and binding: The Pisa lectures}. Dordrecht: Foris.
% 
% Chomsky, Noam. 1986. \textit{Knowledge of language: Its nature, origins, and use}. New York: Praeger.
% 
% Chomsky, Noam. 1995. \textit{The Minimalist Program}. Cambridge, MA: MIT Press.
% 
% Chomsky, Noam. 2000. Minimalist inquiries: The framework. In Martin, Roger \& Michaels, David \& Uriagereka, Juan (eds.), \textit{Step by step: Essays on Minimalist syntax in honor of Howard Lasnik}, 89–155. Cambridge, MA: MIT Press.
% 
% Chomsky, Noam. 2001. Derivation by phase. In Kenstowicz, Michael (ed.), \textit{Ken Hale: A life in language}, 1–52. Cambridge, MA: MIT Press.
% 
% Chomsky, Noam. 2004. Beyond explanatory adequacy. In Belletti, Adriana (ed.), \textit{Structures and beyond: The cartography of syntactic structures}, 104–131. Oxford: Oxford University Press.
% 
% Demonte, Violeta. 2005. Meaning-form correlations and the order of adjectives in Spanish. In McNally, Louise \& Kennedy, Chris (eds.), \textit{Adjectives and adverbs: Syntax, semantics, and discourse}, 71–100. Oxford: Oxford University Press.
% 
% Floripi, Simone. 2003. Argumentos nulos dentro de DPs em português brasileiro. Ms. (Campinas: Universidade Estadual de Campinas.)
% 
% Floripi, Simone \& Nunes, Jairo. 2009. Movement and resumption in null possessor constructions in Brazilian Portuguese. In Nunes, Jairo (ed.), \textit{Minimalist essays on Brazilian Portuguese syntax}, 51–68. Amsterdam: John Benjamins.
% 
% Hawkins, John. 1978. \textit{Definiteness and indefiniteness: A study in reference and grammaticality prediction}. London: Croom Helm.
% 
% Heim, Irene. 1991. Artikel und Definitheit. In von Stechow, Arnim \& Wunderlich, Dieter (eds.), \textit{Semantics: An international handbook of contemporary research}, 487–535. Berlin: Walter de Gruyter.
% 
% Hicks, Glyn. 2009. \textit{The derivation of anaphoric relations}. Amsterdam: John Benjamins.
% 
% Horn, Laurence R. 1969. A presuppositional analysis of \textit{only} and \textit{even}. In Binnick, Robert I. \& Davison, Alice \& Green, Georgia M. \& Morgan, Jerry L. (eds.), \textit{Papers from the Fifth Regional Meeting of the Chicago Linguistic Society}, 98–107. Chicago: University of Chicago.
% 
% Hornstein, Norbert. 2001. \textit{Move! A Minimalist theory of construal}. Malden, MA: Blackwell.
% 
% Hornstein, Norbert. 2007. Pronouns in a Minimalist setting. In Corver, Norbert \& Nunes, Jairo (eds.), \textit{The copy theory of movement}, 351–385. Amsterdam: John Benjamins.
% 
% Kayne, Richard S. 2002. Pronouns and their antecedents. In Epstein, Samuel David \& Seely, T. Daniel (eds.), \textit{Derivation and explanation in the Minimalist Program}, 133–166. Malden, MA: Blackwell.
% 
% Kupisch, Tanja \& Rinke, Esther. 2011. The diachronic development of article-possessor complementarity in the history of Italian and Portuguese. In Siemund, Peter (ed.), \textit{Linguistic universals and language variation}, 92–127. Berlin: De Gruyter Mouton.
% 
% Lebeaux, David. 1985. Locality and anaphoric binding. \textit{The Linguistic Review} 4. 343–363.
% 
% Löbner, Sebastian. 2011. Concept types and determination. \textit{Journal of Semantics} 28. 279–333.
% 
% Lobo, Maria. 2013. Dependências referenciais. In Raposo, Eduardo P. \& Nascimento, M. Fernanda \& da Mota, M. Antónia \& Segura, Luísa \& Mendes, Amália (eds.), \textit{Gramática do português}, 2177–2230. Coimbra: Gráfica de Coimbra.
% 
% Mateus, Maria Helena \& Brito, Ana Maria \& Duarte, Inês \& Faria, Isabel Hub \& Frota, Sónia \& Matos, Gabriela \& Oliveira, Fátima \& Vigário, Marina \& Villalva, Alina (eds.). 2003. \textit{Gramática da língua portuguesa}. Lisbon: Editorial Caminho.
% 
% Miguel, Matilde. 2002a. O estatuto categorial dos possessivos: Possessivos e adjectivos. In Duarte, Inês et al. (eds.), \textit{Actas do Encontro Comemorativo dos 25 anos do Centro da Linguística da Universidade do Porto}, 191–202. Porto: CLUP.
% 
% Miguel, Matilde. 2002b. Para uma tipologia dos possessivos. In Duarte, Inês \& Faria, Isabel (eds.), \textit{Actas do XVII Encontro da Associação Portuguesa de Linguística}, 287–299. Lisbon: APL/Colibri.
% 
% Miguel, Matilde. 2004. O sintagma nominal em Português Europeu: Posições de sujeito. Ms. (Lisbon: Universidade de Lisboa.)
% 
% Montalbetti, Mario. 1984. After binding: On the interpretation of pronouns. Cambridge, MA: MIT. (Doctoral dissertation.)
% 
% Pagin, Peter. 2016. Assertion. In Zalta, Edward. N. (ed.) \textit{The Stanford encyclopedia of philosophy}. Stanford, CA: Stanford University. (https://plato.stanford.edu/archives/win2016/entries/assertion/)
% 
% Parodi, Claudia. 1994. On Case and agreement in Spanish and English DPs. In Mazzola, Michael L. (ed.), \textit{Issues and theory in Romance linguistics: Selected papers from the Linguistic Symposium on Romance Languages XXIII, April 1-4}, 403–416. Washington, D. C.: Georgetown University.
% 
% Partee, Barbara H. 1997. Genitives: A case study. Appendix to Janssen, Theo. 1997. Compositionality. In van Benthem, Johan \& ter Meulen, Alice (eds.), \textit{Handbook of logic and language}, 464–470. Amsterdam: Elsevier.
% 
% Reinhart, Tanya. 1983. \textit{Anaphora and semantic interpretation}. London: Croom Helm.
% 
% Reinhart, Tanya. 2006. \textit{Interface strategies: Optimal and costly computations}. Cambridge, MA: MIT Press.
% 
% Reinhart, Tanya \& Reuland, Eric. 1993. Reflexivity. \textit{Linguistic Inquiry} 24. 657–720.
% 
% Reinhart, Tanya \& Reuland, Eric. 1995. Pronouns, anaphors and case. In Haider, Hubert \& Olsen, Susan \& Vikner, Sten (eds.), \textit{Studies in comparative Germanic syntax}, 241–268. Dordrecht: Kluwer.
% 
% Reuland, Eric. 2001. Primitives of binding. \textit{Linguistic Inquiry} 32. 439–492.
% 
% Reuland, Eric. 2011. \textit{Anaphora and language design}. Cambridge, MA: MIT Press.
% 
% Rizzi, Luigi. 1986. Null objects in Italian and the theory of \textit{pro}. \textit{Linguistic Inquiry} 17. 501–557.
% 
% Rodrigues, Cilene. 2010. Possessor raising through thematic positions. In Hornstein, Norbert \& Polinsky, Maria (eds.), \textit{Movement theory of control}, 119–146. Amsterdam: John Benjamins.
% 
% Ross, John R. 1967. \textit{Constraints on variables in syntax}. Cambridge, MA: MIT. (Doctoral dissertation.)
% 
% Ross, John R. 1969. Guess who? In Binnick, Robert I. \& Davidson, Alice \& Green, Georgia M. \& Morgan, Jerry L. (eds.), \textit{Papers from the Fifth Regional Meeting of the Chicago Linguistic Society}, 252–297. Chicago: University of Chicago.
% 
% Runner, Jeffrey T. \& Kaiser, Elsi. 2005. Binding in picture noun phrases: Implications for Binding Theory. In Müller, Stefan (ed.), \textit{Proceedings of the 12th International Conference on Head-Driven Phrase Structure Grammar}, 594–613. Stanford, CA: CSLI Publications.
% 
% Sag, Ivan. 1980. \textit{Deletion and Logical Form}. New York: Garland.
% 
% Speas, Margaret. 2004. Evidentiality, logophoricity and the syntactic representation of pragmatic features. \textit{Lingua} 114. 255–276. 
% 
% Zwart, Jan-Wouter. 2002. Issues relating to a derivational theory of binding. In Epstein, Samuel David \& Seely, T. Daniel (eds.), \textit{Derivation and explanation in the Minimalist Program}, 269–304. Malden, MA: Blackwell.
% 
% 
% \begin{verbatim}%%move bib entries to  localbibliography.bib
% \end{verbatim} 

\sloppy
\printbibliography[heading=subbibliography,notkeyword=this] 
\end{document}
