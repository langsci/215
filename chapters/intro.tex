\documentclass[output=paper]{langsci/langscibook} 
\ChapterDOI{10.5281/zenodo.3458060}
\author{Matthew Reeve\affiliation{Zhejiang University}\and Mihaela Marchis Moreno\affiliation{FCSH, Universidade Nova de Lisboa}\lastand Ludovico Franco\affiliation{Università degli Studi di Firenze}}
\title{Introduction}


\abstract{\noabstract}
\maketitle

\begin{document}

\section{Opening remarks}

The past two decades or so have seen a considerable amount of investigation into the nature of syntactic dependencies involving the operation \isi{Agree}. In particular, there has been much discussion of the relations between \isi{Agree} and its morphological realisations (agreement and case), and between \isi{Agree} and other syntactic dependencies (e.g., movement, binding, control). The chapters in this volume examine a diverse set of cross-linguistic phenomena involving agreement and case from a variety of theoretical perspectives, with a view to elucidating the nature of the abstract operations (in particular, \isi{Agree}) that underlie them.\footnote{The chapters in this volume derive from a workshop organised by the editors, entitled \textit{Local and non-local dependencies in the nominal and verbal domains} (Faculdade de Ciências Sociais e Humanas (FCSH), Universidade Nova de Lisboa, 13 November 2015).} The phenomena discussed include \is{control!backward control}backward control, passivisation, progressive aspectual constructions, extraction from nominals, possessives, relative clauses and the phasal status of PPs. In this introductory chapter, we provide a brief overview of recent research on \isi{Agree}, and its involvement in other syntactic dependencies, in order to provide a background for the chapters that follow. We do not aim to give an exhaustive treatment of the theories of Agreement and Case here, as there already exist more comprehensive overviews, to which we refer the reader (e.g., \citealt{Bobaljik2008Case}; \citealt{Polinsky2014}).

\section{Case and agreement: Their location, interrelation and realisation}
Our starting point – because of its relative familiarity – is the treatment of case and agreement in more recent versions of Minimalism (esp. \citealt{Chomsky2000,Chomsky2000,Pesetsky2001,Pesetsky2007}). As in earlier GB and \isi{Minimalist} approaches (e.g., \citealt{Chomsky1980,Chomsky1981,Chomsky1995}), both Case and Agreement (which we capitalise here to distinguish them from the relevant morphological notions) are ``abstract'' in the sense that, while they do bear a relation to the morphological phenomena of case and agreement, this relation is only indirect. In other words, Case and Agreement within Minimalism are concerned primarily with the distribution of DPs, rather than with morphology (cf. \citealt{Bobaljik2008Case}). The basis of the approach is the operation \isi{Agree}, which relates a head (a ``probe'', such as T or \textit{v}) bearing uninterpretable (and\slash or ``unvalued'') \is{feature!phi-feature}phi-features to a ``goal'' DP, c-commanded by the probe, that bears counterparts of one or more of those features. This results in deletion at LF of the \is{feature!uninterpretable feature}\is{feature!unvalued feature}uninterpretable\slash unvalued features \is{feature!unvalued feature}on the probe, ensuring ``legibility'' at LF. Thus, in a transitive sentence the functional heads T and \textit{v}, both bearing uninterpretable \is{feature!phi-feature}phi-features and Case, initiate \isi{Agree} with the DPs they most immediately \isi{c-command}, the subject and direct object respectively:\is{feature!phi-feature}

\ea%1
    \label{ex:intro:1}
[\textsubscript{TP} Sue T\textsubscript{[}\textit{\textsubscript{u}}\textsubscript{$\varphi $ Nom, EPP]} [\textit{\textsubscript{v}}\textsubscript{P} Sue\textsubscript{[}\textit{\textsubscript{u}}\textsubscript{$\varphi $, Nom]} \textit{v}\textsubscript{[}\textit{\textsubscript{u}}\textsubscript{$\varphi $, Acc]} [\textsubscript{VP} likes cake\textsubscript{[}\textit{\textsubscript{u}}\textsubscript{$\varphi $, Acc]}]]]
    \z
 
%%please move the includegraphics inside the {figure} environment
%%\includegraphics[width=\textwidth]{OGSVolumeAug2018Introduction-img3}

 
%%please move the includegraphics inside the {figure} environment
%%\includegraphics[width=\textwidth]{OGSVolumeAug2018Introduction-img4}

The assumption here is that the checking of Case features, which are \is{feature!uninterpretable feature}uninterpretable and hence must be deleted, is dependent on the \isi{Agree} relation established by the \is{feature!phi-feature}phi-feature sets of the functional head and the DP (cf. the discussions of ``Person Case Constraint'' effects in \citealt{Anagnostopoulou2003}; \citealt{Rezac2008}). That is, under this view case is simply a reflex of phi-feature-checking that appears on nominal constituents. As it is presented in (1), Chomsky’s proposal only directly covers nominative and accusative (reflexes of \is{feature!phi-feature}phi-feature checking on T and \textit{v} respectively). As for \is{case!oblique case}oblique cases such as dative, it has recently been argued that these are checked by a functional head such as Appl (e.g., \citealt{Cuervo2003,Pylkkaenen2008}). More specifically, one possibility is that datives\slash obliques are simply the reflex of \is{feature!phi-feature}phi-feature agreement between Appl and a DP (see \citealt{Marchis2017}).

An important difference between the model in (1) and previous GB and \isi{Minimalist} models is that movement to the specifier of TP, previously held to be crucial for feature-checking \citep{Chomsky1995}, is now triggered by a distinct feature (an EPP-feature) on the probe. Thus, \isi{Agree} need not entail the movement of the goal to the probe’s specifier, but merely makes this movement available in principle via the EPP-feature that it licenses (cf. \citealt{Pesetsky2001}, who treat EPP as a ``subfeature'' of an \is{feature!uninterpretable feature}uninterpretable feature). The \isi{Agree} relation is thus intended to account for the distribution of DPs in two senses: a DP must at some point be local enough to an appropriate probe in order for \isi{Agree} to be established and the relevant \is{feature!uninterpretable feature}uninterpretable features to be checked, and \isi{Agree} additionally allows for movement of the DP to the probe’s specifier if an EPP-feature is present.

One recent debate about \isi{Agree} has concerned the directionality of the operation; that is, whether \isi{Agree} must always be ``downward'', as in the above presentation (e.g., \citealt{Chomsky2000,Chomsky2001Derivation}; \citealt{Preminger2013}), or whether it may or must operate upwards (e.g., \citealt{Zeijlstra2012}; \citealt{Ackema2018}). A further debate has concerned the extent to which \isi{Agree} is involved in mediating other grammatical dependencies. For example, \citet{Reuland2001}, \citet{Hicks2009} and \citet{Rooryck2011} argue that \isi{Agree} plays a central role in anaphoric relations (though see \citealt{Safir2014} for a dissenting view). \citet{Landau2000} argues that the control relation is mediated by \isi{Agree} relations between the controller, PRO and one or more functional heads in the clause. This approach can be contrasted with the movement-based approach to control \citep{Hornstein1999,Hornstein2010}. One piece of evidence favouring an Agree-based approach is the existence of partial and finite control, which had proven problematic for previous approaches (\citealt{Landau2013}: 65ff.).

Under the approaches outlined above, Case and Agreement are both ``narrow-syntactic'' phenomena that may or may not have an effect at the PF interface, resulting in \is{case!morphological case}morphological case and agreement respectively. This view can usefully be contrasted with an approach that was first proposed by \citet{Marantz1991} and has since had considerable influence (e.g., \citealt{Harley1995,Schütze1997,McFadden2004,Bobaljik2008Phi,Baker2010,Titov2012}). Marantz argues that generalisations about C\slash case, such as Burzio’s generalisation \citep{Burzio1986} and certain restrictions on \is{case!ergative case}ergative case assignment in languages such as \ili{Georgian} and \ili{Hindi}, are about \is{case!morphological case}morphological case (m-case), not about Abstract Case. Furthermore, he argues on the basis of \ili{Icelandic} ``\is{case!quirky case}quirky case'' (cf. \citealt{Zaenen1985}) that there is no relation between the positional licensing of DPs and the \is{case!morphological case}morphological case that they bear. His overall message is that DP-licensing is not about case, and hence that Abstract Case should be eliminated from the theory of syntax. Instead, DP-licensing should be handled entirely by the mapping between thematic roles and argument positions, supplemented by the Extended Projection Principle.

Under Marantz’s model, m-case, as well as agreement morphemes, are assigned at a level of ``Morphological Structure'' (MS) intervening between S-Struc\-ture and PF. Thus, in this model both case and agreement are ``\isi{post-syntactic}'' phenomena that do not enter into the licensing of DP\slash NPs. M-cases are assigned according to a \is{case!case hierarchy}case hierarchy (cf. \citealt{Yip1987}); at the top of the hierarchy are the ``lexically governed'' cases (e.g., ``quirky'' and inherent cases), followed by the dependent cases (accusative, ergative), followed by the unmarked cases (nominative or absolutive in clauses; genitive in DP\slash NP). Finally, there is a ``default'' case (e.g., accusative in \ili{English}) that applies when no other case realisation is possible. Indeed, Marantz emphasises that the provision of a default form when no other form is available is characteristic of morphology; a sentence will never be ungrammatical because no features are assigned to a case affix. Case “merely interprets syntactic structures and does not filter them” \citep[24]{Marantz1991}. Marantz suggests that a similar hierarchy applies in the determination of agreement, but he allows for a relatively flexible relation between case and agreement in order to account for certain case-agreement ``mismatches'' that are found in split ergative systems.

\citet{Bobaljik2008Phi} takes up the question of how agreement is determined in the context of Marantz’s proposal. His main idea is in a sense the opposite of Chomsky’s (\citeyear{Chomsky2000,Chomsky2001Derivation}), namely that agreement is parasitic on case (cf. \citealt{Bittner1996}). Thus, if Marantz’s argument that m-case is \isi{post-syntactic} is correct, then agreement must also be \isi{post-syntactic}. More specifically, Bobaljik argues that the finite verb (or other head) agrees with the highest ``accessible'' NP in its ``domain'', where ``accessibility'' is defined in terms of the \is{case!case hierarchy}case hierarchy proposed by Marantz (see also \citealt{McFadden2004}). In the spirit of \citet{Moravcsik1974} (who stated the hierarchy in terms of grammatical functions rather than cases), the unmarked cases (nominative or absolutive in clauses; genitive in DP\slash NP) are said to be maximally accessible, with the dependent cases (accusative, ergative) being less accessible, and the ``lexically governed'' (e.g., ``quirky'' and inherent cases) being the least accessible. Among other things, this hierarchy accounts for the fact that, in nominative-accusative languages, if a verb agrees with any DP, it at least agrees with subjects (e.g., \citealt{Moravcsik1974}; \citealt{Gilligan1987}), while in ergative-absolutive languages, if a verb agrees with any DP, it at least agrees with absolutive DPs (e.g., \citealt{Croft1990}). Further evidence comes from mismatches between case and grammatical function in \ili{Icelandic}, where it is case, not grammatical function, that turns out to determine the agreement controller \citep{Sigurðsson1993}. Finally, \is{agreement!long-distance agreement}long-distance agreement in languages such as Tsez \citep{Polinsky2001} suggests that there is no need for a particular grammatical relation with the agreement target beyond locality (i.e., only ``accessibility'' and ``domain'' are relevant).

Other ``\isi{post-syntactic}'' treatments of case and agreement can be found in \citet{Embick2006} and \citet{Marchis2015,Marchis2018}. These authors argue that case and agreement nodes\slash features are added after syntax in accordance with language-specific requirements, and are never essential to semantic interpretation. One advantage of this type of approach is that it could explain certain mismatches at the syntax-morphology interface that arise with certain word categories that are in complementary distribution, such as denominal relational \is{adjective}adjectives and prepositional genitives in \ili{Romance}. Semantically and syntactically, these are nouns, but morphologically they instantiate different word categories with different \is{case!case assignment}case assignment requirements \citep{Marchis2018}. In the spirit of \citet{Embick2006},  \citet{Marchis2015,Marchis2018} argues that the Case features of the underlying nouns in the structure of thematic relational \is{adjective}adjectives are relevant only at PF, and that their countability (or lack thereof) in the syntax conditions the choice of Vocabulary Items expressing Case. That is, their underspecification for number triggers deficient Case features on thematic relational adjectives that are valued only at PF, determining the introduction of an Agreement node (AGR) that turns the noun into an \isi{adjective} through suffixation, instead of introducing the Genitive Case feature, spelled out as the \isi{preposition} \textit{de} in \ili{Romance} languages. 

An interesting contrast is provided by the work of \citet{Preminger2014}, who argues against the ``\isi{post-syntactic}'' view of agreement and case, but agrees with Bobaljik that phi-agreement is sensitive to \is{case!morphological case}morphological case. Preminger notes that Marantz’s argument for a \isi{post-syntactic} treatment of case is based on the purported absence of grammatical processes that refer to case. Preminger argues, however, that the distinction between ``quirky-subject'' and ``non-quirky-subject'' languages with respect to raising and agreement over \is{experiencer}experiencers exemplifies such a process. More specifically, he argues that movement to subject position is ``case-discriminating'' in languages such as \ili{English} and \ili{French}, and hence that case must be part of syntax proper. Nevertheless, Preminger makes crucial use of Marantz’s \is{case!case hierarchy}case hierarchy, which he attempts to derive from independently established principles of syntactic structure-building.

A quite different approach to case and agreement is found in the work of \citet{Manzini2016}, \citet{Franco2017} and Manzini et al. (this volume). These authors question the idea of an ``accessibility hierarchy'' of cases, arguing that such a hierarchy has no special advantage over a pure stipulation of the facts, such as the VIVA (Visibility of Inherent Case to Verbal Agreement) parameter of \citet{Anand2006}. Furthermore, they argue that it is both unnecessary and unprofitable to define \isi{Agree} in terms of (un)interpretable and (un)valued features (cf. \citealt{Brody1997}). Finally, they argue that certain types of case are unsuited to treatment in terms of \is{feature!uninterpretable feature}uninterpretable features, as they actually have inherent semantic content. For example, they propose that ``oblique'' cases should be analysed in terms of what they call an ``elementary \isi{relator}'' with a ``part\slash whole'' semantic content. The general approach proposed in these works is adopted in \citet{Reeve2018}, which argues that extraction from DP\slash NP cross-linguistically is dependent on the \isi{Agree} operation, where \isi{Agree} relates sets of \is{feature!interpretable feature}interpretable features as in the above works. However, \isi{Agree} is only possible where the language independently shows overt evidence of agreement. This accounts for the observation that languages with left-branch extraction tend to be languages with overt agreement in DP\slash NP (cf. \citealt[237--238]{Ross1967}; \citealt[188]{Horn1983}). (See Mensching’s chapter for an alternative analysis of extraction from DP\slash NP.)

A final prominent issue in research on case and agreement is the analysis of \isi{syncretism} – the phenomenon whereby two morphosyntactically distinct categories may receive identical morphophonological realisations. Case \isi{syncretism} has been analysed in terms of implicational hierarchies of the type discussed above with respect to \citegen{Marantz1991} proposal. \citet{Blake2001} proposes the implicational hierarchy in (2), such that cases on the right are progressively less likely to occur. \citet{Caha2009} modifies Blake’s hierarchy (not taking ergative into account) as in (3), conceived of as an f-sequence in the Nanosyntactic framework. His main reason for adopting this particular hierarchy is that it can account for possible \is{syncretism}syncretisms between cases, given a constraint blocking non-accidental \isi{syncretism} between non-adjacent categories (cf. the *ABA constraint of \citealt{Bobaljik2012}).

\ea%2
\label{ex:intro:2}\citep[156]{Blake2001}\\
\textsc{nominative > accusative / ergative > genitive > dative > \isi{locative} > ablative\slash instrumental > other}    
 \z

\ea%3
\label{ex:intro:3}\citep[32]{Caha2009}\\
\textsc{nom > acc > loc1 > gen\slash part > loc2 > dat > loc3 > ins\slash com}
\z

A related approach is that of \citet{Calabrese2008}, who adopts the tenets of Distributed Morphology (\citealt{Halle1993}, \citealt{Embick2006}, among others). Calabrese is specifically interested in absolute \isi{syncretism} – i.e., in the fact that certain cases or case oppositions are missing altogether in some languages. He assumes that functional categories are represented by abstract feature clusters in syntax, which are only realised by actual exponents at the PF interface. His key proposal is that there is a markedness hierarchy of cases, not unlike the descriptive hierarchies in (2)–(3). Following \citet{Blake2001}, lower cases in the hierarchy are more likely to be blocked. If they are, the corresponding feature cluster cannot surface at PF, but must be readjusted by the morphological component (including the key rule of Impoverishment) yielding surface \isi{syncretism}.

In a series of recent works, \citet{Manzini2011Grammatical}, \citet{Manzini2016} and \citet{Franco2017} reject these approaches, arguing that they leave the traditional cases, and the traditional notion of case itself, unanalysed. The latter series of works instead analyses (oblique) case as the inflectional realisation of elementary predicative content (‘includes’\slash ‘is included by’) on a noun. Correspondingly, there is no externally imposed hierarchy ordering the relevant primitives, but rather a conceptual network determined by the primitive predicates we use and the relations they entertain with each other. These authors argue that neither Calabrese’s markedness hierarchies nor Caha’s nanosyntactic functional hierarchies are necessary, because \isi{syncretism} depends essentially on natural class \citep{Müller2007}. Seen from this perspective, \is{case!case hierarchy}case hierarchies essentially reduce to a binary split between direct case (reduced to the agreement system; \citealt{Chomsky2001Derivation}) and \is{case!oblique case}oblique case, reducing to part-whole operators. Other so-called cases are analysable into a case core (typically oblique) and some additional structure, yielding something similar to the internally articulated PPs of \citet{Svenonius2006}.

Syncretism has also been shown to have effects on other aspects of the grammar. For example, it has been reported to have the property of repairing violations of syntactic constraints; for example, with agreement \citep{Schütze2003,Bhatt2013} or case-matching \citep{Citko2005,Craenenbroeck2012,Hein2016}. On the face of it, this property of \isi{syncretism} appears to pose a challenge to \isi{post-syntactic} views of morphology such as DM. \citet{Citko2005} and \citet{Asarina2011} attempt to maintain a DM view by appealing to underspecification. However, \citet{Hein2016} argue on the basis of Polish data that underspecification approaches cannot account for the repair effect of \isi{syncretism} on violations of the case-matching requirement in Across-the-Board (ATB) constructions, and that the problem for DM remains.

\section{Issues arising in this volume}

We will now outline a few issues in the syntax of case and agreement that have become prominent in the literature and are discussed in one or more contributions to the present volume. Our aim here is to identify a number of common issues and perspectives among the chapters, which on the face of it are quite diverse in their content.

The first such issue is the question of what the relation is between A\slash agreement and C\slash case. As we have seen, in Chomsky’s \isi{probe-goal} system Case-checking\slash valuation is dependent on the application of \isi{Agree}, while in approaches such as \citet{Bobaljik2008Phi} and \citet{Preminger2014}, agreement depends on the output of C\slash case-assignment. In other approaches, such as \citet{Baker2015} and \citet{Manzini2016}, C\slash case and A\slash agreement are essentially independent. A number of contributions to this volume could be said to argue in favour of a tight relation between case and agreement. Marchis Moreno’s chapter argues that backward \is{control!object control}object control in Brazilian \ili{Portuguese} occurs only in the presence of an inflected \isi{infinitive}, and that this inflection diagnoses the percolation of default \is{case!nominative case}nominative case onto embedded T, which must then be assigned to an overt DP in SpecTP. Such an analysis is only feasible if C\slash case and agreement go hand in hand. Giurgea’s chapter argues that the ``person constraint'' on \is{se-passive}\textit{se}-passives in Romanian can be accounted for if a person feature intervenes to block case-assignment by V to its internal argument. Again, this presupposes that person features are of the ``same type'' as Case features, in the sense that one can block an operation targeting the other.

Other chapters argue for or suggest that the relation between case and agreement goes in one or the other direction. Łęska’s chapter focuses on the nature of ``Case attraction'' in Polish relative clauses, arguing that the \isi{Agree} relation occurring between a \isi{numeral} \isi{quantifier} and a relative pronoun may optionally result in transmission of the \isi{numeral} \isi{quantifier}’s Case onto the relative pronoun. On the other hand, because agreement (full vs. default) on the relative clause predicate depends on whether Case transmission has taken place, \isi{Agree} must be able to detect the output of Case attraction; in other words, agreement must be parasitic on C\slash case, as in the work of \citet{Marantz1991} and \citet{Preminger2014}. By contrast, Mensching’s chapter argues that \isi{Agree} (in the Chomskyan sense) is crucially involved in licensing extraction from nominals, in that an XP must undergo \isi{Agree} with D in order to be extracted from DP. In particular, he argues that the argument\slash \isi{adjunct} asymmetry in extraction can be accounted for if arguments undergo \isi{Agree} with D to value Case, while \is{adjunct}adjuncts cannot. Thus, extraction depends on Case, which depends on \isi{Agree}(ment). Finally, Manzini, Franco \& Savoia argue that, while the so-called ``direct cases'' (e.g., nominative, accusative) are parasitic on agreement, as in Chomsky’s work, ``oblique cases''\is{case!oblique case} (dative, genitive, instrumental) are a different type of phenomenon. They argue that it is problematic to adopt an \isi{Agree} approach to ``\isi{concord}'' within DP (e.g., \citealt{Carstens2001}), involving one goal (N) checking multiple probes (agreeing \is{determiner}determiners and modifiers). Instead, as noted above, they propose that oblique involves an ``elementary \isi{relator}'' with a ``part\slash whole'' semantic content.

A second prominent topic in this volume concerns the extent to which the operation \isi{Agree} is crucially involved in establishing other grammatical dependencies. Alexiadou \& Anagnostopoulou and Marchis Moreno both argue that \is{control!backward control}backward control (in \ili{Greek} and Brazilian \ili{Portuguese} respectively) relies on an \isi{Agree} relation between a head in the control predicate’s clause and a head in the clause embedded by that predicate. This relation enables the realization of either the higher copy in forward control or the lower copy in \is{control!backward control}backward control. Lorusso argues that agreement in aspectual constructions coincides with the semantic operation of event identification, which is responsible for a number of syntactic and semantic properties of these constructions, as compared with similar constructions lacking agreement. Mensching argues – following the general framework of \citet{Chomsky2000,Chomsky2001Derivation} – that \isi{Agree}, and the Case-valuation that goes along with it, are crucially involved in movement dependencies, specifically extraction from nominals. Manzini, Franco \& Savoia argue that \isi{Agree} is also involved in the mediation of thematic dependencies. They focus on what is often called ``\isi{concord}'' – agreement in the nominal domain – arguing that this type of agreement is a morphological equivalent of \citegen{Higginbotham1985} theta-binding relation. Finally, a contrastive perspective is provided by Weingart’s chapter, which argues that \is{pronoun!null possessive}null possessive pronominals in \ili{Portuguese} should not be derived in terms of \isi{Agree} (pace \citealt{Hicks2009}) or Move (pace \citealt{Floripi2009,Rodrigues2010}).

Locality conditions on \isi{Agree} play an important role in several chapters in this volume. Mensching argues, in common with a number of other authors (e.g., \citealt{Svenonius2004,Bošković2005,Heck2009,Reeve2018}), that DP is a phase, which means that extraction from DP is blocked unless the moving item first moves to SpecDP. In particular, Mensching argues that this, in conjunction with the proposal that SpecDP is only accessible to items that agree with D, can account for the often-observed argument\slash \isi{adjunct} asymmetry in extraction from DP. Gallego argues that PP is a phase \citep{Abels2003,Abels2012}, and that this normally blocks \isi{Agree} between a verb and a DP within PP. As well as accounting for the general lack of overt agreement, this can account for the ban on preposition-stranding and \is{quantifier}pseudopassives in the majority of languages, including (most) \ili{Spanish} \citep{Law2006}. However, Gallego argues that cases of agreement between V and PP’s complement in certain dialects of \ili{Spanish} can be accounted for if P incorporates with the verb (cf. \citealt{Hornstein1981,Law2006}). Ackema \& Neeleman’s chapter can be seen as providing something of a contrast, in that it argues for a relatively reduced role for locality in restricting agreement possibilities. In particular, they argue against \citegen{Preminger2014} claim that the phenomenon of ``\is{agreement!omnivorous agreement}omnivorous agreement'' is regulated by relativised minimality conditions on \isi{Agree}. Instead, they argue that it is necessary for both syntactic and morphological accounts of agreement to postulate cross-linguistic distinctions in feature hierarchies; thus, the syntactic account has no special advantage here. Similarly, Weingart’s chapter argues that \is{pronoun!null possessive}null \is{possessive pronoun}possessive pronouns in \ili{Portuguese} are not restricted by locality conditions, as part of her overall argument that they should not be derived in terms of \isi{Agree} or Move.

Another prominent topic in this volume is the specific nature of the features related by \isi{Agree}. One issue already touched on here is the question of whether \is{feature!phi-feature}phi-features are \is{feature!uninterpretable feature}\is{feature!phi-feature}uninterpretable features, as in most of the contributions here, or \is{feature!interpretable feature}interpretable features, as Manzini, Franco \& Savoia argue. They also argue against the idea, developed in particular in \citet{Chomsky2000} and \citet{Pesetsky2007}, that features should be distinguished in terms of whether they enter the derivation as valued or unvalued. The structure of \is{feature!phi-feature}phi-features is also the central topic of Ackema \& Neeleman’s chapter, which focuses on distinctions between person and number: in particular, that agreement conflicts between third person and first\slash second person result in ungrammaticality, while conflicts between singular and plural number do not, but result in a default. Mensching’s chapter crucially proposes a particular feature structure for Ds that license extraction from DP, involving an unvalued phi-set that probes the head noun, together with an optional second probe with a case-assigning property, enriched with an unvalued operator feature associated with an EPP-feature.

Finally, the issue of \isi{syncretism}, discussed at the end of §2, becomes relevant in two chapters in this volume. In their discussion of \is{agreement!omnivorous agreement}omnivorous agreement, Ackema \& Neeleman note that although feature clashes between the \is{feature!phi-feature}phi-features of the subject and object may prevent the realisation of agreement in such systems, the problem may be averted if the two feature-sets give rise to identical morphophonological realisations. (They give examples from agreement with nominative objects in \ili{Icelandic} and agreement with the focus in Dutch clefts.) In Łęska’s chapter, case \isi{syncretism} between a relative operator and a \isi{numeral} \isi{quantifier} is a precondition for Case transmission from the \isi{numeral} to the relative operator, resulting in \is{agreement!default agreement}default agreement on the relative clause predicate.

\section{Summary of the chapters}

We now provide a summary of each chapter in this volume. In the first chapter, Alexiadou \& Anagnostopoulou discuss an asymmetry between backward subject and backward \is{control!object control}object control in \ili{Greek}: backward subject control is fully productive, while backward \is{control!object control}object control is limited. They argue, following \citet{Tsakali2017}, that \is{control!backward control}backward control in \ili{Greek} is derived not through movement, but through the formation of a chain between the \is{feature!phi-feature}phi-features of the controller (and ultimately the head licensing it) and those of a functional head in the matrix clause. While a chain can be formed between matrix T and the embedded subject and T, allowing for backward subject control, chain-formation between a higher \isi{Voice}/\textit{v}Appl and the embedded subject is generally impossible, presumably because T has pronominal \is{feature!phi-feature}phi-features while \isi{Voice} does not. Backward \is{control!object control}object control is thus normally ruled out in \ili{Greek}. This restriction, however, can be overridden in cases where an \isi{experiencer} argument in the embedded clause is doubled by a dative or accusative clitic and matrix \isi{Voice} also hosts a dative or accusative clitic (i.e., in cases of ``resumption''). The authors hypothesise that this is due to a condition on Backward \isi{Agree} requiring it to apply to heads of the same type – T in the case of backward subject control; dative\slash accusative clitics in the case of backward \is{control!object control}object control.

In the same vein, Marchis Moreno focuses on backward \is{control!object control}object control, providing evidence that such control is possible in Brazilian \ili{Portuguese} because both the external and internal copies are marked with default \is{case!nominative case}nominative case; hence there is no case mismatch and no case competition. Specifically, the paper argues that the inflected \isi{infinitive} can be regarded as a diagnostic for backward \is{control!object control}object control patterns, because the percolation of default \is{case!nominative case}nominative case from the matrix T to the embedded T requires a local checking relation with an overt DP in the absence of a \isi{preposition}. The overt realization of the lower copy in \is{control!backward control}backward control is made possible by the loss of the [+person] feature. According to \citet{Cyrino2010}, the absence of the [+person] feature both in finite and non-finite domains allows nominative subjects to occupy the Spec of the inflected infinitival T, just as in finite clauses.

The relation between person and \is{case!case feature}case features constitutes the focus of Ion Giurgea’s chapter. He shows that the ``person constraint'' on \textit{se-}passives in Romanian and other \ili{Romance} languages can be accounted for on the basis of the intervening person feature associated with the external argument. Giurgea documents the crosslinguistic variation in ``\isi{impersonal}'' \textit{se} constructions in \ili{Romance} and shows that Romanian only allows a \textit{se}{}-passive construction where the verb agrees with the internal argument and the accusative cannot be assigned. Building on \citet{Cornilescu1998}, Giurgea provides additional evidence that the person constraint on \is{se-passive}\textit{se}-passives does not exclusively involve [+participant] pronouns (1\textsuperscript{st} or 2\textsuperscript{nd} person), but also affects DPs that require differential object-marking and are high on the person\slash animacy\slash definiteness hierarchy. From this, Giurgea derives an intervention-based account of passive \textit{se} according to which the person feature triggered by the external argument (syntactically projected as a null arbitrary PRO in \textit{se}{}-passives) intervenes in the case-licensing of internal arguments bearing a [Person] feature. By contrast, \textit{by}{}-phrases\is{by-phrase} do not count as interveners, as they do not have a Case to check.

Ackema \& Neeleman’s chapter discusses the feature structure of agreement and, in particular, a curious difference between person and number: while both third person and singular number may behave as defaults, third person gives rise to feature clashes that singular does not. The authors argue that this difference can be accounted for if third person has feature content while singular number does not (see also \citealt{Nevins2007,Nevins2011}). Specifically, third person is characterised by a feature \textsc{dist} that is shared with second person (which also bears \textsc{prox}, a feature shared with first person). What allows third person to act as a default is that it can deliver an empty set of referents: this follows if \textsc{dist} operates on the set of discourse referents, eliminating the speaker and \isi{addressee} and their ``associates'', leaving a subset that only optionally contains referents. As singular number lacks features imposing a cardinality on the output of the person system, it may also deliver an empty set and hence act as a default. Ackema \& Neeleman show that this difference in feature content between third person and singular number can account for cases of omnivorous \is{agreement!number agreement}number agreement in languages such as \ili{Dutch}, \ili{Icelandic} and \ili{Eastern Abruzzese}, and they argue that their account also has advantages over a locality-based \isi{Agree} account (e.g., \citealt{Preminger2014}) with respect to capturing omnivorous \is{agreement!person agreement}person agreement in languages such as \ili{Ojibwe} and \ili{Kaqchikel}. Their contribution thus bears on both the feature makeup of agreement and the morphosyntactic mechanisms that give rise to agreement.

The effects of person and number features on agreement patterns also constitute the main topic of Lorusso’s paper, which explores the patterns of agreement with progressive aspect in Apulian dialects. In many of these varieties, the present continuous is expressed through an aspectual inflected construction formed by an inflected \isi{stative} verb, an optional prepositional element and a lexical verb that either appears in a present indicative form, agreeing in person and number with the matrix verb, or in a non-agreeing infinitival form. Lorusso argues that both constructions involve a \isi{locative} derivation, but that in the inflected construction the \isi{preposition} selects a full IP, while in the \isi{uninflected} construction the \isi{preposition} selects an ``\isi{indefinite} CP' (CP\textsubscript{I} in the terms of \citealt{Manzini2003}). He uses this syntactic difference to account for a number of differences between the two constructions (e.g., placement of frequency adverbs). The inflected construction seems to involve an instance of event identification \citep{Kratzer1996} between the auxiliary and the lexical verb, and shows a number of properties in common with restructuring or serial verb constructions (e.g. clitic-climbing). By contrast, the \isi{uninflected} construction gives rise to a \isi{frequentative} reading which is not found with genuine progressive constructions \citep{Chierchia1995}, and shows properties in common with control\slash aspectual verbs. The author further describes and discusses person splits and number asymmetries that occur in the inflected construction, suggesting an analysis along the lines of \citet{Bobaljik2008Phi} and \citet{Manzini2007,Manzini2011Grammatical}.

The tight link between case and agreement proposed in Chomsky’s (\citeyear{Chomsky2000,Chomsky2001Derivation}) \isi{probe-goal} system is the focus of Mensching’s contribution. He reopens a topic that has been debated ever since \citegen{Ross1967} dissertation: how to constrain extraction from nominals. The empirical focus is on PP-extraction from DP in \ili{French}, and specifically on the question of why certain types of \textit{de}{}-PPs can be extracted from DP, while other types of \textit{de}{}-PP, along with \isi{adjunct} PPs, cannot. For example, if a DP contains both a Possessor \textit{de}{}-PP and an Agent \textit{de}{}-PP, only the Possessor can be extracted. His solution is based on \citegen{Kolliakou1999} proposal that extraction is restricted by the semantics of the \textit{de}{}-PP, which has the consequence that if there are two \textit{de}{}-PPs, only one can be an argument; the other must be an \isi{adjunct}. The argument\slash \isi{adjunct} distinction in extraction is then accounted for in terms of case-valuation: DP-internal arguments have their \is{case!case feature}case feature valued as genitive under \isi{Agree} with D, while DP-internal \is{adjunct}adjuncts do not enter into case-valuation. Given the idea that SpecDP is an ``escape hatch'' for movement that only accommodates XPs that enter an \isi{Agree} relation with D, only arguments will be able to move to SpecDP and hence out of DP. Mensching’s paper can thus be seen as an an argument in favour of the \isi{probe-goal} theory of Case and \isi{Agree} in terms of its ability to constrain extraction.

The topic of possessives is also discussed in Weingart’s paper, but from a very different perspective. Weingart shows, on the basis of a full set of clear diagnostics, that null (and simple) \is{possessive pronoun}possessive pronouns in \ili{Portuguese} have apparently contradictory properties that argue against analyses in terms of \isi{Agree} (e.g., \citealt{Hicks2009}) or Move (e.g., \citealt{Floripi2009,Rodrigues2010}), or in terms of an operation on predicates (e.g., \citealt{Reinhart2006}). Specifically, \is{pronoun!null possessive}null possessives appear to have something in between a bound variable and an indexical interpretation. Weingart thus suggests that they should be classified as logophoric \textit{pro}, and outlines a syntactic proposal, based on the semantic analysis of \citet{Partee1997}, to account for their restriction to relational nouns.

Łęska’s paper analyses the patterns of \is{agreement!subject-verb agreement}subject-verb agreement resulting from the interaction of Genitive of Quantification (GoQ) and relativisation in Polish. She shows that relative clauses modifying GoQ head nouns show distinct agreement patterns depending on whether the head noun is a subject or an object. When it is a subject, GoQ forces \is{agreement!default agreement}default agreement on the relative clause predicate (cf. \citealt{Leska2016}), but when it is an object, agreement may vary between default and full agreement, depending on the type of relative clause (introduced by \textit{który} vs. \textit{co}) and the gender of the head noun. Łęska argues that the option of \is{agreement!default agreement}default agreement is due to ``Case attraction'' \citep{Bader2006}: provided the morphological form of the relative pronoun is compatible with the case required by the \isi{numeral}, the Case feature of the \isi{quantifier} may be shared with the relative pronoun (or null operator), resulting in \is{agreement!default agreement}default agreement on the relative clause predicate. Because such extension is only seen when the head noun is a subject, however, the mechanism of \is{case!case attraction}case attraction must be restricted so that it does not overgenerate.

Gallego’s chapter focuses on dialects of \ili{Spanish} that exhibit \is{agreement!long-distance agreement}long-distance agreement between T and a DP inside a PP. Given the standard assumption that phi-probes cannot probe inside a PP in \ili{Spanish}, which is held to be responsible for the ban on preposition-stranding and \is{quantifier}pseudopassives (cf. \citealt{Law2006}), the existence of such \is{agreement!long-distance agreement}long-distance agreement is unexpected. Gallego compares this phenomenon with similar evidence concerning the differential object marker \textit{a} (e.g., \citealt{Torrego1998,López2012}), arguing that there are three types of \is{preposition}prepositions: P is merged external to TP; P is inserted at PF; P is reanalysed with V. While the differential object marker \textit{a} is plausibly of the first type, allowing T to probe the DP object directly, this and the second option are less plausible for \is{preposition}prepositions with a more ``semantic'' flavour. Gallego thus suggests that such \is{preposition}prepositions may reanalyse or incorporate with the verb, allowing the DP to be probed by T. His findings have implications for the typology of \is{preposition}prepositions in \ili{Spanish}, and more generally for the interaction of micro- and macro-parameters.

Almost all of the authors discussing the tight relation between case and agreement acknowledge that \is{case!oblique case}oblique case represents a distinct phenomenon, with no syntactic theory offering a satisfactory analysis. Manzini, Franco \& Savoia attempt to fill this gap, offering an overview of \is{case!oblique case}oblique case and a set of phenomena discussed in the typological literature under the label of ``\isi{Suffixaufnahme}''. The theoretical focus of the contribution is on the \isi{Minimalist} operation \isi{Agree} and the notion of case, specifically \is{case!oblique case}oblique case. The authors question the necessity of referring to [interpretable] and [valued] features in the formulation of \isi{Agree}. They suggest that a more primitive syntactic notion underlies the descriptive label ``oblique'', specifically that of an elementary \isi{relator} with a part\slash whole content. Thus, a DP embedded under a \is{case!genitive case}genitive case morpheme or adposition is interpreted as a possessor or ``whole'' with respect to a local superordinate DP (the possessum or ``part''). They argue that case\slash agreement-stacking in languages such as \ili{Lardil} (also discussed in Łęska’s chapter) corresponds crosslinguistically to the presence of a partial copy of this second argument within the phrasal projection of the \isi{relator}.

\section*{Acknowledgements}
This work has been supported by a grant (IF/00846/2013: ``The case of grammatical relations'') from the Fundação~para~a Ciência~e a Tecnologia (FCT).

{\sloppy\printbibliography[heading=subbibliography,notkeyword=this] }
\end{document}
