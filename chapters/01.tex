\documentclass[output=paper]{langsci/langscibook} 
\title{Default person versus default number in agreement} 
\author{Peter Ackema\affiliation{University of Edinburgh}\lastand Ad Neeleman\affiliation{UCL}}
% \chapterDOI{} %will be filled in at production

% % \epigram{Change epigram in chapters/01.tex or remove it there }

\abstract{In this paper, we compare the behaviour of the default in the person system (third person) with the default in the number system (singular). We argue, following Nevins (2007; 2011), that third person pronouns have person features, while singular DPs lack number features. The evidence for these claims comes from situations in which a single head agrees with multiple DPs that have contrasting person and number specifications. In case the number of morphological slots in which agreement can be realized is lower than the number of agreement relations established in syntax, such contrasting specification may prove problematic. As it turns out, conflicts between singular and plural do not result in ungrammaticality, but conflicts between third person and first or second person do. Such person clashes can be avoided if the morphological realization of the relevant person features is syncretic. Alternatively, languages may make use of a person hierarchy that regulates the morphological realization of conflicting specifications for person. The argument we present is rooted in, and supports, the theory of person developed in Ackema \& Neeleman (2013; to appear).}

\maketitle
\begin{document}
 
%%please move the includegraphics inside the {figure} environment
%%\includegraphics[width=\textwidth]{OGSVolumeAug2018AckemaNeeleman-img1.jpg}

 
%%please move the includegraphics inside the {figure} environment
%%\includegraphics[width=\textwidth]{OGSVolumeAug2018AckemaNeeleman-img2.jpg}

 
%%please move the includegraphics inside the {figure} environment
%%\includegraphics[width=\textwidth]{OGSVolumeAug2018AckemaNeeleman-img3.jpg}

 
%%please move the includegraphics inside the {figure} environment
%%\includegraphics[width=\textwidth]{OGSVolumeAug2018AckemaNeeleman-img4.jpg}

 
%%please move the includegraphics inside the {figure} environment
%%\includegraphics[width=\textwidth]{OGSVolumeAug2018AckemaNeeleman-img5.jpg}

\section{Introduction}

The problem addressed in this paper is an apparent paradox involving singular number and third person. On the one hand, there is evidence that in the person system the default is third person, while in the number system the default is singular. For example, dummy pronouns and verbs that fail to agree (as in impersonal passives) show up in the third person singular:

\ea 
  \ea[] It seems that a solution is hard to find.
  \ex[*] I/you/they seem(s) that a solution is hard to find.
  \z
\z

\ea Dutch \\
\gll Nog  jaren  is/*ben/*bent/*zijn              naar een oplossing gezocht.\\
  still years  be-\textsc{3sg}/be.\textsc{1sg}/be.\textsc{2sg}/be.\textsc{pl} for   a    solution   searched\\
\glt ‘People searched for a solution for many years.’
\z

On the other hand, singular agreement can be overwritten by plural agreement in certain contexts, but in those same contexts third person remains robustly in place. For example, in (3) the expected singular agreement with the subject pronoun is replaced by plural agreement if the clefted constituent is plural, but not by first person or second person agreement if the clefted constituent is a first person or second person pronoun.

\ea (\label{bkm:Ref328731088}\label{bkm:Ref294426209})  Dutch
\ea \textsc{pl} overwrites \textsc{sg}
\gll Het zijn    zij    die  de  whisky gestolen hebben.\\
    it    are.\textsc{pl} they who the whisky stolen     have\\
\glt    ‘It’s them who stole the whisky.’
\ex  1\textsuperscript{st} clashes with 3\textsuperscript{rd}\\
\gll * Het ben ik die  de  whisky gestolen heeft.  \\
    {} it     am  I  who the whisky stolen     has\\
\glt ‘It’s me who stole the whisky.’
\ex 2\textsuperscript{nd} clashes with 3\textsuperscript{rd}\\
\gll * Het ben      jij        die    de whisky gestolen heeft.\\
    {} it     are.\textsc{sg} you.\textsc{sg} who the whisky stolen     has\\
\glt ‘It’s you who stole the whisky.’
\ex No overwriting \\
\gll Het is hij die   de whisky gestolen heeft.\\
    it     is he who the whisky stolen     has\\
\glt    ‘It’s him who stole the whisky.’
\z 
\z

Nevins (2007; 2011) argues that singular is the absence of plural, while third person is not the absence of person but does in fact have a feature specification (see also \citealt{Kerstens1993}; \citealt{Halle1997}; contra \citealt{Forchheimer1953}; \citealt{Kayne1993}; Harley \& \citealt{Ritter2002}; Béjar \& \citealt{Rezac2003}; \citealt{Cysouw2003}; \citealt{Anagnostopoulou2005}; Adger \& \citealt{Harbour2007}). We agree with this (see Ackema \& \citealt{Neeleman2013}; to appear). But if there is this asymmetry between singular number and third person, the question arises how can we account for the fact that both singular and third person are defaults. This would follow naturally from the idea, rejected here, that third person, like singular, is a name for the absence of information.

In this paper we will account for the fact that the default in the person system has feature content while the default in the number system does not. We will show that our proposal captures data from various languages that involve the realization of a single agreement slot when there is agreement with multiple arguments, as in the examples in (3). The paper is organized as follows. In §2, we introduce a system of privative person features, in which third person has a specification. In §3, we introduce a system of privative number features, in which singular has no specification. We set out our theory of defaults in §4. We will argue that the default is that feature specification that allows reference to the empty set. In §5 and §6 we confront this theory with data in which multiple arguments agree with a single verbal head. §7 concludes.

\section{The person system}
Our starting in exploring the person system is a generalization about the pattern of syncretisms found in the morphological realization of person. The relevant generalization was noted by \citet[59]{BaermanEtAl2005} and Baerman \& \citet{Brown2011} and is given in (4) 

\ea (\label{bkm:Ref254353272})  1-2 and 2-3 syncretisms are far more common than 1-3 syncretisms.\z

The asymmetry expressed in (4) suggests that the system of person features is organised as in (5) (compare \citealt{Kerstens1993}; \citealt{Halle1997}; Bennis \& Mac\citealt{Lean2006}; Aalberse \& \citealt{Don2009}; 2011): 

\begin{tabularx}{\textwidth}{XXXX}
\lsptoprule
% \hhline{~---}
(\label{bkm:Ref295641111}) & \textit{First person} & \textit{Second person} & \textit{Third person}\\
% \hhline{~---} 
& [F\textsubscript{1}] & [F\textsubscript{1} F\textsubscript{2}] & [F\textsubscript{2}]\\
\lspbottomrule
\end{tabularx}

In line with this, we propose in Ackema \& \citealt{Neeleman2013} that there are two person features, \textsc{prox} and \textsc{dist}. \textsc{Prox} is shared by first and second person; \textsc{dist} is shared by second and third person. Following insights in \citealt{Harbour2016}, we interpret these features as functions. Both operate on an input set to deliver a subset as output. 

The basic input set for the person system, which we call S\textit{\textsubscript{i}}\textsubscript{+}\textit{\textsubscript{u}}\textsubscript{+}\textit{\textsubscript{o}}, contains a subset S\textit{\textsubscript{i}}\textsubscript{+}\textit{\textsubscript{u}}, which in turn contains a subset S\textit{\textsubscript{i}}. S\textit{\textsubscript{i}} contains the speaker, which we will represent as \textit{i}, and any associates of the speaker, represented as \textit{a\textsubscript{i}}. S\textit{\textsubscript{i}}\textsubscript{+}\textit{\textsubscript{u}} additionally contains the addressee(s), represented as \textit{u}, and any associates of the addressee (\textit{a\textsubscript{u}}). Finally, S\textit{\textsubscript{i}}\textsubscript{+}\textit{\textsubscript{u}}\textsubscript{+}\textit{\textsubscript{o}} contains additional members that are neither associates of the speaker nor of the addressee(s); these other members are represented as \textit{o}.\footnote{For the purposes of this paper, the difference between associates and others is irrelevant. A detailed discussion of this distinction can be found in Ackema \& Neeleman (to appear).} The only obligatory members of  S\textit{\textsubscript{i}}\textsubscript{+}\textit{\textsubscript{u}}\textsubscript{+}\textit{\textsubscript{o}} are one \textit{i} and one \textit{u}:

\ea (\label{bkm:Ref328731676}\label{bkm:Ref215574498}\label{bkm:Ref229151282}) 
\ea 
\begin{forest}
for tree={no edge,l=5mm,s sep=10mm}
[{\textbf{\textit{i}} (\textit{a\textsubscript{i}})}
         [{S\textit{\textsubscript{i}}} [
                {S\textit{\textsubscript{i}}\textsubscript{+}\textit{\textsubscript{u}}} [
                    {S\textit{\textsubscript{i}}\textsubscript{+}\textit{\textsubscript{u}}\textsubscript{+}\textit{\textsubscript{o}}} ] [] ] [] ]
          [{\textbf{\textit{u}} (\textit{a\textsubscript{u}})} [] [{(\textit{o})}]]
]
\end{forest}\todo{draw circles}
\ex \textsc{pred}(S\textit{\textsubscript{i}}\textsubscript{+}\textit{\textsubscript{u}}\textsubscript{+}\textit{\textsubscript{o}}) = S\textit{\textsubscript{i}}\textsubscript{+}\textit{\textsubscript{u}}
\ex \textsc{pred}(S\textit{\textsubscript{i}}\textsubscript{+}\textit{\textsubscript{u}}) = S\textit{\textsubscript{i}}
\ex \textsc{prox}(S) = \textsc{pred}(S)
\ex \textsc{dist}(S) = S – \textsc{pred}(S)
\z \z

The two person features are defined in terms of a function \textsc{pred} (for ‘predecessor’) given in (6b,c). \textsc{Prox}, whose definition is given in (6d), discards the outer layer of the input set; applied to S\textit{\textsubscript{i}}\textsubscript{+}\textit{\textsubscript{u}}\textsubscript{+}\textit{\textsubscript{o}} it delivers S\textit{\textsubscript{i}}\textsubscript{+}\textit{\textsubscript{u}}. \textsc{Dist}, whose definition is given in (6e), selects the outer layer; applied to S\textit{\textsubscript{i}}\textsubscript{+}\textit{\textsubscript{u}}\textsubscript{+}\textit{\textsubscript{o}} it delivers S\textit{\textsubscript{i}}\textsubscript{+}\textit{\textsubscript{u}}\textsubscript{+}\textit{\textsubscript{o}} [F02D?] S\textit{\textsubscript{i}}\textsubscript{+}\textit{\textsubscript{u}}.

We now consider how first, second and third person readings are derived, starting with the singular. The specification of the third person singular is straightforward: it should be [\textsc{dist}], as this feature will give S\textit{\textsubscript{i}}\textsubscript{+}\textit{\textsubscript{u}}\textsubscript{+}\textit{\textsubscript{o}} [F02D?] S\textit{\textsubscript{i}}\textsubscript{+}\textit{\textsubscript{u}}, a set that excludes the speaker and any addressees. 

 The first person singular is derived by two applications of \textsc{prox}. It first applies to S\textit{\textsubscript{i}}\textsubscript{+}\textit{\textsubscript{u}}\textsubscript{+}\textit{\textsubscript{o}}, delivering S\textit{\textsubscript{i}}\textsubscript{+}\textit{\textsubscript{u}}; it then applies to the latter set, delivering S\textit{\textsubscript{i}}. The only obligatory member of S\textit{\textsubscript{i}} is the speaker, yielding the correct interpretation in the singular:

\ea 
    \textsc{prox}(\textsc{prox}(S\textit{\textsubscript{i}}\textsubscript{+}\textit{\textsubscript{u}}\textsubscript{+}\textit{\textsubscript{o}}))    

    = \textsc{prox}(S\textit{\textsubscript{i}}\textsubscript{+}\textit{\textsubscript{u}})        by (6d)

    = S\textit{\textsubscript{i}}          by (6d)
\z

The second person singular is generated by applying both \textsc{prox} and \textsc{dist}. \textsc{Prox} is applied first, so that S\textit{\textsubscript{i}}\textsubscript{+}\textit{\textsubscript{u}} is selected. Applying \textsc{dist} to this set removes S\textit{\textsubscript{i}}, leaving a set with \textit{u} as the only obligatory member:

\ea 
     \textsc{dist}(\textsc{prox}(S\textit{\textsubscript{i}}\textsubscript{+}\textit{\textsubscript{u}}\textsubscript{+}\textit{\textsubscript{o}}))    

    = \textsc{dist}(S\textit{\textsubscript{i}}\textsubscript{+}\textit{\textsubscript{u}})        by (6d)

    = S\textit{\textsubscript{i}}\textsubscript{+}\textit{\textsubscript{u}} [F02D?] S\textit{\textsubscript{i}}        by (6e)

    = S\textit{\textsubscript{u}}
\z



Note that the opposite order of function application (first \textsc{dist}, then \textsc{prox}) is not coherent. \textsc{Dist} applied to S\textit{\textsubscript{i}}\textsubscript{+}\textit{\textsubscript{u}}\textsubscript{+}\textit{\textsubscript{o}} yields S\textit{\textsubscript{i}}\textsubscript{+}\textit{\textsubscript{u}}\textsubscript{+}\textit{\textsubscript{o}} [F02D?] S\textit{\textsubscript{i}}\textsubscript{+}\textit{\textsubscript{u}}. But as this set is not layered, \textsc{prox} cannot apply to it.

We assume that the ‘person space’ in (6a) is introduced by a node we refer to as N\textsubscript{$\Pi $}. Person features are introduced in a \textsc{prs} node that selects N\textsubscript{$\Pi $}. The basic semantics of this node is the identity function [F06C?]P.P, but this specification can be enriched through function composition if \textsc{prox} and/or \textsc{dist} are added. The order of function application is reflected in syntax. The notation we use for this is borrowed from feature geometry (Gazdar \& \citealt{Pullum1982}; Harley \& \citealt{Ritter2002}): features representing functions applied later are dominated by features representing functions applied earlier:

 
%%please move the includegraphics inside the {figure} environment
%%\includegraphics[width=\textwidth]{OGSVolumeAug2018AckemaNeeleman-img6.jpg}

(\label{bkm:Ref254355362})

We now turn to plural pronouns. For now, we assume that number is encoded through an \textsc{nmb} node, which is merged above \textsc{prs} and which can host a feature \textsc{pl} (but see §3). If this feature is present, the cardinality of the output set of the person system must be larger than one.

In the second and third person, the person specification in the plural is the same as the person specification in the singular. In the first person, however, there are two options. Suppose that the plural feature is simply added to the singular form in (6a), where \textsc{prox} is applied twice. This delivers S\textit{\textsubscript{i}}, a set containing the speaker and in the plural also any contextually given associates, but no addressee. The result is an exclusive first person pronoun. Another option is to apply \textsc{prox} only once. This delivers S\textit{\textsubscript{i}}\textsubscript{+}\textit{\textsubscript{u}}, a set containing the speaker, at least one addressee, and any associates. The resulting pronoun is a first person inclusive:

(\label{bkm:Ref328732770}\label{bkm:Ref254355366})

 
%%please move the includegraphics inside the {figure} environment
%%\includegraphics[width=\textwidth]{OGSVolumeAug2018AckemaNeeleman-img7.jpg}

Note that the option of applying \textsc{prox} only once in the first person is incompatible with a singular reading. Such a derivation has as its output S\textit{\textsubscript{i}}\textsubscript{+}\textit{\textsubscript{u}}, a set with two obligatory members.

The system just outlined exhausts the feature structures made available by the person system. No structures other than those in (9) and (10) deliver an interpretable output. Consider why. Both \textsc{prox} and \textsc{dist} require a layered input set. Given that S\textit{\textsubscript{i}}\textsubscript{+}\textit{\textsubscript{u}}\textsubscript{+}\textit{\textsubscript{o}} has only three layers, the number of possible feature combinations is restricted. If \textsc{dist} is applied first, this delivers an unstructured set (S\textit{\textsubscript{i}}\textsubscript{+}\textit{\textsubscript{u}}\textsubscript{+}\textit{\textsubscript{o}} [F02D?] S\textit{\textsubscript{i}}\textsubscript{+}\textit{\textsubscript{u}}), and hence neither \textsc{prox} nor \textsc{dist} can apply subsequently. If \textsc{prox} is applied first, the output is a layered set (S\textit{\textsubscript{i}}\textsubscript{+}\textit{\textsubscript{u}}). This leaves open three possibilities: (i) \textsc{prox} applies again, which yields an unstructured set (S\textit{\textsubscript{i}})), or (ii) \textsc{dist} applies, which again yields an unstructured set (S\textit{\textsubscript{i}}\textsubscript{+}\textit{\textsubscript{u}} [F02D?] S\textit{\textsubscript{i}}), or (iii) neither \textsc{prox} nor \textsc{dist} applies, which delivers the first person inclusive.

As a result, the following generalizations about person distinctions expressed in pronouns follow (adapted from \citealt{Bobaljik2008}):

\ea 
    \ea  No language distinguishes pronouns expressing \textit{i+i} and \textit{i+a\textsubscript{i}}.
    \ex No language distinguishes pronouns expressing \textit{u+u} and \textit{u+a\textsubscript{u}}.
    \ex No language distinguishes pronouns expressing \textit{i}+\textit{i}+\textit{u}, \textit{i}+\textit{u}+\textit{u} and \textit{i}+\textit{u}+\textit{a\textsubscript{i}}\textsubscript{/}\textit{\textsubscript{u}}.
    \z
\z



In the system just outlined, the first person (inclusive or exclusive) does not form a natural class with the third person to the exclusion of the second person. Similarly, the first person inclusive does not form a natural class with the second person to the exclusion of the first person exclusive. This is relevant in view of the results of a large-scale study reported in \citealt{Harbour2016}. Harbour looked at which systematic patterns of syncretism are attested cross-linguistically, where a systematic pattern of syncretism is a syncretism characteristic of all paradigms of a given language. He found that no language had a systematic syncretism for first and third person, or for first person inclusive and second person. On the assumption that the distribution of systematic syncretisms reflects the underlying distribution of features, this shows that no set of features is shared uniquely by the relevant combinations of persons.

The absence of systematic syncretisms for first person inclusive and second person is line with a typological generalization discussed by \citet{Zwicky1977}. Zwicky argues that in languages that lack the distinction between inclusive and exclusive first person pronouns, the inclusive reading is systematically expressed by the first person, rather than the second person plural pronoun – this despite the fact that the inclusive reading covers both speaker and addressee. An account for this observation would be impossible if first person inclusive and the second person did form a natural class to the exclusion of the first person exclusive.\footnote{Strictly speaking, in order to capture Zwicky’s generalization, not only the syntactic feature system, but also the system of morphological realization (spell out) must be considered. In fact, there is a way of constructing grammars that violate the generalization in our system, namely by impoverishment of \textsc{dist} in in the context of both \textsc{pl} and \textsc{prox (}so in the second person plural). In a language that has distinct spell-out rules that apply to the feature structures [\textsc{prox]} and [\textsc{prox–prox}], this will create a formal opposition between first person exclusive on the one hand, and first person inclusive and second person on the other. Interestingly, Sanuma appears to have a pronominal spell-out system of this type (see \citealt{Borgman1990}:149 and \citealt{Simon2005}:127; see Perri \citealt{Ferreira2013} for critical discussion of Borgman’s observations). However, in the absence of the particular set of circumstances described above, we expect Zwicky’s generalization to hold, and we therefore expect it to be valid at least as a statistical universal.}

For the purposes of this paper, the main characteristic of our person system is that third person has a person specification, namely [\textsc{dist}].  We should note that this does not mean that there are no pronouns that lack person features. One would expect there to be such pronouns, especially in an analysis based on privative features. In Ackema and Neeleman (to appear), we argue that a particular type of generic pronoun should be analyzed in this way (see also \citealt{Egerland2003} and D’\citealt{Alessandro2007}). English \textit{one}, West Frisian \textit{men} \citep{Hoekstra2010} and Icelandic \textit{maður} (Sigurðsson \& \citealt{Egerland2009}) are examples: in the absence of person features, the generic operator contained in them ranges over the entire person space (S\textit{\textsubscript{i}}\textsubscript{+}\textit{\textsubscript{u+o}}).

 
%%please move the includegraphics inside the {figure} environment
%%\includegraphics[width=\textwidth]{OGSVolumeAug2018AckemaNeeleman-img8.jpg}

(\label{bkm:Ref295390383})

\section{The number system}

We now turn to the number system. We will argue that, like the person system, it is based on privative features that are interpreted as functions. We will show that in this system there cannot be a feature that encodes singularity. Rather, singular is one of the interpretations that results from the absence of a number feature specification.

  In languages that make a distinction between inclusive and exclusive first person pronouns, two types of number system are found. The difference between these systems involves the interpretation of number in the inclusive. In what we will call absolute number systems, the inclusive is always marked as either dual or plural. Maori provides an example: 

(\label{bkm:Ref328732387})    Maori pronouns

\begin{tabularx}{\textwidth}{XXXXX} 
\lsptoprule
&  & Singular [ ] & Plural [\textsc{pl}] & Dual [\textsc{pl} \textsc{min}]\\
% \hhline{~----} 
& 1 inclusive & {}- & t\=a-ua & t\=a-tou\\
& 1 exclusive & au & \=a-ua & m\=a-tou\\
& 2 & koe & k\=or-ua & kou-tou\\
& 3 & ia & r\=a-ua & r\=a-tou\\
\lspbottomrule
\end{tabularx}

As indicated, absolute number systems can in principle be analyzed using two features, \textsc{pl} (for ‘plural’) and \textsc{min} (for ‘minimal’), which we take to be hosted by a dedicated functional head \textsc{nmb}. \textsc{Pl} encodes that the cardinality of the set referred to, which we will represent as \textit{n}, exceeds 1 (\textit{n}>1). \textsc{Min} selects the minimal plural (\textit{n}=2).

There is a second type of number system, however, which we will refer to as a relative number system. In such a system, the interpretation of number marking seems dependent on person, with a shift in the inclusive that is absent in the other persons. In particular, the inclusive pronoun need not be inflected for number. If it is, its cardinality is larger than two, whereas in other pronouns, number marking implies a cardinality larger than one. The Rembarrnga paradigm in (14) illustrates the point:

(\label{bkm:Ref451950792})    Rembarrnga pronouns

\begin{tabularx}{\textwidth}{XXXXXX} &  & Singular & Plural & Dual & Trial\\
\lsptoprule
% \hhline{~-----}
& 1 inclusive & {}- & yukku & ngakorru & ngakorr-bbarrah\\
& 1 exclusive & ngunu & yarru & yarr-bbarrah & \\
& 2 & ku & nakorru & nakorr-bbarrah & \\
& 3 & nawu/ngadu & barru & barr-bbarrah & \\
\lspbottomrule
\end{tabularx}
Such number systems are typically analyzed using the \textsc{min} feature already mentioned and – instead of \textsc{pl} – a feature \textsc{aug} for ‘augmented’ (see \citealt{Bobaljik2008} and \citealt{Cysouw2011}, and references mentioned there). \textsc{Aug} indicates that \textit{n} is larger than the minimal cardinality allowed by the person system. Except in the inclusive, the minimal cardinality allowed by the person system is one, and so \textsc{aug} delivers \textit{n}>1. In the inclusive, however, the minimal cardinality allowed by the person system is two, so \textsc{aug} delivers \textit{n}>2. On this analysis, the Rembarrnga paradigm looks much more elegant:

(\label{bkm:Ref328732397}\label{bkm:Ref295309335})    Rembarrnga pronouns

\begin{tabularx}{\textwidth}{XXXXX}
\lsptoprule
 &  & Non-aug. [ ] & Augmented [\textsc{aug}] & Unit-augmented [\textsc{aug} \textsc{min}]\\
% \hhline{~----}
& 1 inclusive & yukku & ngakorru & ngakorr-bbarrah\\
& 1 exclusive & ngunu & yarru & yarr-bbarrah\\
& 2 & ku & nakorru & nakorr-bbarrah\\
& 3 & nawu/ngadu & barru & barr-bbarrah\\
\lspbottomrule
\end{tabularx}

If we were to accept both the feature systems in (13) and (15), the resulting proposal would model parametric variation between absolute and relative number systems as a choice between features (\textsc{pl} versus \textsc{aug}). However, this would make the parametrization of the number system something of an oddity. Our impression is that in other cases where feature systems are parametrized, languages select more or fewer features from a fixed inventory, rather than choosing between features that cannot co-occur in the same grammar. We propose to fix this problem by assuming that \textsc{aug} is universal and that \textsc{pl} does not exist. However, the effects of \textsc{aug} are dependent on information from the person system. If \textsc{aug} has no access to the person system, then its interpretation defaults to the interpretation normally assumed for \textsc{pl}. This idea can be worked out as follows.

The input set for the number system is $\mathbb{N}$. The features \textsc{aug} and \textsc{min} select a subset from $\mathbb{N}$ in accordance with the definitions in (16a,b). The cardinality of the set delivered by the person system must be an element of this subset. 

\ea 
\ea (\label{bkm:Ref234211253})   \textsc{aug}(S)  =  S’,  S’ ${\subseteq}$ S,  \textit{n} ${\in}$ S’ $\Leftrightarrow $ \textit{n} > \textit{n}\textsubscript{R}
\ex  \textsc{min}(S)  =  S’,  S’ ${\subseteq}$ S,  \textit{n} ${\in}$ S’ $\Leftrightarrow $ \textit{n} > 0 ${\wedge}$ ${\nexists}$\textit{n}’, \textit{n}’ ${\in}$ S ${\wedge}$ \textit{n}’ < \textit{n}
\z
\z

As indicated in (16a), \textsc{aug} refers to a reference number \textit{n}\textsubscript{R}, whose value is determined by the following procedure (S\textsubscript{person} is the output of the person system):

\ea (\label{bkm:Ref453320625})
\ea \textit{n}\textsubscript{R} = \textit{n}\textsubscript{person} iff \textit{n}\textsubscript{person} is accessible and \textit{n}\textsubscript{person} > 0; otherwise \textit{n}\textsubscript{R} = 1
\ex \textit{n}\textsubscript{person} = {\textbar}strip(S\textsubscript{person}){\textbar}
\ex strip(S\textsubscript{person})  =  S’,  S’ ${\subseteq}$ S\textsubscript{person},  \textit{p} ${\in}$\{\textit{i}, \textit{u}\} $\Leftrightarrow $ \textit{p} ${\in}$ S’
\z
\z

The accessibility of person information depends on the functional structure of the pronoun. We assume, following \citet{Platzack1983} and others, that there is parametric variation in whether certain functional heads project separately or conflate and project together. Applied to \textsc{nmb} and \textsc{prs}, this gives the possible structures for pronouns in (18).

\ea
\ea
\begin{forest} for tree={font=\scshape}
[nmb [nmb] [prs [prs] [n\textsubscript{Π}]]]
\end{forest}
\ex
\begin{forest} for tree={font=\scshape}
[nmb\slash prs [nmb\slash prs] [n\textsubscript{Π}]]
\end{forest}
\z
\z\todo{two column mode}


%%please move the includegraphics inside the {figure} environment
%%\includegraphics[width=\textwidth]{OGSVolumeAug2018AckemaNeeleman-img9.jpg}

Our hypothesis is that \textit{n}\textsubscript{person} is accessible to \textsc{aug} if and only if \textsc{nmb} and \textsc{prs} conflate, so that \textsc{aug} is located in the same node as the person features that deliver S\textsubscript{person}. Given the definitions in (17), this means that only in (18b) can \textit{n}\textsubscript{R} assume a value other than 1.

  Consider how this plays out in absolute and relative number systems, respectively. The situation in absolute number systems is straightforward, as \textit{n}\textsubscript{R} is always 1 (by default, as \textsc{aug} has no access to person information):

\ea 
 \gll \\
   \\
 \glt
\z

    \textit{Absolute number system} – (18a)

\begin{itemize}
\item \textit{n}\textsubscript{R} = 1 (by default)
\item \textsc{nmb}–\textsc{aug}: \textit{n} > 1
\item \textsc{nmb}–\textsc{aug}–\textsc{min}: \textit{n} = 2
\end{itemize}

In relative number systems, \textsc{aug} does have access to the person system, which means that \textit{n}\textsubscript{R} varies depending on person, along the following lines:

\ea \textit{Relative number system} – (18b)\\
\ea First person inclusive:\\

\begin{itemize}
\item \textit{n}\textsubscript{person}  =  {\textbar}strip(\{\textit{i}, \textit{a}\textsubscript{i}+, \textit{u}, \textit{a}\textsubscript{u}+\}){\textbar}  =  {\textbar}\{\textit{i}, \textit{u}\}{\textbar}  =  2
\item \textit{n}\textsubscript{R}  =  \textit{n}\textsubscript{person}  =  2
\item \textsc{nmb}–\textsc{aug}: \textit{n} > 2
\item \textsc{nmb}–\textsc{aug}–\textsc{min}: \textit{n} = 3
\end{itemize}

  \ex First person exclusive: \\

\begin{itemize}
\item \textit{n}\textsubscript{person}  =  {\textbar}strip(\{\textit{i}, \textit{a}\textsubscript{i}+\}){\textbar}  =  {\textbar}\{\textit{i}\}{\textbar}  =  1
\item \textit{n}\textsubscript{R}  =  \textit{n}\textsubscript{person}  =  1
\item \textsc{nmb}–\textsc{aug}: \textit{n} > 1
\item \textsc{nmb}–\textsc{aug}–\textsc{min}: \textit{n} = 2
\end{itemize}

  \ex Second person: \\

\begin{itemize}
\item \textit{n}\textsubscript{person} = {\textbar}strip(\{\textit{u}, \textit{a}\textsubscript{u}+\}){\textbar}  =  {\textbar}\{\textit{u}\}{\textbar}  =  1
\item \textit{n}\textsubscript{R}  =  \textit{n}\textsubscript{person}  =  1
\item \textsc{nmb}–\textsc{aug}: \textit{n} > 1
\item \textsc{nmb}–\textsc{aug}–\textsc{min}: \textit{n} = 2
\end{itemize}

  \ex Third person:\\

\begin{itemize}
\item \textit{n}\textsubscript{person} = {\textbar}strip(\{o+\}){\textbar}  =  {\textbar}\{ \}{\textbar}  =  0
\item \textit{n}\textsubscript{R}  =  1 (by default)
\item \textsc{nmb}–\textsc{aug}: \textit{n} > 1
\item \textsc{nmb}–\textsc{aug}–\textsc{min}: \textit{n} = 2
\end{itemize}
\z
\z

When the semantics of number in (18b) is computed, the value of \textit{n}\textsubscript{person} is accessible to \textsc{aug}, because \textsc{prs} is part of the same terminal node. This has an effect for the interpretation of number in the first person inclusive. Since applying \textsc{prox} once delivers a set with \textit{i} and \textit{u} as obligatory members (see (10a)), \textit{n}\textsubscript{R} = \textit{n}\textsubscript{person} = 2 here. The consequence is that \textsc{aug} requires that \textit{n} > 2. When the semantics of the terminal containing \textsc{aug} in the structures in (18a) is computed, however, the value of \textit{n}\textsubscript{person} is not accessible, because [\textsc{prs}–\textsc{prox}] is generated in a sister node. This means that \textit{n}\textsubscript{R} assumes its default value of 1, also in the first person inclusive, so that \textsc{aug} now requires that \textit{n} > 1.

Our analysis makes a crucial prediction about the morphological form of pronominal number. In absolute systems, plural can be either agglutinative or fusional. If the terminals introducing person and number are spelled out separately, an agglutinative number paradigm will emerge; if spell-out targets a string of terminals or a non-terminal node (on a par with \{go past\} [F0DB?] \textit{went}), the number morphology will be fused with the person morphology. If person and number are introduced in the same terminal, however, as is the case in relative systems, they \textit{must} be fusional (there is no position in which a distinct number morpheme could be anchored).\footnote{This is under the assumption that an operation like fission, as used in Distributed Morphology (see Halle \& \citealt{Marantz1993} and \citealt{Noyer1997}), either does not exist or must give rise to instances of multiple exponence, which is not at issue here.} We predict, then, that if number marking is agglutinative in pronouns, the number system must be of the absolute type. This prediction appears to be confirmed by the discussion in Cysouw (2003: 89, 263), where it is noted that languages that have a relative number system and are agglutinative for \textsc{aug} are extremely rare, if they exist at all (see also \citealt{Greenberg1988}).

Note that it \textit{is} possible for a relative number system to be agglutinating for \textsc{min}, as \textsc{min} need not have access to person information, but only to the output of \textsc{aug}. Hence, a language can have an interpretable structure in which \textsc{nmb} and \textsc{prs} are partially conflated, as in (21).

\ea (\label{bkm:Ref453322239})
\begin{forest} for tree={font=\scshape}
[nmb
    [nmb
        [min]
    ]
    [nmb\slash prs
        [nmb\slash prs
            [aug]
        ]
        [n\textsubscript{Π}]
    ]
]
\end{forest}
\z
 
%%please move the includegraphics inside the {figure} environment
%%\includegraphics[width=\textwidth]{OGSVolumeAug2018AckemaNeeleman-img10.jpg}

Languages with a relative number system that have agglutinative morphology for \textsc{min} indeed exist; the Rembarrnga paradigm in (15) provides an example.

  In sum, the \textsc{aug} feature is shared by all number systems, but its interpretive effects depend on whether or not it has access to information delivered by the person features, which in turn depends on the syntactic structure of pronouns. Notice that in this system singular and non-augmented must both equal the absence of \textsc{aug}. There cannot be a contentful privative feature that characterizes singular and non-augmented number, given that the interpretation of these numbers as \textit{n}=1 or \textit{n}=2 is determined fully by the interpretation of \textsc{aug}. Therefore, the default in the number system is characterized by the absence of a feature specification. 

\section{Defaults}

If we are correct in assuming that singular is a non-number, while third person has a feature specification, the question arises why both are defaults. In order to address this question, we must first consider what a default is. There are several views of this; the following three are probably the most common. (i) Defaults are the most frequent forms. It is not clear what insight that can provide here. (ii) Defaults correspond to absence of features. This is an attractive idea, but it cannot work on our view of person, as the third person has feature content. (iii) Defaults correspond to feature structures that do not force an interpretation. This is the view we will defend.

  Our core assumption is that only if a $\varphi $-feature structure may denote an empty set can it fail to be interpreted, and hence act as default. In the person system, [\textsc{dist}] is the only feature structure that can deliver an empty set. \textsc{Dist} selects the outer layer in (6), discarding the only obligatory members of S\textit{\textsubscript{i}}\textsubscript{+}\textit{\textsubscript{u}}\textsubscript{+}\textit{\textsubscript{o}}, speaker and addressee. As \textit{o} is optional, [\textsc{dist}] may deliver an empty set. All other specifications deliver a set that contains either \textit{i} or \textit{u} or both and can therefore not act as a default. This holds, even, for a specification in which \textsc{prs} does not contain person features, as this delivers a generic impersonal pronoun that ranges over the entire S\textit{\textsubscript{i}}\textsubscript{+}\textit{\textsubscript{u}}\textsubscript{+}\textit{\textsubscript{o}} input set, see (12).

In the number system, [ ] is the only feature structure that can deliver an empty set. [\textsc{aug}] and [\textsc{aug}–\textsc{min}] impose a positive cardinality on the output of the person system. However, [ ] does not, and is therefore compatible with a cardinality of 0 in both absolute and relative number systems, regardless of person specification. 

\section{Multiple agreement, single spell-out}

We have argued that third person has a feature specification, as opposed to singular number, and explained why nevertheless both can function as defaults. We now show how the asymmetry in feature specification plays out in agreement.

  \citet{Nevins2011} discusses so-called omnivorous number systems, in which a verb shows plural agreement when either subject or object is plural (see (22)).

\ea (\label{bkm:Ref295394739}\label{bkm:Ref295395935})  Eastern Abruzzese (D’Alessandro \& \citealt{Roberts2010})\\
\ea \gll Giuwanne a    pittate        nu mure.\\
    John         has painted.\textsc{sg} a   wall\\  
\ex \gll Giuwanne e     Mmarije a      pittite        nu mure.\\
    John          and Mary     have painted.\textsc{pl} a  wall \\
\ex \gll  Giuwanne a    pittite        ddu mure. \\
    John         has painted.\textsc{pl} two walls \\
\ex \gll Giuwanne e      Mmarije a      pittite        ddu mure. \\
    John          and Mary      have painted.\textsc{pl} two walls \\
\z
\z
    
Like Nevins, we assume that data like (22) involve multiple agreement. We further assume that this leads to a situation in which one morpho-phonological agreement slot must realize two distinct feature bundles: 

\ea 
  \ea  DP\textsubscript{1} … V-$\varphi $\textsubscript{1}{}-$\varphi $\textsubscript{2} … DP\textsubscript{2}
  \ex  V-$\varphi $\textsubscript{1}{}-$\varphi $\textsubscript{2} ⟺ /V/-/affix/
\z
\z

In general, where one form realizes two feature bundles either unification is necessary or arbitration by rules of resolution. We begin by discussing unification. In the next section, we will discuss resolution rules.

We assume that unification is either unification of sets of syntactic feature structures or of phonological forms. The syntactic unifications relevant to the data in (22) are given below. These can all be realized without difficulty, as a singular form in (24a) and a plural form in (24b-d): 
  
\ea (\label{bkm:Ref295571427})
\ea V-[ ]\textsubscript{1}{}-[ ]\textsubscript{2} ⟶ V-[ ]\textsubscript{1+2}
\ex V-[\textsc{aug}]\textsubscript{1}{}-[ ]\textsubscript{2} ⟶ V-[\textsc{aug}]\textsubscript{1+2}
\ex V-[ ]\textsubscript{1}{}-[\textsc{aug}]\textsubscript{2} ⟶ V-[\textsc{aug}]\textsubscript{1+2}
\ex V-[\textsc{aug}]\textsubscript{1}{}-[\textsc{aug}]\textsubscript{2} ⟶ V-[\textsc{aug}]\textsubscript{1+2}
\z
\z

Given that third person is different from singular in that it does have feature content, syntactic unification in parallel cases involving person can result in feature bundles with multiple person specifications:

\ea (\label{bkm:Ref328733186}\label{bkm:Ref295571842}\label{bkm:Ref295397399}\label{bkm:Ref298060525})
\ea V-[\textsc{dist}]\textsubscript{1}{}-[\textsc{dist}]\textsubscript{2} ⟶ V-[\textsc{dist}]\textsubscript{1+2}
\ex V-[\textsc{dist}]\textsubscript{1}{}-[\textsc{prox} (…)]\textsubscript{2} ⟶ V-[\textsc{dist prox} (…)]\textsubscript{1+2}
\z
\z

While realization of the output in (25a) is unproblematic, the feature specification in (25b) makes spell-out impossible, on the assumption that the process is blocked if a single agreement slot contains multiple feature bundles for the same class of $\varphi $-features.\footnote{Note that there is a fundamental difference between the feature specification [\textsc{dist prox}] in (25b) on the one hand and the feature specification [\textsc{prox–dist}] (second person) on the other. The former contains two (simplex) feature bundles (for third and first person), with the result that spell-out is blocked.} This means that where the input contains conflicting person specifications, spell-out cannot proceed on the basis of syntactic unification. Instead, phonological unification is necessary. Hence the structure in (25b) can be realized only if the spell-out rules for [\textsc{dist}] and [\textsc{prox} (…)] deliver the same phonological form:

\ea (\label{bkm:Ref295480264})  
\ea  \{\textsc{dist}\} [F0DB?] /aaa/
\ex  \{\textsc{prox} (…)\} [F0DB?] /aaa/
\ex V-[\textsc{dist}]\textsubscript{1}{}-[\textsc{prox} (…)]\textsubscript{2} [F0DB?] /V/-/aaa/
\z \z

There are other situations in which a derivation converges if a single phonological element can realize multiple conflicting syntactic feature bundles; an example involves case morphology on free relatives in German, see Groos \& Van \citealt{Riemsdijk1981}.

We will now discuss instances of (25) and (26). In particular, we will consider two structures in which a low DP must have the same person specification as imposed on the verb by the subject in a double agreement structure.\footnote{In contrast, there are no similar cases in which a low DP must have the same number specification as the subject. This follows from the fact that singular is absence of number features. \citet{Nevins2011} proposes an analysis of relevant person-number contrasts along similar lines. His account assumes that the person system is built on bivalent features, while features in the number system are privative, with singular lacking number. The above preserves the insights of Nevins’ proposal while avoiding this duality of design. Both the person system and the number system have privative features, and there is a principled reason why singular is featureless while third person has content.} One is the Dutch cleft construction already introduced in (3). The other involves the well-known case of nominative objects in Icelandic. Let us start with the latter.

Agreement with nominative objects in Icelandic is possible when the subject carries quirky case. However, such agreement is usually impossible with first or second person objects:\footnote{D’\citet{Alessandro2007} shows that impersonal \textit{si} constructions in Italian behave in a fashion parallel to the Icelandic examples discussed below: \textit{si} triggers default third person singular agreement, and when the object is nominative the verb agrees in number with it. Crucially, in the latter case the object cannot be first or second person. Any adequate analysis proposed for Icelandic can therefore be extended to Italian impersonal constructions, as indeed argued by D’Alessandro.}

\ea
(\label{bkm:Ref295642002})  Icelandic (Sigurðsson \& \citealt{Holmberg2008})\\
\ea
\gll  *Honum   líkum     við.\\  
    him.\textsc{dat} like-\textsc{1pl} we.\textsc{nom}\\ 
\glt    ‘He likes us.’
\ex 
\gll *Honum  líkið       þið.\\
    him.\textsc{dat} like-\textsc{2pl} you.\textsc{pl}.\textsc{nom}\\
\glt    ‘He likes you all.’
\ex
\gll Honum  líka        þeir.\\
     him.\textsc{dat} like-\textsc{3pl} they.\textsc{nom}\\
\glt    ‘He likes them.’
\z
\z

We follow a strand in the literature according to which the verb agrees with both the quirky subject and the nominative object (see \citealt{Burzio2000}, Schütze 2003, and \citealt{Ussery2013}). Thus, Icelandic agreement is regulated by two rules: (i) agree with the subject; (ii) agree with nominatives. Non-nominative DPs trigger default third person singular agreement, presumably because they differ from nominatives in having a Case shell which prevents access to their $\varphi $-features. Therefore, quirky subjects behave just like other categories that lack $\varphi $-features, such as clausal subjects. Indeed, in examples with a quirky subject in which the object is not nominative, the verb must carry third person singular inflection:

\ea 
 \gll    Mig      hefur/*hef/*hafa     vantað mýts.  (Schütze 2003)\\
         me.\textsc{acc} has-\textsc{3sg}/*\textsc{1sg}/*\textsc{3pl} lacked mice.\textsc{acc}\\
 \glt    ‘I have lacked mice.’ 
\z

Structures like those in (27), which involve agreement with both a quirky subject and a nominative object, will then have a verb that carries two distinct $\varphi $-feature bundles, one of which will be [\textsc{dist}] (KP stands for ‘Case Phrase’, in this structure the quirky subject):

\ea (\label{bkm:Ref295466963})  KP\textsubscript{1} … V-[\textsc{dist}]\textsubscript{1}{}-$\varphi $\textsubscript{2} … DP\textsubscript{2} \z

Whether or not (29) can be realized depends on the content of $\varphi $\textsubscript{2}. Consider the various possibilities listed in (30).

\ea (\label{bkm:Ref295570245}\label{bkm:Ref295481259})  
\ea  KP\textsubscript{1} … V-[\textsc{dist}]\textsubscript{1}{}-[\textsc{dist}]\textsubscript{2} … DP\textsubscript{2}
\ex  KP\textsubscript{1} … V-[\textsc{dist}]\textsubscript{1}{}-[\textsc{dist} \textsc{aug}]\textsubscript{2} … DP\textsubscript{2}
\ex  KP\textsubscript{1} … V-[\textsc{dist}]\textsubscript{1}{}-[\textsc{prox} (…)]\textsubscript{2} … DP\textsubscript{2}
\z
\z

Syntactic unification of feature bundles applied to these structures yields the following:

\ea (\label{bkm:Ref295480997})  
\ea KP\textsubscript{1} … V-[\textsc{dist}]\textsubscript{1+2} … DP\textsubscript{2}
\ex KP\textsubscript{1} … V-[\textsc{dist} \textsc{aug}]\textsubscript{1+2} … DP\textsubscript{2}
\ex KP\textsubscript{1} … V-[\textsc{dist prox} (…)]\textsubscript{1+2} … DP\textsubscript{2}
\z
\z

The feature bundles in (31a) and (31b) are unproblematic as far as spell-out is concerned. The feature bundle in (31c) is not, however, as it contains contradictory values for person. This means that spell-out must proceed on the basis of the non-unified structure in (30c). But that will only meet the condition that there be a single affix if phonological unification is possible, which is only the case if the phonological realization of [\textsc{dist}]\textsubscript{1} is identical to the phonological realization of [\textsc{prox} (…)]\textsubscript{2}.

Indeed, Sigurðsson (1996) observes that the person restriction on object agreement is lifted (for many speakers) when the first/second person form of the verb is syncretic with the third person form:\footnote{Note that the fact that syncretism prevents the problem with conflicting person features indicates that the solution should not be sought in syntax proper. This rules out a number of accounts that attempt to deal with such data in terms of an intervention effect, such as Sigurðsson \& \citealt{Holmberg2008}. While the relevant syncretism in Icelandic is a relatively rare phenomenon, we will see below that in a similar situation in Dutch clefts, syncretism indeed systematically ameliorates person clashes. An analysis should therefore not centre on a putative problem with syntactically establishing the agreement relation(s) in question, but on a problem with how these relations are expressed on the verb.}

\ea 
 \gll \\
   \\
 \glt
\z

\ea \ea bored.at-3sg [F0DB?] /leiddist/  (Sigurðsson 1996)
\ex \gll  *Henni    leiddumst      við. \\
      her.\textsc{dat} bored.at-1\textsc{pl} we.\textsc{nom} \\
\ex\gll \%Henni   leiddust         þið. \\
       her.\textsc{dat} bored.at-2\textsc{pl} you-\textsc{pl.nom} \\
\ex\gll  ?Henni    leiddist          ég. \\
      her.\textsc{dat} bored.at-1\textsc{sg} I.\textsc{nom} \\
\ex\gll  ?Henni    leiddist          þú. \\
      her.\textsc{dat} bored.at-2\textsc{sg} you-sg.\textsc{nom}\\
\z \z

Agreement with lower nominative DPs does not only occur in mono-clausal, but also in bi-clausal structures with a raising verb. In such structures, the same person restriction is observed as in mono-clausal structures (see (33)). 

\ea (\label{bkm:Ref295483138}) 
\ea (Sigurðsson \& \citealt{Holmberg2008})\\
\gll *Honum   mundum  virðast við        (vera) hæfir.\\
      him.\textsc{dat} would.1\textsc{pl} seem   we.\textsc{nom} (be)    competent \\
\ex 
\gll *Honum  munduð    virðast þið              (vera) hæfir.\\
      him.\textsc{dat} would.2\textsc{pl} seem    you.\textsc{pl.nom} (be)   competent\\
\ex
\gll Honum   mundu     virðast þeir         (vera) hæfir.\\
    him.\textsc{dat} would.3\textsc{pl} seem   they.\textsc{nom} (be)   competent\\
\glt  ‘They would seem to be competent to him.’
\z
\z

Interestingly, many speakers allow suspension of agreement with the nominative in the bi-clausal construction. Crucially, the person restriction disappears in that case (see (34)). This is as expected: if there is only agreement with the quirky subject, there cannot be conflicting feature bundles in the verb.

\ea (\label{bkm:Ref295483205})  (Sigurðsson \& \citealt{Holmberg2008})\\
 \ea \gll  Honum  mundi        virðast við         (vera) hæfir.\\
 him.\textsc{dat} would.3\textsc{sg} seem    we.\textsc{nom} (be)    competent   \\
  \ex \gll  Honum  mundi       virðast þið               (vera) hæfir.\\
  him.\textsc{dat} would.3\textsc{sg} seem   you.\textsc{pl.nom} (be)    competent\\
  \ex \gll  Honum  mundi       virðast þeir          (vera) hæfir.\\
  him.\textsc{dat} would.3\textsc{sg} seem   they.\textsc{nom} (be)    competent\\
\z
\z

Sigurðsson \& \citet{Holmberg2008} observe that there is considerable variation in whether suspension of agreement is allowed, preferred or required. In one variant (their Icelandic C), agreement with low nominatives is dispreferred in general, even in mono-clausal constructions. We predict that in that variant there should not be a person restriction on nominative objects at all. This appears to be in line with Sigurðsson and Holmberg’s assessment of the relevant data.

Dutch clefts show almost the same pattern of core observations as Icelandic quirky subject constructions (see also Den \citealt{Dikken2014}). They have the following properties. (i) Number agreement with a clefted nominative is obligatory (see (35)). (ii) If there is unambiguous person agreement, first and second person nominatives cannot be clefted (see (36)). (iii) Some speakers allow suspension of person agreement with clefted nominatives. In that case, there is no person restriction (hence the \%-sign on the variants with third singular \textit{is} in (36a,b)). (iv) Where the verb forms triggered by the pronoun in subject position (\textit{het} ‘it’) and by the clefted nominative DP are identical, the person restriction is lifted for all speakers. This is the case with some modal verbs and in the past tense (see (37)).

\ea (\label{bkm:Ref295486016})  Dutch\\
 \gll Het zijn/*is   zij   die   de whisky gestolen hebben.\\
    it    are.\textsc{pl}/is they that the whisky stolen     have\\
\glt   ‘It’s them who stole the whisky.’  
\z

\ea (\label{bkm:Ref295486091})  
\ea \gll Het \textsuperscript{\%}is/*ben ik die  de  whisky gestolen heeft.\\
    it       is/am    I  that the whisky stolen      has\\
\glt    ‘It’s me who stole the whisky.’
\ex 
\gll Het \textsuperscript{\%}is/*ben(t) jij         die  de  whisky gestolen heeft.\\
    it      is/are.\textsc{sg}   you.\textsc{sg} that the whisky stolen      has\\
\glt    ‘It’s you who stole the whisky.’
\ex 
\gll Het is hij die  de  whisky gestolen heeft.\\
    it     is he that the whisky stolen     has\\
\glt  ‘It’s him who stole the whisky.’
\z
\z


\ea (\label{bkm:Ref295486338})  
\ea \gll Het zal  ik/jij      wel      geweest zijn die  de  whisky gestolen heeft.\\
    it    will I/you.\textsc{sg} indeed been      be   who the whisky stolen     has\\
\glt    ‘It is likely that it was me/you who stole the whisky.’
\ex \gll  Het was ik/jij        die  de  whisky gestolen heeft.\\
    it     was I/you.\textsc{sg} who the whisky stolen      has\\
\glt  ‘It was I/you who stole the whisky.’
\z
\z

These data allow an analysis similar to that proposed for Icelandic. Dutch requires agreement with the subject and (usually) agreement with nominatives. If the clefted constituent is a nominative DP, this yields the following representation: 

\ea 
 \textit{het}\textsubscript{1} … V-[\textsc{dist}]\textsubscript{1}{}-$\varphi $\textsubscript{2} … DP\textsubscript{2} [\textsubscript{CP} (\textit{Op}\textsubscript{2}) … \textit{t}\textsubscript{2} …]
\z

This structure can be realized without problems if the syntactic unification of [\textsc{dist}]\textsubscript{1} and $\varphi $\textsubscript{2} delivers a feature bundle that does not contain multiple person specifications (i.e. when $\varphi $\textsubscript{2} is [\textsc{dist} (\textsc{aug})]). Where syntactic unification does not lead to such a feature bundle, the derivation may converge under phonological unification (i.e. when /[\textsc{dist}]\textsubscript{1}/ = /$\varphi $\textsubscript{2}/). If neither type of unification allows spell-out, the derivation crashes. This accounts for the person restriction observed in (36). Some speakers allow agreement with the clefted nominative to be suspended under these circumstances (through deletion of $\varphi $\textsubscript{2}). For those speakers, first and second person singular clefted nominatives may show up with a third person singular copula:\footnote{In Icelandic clefts, there is always full agreement between the copula and the clefted constituent. In contrast to sentences with a quirky subject and nominative object, there is no evidence for a person clash (Jóhannes Jónsson, Sigríður Sigurjónsdóttir and Höskuldur Þráinsson, p.c.):(i)  Í gær        varst       það þú   sem tókst       bókina.  yesterday was.2\textsc{sg} it     you that took.2\textsc{sg} book.\textsc{def}  ‘Yesterday it was you who took the book.’Apparently, then, Icelandic clefts also permit deletion of one of the $\varphi $-feature bundles in the verb before spell-out, but as opposed to the relevant variety of Dutch, it is the agreement with the subject that is suppressed in Icelandic, rather than the agreement with the nominative predicate. This gives rise to the question why the same deletion is not allowed in quirky subject constructions. One possibility is that this is related to the fact that the agreement induced by such a subject is default agreement. Arguably, default agreement cannot be deleted because it is not recoverable, as opposed to regular agreement, which reflects features of the controller.}

\ea 
 \textit{het}\textsubscript{1} … V-[\textsc{dist}]\textsubscript{1} … DP\textsubscript{2} [\textsubscript{CP} (\textit{Op}\textsubscript{2}) … \textit{t}\textsubscript{2} …]
\z

There is an interesting twist in the plural. Here, all speakers require number agreement, but there are no effects of the person restriction:

\ea 
 \gll  Het zijn/*is   wij/jullie  die  de  whisky gestolen hebben.\\
      it    are.\textsc{pl}/is we/you.\textsc{pl} that the whisky stolen     have\\
 \glt    ‘It’s we/you who stole the whisky.’
\z



These data have no parallel in Icelandic quirky subject constructions and cannot be accounted for through phonological unification, since the third person singular form of the copula is \textit{is} and the first/second person plural form is \textit{zijn}. However, in contrast to Icelandic, Dutch shows full neutralization of person distinctions in the plural, as illustrated for the copula in (41). This fact can be accounted for in terms of two rules of impoverishment that delete person features in the context of \textsc{aug}, as in (42).

\ea (\label{bkm:Ref295574585})  
\ea \gll Ik ben even              weg.\\
    I   am  momentarily away\\
\glt ‘I am out at the moment.’
\ex \gll Jij   bent even              weg.\\
    you are   momentarily away\\
\ex \gll  Hij is even             weg.\\
He is momentarily away\\
\ex \gll  Wij/jullie/zij     zijn even              weg.\\
    we/you.\textsc{pl}/they are   momentarily away\\
\z \z

\ea (\label{bkm:Ref328734023}\label{bkm:Ref295642138})  
\ea \textsc{prox} → Ø / \_\_\_ [\textsc{aug}]
\ex \textsc{dist} → Ø / \_\_\_ [\textsc{aug}]
\z
\z

If the rules in (42) apply to the output of syntactic unification of the two feature bundles on the verb, they will remove the conflicting person specifications, leaving only [\textsc{aug}], and therefore the structure will be realized with the plural form of the copula. We give the derivation for a case with a clefted first person plural pronoun in (43).\footnote{The person restriction discussed above for Dutch clefts is also absent when the pronoun used as subject is not the weak pronoun \textit{het} ‘it’ but the strong pronoun \textit{dat} ‘that’. Arguably, this is because the strong pronoun is a fronted (accusative) predicate, so that in this construction the postverbal DP (the subject) is the only agreeing element; see Ackema and Neeleman (to appear) for discussion.}

\ea (\label{bkm:Ref295575416})  

\ea \textit{het}\textsubscript{1} … V-[dist]\textsubscript{1}{}-[\textsc{prox} \textsc{aug}]\textsubscript{2} … DP\textsubscript{2} [\textsubscript{CP} (\textit{Op}\textsubscript{2}) … \textit{t}\textsubscript{2} …]   (syntactic output)
\ex \textit{het}\textsubscript{1} … V-[\textsc{dist prox} \textsc{aug}]\textsubscript{1+2} … DP\textsubscript{2} [\textsubscript{CP} (\textit{Op}\textsubscript{2}) … \textit{t}\textsubscript{2} …]   (after unification)
\ex  \textit{het}\textsubscript{1} … V-[\textsc{aug}]\textsubscript{1+2} … DP\textsubscript{2} [\textsubscript{CP} (\textit{Op}\textsubscript{2}) … \textit{t}\textsubscript{2} …]     (after application of (42))
\z \z

In summary, third person agreement can induce a person clash in cases of multiple agreement, while singular number agreement never induces a number clash. This confirms that third person has a feature specification, while singular number does not. However, not all cases of multiple agreement give rise to person clashes. Sometimes, conflicts in person specification are resolved by rules that operate before spell-out, which delete one of the problematic feature bundles. In the next section, we will explore such rules of resolution.

\section{ 6. Omnivorous person agreement}

While we have seen that there is an asymmetry between person and number in that person clashes in agreement exist, but number clashes do not, it is not the case that multiple agreement for different persons necessarily leads to ungrammaticality. Some languages allow resolution of a potential clash on the basis of a person hierarchy: the feature structure highest on the hierarchy is realized, while the feature structure lower on the hierarchy is not. 

  A good example is the agreement system in Ojibwe, which is sensitive to a person hierarchy 2 > 1 > 3 (see \citealt{Valentine2001}, among others). The agreement morphology on the Ojibwe verb reflects features of both its subject and object. That there must be simultaneous subject and object agreement is clearest when considering the so-called theme sign on the verb. This is a suffix that expresses the relative position of subject and object on the person hierarchy. In particular, when the subject is higher on this hierarchy than the object, a ‘direct’ theme-sign appears, while an ‘inverse’ form appears when the object is higher on the hierarchy. The form of the theme sign is also determined by whether or not both arguments are ‘local’ persons (first or second) or only one of them is. Thus, the following distribution of theme signs obtains (adapted from \citealt{Lochbihler2008}).

\ea 
   Ojibwe theme signs

\begin{tabularx}{\textwidth}{XXX} 
\lsptoprule
& Subject outranks object on 2 > 1 > 3. & Object outranks subject on 2 > 1 > 3.\\
Both subject and object are 1 or 2. & {}-\textit{i} & {}-\textit{in(i)}\\
Either subject or object is 3. & \textit{{}-aa} & \textit{{}-igw} (and allomorphs)\\
\lspbottomrule
\end{tabularx}
\z

 
This simultaneous sensitivity to the features of subject and object can only be accounted for under the assumption that both agree with the verb. Only if the features of both arguments are represented in the verb is it possible to have a spell-out system for the verbal agreement that is based on a comparison of their position on the person hierarchy. For the theme-sign suffixes, then, resolution of person clashes is achieved by spell-out rules that insert a single morpheme as the realization of pairs of feature bundles.

  In addition to the theme-sign suffix, the Ojibwe verb also carries a prefix that expresses person agreement. Interestingly, this prefix shows omnivorous person effects: it expresses agreement with the argument that is highest on the person hierarchy, regardless of whether this is the subject or the object (\textit{g}{}- realizes second person, \textit{n}{}- first person, \textit{w}{}-/${\emptyset}${}- third person). Given the discussion above, we know that the person features of both subject and object are represented in the verb. Hence, the behavior of the Ojibwe prefix shows that resolution of a person clash can also consist of non-realisation of the feature structure lower on the person hierarchy. The following examples illustrate the system (from \citealt{Valentine2001}, cited here from \citealt{Lochbihler2008}):

\ea      Ojibwe\\
 \ea
 \gll  \textit{n-waabm-aa} \\
       1-see-\textsc{dir}    \\
 \glt  ‘I see him.’
\ex
  \gll     \textit{n-waabm-ig}\\
           1-see-\textsc{inv}  \\
  \glt         ‘He sees me.’
\z
 \z

\ea 
  \ea
 \gll     \textit{g-waabam-i}      \\
           2-see-\textsc{dir}(local)  \\
 \glt      ‘You see me.’
 \ex
   \gll  \textit{g-waabm-in}\\
         2-see-\textsc{inv}(local)\\
   \glt   ‘I see you.’
\z
  \z

Not all languages that allow resolution of person clashes on the basis of a hierarchy make use of the same hierarchy. There is one cross-linguistic constant, though: third person is outranked by both first and second. The variation lies in the ranking of first and second person, as follows:

\ea 
\ea  2 > 1 > 3  (example: Ojibwe, see above)
\ex  1 > 2 > 3   (example: Nocte, see below)
\ex  1,2 > 3  (example: Kaqchikel, see below)
\z
\z

We suggest that this cross-linguistic variation comes about through variation in weighting of the two conditions in (48). (For the purpose of (48b), a feature structure is less uniform if it contains instances of more features.)

\ea (\label{bkm:Ref453928502})  
\ea  \textsc{Prox} outranks \textsc{dist}.
\ex  Less uniform feature structures outrank more uniform feature structures.
\z
\z

A constraint equivalent to (48a) is present in some form or other in most any theory of person hierarchies, sometimes expressed directly and sometimes expressed in the order of functional projections, or in the order of probing of features (see below). The constraint in (48b) may look unfamiliar, but it is an instantiation of the general idea that feature structures containing more features are marked compared to feature bundles containing fewer. The only innovation is that markedness is assumed not to increase with repetition of the same feature, as in the first person exclusive (characterized by [\textsc{prox}–\textsc{prox}], see §2).

If the first condition in (48) is more important than the second, the resulting hierarchy will be 1 > 2 > 3. This is because first person is maximally marked according to this principle, as it contains only instances of \textsc{prox}. By contrast, third person is maximally unmarked, as it contains only \textsc{dist}. Second person is in between, as it contains both \textsc{prox} and \textsc{dist}. If the second condition in (48) is more important, second person will be highest in the hierarchy, as this is the only person with a non-uniform feature structure. The relative ranking of first and third person is still determined by the first condition, so that the result is a hierarchy 2 > 1 > 3. Finally, if the two conditions are equally weighted, a hierarchy results in which first and second person are ranked equally, and are both ranked above third person.

Nocte is an example of a language that is like Ojibwe, but with first and second person reversed on the hierarchy (that is, it uses a 1 > 2 > 3 hierarchy). The following data (from De\citealt{Lancey1981}:641, cited here from \citealt{Croft2003}:172) illustrate this:

\ea (\label{bkm:Ref328735211})  Nocte\\
\ea\gll  Nga-ma  ate  hetho-ang.  \\
    1\textsc{sg-erg 3sg} teach-1\\
\glt    ‘I will teach him.’
\ex \gll  Ate-ma   nga-nang hetho-h-ang.\\
      3\textsc{sg-erg 1sg-acc} teach-\textsc{inv-1}\\
\glt     ‘He will teach me.’
\ex \gll Nang-ma nga hetho-h-ang.\\
      2\textsc{sg-erg  1sg} teach-\textsc{inv-1}\\
\glt     ‘You will teach me.’
\ex \gll Nga-ma  nang hetho-e.\\
     1\textsc{sg}{}-\textsc{erg} 2\textsc{sg} teach-1\textsc{pl}\\
\glt    ‘I will teach you.’
\z \z

As in Ojibwe, an inverse marker appears on the verb in case the object is higher on the person hierarchy than the subject, the only difference being that, since the hierarchy is 1 > 2 > 3 in Nocte, the inverse marker is used when the subject is second person and the object first person.  As before, the presence of this kind of morphology can only be understood if there is double agreement, so that the features of both subject and object are represented in the verb. Also as in Ojibwe, there is a second morpheme, in this case a suffix, that agrees in person with that argument whose feature specification is highest on the hierarchy (the omnivorous person effect). There is an interesting twist when the subject is first person and the object second person, as in (49d). As expected, the person agreement shown by the relevant suffix is with first person. However, the number expressed is an unexpected inclusive plural, rather than the singular. We will not attempt to analyse this observation, but it is another indication that the agreement morphology reflects agreement with both subject and object.

The final possibility of the system outlined above is a person hierarchy in which first and second person are equally ranked. This should result in a language that allows resolution of clashes between third person and either first or second person, but not resolution of clashes between first and second person. An example of such a language is Kaqchikel, as discussed in \citealt{Preminger2014} (all Kaqchikel data below are taken from this source). In ordinary transitive clauses, the verb agrees with both subject and object, and this configuration of multiple agreement is reflected in two distinct agreement morphemes:

\ea Kaqchikel
\ea 
 \gll rat  x-Ø-aw-ax-aj             ri   achin.    \\
      you.\textsc{sg} \textsc{com}{}-\textsc{3sg.abs-2sg.erg}{}-hear-\textsc{act} the man\\
 \glt ‘You heard the man.’
  \gll       ri achin  x-a-r-ax-aj              rat.\\
         the man \textsc{com-2sg.abs-3sg.erg}{}-hear-\textsc{act} you.\textsc{sg}\\
\glt ‘The man heard you.’
\z
\z

The interesting twist in Kaqchikel is that there is a construction, known as the Agent Focus construction, in which the number of agreement slots on the verb is reduced to one. This, of course, creates a situation in which person clashes arise. When one of the arguments of the verb is third person and the other one is not, the clash is resolved in favour of the non-third person argument. This is illustrated in (51) for a combination of a first person and third person argument, and in (52) for a combination of a second person and third person argument.

\ea (\label{bkm:Ref454183795})  
\ea \gll   ja    yïn x-in/*Ø-ax-an              ri   achin.\\
    \textsc{foc} me \textsc{com-1sg/*3sg.abs}{}-hear-\textsc{af} the man\\
\glt    ‘It was me that heard the man.’
\ex
\gll  ja    ri    achin x-in/*Ø-ax-an                      yïn.\\
    \textsc{foc} the man  \textsc{com-1sg/*3sg.abs}{}-hear-\textsc{af} me\\
\glt    ‘It was the man that heard me.’
\z
\z

\ea (\label{bkm:Ref454183994}) 
\ea \gll  ja     rat x-at/*Ø-ax-an          ri   achin.\\
    \textsc{foc} you.\textsc{sg} \textsc{com-2sg/*3sg.abs}{}-hear-\textsc{af} the man\\
\glt    ‘It was you that heard the man.’
\ex
\gll ja    ri    achin x-at/*Ø-ax-an                        rat.\\
    \textsc{foc} the man   \textsc{com}{}-\textsc{2sg/*3sg.abs}{}-hear-\textsc{af} you.\textsc{sg}\\
\glt    ‘It was the man that heard you.’
\z
\z

This indicates that there is a person hierarchy in Kaqchikel on which both first and second person outrank third person.\footnote{When both arguments in the Agent Focus construction are third person, the result is third person agreement. If one of the third person arguments is plural and the other singular, we get plural agreement (omnivorous number). This indicates that, as expected, when unification is possible, this is used as the strategy for determining the spell-out of a single agreement slot for two feature bundles. When one of the arguments is first or second person and the other argument is third person, the first or second person argument will be agreed with not only for person but also for number (no omnivorous number in this case; see \citealt{Preminger2014}:20). This shows that ‘partial unification’ is impossible: either there is unification for all $\varphi $-features, or no unification at all, and that, when unification fails, the person hierarchy determines which argument’s features are realized. This is a property of unification in general: if there is a clash in any feature, it fails.} That first and second person are not ranked with respect to each other on this hierarchy is shown by the fact that, in the Agent Focus construction, no resolution is possible in case both arguments are local. As in Icelandic and elsewhere, unresolved clashes result in ungrammaticality. Thus, the following are impossible, regardless of the choice of agreement on the verb, whether first person, second person, or (default) third person.

\ea 
\ea
\gll *ja   rat        x-in/at/Ø-ax-an                          yïn.\\
      \textsc{foc} you.\textsc{sg} \textsc{com-1sg/2sg/3sg.abs}{}-hear-\textsc{af} me\\
\glt Intended: ‘It was you that heard me.’
\ex
\gll *ja    yïn x-in/at/Ø-ax-an                           rat.\\
     \textsc{foc} me \textsc{com-1sg/2sg/3sg.abs}{}-hear-\textsc{af} you.\textsc{sg}\\
\glt    Intended: ‘It was me that heard you.’
\z
\z

\citet{Preminger2014} argues that it is undesirable to appeal to person hierarchies to deal with the Kaqchikel data. He proposes a syntactic account which he claims to be motivated independently, and which derives the effects of the person hierarchy. The account is based on a Probe-Goal system of syntactic agreement regulated by relativized minimality. In the Kaqchikel Agent Focus construction, there is one functional head that acts as a Probe for person features. This head specifically probes for a participant feature. Given relativized minimality, the highest DP that has a participant feature will act as the Goal. However, Preminger assumes, following Béjar \& \citet{Rezac2003}, that all first or second person features in DPs must be licensed by entering an agreement relation:\footnote{Bejar \& \citet{Rezac2003} invoke this condition in an account of the so-called Person Case Constraint (PCC). This is a constraint on the possible features of an accusative clitic or weak pronoun in the presence of a dative clitic or weak pronoun. There is language variation in what is prohibited, but a common form of the constraint is that the accusative pronominal cannot be first or second person in the context of any dative pronominal. We think that PCC effects should not be linked to agreement, however, simply because in most of the languages that show PCC effects, neither dative nor accusative objects agree with the verb. At the least, this shows that the Agree operation invoked in (54) cannot be equated with actual agreement, but it is the latter in which we are interested here. For accounts of the PCC that are not based on Agree, see \citealt{Haspelmath2004}, Runić 2013, and \citealt{Kiss2015}, among others.}

(\label{bkm:Ref454187927})  Person Licensing Condition (Béjar \& \citealt{Rezac2003})

  Interpretable 1st/2nd person features must be licensed by entering into an Agree 

  relation with an appropriate functional category.

The consequence of this is that the lower DP in the Agent Focus construction cannot be licensed if it, too, is first or second person. In contrast, if the subject is third person, this is skipped in the Probe’s search for a participant feature, and agreement will be with the first or second person object.

  Whether or not an account that appeals to a person hierarchy is more stipulative than this syntactic account can only be evaluated properly when cross-linguistic variation in the effects of person hierarchies is considered. After all, we have seen that it is certainly not always the case that a clash between first and second person results in ungrammaticality. In some languages, these clashes are resolved as well, sometimes in favour of first person and sometimes in favour of second person (see above). It seems to us that the only way in which the syntactic account just outlined can deal with such variation is by specifying the features that the Probe is searching for. However, the language variation implies that it is not sufficient to specify a fixed feature content for the Probe per language. Probes must be allowed to search for different features, and in addition the features searched for must be ordered such that agreement with some is preferred over agreement with others.

  Consider a language with a 2 > 1 > 3 hierarchy, for instance. Given that second person defeats first person in a clash, the verbal head must probe specifically for a feature that is unique to second person, say addressee. Otherwise, it should not be able to skip a first person argument in its search. However, if the Probe is specified as addressee also in a context where there is a clash between a first person and a third person argument, the situation would be unresolvable. In order to explain why the third person is ignored in favour of the first person argument, the feature content of the Probe must be different. In particular, the Probe must search for a feature that distinguishes first and third person, that is, either a Speaker feature or a more general participant feature. But in the 1 vs 2 situation, the Probe cannot be permitted to search for either of these features. The implication is that there is a hierarchy that determines which features are preferably selected as the specification of the Probe. Clearly, this is simply the counterpart of the 2 > 1 > 3 person hierarchy. Given the attested language variation, it must be the case that this hierarchy of preferred feature content can vary from language to language. We conclude that there is no difference between the syntactic account and the morphological account proposed here in terms of the necessity of stipulating a language-particular feature hierarchy.\footnote{Preminger argues that the syntactic account, but not an account based on a person hierarchy directly, provides insight into the morphology of the agreement markers in Kaqchikel. In particular, first and second person agreement markers are reduced versions of strong pronouns, while third person agreement markers are not. Moreover, the third person marker is a number marker; third person singular is null. Preminger’s account for this is that probing by the person head results in clitic doubling of the Goal, while probing by the number head does not. Since the person head does not probe a third person DP, we get only number agreement when a third person DP agrees, and therefore not a clitic. (This holds both in the Agent Focus construction and in ordinary transitive clauses, so is not related to the occurrence of a person clash.) Of course, the generalisation that agreement with first and second person takes the form of a clitic can be made in any theory that can generalise over first and second person. In our account, one could say that agreement for \textsc{prox} takes the form of a clitic. Neither of these accounts provides insight for why this should be so. It is a well-known observation that in a number of languages the morphology of first and second person agreement markers diachronically developed from pronouns, while the morphology of third person agreement markers did not (see Fuß 2005 and references mentioned there). This may not have anything to do with the internal logical of the person feature system, but rather with the high accessibility in discourse of first and second person, which \citet{Ariel2000} argues favours reduction of the pronominal markers expressing these persons to clitics and subsequently to agreement markers.}

  The main objection to the syntactic alternative, however, is that it fails to account for those situations in which third person DPs are involved in person clashes. As we have seen in the previous sections, the agreement data from Icelandic quirky subject constructions and Dutch clefts can be understood as the result of just such a clash. If the person clash in the Kaqchikel Agent Focus construction is the result of the Person Licensing Condition in (54), third persons should never lead to a similar problem. At the least, then, this implies that a unified account of all the data discussed in this paper is not possible on a syntactic account based on this particular constellation of assumptions.

\section{Conclusion}
In this paper we have shown that there is a fundamental distinction between default person and default number. Third person has a feature specification, while singular number does not. The argument is based on configurations in which two $\varphi $-feature bundles compete for spell-out. In the case of number, this never results in a clash. Instead, there will be omnivorous number: the verb shows plural agreement whenever at least one of the feature bundles is specified as plural. In contrast, in the case of person this situation can lead to a clash. This accounts for the impossibility of having a lower nominative with a different person specification than the subject in both Icelandic quirky subject constructions and Dutch clefts. Those cases where a verb does show omnivorous person agreement are the result of language-specific person hierarchies used for resolution. We have presented an account of such hierarchies that is in line with the assumption that third person is not feature-less. 

\section*{Acknowledgements}

For discussion and/or comments, we would like to thank Caroline Heycock, Andrew Nevins, Omer Preminger, the anonymous reviewers and editors for this volume, the audience at the \textit{Agreement without Borders} conference 2015 at the University of Zadar, and the students at the LOT Summer \citealt{School2016} at Utrecht University.

\begin{verbatim}%%move bib entries to  localbibliography.bib
\section{ References}

Aalberse, Suzanne \& Don, Jan. 2009. Syncretism in Dutch dialects. \textit{Morphology} 19. 3–14.

Aalberse, Suzanne \& Don, Jan. 2011. Person and number syncretisms in Dutch. \textit{Morphology} 21. 327–350.

Ackema, Peter \& Neeleman, Ad. 2013. Person features and syncretism. \textit{Natural Language \& Linguistic Theory} 31. 901–950.

Ackema, Peter \& Neeleman, Ad. To appear. \textit{Features of person}. Cambridge, MA: MIT Press.

Adger, David \& Harbour, Daniel. 2007. Syntax and syncretisms of the Person Case Constraint. \textit{Syntax} 10. 2–37.

Anagnostopoulou, Elena. 2005. Strong and weak person restrictions: A feature checking analysis. In Heggie, Lorie \& Ordoñez, Francisco (eds.), \textit{Clitics and affix combinations: Theoretical perspectives}, 199–235. Amsterdam: John Benjamins.

Ariel, Mira. 2000. The development of person agreement markers: From pronouns to higher accessibility markers. In Barlow, Michael \& Kemmer, Suzanne (eds.), \textit{Usage-based models of language}, 197–260. Stanford, CA: CSLI Publications.

Baerman, Matthew \& Brown, Dunstan. 2011. Syncretism in verbal person/number marking. In Dryer, Matthew \& Haspelmath, Martin (eds.), \textit{The world atlas of language structures online}. Munich: Max Planck Digital Library, chapter 29. (http://wals.info/chapter/29) (Accessed on 2011-08-19.)

Baerman, Matthew \& Brown, Dunstan \& Corbett, Greville. 2005. \textit{The syntax-morphology interface: A study of syncretism}. Cambridge: Cambridge University Press.

Béjar, Susana \& Rezac, Milan. 2003. Person licensing and the derivation of PCC effects. In: Pérez-Leroux, Ana-Teresa \& Roberge, Yves (eds.), \textit{Romance linguistics: theory and acquisition}, 49–61. Amsterdam: John Benjamins.

Bennis, Hans \& MacLean, Alies. 2006. Variation in verbal inflection in Dutch dialects. \textit{Morphology} 16. 291–312.

Bobaljik, Jonathan. 2008. Missing persons: A case study in morphological universals. \textit{The Linguistic Review} 25. 203–230.

Borgman, Donald. 1990. Sanuma. In Derbyshire, Desmond \& Pullum, Geoffrey (eds.), \textit{Handbook of Amazonian languages}, vol. 2, 15–248. Berlin: Mouton de Gruyter.

Burzio, Luigi. 2000. Anatomy of a generalization. In Reuland, Eric (ed.), \textit{Arguments and case: Explaining Burzio’s generalization}, 195–240. Amsterdam: John Benjamins.

Croft, William. 2003. \textit{Typology and universals}, 2nd ed. Cambridge: Cambridge University Press.

Cysouw, Michael. 2003. \textit{The paradigmatic structure of person marking}. Oxford: Oxford University Press.

Cysouw, Michael. 2011. The expression of person and number: A typologist’s perspective. \textit{Morphology} 21. 419–443.

D’Alessandro, Roberta. 2007. \textit{Impersonal} si\textit{{}-constructions}. Berlin: Mouton de Gruyter.

D'Alessandro, Roberta \& Roberts, Ian. 2010. Past participle agreement in Abruzzese: Split auxiliary selection and the null-subject parameter. \textit{Natural Language \& Linguistic Theory} 28. 41–72.

DeLancey, Scott. 1981. An interpretation of split ergativity and related patterns. \textit{Language} 57. 626–657.

Den Dikken, Marcel. 2014. The attractions of agreement. Ms. (New York, NY: CUNY.)

Egerland, Verner. 2003. Impersonal pronouns in Scandinavian and Romance. \textit{Working Papers in Scandinavian Syntax} 71. 75–102.

Forchheimer, Paul. 1953. \textit{The category of person in language}. Berlin: Walter de Gruyter.

Fuß, Eric. 2005. \textit{The rise of agreement: A formal approach to the syntax and grammaticalization of verbal inflection}. Amsterdam: John Benjamins.

Gazdar, Gerald \& Pullum, Geoffrey. 1982. Generalized phrase structure grammar: A theoretical synopsis. Bloomington, IN: Indiana University Linguistics Club.

Greenberg, Joseph. 1988. The first person inclusive dual as an ambiguous category. \textit{Studies in Language} 12. 1–18.

Groos, Anneke \& van Riemsdijk, Henk. 1981. Matching effects with free relatives: A parameter of core grammar. In: Belletti, Adriana \& Brandi, Luciana \& Rizzi, Luigi (eds.), \textit{Theory of markedness in generative grammar}, 171–216. Pisa: Scuola Normale Superiore.

Halle, Morris. 1997. Distributed morphology: Impoverishment and fission. \textit{MIT Working Papers in Linguistics} 30. 425–449.

Halle, Morris \& Marantz, Alec. 1993. Distributed morphology and the pieces of inflection. In Hale, Kenneth \& Keyser, Samuel Jay (eds.), \textit{The view from Building 20}, 111–176. Cambridge, MA: MIT Press.

Harbour, Daniel. 2016. \textit{Impossible persons}. Cambridge, MA: MIT Press.

Harley, Heidi \& Ritter, Elizabeth. 2002. Person and number in pronouns: A feature-geometric analysis. \textit{Language} 78. 482–526.

Haspelmath, Martin. 2004. Explaining the ditransitive person-role constraint: A usage-based account. \textit{Constructions} 2. 1–71.

Hoekstra, Jarich. 2010. On the impersonal pronoun \textit{men} in Modern West Frisian. \textit{Journal of Comparative Germanic Linguistics} 13. 31–59.

Kayne, Richard. 1993. Toward a modular theory of auxiliary selection. \textit{Studia Linguistica} 47. 3–31.

Kerstens, Johan. 1993. \textit{The syntax of person, number and gender}. Berlin: Mouton de Gruyter.

Kiss, Katalin É. 2015. The Person–Case Constraint and the Inverse Agreement Constraint are manifestations of the same information-structural restriction. (Paper presented at GLOW 38, Paris.)

Lochbihler, Bethany. 2008. Person encoding in the Ojibwe inverse system. In Richards, Marc \& Malchukov, Andrej (eds.), \textit{Scales}, 295–315. Leipzig: Institut für Linguistik, University of Leipzig.

Nevins, Andrew. 2007. The representation of third person and its consequences for person-case effects. \textit{Natural Language \& Linguistic Theory} 25. 273–313.

Nevins, Andrew. 2011. Multiple Agree with clitics: Person complementarity vs omnivorous number. \textit{Natural Language \& Linguistic Theory} 29. 939–971.

Noyer, Rolf. 1997. \textit{Features, positions, and affixes in autonomous morphological structure}. New York: Garland.

Perri Ferreira, Helder. 2013. Patrones de marcación argumental en las lenguas de la familia Yanomami. (Paper presented at the Conference on Indigenous Languages of Latin America VI, University of Texas at Austin.)

Platzack, Christer. 1983. Germanic word order and the COMP/INFL parameter. \textit{Working Papers in Scandinavian Syntax} 2. 1–45.

Preminger, Omer. 2014. \textit{Agreement and its failures}. Cambridge, MA: MIT Press.

Runić, Jelena. 2013. The Person-Case Constraint: A morphological consensus. (Poster presented at the 87th Meeting of the Linguistic Society of America, Boston.)

Sch\"{u}tze, Carson. 2003. Syncretism and double agreement with Icelandic nominative objects. In Delsing, Lars-Olof \& Falk, Cecilia \& Josefsson, Gunlög \& Sigurðsson, Halldór Ármann (eds.), \textit{Grammar in focus: Festschrift for Christer Platzack}, 295–303. Lund: Department of Scandinavian Languages.

Sigurðsson, Halldór Ármann. 1996. Icelandic finite verb agreement. \textit{Working Papers in Scandinavian Syntax} 57. 1–46.

Sigurðsson, Halldór Ármann \& Egerland, Verner. 2009. Impersonal null-subjects in Icelandic and elsewhere. \textit{Studia Linguistica} 63. 158–185.

Sigurðsson, Halldór Ármann \& Holmberg, Anders. 2008. Icelandic dative intervention: Person and number are separate probes. In D’Alessandro, Roberta \& Fischer, Susann \& Hrafnbjargarson, Gunnar Hrafn (eds.), \textit{Agreement restrictions}, 251–280. Berlin: Mouton de Gruyter.

Simon, Horst. 2005. Only you? Philological investigations into the alleged inclusive-exclusive distinction in the second person plural. In Filimonova, Elena (ed.), \textit{Clusivity: Typology and case studies of the inclusive-exclusive distinction}, 113–150. Amsterdam: John Benjamins.

Ussery, Cherlon. 2013. The syntax of optional agreement in Icelandic. Ms. (Northfield, MN: Carleton College.)

Valentine, Randolph. 2001. \textit{Nishnaabemwin reference grammar}. Toronto: University of Toronto Press.

Zwicky, Arnold. 1977. Hierarchies of person. In Beach, Woodford \& Fox, Samuel \& Philosoph, Shulamith (eds.), \textit{Papers from the Thirteenth Annual Meeting of the Chicago Linguistic Society}, 714–733. Chicago: Chicago Linguistic Society.


\end{verbatim}  
{\sloppy
\printbibliography[heading=subbibliography,notkeyword=this]
}
\end{document}
