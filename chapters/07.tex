\documentclass[output=paper]{langsci/langscibook} 
\ChapterDOI{10.5281/zenodo.3458074}
\author{M. Rita Manzini\affiliation{Università degli Studi di Firenze}\and Ludovico Franco\affiliation{Università degli Studi di Firenze}\lastand Leonardo Savoia\affiliation{Università degli Studi di Firenze}}
\title{Suffixaufnahme, oblique case and Agree}


\abstract{The present contribution focuses on a set of phenomena which are unified by the typological literature under the label of Suffixaufnahme. The theoretical focus of the contribution is the minimalist rule of Agree and the notion of case, specifically oblique case. We question the necessity of [interpretable] and [valued] features for the formulation of Agree. We suggest that more primitive syntactic notions underlie the descriptive label ‘oblique’, specifically that of an elementary relator with a part\slash whole content. Thus, the DP embedded under a genitive case\slash adposition is interpreted as a possessor\slash whole with respect to a local superordinate DP (the possessum\slash part). We argue that case\slash agreement stacking corresponds to the presence of a partial copy of this second argument within the phrasal projection of the relator. In §2 we apply this analysis to linkers, using Albanian as a case study; we then go on to case\slash agreement stacking in Punjabi (§3) and in the Australian languages (e.g. Lardil), which are often taken as the core instantiation of the phenomenon.}

\maketitle
\begin{document}


 
%%please move the includegraphics inside the {figure} environment
%%\includegraphics[width=\textwidth]{OGSVolumeAug2018ManziniFrancoSavoia-img1.jpg}

 
%%please move the includegraphics inside the {figure} environment
%%\includegraphics[width=\textwidth]{OGSVolumeAug2018ManziniFrancoSavoia-img2.jpg}

 
%%please move the includegraphics inside the {figure} environment
%%\includegraphics[width=\textwidth]{OGSVolumeAug2018ManziniFrancoSavoia-img3.jpg}

 
%%please move the includegraphics inside the {figure} environment
%%\includegraphics[width=\textwidth]{OGSVolumeAug2018ManziniFrancoSavoia-img4.jpg}

 
%%please move the includegraphics inside the {figure} environment
%%\includegraphics[width=\textwidth]{OGSVolumeAug2018ManziniFrancoSavoia-img5.jpg}

 
%%please move the includegraphics inside the {figure} environment
%%\includegraphics[width=\textwidth]{OGSVolumeAug2018ManziniFrancoSavoia-img6.jpg}

 
%%please move the includegraphics inside the {figure} environment
%%\includegraphics[width=\textwidth]{OGSVolumeAug2018ManziniFrancoSavoia-img7.jpg}

 
%%please move the includegraphics inside the {figure} environment
%%\includegraphics[width=\textwidth]{OGSVolumeAug2018ManziniFrancoSavoia-img8.jpg}

 
%%please move the includegraphics inside the {figure} environment
%%\includegraphics[width=\textwidth]{OGSVolumeAug2018ManziniFrancoSavoia-img9.jpg}

 
%%please move the includegraphics inside the {figure} environment
%%\includegraphics[width=\textwidth]{OGSVolumeAug2018ManziniFrancoSavoia-img10.jpg}

 
%%please move the includegraphics inside the {figure} environment
%%\includegraphics[width=\textwidth]{OGSVolumeAug2018ManziniFrancoSavoia-img11.jpg}

\section{The core phenomena and the role of Agree}% 1. 
\subsection{Case/agreement stacking and linkers} % 1.1. 

A core instance of what the typological literature labels \isi{Suffixaufnahme} \citep{Plank1995} is case stacking. \ili{Lardil} is cited by \citet{Richards2013} as a case in point, as in (1). In (1), the DP \textit{marun-ngan-ku} ‘boy-\textsc{gen-instr}’ is inflected both for genitive and for instrumental cases, reflecting its status as the possessor (\textsc{gen}) of the instrumental nominal \textit{maarnku} ‘spear-\textsc{instr}’. For \citet[62]{Merchant2006}, case stacking amounts to the fact that “a single DP may be the goal for multiple probes.” \citet{Richards2013} in turn speaks of ‘\isi{concord}’ as the process responsible for case stacking configurations, where \isi{concord} is “a series of \isi{Agree} operations” with the same c-commanding probe. Thus, stacking implies that the same probe can attract several goals.

\ea%1
\ili{Lardil}, Pama-Nyungam \citep[43]{Richards2013}\\
\gll  Ngada latha   karnjin-i   marun-ngan-ku   maarn-ku.\\
I   spear   wallaby-\textsc{acc} boy-\textsc{gen-instr}  spear-\textsc{instr}\\
\glt ‘I speared the wallaby with the boy’s spear.’        
\z

\citet{Plank1995} points to a close similarity between case stacking and \is{linker}linkers – which have their own tradition of studies in the generative framework. The non-agreeing \ili{Persian} \textit{ezafe} is often at the center of discussions of \is{linker}linkers (\citealt{Dikken2004}; \citealt{Larson2008}; \citealt{Richards2010}). On the other hand, \citet{Franco2015} exemplify \is{linker}linkers with data from \ili{Albanian}, where the pre-genitival \isi{linker} varies according to the \is{feature!phi-feature}phi-features and also according to the case of the head DP. Agreement in \is{feature!phi-feature}phi-features is illustrated in (2a--2b), while (2c) illustrates agreement in case. Here and throughout we use data from the Geg (Northern \ili{Albanian}) variety of Shkodër.

\ea%2
    Shkodër, Geg \ili{Albanian}  (\citealt{Manzini2011Reducing}: 105)\label{ex:manzini:2}\\
    \ea
    \gll libr-i          i       msus-ɛs   \\
            book-\textsc{m.sg.nom.def}  \textsc{lkr.m.sg.nom.def}   teacher-\textsc{f.sg.obl.def} \\
    \glt    ‘the book of the teacher’
    \ex
    \gll kɑ:m-a      ɛ          tʃɛn-it                \\
         paw-\textsc{f.sg.nom.def}   \textsc{lkr.f.sg.nom.def}  dog-\textsc{m.sg.obl.def}  \\
    \glt ‘the paw of the dog’\\
    \ex
    \gll paɾa     libr-it       t       msus-ɛs\\
         before   book-\textsc{m.sg.obl.def}  \textsc{lkr.m.sg.obl.def}  teacher-\textsc{f.sg.obl.def}\\
    \glt ‘in front of the book of the teacher’ 
    \z
\z 

The examples of adnominal modification in (1--2) include essentially the same ingredients, though differently arranged. In (1) both genitive and instrumental are suffixed to the possessor. In (2) the possessor has a single genitive suffix and it is preceded by a head bearing a case agreeing with that of the possessum. As it turns out, most of the generative theories of \is{linker}linkers do not extend to stacking. Theories of \is{linker}linkers are easily classified into a few major subtypes. \citet{Richards2010} argues that the \ili{Persian} \textit{ezafe} is a PF device to ensure N-N identity avoidance (cf. \citealt{Ghomeshi1997}). Stacked case could not be a means to the same end, since the instrumental N and the genitive N are adjacent in (1). Incidentally, this is an obvious demonstration of the lack of perceived unity between case stacking and \is{linker}linkers from the point of view of the same author, Norvin Richards. A second stream of theoretical literature (\citealt{Dikken2004}; \citealt{Campos2005}) treats \is{linker}linkers as (the counterpart of) copulas in the DP domain. But it is hard to see how stacked case could fit into this definition. \citegen{Larson2008} conclusion that \is{linker}linkers are to be explained in terms of case (cf. \citealt{Samiian1994} on \ili{Persian}) seems to hold some promise towards the unification of \is{linker}linkers with case stacking – except that these authors argue that \is{linker}linkers play a role as case assigners, allowing Ns, which do not normally license case, to be construed with DP complements and AP modifiers. On the other hand, a stacked case is a case being assigned on top of another. 

This then leaves proposals \citep{Philip2012} that \is{linker}linkers should be understood as agreements, though represented by heads, rather than by agreement suffixes; if predication is involved, then \is{linker}linkers are subjects of predication \citep{Franco2015}, rather than copulas. Specifically, in \ili{Albanian} (2) the \isi{linker} agrees in case, as well as in \is{feature!phi-feature}phi-features, with the head of the possession construct, providing an obvious link with case stacking, described by \citet{Richards2013} himself in term of ‘\isi{concord}’. Summarizing, we have on the one hand a typological similarity between case stacking in (1) and \is{linker}linkers in (2) and on the other hand a rough theoretical compatibility, at least for a particular subset of analyses of case stacking and \is{linker}linkers – treating both phenomena as connected to \isi{Agree}. 

Though stacking of suffixal material in \ili{Lardil} (1) involves case, \is{feature!phi-feature}phi-features may also in principle be stacked. A clear example of this configuration is provided by \ili{Punjabi} (Indo-Aryan). \ili{Punjabi} Ns have \is{feature!phi-feature}phi-feature inflections sensitive to a direct\slash \is{case!oblique case}oblique case distinction in the masculine singular, followed by \is{postposition}postpositions. In addition, possessor phrases in \ili{Punjabi} require the \is{feature!phi-feature}phi-features of the head noun (the possessum) to be stacked on top of the genitive \textit{d-} \isi{postposition}. Thus, consider \textit{muɳɖ}{}- ‘boy’ in (3), on which we find, from left to right, the oblique \is{feature!phi-feature}phi-feature inflection -\textit{e}, the \textit{d-} genitive \isi{postposition} and finally a \is{feature!phi-feature}phi-feature inflection \textit{{}-}\textit{i/-ĩã} agreeing with the head noun. As we will see in §3, if the head noun is masculine, this outer inflection is also sensitive to the direct\slash \is{case!oblique case}oblique case distinction. At first sight, \ili{Punjabi} (3) is essentially like \ili{Lardil} (1), modulo the presence of stacked case in (1) and of stacked agreement in (3). It is also similar to \ili{Albanian} (2), modulo the fact that agreement with the head N is externalized by a head (the \isi{linker}) in \ili{Albanian} and by a \isi{postposition} in \ili{Punjabi}. 

\ea%3
\ili{Punjabi} \citep[316]{Manzini2015}\\
\gll muɳɖ- e-  d-  i / -ĩã    kita:b   /   kitabb-a\\
     boy- \textsc{m.sg-} \textsc{gen-}  \textsc{f.sg} / \textsc{-f.pl}  book.\textsc{abs.f.sg} /   book-\textsc{abs.f.pl}\\
\glt ‘the book/the books of the boy’
\z

The interest of an inquiry into (1--3) from a theoretical perspective is represented in part by the potential implications for \isi{Agree}, one of the core rules of \isi{Minimalist} syntax. A quick survey of the formalizations proposed for both stacking and \is{linker}linkers in terms of \isi{Agree} reveals some potential difficulties for this rule. Recall that \citegen{Merchant2006} idea is that in case stacking the same set of interpretable nominal features are able to check more than one probe. For instance, in (1), ‘boy’ could check both a \is{case!genitive case}genitive case probe and an instrumental probe. In (3) then the agreement suffixes \textit{{}-}\textit{i/-ĩã} would have to be the result of checking some probe associated with the head noun ‘book(s)’, say an abstract D. Unfortunately they can’t be, because ‘boy’ has its own interpretable set of features, which definitely cannot be made to agree with the equally interpretable, different features of ‘book(s)’.\footnote{The empirical evidence points to the agreeing feature sets being associated with the \isi{postposition}, construed as a syntactic head (§3.1). We could then deny that stacking is involved at all in languages like \ili{Punjabi} – except that the formal (not merely functional) continuity between the various phenomena here briefly introduced can in our view be modelled by (the appropriate version of) \isi{Agree}.} Next consider \is{linker}linkers. Suppose that the \ili{Albanian} \isi{linker} in (2) is an agreement head. Then, as discussed in particular by \citet{Philip2012}, we are forced to diverge from a standard tenet of Minimalism, namely that heads are contentive elements – since their deletion at LF under Full Interpretation would amount to the destruction of structure (contravening Inclusiveness; \citealt{Chomsky1995}). 

These problems may be taken to determine one of two logical outcomes. First, \isi{Suffixaufnahme} cannot have any theoretical significance given that aligning it with \isi{Agree} seems to involve difficulties for this rule. Alternatively, we will have to reconsider the formulations of \isi{Agree} that are standardly used. In fact, the core context for \isi{Suffixaufnahme} is adnominal modification – and we independently know that \isi{Minimalist} \isi{Agree}, developed by \citet{Chomsky2000,Chomsky2001Derivation} for verb agreement, cannot straightforwardly be applied within DPs \citep{Carstens2001}. In §1.2 we will argue for a retreat from rich current models of \isi{Agree} to an impoverished model characterized by the absence of such constructs as multiple probes\slash goals or multiple directionality. In §2-3, we will address the evidence in (1--3) – shifting our theoretical focus to the nature of case, and specifically \is{case!oblique case}oblique case.

\subsection{Minimal Agree} % 1.2. 

The basic statement of \isi{Agree} is provided by \citet[122]{Chomsky2000} as follows: “Matching is a relation that holds of a probe P and a goal G. Not every matching pair induces \isi{Agree}. To do so, G must (at least) be in the domain D(P) of P and satisfy locality conditions. The simplest assumptions for the \isi{probe-goal} system are shown in (4).

\ea%4
    \label{ex:manzini:4}
    \ea Matching is feature identity.
    \ex D(P) is the sister of P.
    \ex Locality reduces to closest \isi{c-command}. 
\z
\z

Thus, D(P) is the \isi{c-command} domain of P, and a matching feature G is closest to P if there is no G' in D(P) matching P such that G is in D(G').” 

In the statement of the conditions for \isi{Agree} in (4), the absence of any mention of [interpretable]/[valued] features is rather striking, when compared to current \isi{Minimalist} practice. In the text surrounding (4) we are told, on the other hand, that “the erasure of \is{feature!uninterpretable feature}\is{feature!uninterpretable feature}uninterpretable features of probe and goal is the operation we call \isi{Agree}” \citep[122]{Chomsky2000}. It is in \citet{Chomsky2001Derivation} that the (un)interpretability asymmetry takes on a paramount role in the definition of \isi{Agree}: “uninterpretable features … constitute the probe [K] that seeks a matching goal – another collection of features – within the domain of … K. What is the relation Match? The optimal candidate is identity; we therefore take Match to be Identity” (5). The latter is the definition of \isi{Agree} adopted as the \isi{Minimalist} standard.

Furthermore “the natural principle is that the \is{feature!uninterpretable feature}uninterpretable features, and only these, enter the derivation without values, and are distinguished from \is{feature!interpretable feature}interpretable features by virtue of this property” so that a further probe\slash goal asymmetry in terms of the property [valued] is superimposed on the original [interpretable] one. As far as we can tell the two are equivalent in Chomsky’s work, though in the subsequent literature, they are sometimes treated as independent features (\citealt{Pesetsky2007}), or [interpretable] is abandoned in favour of [valued] \citep{Preminger2014}. 

 Now, life may be simpler without [interpretable]/[valued] features, as in (4), but as \citet[4]{Chomsky2001Derivation} points out “the existence of these features is a question of fact: does L have these properties or not? If it does (as appears to be the case), we have to recognize the fact and seek to explain it.” In other words, the question is whether there is independent evidence for these features – or, to be more precise, for their negative value, given that we take it for granted that there are interpretable, valued features (e.g. 1P, plural, etc.).

For \citet{Chomsky1995} the crucial empirical argument in favor of uninterpretability is that while verbs are routinely associated with singular or plural features, there is no sense in which the event they denote is singular or plural. However, \citet[21]{Manzini2007} argue “that the so-called agreement inflection of the verb is categorized exactly as a subject clitic; what is more, it bears a structural relation to the verb root which parallels that of a subject clitic (or any other subject) to the verb.” Therefore, Manzini \& Savoia’s counterargument is that inflections can in fact be interpreted if construed as (EPP) arguments of the verb\slash event. As they emphasize, two different conceptions of the syntax-morphology interface are implied by the two views of the verb inflection. For \citet{Chomsky1995} syntactic Merge takes entire words labelled by categories and sets of features. For Manzini \& Savoia Merge takes morphemes as its input and single morphemes are visible to syntactic computation.\footnote{Distributed Morphology (\citealt{Halle1993}) strikes a somewhat intermediate position, since it adopts the view that morphological structures are formed by Merger of morphemes, but at the same time it insulates Morphological Structure from syntax. Incidentally the notion of word can only be reconstructed in these frameworks as a derived notion, for instance through the notion of a word phase \citep{Marantz2007}; the visibility of agreement morphology suggested in the text could mean that it sits in an edge position.} 

\citet[4]{Baker2008} is generally critical of “the Chomskyan tradition,” where “the (often tacit) state of the art has been simply to stipulate which feature slots are present but unvalued on a particular lexical item, thereby specifying explicitly its agreement potential (\citealt{Chomsky2000}, \citealt{Chomsky2001Derivation}).” As he notes, “the idea behind this is that the probe has certain predefined feature slots that need to receive a value from some other phrase in the structure that has specified values for those same features … Instead, I am foregrounding the idea that all Fs are potential agreers and they agree with whatever features they can find in their environment according to structural principles” \citep[44]{Baker2008}.{}  

Nevertheless, Baker’s concern is that “stipulating what \is{feature!unvalued feature}unvalued features a given head has on a case-by-case basis does not capture the systematic differences in how verbs, \is{adjective}adjectives, and nouns behave with respect to agreement” – they do not have to do with the notion of valuation itself. Thus he adopts the principle in (5), which, as he notes, is simply a rather more explicit version of what Chomsky assumes. However, (5) does not change the empirical picture. The question concerning the status of the verb inflection is still in the same terms posed by \citet{Manzini2007}. If the verb inflection is a pronominal realization (or a pronominal copy) of the EPP argument, then it is referential and will carry interpretable\slash valued features.

\ea%5
    \label{ex:manzini:5}
    XP can have intrinsic $\varphi $-features (pre-specified values for person, number, and gender) only if XP has a referential index.
    \z

           

In fact, the transition from (4) to standard \isi{Minimalist} \isi{Agree} involves two steps. The first step involves introducing the [interpretable]/[valued] properties – and proposing that they pair up with \is{feature!phi-feature}phi-feature sets in the sentential domain as just discussed. The second step is that probes are \is{feature!uninterpretable feature}\is{feature!unvalued feature}uninterpretable\slash unvalued – though for \citet{Chomsky2001Derivation} goals must also be ‘active’ (i.e. have an uninterpretable \is{case!case feature}case feature). Most of the issues connected with \isi{Agree} discussed in the \isi{Minimalist} literature do not stem from any of the core properties in (4), namely identity, \isi{c-command} and locality, nor do they interact with such properties – rather, they are connected with the [interpretable]/[valued] properties and the identification of probes with uninterpretable\slash \is{feature!unvalued feature}unvalued feature sets. 

First, pre-encoding of probes allows for both downward and upward \isi{Agree} to be expressed (probe higher than goal or lower than goal; cf. \citealt{Zeijlstra2012}); the two directions are not expressible in the absence of pre-encoding. Thus, consider $\alpha $ and $\beta $ such that $\alpha $ \is{c-command}c-commands $\beta $; if there is no pre-encoding of probe\slash goal status on them, then it is logical to enforce the general direction of operations\slash relations defined by \isi{c-command}, creating an ordered pair ($\alpha $, $\beta $). However, only \isi{c-command} orders the two elements, and independently of \isi{c-command} the \isi{Agree} relation is perfectly symmetric. In other words, not only is the original statement in (4) simpler than standard versions of \isi{Agree} – it is also more restrictive, in the sense that it has less expressive power than a theory inclusive of [interpretable]/[valued] properties.

Similarly, pre-encoding probes and goals allows many-to-one or one-to-many \isi{Agree} to be expressed (one probe, many goals – or many probes, one goal); in the absence of pre-encoding, a set of features $\alpha $ simply acts as a probe on a set of features it immediately \is{c-command}c-commands $\alpha $ and $\beta $. If $\beta $ in turns acts as a probe for $\gamma $, the, we have a sequence of agreement pairs ($\alpha $, $\beta $), ($\beta $, $\gamma $) and so on. It is only the pre-encoding of \isi{probe-goal} status that allows the issue of multiple \isi{Agree} to be defined (one goal, several probes – cf. \citealt{Carstens2001} – or vice versa). Again, (4) not only is simpler than its current versions in cutting out certain extra assumptions – it also has less expressive power; i.e. it is more restrictive.

These apparently abstract questions take on empirical significance when we consider agreement within the DP – which is what concerns us here directly. Having a concrete example at hand, for instance \ili{Italian} (6), may help here. 

\ea%6
    \label{ex:manzini:6}
    \gll l-e    molt-e   bell-e    region-i  italian-e\\
         the-\textsc{f.pl}   many-\textsc{f.pl}   nice-\textsc{f.pl}   region-\textsc{pl}    \ili{Italian}-\textsc{f.pl} \\
    \glt ‘the many nice \ili{Italian} regions’
\z

A preliminary question is whether what is sometimes called \isi{concord} (DP-internally) is in fact the same phenomenon\slash rule as (sentence-internal) \isi{Agree}. Some theorists recognize two separate phenomena, subject to two different rules: sentential agreement is deemed to fall under Chomskyan (\isi{probe-goal}) \isi{Agree}, while DP-internal \isi{concord} responds to different processes; see for instance \citet{Giusti2008}. However, sentential and DP-internal agreement proper obviously share what Chomsky calls Matching – i.e. ‘the identity relation’ – and the locality conditions on it – indeed as laid out in (4). In this sense, it goes against commonly held measures of simplicity to postulate separate processes. 

Nor is there much PF evidence for the separation. Consider \ili{Punjabi}. In this language, verbs are participial forms agreeing in number and gender\slash nominal class. Therefore, morphologically it is impossible to separate agreement proper from \isi{concord} (contrary to \ili{English}). The same is true for nominal class agreement in Bantu \citep{Baker2008}. Even in familiar Western European languages, where it would appear that there is a strong realizational asymmetry between agreement on verbs and nouns, it can be shown that, given a statistically significant sample of varieties, (pro)nominal and verbal inflections admit of common lexicalizations (\citealt{Manzini2007} on Italo-\ili{Romance}).

Nevertheless, theorists arguing for a single \isi{Agree} process are faced by the issues of multiple probes\slash goals and directionality. First, while canonical sentential agreement involves one probe and one goal, DP-internal agreement may involve \textit{n} categories, for arbitrary \textit{n} (the head noun, its \isi{determiner}, its \isi{quantifier}, its adjectival modifier in (6)). \citet{Carstens2001} argues that this type of agreement should be modelled by allowing one goal – i.e. N – to check several probes – i.e. the set of \is{determiner}determiners and modifiers of N. This incidentally maintains the correct directionality of \isi{Agree}, with the probe higher than the goal. However, Carstens’ analysis meets considerable difficulties when we consider that if we take \is{feature!interpretable feature}interpretable features to be associated with N, we are forced to conclude that features associated with D are non-interpretable. Yet Ds alone appear perfectly capable of reference, implying the interpretability of their features (see \citealt{Danon2010} for more problems with D-N configurations). We may want to correct this state of affairs by changing the direction of agreement – i.e. having the goal higher than the probe – but then we are facing a further enrichment of the model, plus potential empirical problems having to do with the fact that gender\slash nominal class is clearly determined by N.

If we eliminate the [interpretable]/[valued] properties, or in any event we eliminate their pre-encoding on probes and goals, many potential problems are automatically eliminated, though we otherwise keep to the Chomskyan formulation of \isi{Agree} in (4) in all of its aspects. Put simply, each category within the DP in (6) acts as a probe for the immediately c-commanded category, all the way down from D to N and to the postnominal A. In other words, we surmise that Minimal Search and Match (the \isi{Agree} computation), as in (7a), should be retained, but its connections with [interpretable]/[valued] features should be severed, as in (7b). 

\ea%7
    \label{ex:manzini:7}
    \ea \isi{Agree} is the Minimal Search and Match operation formalized by \citet{Chomsky2000} – cf. (4). 
    \ex Further stipulations about ±interpretable and ±valued features, and their pre-encoding on probe\slash goals \citep{Chomsky2001Derivation} are eliminated.  
    \z
\z


For present purposes, we are interested in the fact that (7) facilitates the discussion of agreement in DP-internal contexts. In other words, in addressing the core concerns of this article – i.e. the unification of \isi{Suffixaufnahme} phenomena and the nature of \is{case!oblique case}oblique case – we will abstract away from any pre-encoding of features and of probe\slash goal status. Thus, in (6) \isi{Agree} creates pairs (\textit{regioni, italiane}), (\textit{belle, regioni}), etc., where the c-commanding element and the c-commanded element serve as probe and goal respectively and their \is{feature!phi-feature}phi-feature sets are identified. We have already provided the reasons why we consider it unlikely that DP-internal ‘\isi{concord}’ is separate from sentential \isi{Agree} – and why we believe that (7) extends to \is{agreement!subject-verb agreement}subject-verb agreement. However, as far as we can tell, the discussion in this article goes through even if something like (7) holds only for DP-internal positions. 

The main remaining problem is that in terms of \citet{Chomsky2000,Chomsky2001Derivation}, [interpretable] properties interact not only with the computational component, as we have been discussing, but also with the \isi{LF interface}. In Chomskyan terms, the deletion of \is{feature!uninterpretable feature}uninterpretable features is necessary because their permanence at the LF interface would violate Full Interpretation. But there is another reason why \isi{Agree} is crucial to Full Interpretation, namely {that at the LF interface there is a single interpreted copy of any \is{feature!phi-feature}phi-feature set, potentially identifying a referential argument; what is more, that copy is associated with the element that is capable of reference (cf. (5)).}

\citet{Preminger2014}{} points out that it is frequent to find \isi{default} inflections on verbs capable in principle of agreement – which it is natural to construe as those verbs not entering into \isi{Agree} at all.{} In Chomskyan terms, this would mean the survival of \is{feature!uninterpretable feature}uninterpretable features and the violation of Full Interpretation – which leads Preminger to reject the [interpretable] feature. He therefore {depends only on [valued] features for his formulation of \isi{Agree} – i.e.} {fi}{nd(f): “Given an \is{feature!unvalued feature}unvalued feature f on a head H0, look for an XP bearing a valued instance of f and assign that value to H0.”}\footnote{\citet{Preminger2014} proposes that when an unvalued probe fails to find a suitable goal, values are simply filled in by default in the morphology. It is worth pointing out that this may be necessary, but it is empirically insufficient. For instance, it has been known for a long time \citep{Kayne1989} that in \ili{French} or \ili{Italian} the perfect \isi{participle} agrees with the internal argument of \is{unaccusative}unaccusatives (eventually left in situ in \ili{Italian}) but does not agree with the internal argument of transitives, surfacing in the default masculine singular form. Evidently, in the second case it is not sufficient to say that there is no possible goal for the perfect \isi{participle} probe – because there is one, namely the internal argument.} { But this means that we are in the dark as to what the Full Interpretation algorithm does with all those valued feature sets. It appears that information about which sets are and are not interpretable is relevant for LF after all – however it is formulated (for instance in terms of (5), from \citealt{Baker2008}).}

   \citet{Manzini2007}, by contrast, are quite explicit about the interaction of their proposal with Full Interpretation at the LF interface. They propose that \isi{Agree} is a syntactic means for establishing equivalence classes between two or more copies of the same \is{feature!phi-feature}phi-feature material – to be interpreted at the interface as individuating a single referent. In other words, \isi{Agree} establishes that two sets of \is{feature!phi-feature}phi-features in fact reduce to two occurrences of the same set. Manzini \& Savoia speak of chain-formation. In reality, the notion of chain is unnecessarily rich – all that is needed is an unordered set. If \isi{Agree} is identity, its representational counterpart is an equivalence class. This equivalence class achieves roughly the same result as the survival of a single \is{feature!phi-feature}phi-feature set in \citet{Chomsky2000,Chomsky2001Derivation}.\footnote{The results are not exactly the same. For instance, {as already discussed, in DPs it is far from clear whether N or D ought to have the unique set of interpretable phi-featu}{r}{es at LF}{ predicted by \citet{Chomsky2001Derivation}. No such issue arises under the present proposal.}}

  In the discussion that follows we will consider data of the type presented in §1.1 on the basis of the ‘minimal’ theory of \isi{Agree} that we have sketched here. In essence the simplified version of the standard model in (7) allows us to tackle the empirical evidence without paying attention to matters such as the (un)interpretable nature of an agreeing node or to the direction of the \isi{Agree} operation. The emphasis will be on: (i) establishing the formal similarity of the internal structure of \is{linker}linkers and stacking; (ii) explaining their characteristic distribution in contexts (roughly) of adnominal modification.   

\section{Linkers} % 2. 

\subsection{Introduction}% 2.1. 

The very fact that case stacking and \is{linker}linkers are differently named points to the existence of surface dissimilarities between them, beyond the fact that the canonical environment for both of them is DP DP\textsubscript{Gen}, when the genitive DP is a modifier of the head DP. Case stacking has the case of the head DP realized as an extra suffix on DP\textsubscript{Gen} as in \ili{Lardil} (1); cf. (8) below. The \isi{linker}, realizing case and \is{feature!phi-feature}phi-feature agreement with the head DP, is an independent morpheme (a clitic of sorts) preceding DP\textsubscript{Gen} as in \ili{Albanian} (2); cf. (12ff.) below. Morphologically, the stacked case in \ili{Lardil} is a pure copy on DP\textsubscript{Gen} of the instrumental case realized on the DP head, while in \ili{Albanian} the \isi{linker} is sensitive to case and to \is{feature!phi-feature}phi-features as well. Importantly, however, the morphological differences cross-cut the basic syntactic difference between suffix (stacking) vs. independent morpheme (\isi{linker}). Thus, in \ili{Punjabi}, the stacked morphology is sensitive to \is{feature!phi-feature}phi-features and case, rather like the \ili{Albanian} \isi{linker}.  

  There are also considerable distributional similarities between stacking and \is{linker}linkers. First, the two phenomena have the same distribution, hinted at in (2) by the fact that a ‘primary’ and a ‘dependent’ are mentioned. Here is what \citet[46--47]{Richards2013} has to say about \ili{Lardil}: “In general, morphology that appears on a nominal is realized not just on the head noun but on everything dominated by the DP, including possessors, demonstratives, and \is{adjective}adjectives”, as in (1), repeated here as (8a), and in (8b). “Similarly, the Case of the head noun is realized on relative clauses modifying that noun”, as in (8c).  

\ea%8
    \ili{Lardil}  \citep[47]{Richards2013}\label{ex:manzini:8}\\
    \ea
    \gll Ngada latha   karnjin-i   marun-ngan-ku   maarn-ku.     \\
         I   spear   wallaby-\textsc{acc} boy-\textsc{gen-instr}  spear-\textsc{instr} \\
    \glt ‘I speared the wallaby with the boy’s spear.’
    \ex
    \gll Kara   nyingki   kurri   kiin-i     mutha-n   thungal-i. \\
         \textsc{q}    you     see   that-\textsc{acc}    big-\textsc{acc}    tree\textsc{{}-acc} \\
    \glt ‘Do you see that big tree?’
    \ex
    \gll Kara   nyingki kurri kiin-i     mutha-n thungal-i,   ngithun-i kirdi-thuru-Ø. \\
         \textsc{q}  you     see   that-\textsc{acc}    big-\textsc{acc}   tree-\textsc{acc}    I.\textsc{gen-acc}  cut-\textsc{fut-acc} \\
    \glt ‘Do you see that big tree, which I am going to cut down?’
    \z
\z

Adjectival modification, genitives and relative clauses are also the three core contexts for the insertion of \is{linker}linkers, as illustrated in (9) with the \ili{Persian} \textit{ezafe}. In standard \ili{Persian}, relative clauses in (9c-d) are introduced by the morpheme -\textit{i,} which is considered to be an allomorph of the \textit{ezafe} morpheme -\textit{e} introducing \is{adjective}adjectives and \is{case!genitive case}genitives in (9a-b) \citep{Samvelian2007}.

\ea%9
    \ili{Persian}\label{ex:manzini:9}\\
    \ea
    \gll asman-\textbf{e}   abi     \\
         sky-\textsc{lkr}    blue\\
    \glt ‘blue sky’        
    \ex  
    \gll ketab-\textbf{e}   Hasan   \\
         book-\textsc{lkr}    Hasan\\
    \glt ‘the book of Hasan’ 
    \ex  
    \gll zæn-\textbf{i}     [ ke  mæn  dust  daræm~]\\
         woman-\textsc{lkr} {}  that   I   love  give.\textsc{1sg.prs}\\
    \glt ‘the woman I love’
    \ex
    \gll bæcce-\textbf{i}   ke   lebas-a-ro     be-heš dad-æm\\
         child-\textsc{lkr}    that   clothes-\textsc{pl-dom}    to-him gave-\textsc{1sg} \\
    \glt ‘the child that I gave the clothes to’
    \z
\z

In the West Iranian language \ili{Kurmanji} \ili{Kurdish}, the same set of phi-feature-inflected \is{linker}linkers are realized in front of \is{adjective}adjectives, \is{case!genitive case}genitives, and relative clauses, as in (10--11).  

\ea%10
    \ili{Kurmanji} \ili{Kurdish}, Bahdini dialect (\citealt{Franco2015}: 279)\label{ex:manzini:10}\\
    \ea
    \gll kurk-(ak-)\textbf{e:}     mazən    \\
         boy-(one){}-\textsc{lkr.m}    big\\
    \glt ‘a big boy’
    \ex  
    \gll dest-\textbf{e}     kurk-i\\
         hand{}-\textsc{lkr.m}    boy{}-\textsc{obl.m} \\
    \glt ‘the hand of the boy’
    \z
\z         

\ea%11
    \ili{Kurmanji} \ili{Kurdish} \citep[203]{McKenzie1961}\label{ex:manzini:11}\\
    \gll aw   ḱas-\textbf{e:}     (ku)   {awwil\=\i} b-e:-t\\
         \textsc{dem}   person{}-\textsc{lkr.m}   (that)  first \textsc{subj}-come-\textsc{3sg.prs}\\
    \glt ‘that person who shall come first’
\z

 The constituent structure of \is{linker}linkers is of particular importance in establishing that they are structurally related to stacking. The literature is unanimous in concluding that the \isi{linker}, while eventually agreeing with the head noun in a modification structure, forms an immediate constituent with the modifier (\is{case!genitive case}genitive, \isi{adjective}, relative clause). In \ili{Albanian}, \is{linker}linkers appear in front of genitives (\is{adjective}adjectives, etc.) in predicative contexts with an overt copular ‘be’, as in (12). Copular sentences provide us with a straightforward argument for constituency, since the \isi{linker} that appears in front of the \is{case!genitive case}genitive DP, following the copula, must be part of the structure of the DP, as shown in (13). For the time being, in (13) we make no commitment to the category label of the ‘article’.

\ea%12
    Shkodër, Geg \ili{Albanian}\label{ex:manzini:12}\\
    \ea
    \gll Ky   ɑʃt   i   diɑl-i-t.     \\
         this  is  \textsc{lkr}  boy-\textsc{obl.sg-def}   \\
    \glt ‘This is of the boy’s.’  
    \ex
    \gll Ky   ɑʃt   i  ɲ{}-i    diɑl-i.\\
         this  is  \textsc{lkr}  a-obl.sg   boy-\textsc{obl.sg}   \\
    \glt ‘This is of a boy’s.’
    \z
\z
          
\ea%13
    \label{ex:manzini:13}
    [\textsubscript{Lkr} t  [\textsubscript{DP} ɲ-i diɑl-i ]]     
\z

The Iranian \textit{ezafe}, despite conventional orthography associating it with the head noun of a complex DP, also forms a constituent with the following modifier \isi{adjective} or \is{case!genitive case}genitive DP, as concluded by \citet{Larson2008} and \citet{Philip2012} among others. One argument in favour of this structure is that in sequences of more than one modifier, the last modifier bears no \textit{ezafe}, while other modifiers are obligatorily associated with it. This is true in \ili{Persian} (14) and in \ili{Kurmanji} \ili{Kurdish} (15), despite other differences, for instance the fact that the \textit{ezafe} is invariable in Persian and agrees with the head noun in \ili{Kurdish}. If the \textit{ezafe} forms a constituent with the following modifier, as indicated by our brackets, the last modifier of the sequence is correctly predicted to be \textit{ezafe}{}-free. 

\ea%14
         \ili{Persian} (\citealt{Samvelian2007}: 606, our brackets)\label{ex:manzini:14}\\
    \gll in   ketâb-[e   kohne-[ye   bi arzeš-[e     maryam]]]\\
         this   book{}-\textsc{lkr}    ancient{}-\textsc{lkr}    without value{}-\textsc{lkr}    Maryam\\
    \glt ‘this ancient worthless book of Maryam’s’     
    \z

  
\ea%15
    \ili{Kurmanji} (\citealt{Yamakido2005}: 121, our brackets)\label{ex:manzini:15}\\
    \ea
    \gll kitêb-ek-[e    bas-[e    nû]]\\
         book-\textsc{indef-lkr}   good-\textsc{lkr}   new\\
    \glt ‘a good new book’
    \ex  
    \gll xani-yek-[î    bas-[î    nû]]\\
         house-\textsc{indef-lkr}   good-\textsc{lkr}   new\\
    \glt ‘a good new house’
    \z
\z    

Further evidence comes from \isi{coordination}. \citet[37ff.]{Philip2012} shows that in \ili{Persian}, when the head noun is coordinated, there can only be one \textit{ezafe} on the coordinated head, next to the modifier, as in (16). In other words, the \textit{ezafe} is an integral part of the modifier, not of the modified noun. Therefore in Iranian, adjectival modifiers have the same structure as in \ili{Albanian}, namely the one indicated in (16) for \ili{Persian}.

\ea%16
         \ili{Persian} \citep[38]{Philip2012}\label{ex:manzini:16}\\
    \gll\relax [kolâh(*-e)   va  lebâs][-e   Maryam] \\
          hat-\textsc{lkr}   and   dress-\textsc{lkr}   Maryam\\
    \glt  ‘Maryam’s hat and dress.’ 
\z
                    
The languages exemplified, namely \ili{Albanian}, \ili{Kurdish} and \ili{Persian}, all display the head-complement\slash modifier order, at least within the DP; they also uniformly have head-final (i.e. suffixal) morphological structures. Therefore we know that the \isi{linker} structure with an independent head in (13) differs from the suffixed structure of \ili{Lardil}. Leaving this aside, \is{linker}linkers and stacking structures involve the presence of a copy of the \is{feature!phi-feature}phi-feature\slash case specifications of a head DP within the projection of a modifier DP\slash AP\slash CP.   

It is also worth returning briefly to the question of morphological differences. \citet{Franco2015} have access to dialect variation data within \ili{Albanian} (\citealt{Manzini2011Grammatical,Manzini2011Reducing}), as well as within \ili{Kurdish}. In the Shkodër Geg \ili{Albanian} variety in (12) the pre-ad\-jec\-ti\-val \isi{linker} varies according to the gender, number and case of the head noun; specifically, it takes the form \textit{i} for the nominative masculine singular, \textit{ɛ} for the nominative feminine singular and for the accusative, and \textit{t} for the oblique, as in (17--18) (cf. \citealt{Solano1972}; \citealt{Camaj1984}; \citealt{Turano2004}; \citealt{Campos2008} for standard \ili{Albanian}).

\ea%17
    Shkodër, Geg \ili{Albanian}\label{ex:manzini:17}\\
    \begin{xlista}
    \ex
    \gll diɑl-i      i     mɑð           \\
         boy-\textsc{nom.m.def}   \textsc{lkr.nom.m}  grown-up.\textsc{m}\\
    \ex
    \gll diɑli-n     e     mɑð    \\
         boy-\textsc{acc.m.def}     \textsc{lkr}    grown-up.\textsc{m}\\
    \ex
    \gll diɑli-t              t     mɑð  \\
         boy-\textsc{obl.m.def}  \textsc{lkr.obl}  grown-up.\textsc{m}\\
    \glt ‘(to) the big boy’
    \exi{a'.}
    \gll vɑiz-a     ɛ     mɑðɛ    \\
         girl-\textsc{nom.f.def}   \textsc{lkr}    grown-up.\textsc{f} \\
    \exi{b'.}
    \gll vɑizə-n     ɛ     mɑðɛ    \\
         girl-acc.f.def    \textsc{lkr}    grown-up.\textsc{f}\\
    \exi{c'.}
    \gll vɑiz-əs    t     mɑðɛ \\
         girl-\textsc{obl.f.def}    \textsc{lkr.obl}  grown-up.\textsc{f}\\
    \glt ‘(to) the big girl’
    \end{xlista}
\z


\ea%18
    Shkodër, Geg \ili{Albanian}\label{ex:manzini:18}\\
    \ea
    \gll diem-t     e   mɑði     \\
         boys-\textsc{dir.pl.def}  \textsc{lkr}  grown-up.\textsc{m.pl}  \\
    \ex
    \gll diem-vɛ    t     mɑði \\
         boys-\textsc{obl.pl}     \textsc{lkr.obl}  grown-up.\textsc{m.pl}\\
    \glt ‘(to) the big boys’  
    \z
\z

In the Arbëresh (Italo-\ili{Albanian}) varieties discussed by \citeauthor{Manzini2011Grammatical}\linebreak (\citeyear{Manzini2011Grammatical}), on the other hand, the pre-ad\-jec\-ti\-val \isi{linker} only agrees with the head noun in \is{feature!phi-feature}phi-features (number, gender) and displays no sensitivity to case. The variation internal to Iranian languages follows the same parameters as the variation between \ili{Albanian} dialects. The \ili{Persian} \textit{ezafe} is a non-agreeing morpheme \textit{e\slash i}. In \ili{Kurmanji} \ili{Kurdish} (19), the \isi{linker} has three realizations, namely \textit{e} for the masculine, \textit{a} for the feminine and \textit{et} for the plural. In Hawrami \ili{Kurdish} in (20), the adjectival \textit{ezafe} has different realizations, \textit{{}-i}, \textit{{}-æ}, \textit{{}-e}, depending on the number and definiteness of the head noun. At the same time, Hawrami \ili{Kurdish} distinguishes the adjectival \textit{ezafe} from the \is{case!genitive case}genitival one, since the latter takes the invariable -\textit{u} form.

\ea%19
    \ili{Kurmanji} \ili{Kurdish} \citep{Franco2015}\label{ex:manzini:19}\\
    \ea
    \gll kurk-(ak-)e:     mazən       \\
         boy-(one)-\textsc{lkr.m}    big     \\
    \glt ‘a/the big boy’
    \ex
    \gll ketʃk-(ak-)ɑ:    mazən      \\
         girl-(one)-\textsc{lkr.f}   big     \\
    \glt ‘a/the big girl’
    \ex
    \gll kurk-e:t / ketʃk-e:t     mazən     \\
         boy-\textsc{lkr.pl} / girl-\textsc{lkr.pl}  big   \\
    \glt ‘the big boys/girls’
    \z
\z


\ea%20
    Hawrami \ili{Kurdish} (\citealt{Holmberg2008}: 132)\label{ex:manzini:20}\\
    \ea
    \gll æsp-i     sya:w      \\
         horse-\textsc{lkr}  black\\
    \glt ‘black horse’
    \ex
    \gll æsp{}-æ     zɪl-ækæ\\
         horse-\textsc{lkr.def}  big-\textsc{def}\\
    \glt ‘the big horse’ 
    \ex
    \gll due   æsp-e    zɪl-e \\
         two   horse-\textsc{lkr.pl}  big-\textsc{pl} \\
    \glt ‘two big horses’
    \ex
    \gll pæl-u     hało-i\\
         feather-\textsc{lkr}   eagle-\textsc{obl}\\
    \glt ‘eagle’s feather’ 
    \z
\z        

In conclusion, stacking (involving a suffix as in (8)) and \is{linker}linkers (involving a head) display the same syntactic distribution (roughly, adnominal modification). Constituency tests also show that the \isi{linker} is internal to the structure of the modifier – no less than stacked suffixes. Finally, \is{linker}linkers can display agreement in \is{feature!phi-feature}phi-features or in case or an invariant form – providing no basis for differentiating them from stacked morphology, which also may involve case (\ili{Lardil}) and\slash or \is{feature!phi-feature}phi-features (\ili{Punjabi}, cf. §1.1).

\subsection{Analysis}

As already mentioned in §1, only a few theorists see \is{linker}linkers as agreement heads, most recently \citet{Philip2012} and \citet{Franco2015}; cf. also \citet{Zwart2006}. Franco et al. provide a detailed survey of why the other construals of \is{linker}linkers proposed in the literature meet empirical difficulties in accounting for \isi{linker} phenomena crosslinguistically. Consider the idea that \is{linker}linkers are a means for avoiding NN sequences, embraced by \citet{Richards2010} for \ili{Persian}. Franco et al. note that in \ili{Albanian} \is{linker}linkers create obvious identical sequences of their own. Thus, consider the oblique singular in (17c), \textit{djali-t t mɑð} ‘the boy \textsc{lkr} big’; the \isi{linker} reproduces the definiteness, case and \is{feature!phi-feature}phi-features of the head noun, yielding a morphological copy of the N’s ending. It is far from clear in what sense the \isi{linker} would contribute to identity avoidance.

  Case theories of \is{linker}linkers (\citealt{Larson2008}) construe the \isi{linker} as a way of assigning case to adnominal modifiers, both DPs and APs, which could not be assigned case by the head N. Again, this idea is difficult to transpose from a language like \ili{Persian} which has very little inflectional morphology (and no inflectional case) to a morphologically rich language like \ili{Albanian}. To take up \textit{djali-t t mɑð} ‘the boy \textsc{lkr} big’ in (17c) again – it is unclear why a \isi{linker} which exactly reproduces a piece of the head N would be able to assign case while the head N is not able to do so.    

  Finally, \citet{Dikken2004} take \is{linker}linkers to be copulas – effectively the counterpart of the verb ‘be’ in the DP domain. It appears, however, that the fact that \is{linker}linkers can be found in predicative contexts, such as \ili{Albanian} (12), weakens this theory considerably; since the copula is already lexicalized, it is hard to see what role the \isi{linker} could play. In fact, the (typologically rare) occurrence of \is{linker}linkers in predicative position provides counterexamples to the theory of \is{linker}linkers as breaking identical *NN sequences – or as assigning case in the presence of an N head. 

  We conclude in favour of the construal of \is{linker}linkers as agreement heads – which is interesting in the context of the present discussion because case stacking is also essentially an agreement phenomenon. As also mentioned in §1, \is{linker}linkers present the standard theory of \isi{Agree} with a considerable challenge. Consider for instance \ili{Albanian} (2b), repeated here as (21a) – with the structure in (21b).

\ea%21
    \label{ex:manzini:21}
    \ea Shkodër, Geg \ili{Albanian} (\citealt{Manzini2011Reducing}: 105)
    \gll kɑ:m-a      ɛ          tʃɛn-it          \\
         paw-\textsc{f.sg.nom.def}   lkr-\textsc{f.sg.nom.def}  dog-\textsc{m.sg.obl.def} \\
    \glt ‘the paw of the dog’
    \ex\relax [\textsubscript{DP} kɑ:ma    [\textsubscript{Lkr} ɛ  [\textsubscript{DP} tʃɛnit ]]] 
    \z
\z

 Within a standard \isi{Minimalist} framework, it is assumed that \is{feature!phi-feature}phi-features are interpretable on nouns, and \is{feature!uninterpretable feature}\is{feature!phi-feature}uninterpretable on functional heads acting as probes for the Noun \citep{Carstens2001}. Therefore in (21a) \textit{kɑ:ma} ‘the paw’ is the goal for a probe associated with the \isi{linker}, conceived of as a pure bundle of \is{feature!phi-feature}phi-features and case.\footnote{In (21b) the potential goal – i.e. \textit{‘}paw’ – \is{c-command}c-commands the potential probe – i.e. the \isi{linker} structure. This change in directionality is allowed under certain models. Furthermore, the literature on languages with ‘post-nominal Ds’ (or definite inflections) consistently assumes that the noun (e.g. \textit{‘}paw’) moves from a lower (post-modifier) position to a higher (pre-modifier) position; see \citet{Turano2002}, \citet{Dimitrova-Vulchanova1998} for different implementations. Under this analysis, in (21b) there is a copy of ‘paw’ lower than the \isi{linker} structure.} Probe status in standard \isi{Minimalist} theory is associated with uninterpretability. Therefore, there is a syntactic head, namely the \isi{linker}, that entirely consists of \is{feature!uninterpretable feature}uninterpretable features. This is actually predicted to be impossible by \citet{Chomsky1995}. For, under Full Interpretation at the C-I interface, we expect uninterpretable material to be deleted; but if a head consists of uninterpretable material, then this leads to the deletion of structure – which violates Inclusiveness. In other words, classical Minimalism requires heads to be interpretable – but \isi{linker} heads must be probes and hence uninterpretable. 

  This leaves the approach taken here in (7) and embraced by \citet{Franco2015}, who assume that \isi{Agree} works on sets of features which are uniformly interpretable. Their approach is best appraised starting with the simpler \isi{linker} structure involving adjectival modification. Consider for instance \ili{Albanian} (17a), repeated as (22) for ease of reference. 

\ea%22
    Shkodër, Geg \ili{Albanian}\label{ex:manzini:22}\\
    \gll diɑl-i       i   mɑð       \\
         boy-\textsc{nom.m.def}   \textsc{lkr.m}  grown-up.\textsc{m}  \\
    \glt ‘(to) the grown-up boy’ 
    \z

\citet{Franco2015} for \ili{Albanian}, as well as \citet{Lekakou2012} for \ili{Greek}, take the category of the \isi{linker} to be D, based (among other things) on its morphological identity with the definite article (\ili{Greek}) or the definite inflection (\ili{Albanian}; cf. table (18)). They further adopt \citegen{Higginbotham1985} proposals as to the interpretation of D-N sequences such as \ili{English} \textit{the boy}. The N \textit{boy} is a predicate denoting the set of individuals with the property ‘boy’; its argumental slot (called the R-role; cf. \citealt{Williams1994}) needs to be saturated by the \isi{determiner}. Suppose we mechanically apply this analysis to \ili{Albanian} (22). The predicate \textit{mɑð} ‘small’ must be satisfied by an argument, which is provided by the D element, as in (23).{}  

\ea%23
    \label{ex:manzini:23}
    \begin{forest}
    [~
        [D\\i\textsubscript{x}]
        [A\\mað\textsubscript{λx}]
    ]
    \end{forest}
\z
%%\includegraphics[width=\textwidth]{OGSVolumeAug2018ManziniFrancoSavoia-img12.jpg}
%%please move the includegraphics inside the {figure} environment

 
This is also the construal provided for \ili{Greek} pre-ad\-jec\-ti\-val \is{linker}linkers by \citet{Lekakou2012}, who distinguish the D category assigned to \is{linker}linkers from the Def category assigned to the definite operator. \citet{Franco2015} maintain the same label D for both, further assuming that all Ds have definiteness properties, besides being associated with nominal class (gender) and number features. Consider their structure (24) for example (22) (slighly simplified). The lower D simply values the argument slot of A, awaiting further quantificational closure. The higher D differs from it in that it is interpreted as a \isi{quantifier}; i.e. as indicating that there is an individual (or set of individuals, or a unique\slash familiar etc. individual, and so on) to which the properties of the NP predicate and those of the sentential predicate both apply (or not). Following \citet{Higginbotham1985}, adjectival modification involves the identification of the \isi{theta-role} of the \isi{adjective} with the R-role of the noun (here x=y); the same argument (the noun phrase’s \isi{determiner}, according to Higginbotham) satisfies both – whence the intersective reading of \isi{adjective} modification. As a result, the \isi{linker} D in (24) is essentially a bound variable of the operator D – much like a resumptive clitic may be a bound variable of a higher definite description.

\ea%24
    \label{ex:manzini:24}
    \begin{forest}
        [DP
            [D\textsubscript{x=y}]
            [NP
                [N\\diali\textsubscript{λy}]
                [DP
                    [D\\i\textsubscript{x}]
                    [A\\mað\textsubscript{λx}]
                ]
            ]        
        ]
    \end{forest}
\z
%%please move the includegraphics inside the {figure} environment
%%\includegraphics[width=\textwidth]{OGSVolumeAug2018ManziniFrancoSavoia-img13.jpg}

Syntactically, theta-unification depends on \isi{Agree}. Recall that following §1, \is{feature!phi-feature}phi-features are not precompiled as \is{feature!uninterpretable feature}\is{feature!interpretable feature}uninterpretable or interpretable (valued\slash unval\-ued etc.) \is{feature!unvalued feature}on certain heads; thus, probe and goal status depend only on the syntactic configuration (ultimately \isi{c-command}). In (24) the N \textit{diali} ‘the boy’, in virtue of the \is{feature!phi-feature}phi-features associated with its D inflection, acts as a probe for the embedded pre-ad\-jec\-ti\-val D \isi{linker} \textit{i}, its (closest) goal – in (24). Under \isi{Agree}, \is{feature!phi-feature}phi-features must be matched; if they are not matched then \isi{Agree} fails – and so ultimately does the interpretation at the LF interface, which returns no single argument satisfying both N and A. 

Adjectival modification, as in (24), is not directly relevant to stacking, but \is{case!genitive case}genitival modification is at the core of it. With the background we have now established on pre-ad\-jec\-ti\-val \is{linker}linkers, we are ready to tackle \is{case!genitive case}pre-genitival \is{linker}linkers. The latter are more complex, because they involve an analysis of \is{case!genitive case}genitive case. The standard \isi{Minimalist} approach to case, namely that case is parasitic on agreement, is formulated by \citet{Chomsky2000,Chomsky2001Derivation} for direct cases; i.e. nominative and accusative. We assume that this view is fundamentally correct (notwithstanding \citealt{Baker2010}). Nevertheless, it does not have any immediate implications for \is{case!oblique case}oblique case, of which \is{case!genitive case}genitive is an example (on morphological grounds, among others). To be more precise, an \isi{Agree} approach could be made to work, at least within the sentential domain, by postulating Appl heads corresponding to dative and instrumental case (\citealt{Pylkkänen2008}) – yet we are not aware of this approach being pursued at all DP-internally.

Following in essence the theory of obliques originally suggested by \citet{Fillmore1968}, we assume that \is{case!oblique case}oblique case inflections, like Ps (prepositions\is{preposition} or \is{postposition}postpositions), have a relational content. ‘Possessor’ in turn is the traditional characterization of \is{case!genitive case}genitives. Following \citet{Belvin1997}, writing on the verb ‘have’, we take the relevant characterization of possession to involve ‘inclusion’. Following \citet{Manzini2011Reducing}, we notate it as ($\subseteq$), to suggest that a part\slash whole interpretation is involved. Putting together this proposal with the proposal on \is{linker}linkers in (24), we obtain the representation in (25) for \ili{Albanian} (21). The \is{case!genitive case}genitive noun is formed by the base \textit{tʃɛni} ‘dog’ merged with the case ending -\textit{t}. The latter encodes a relational part\slash whole content ($\subseteq$), which projects a ($\subseteq$)P complement of the head noun \textit{kɑ:ma} ‘paw’. In imputing a relational content to -\textit{t}, we imply that it connects two arguments. One is the possessor ‘dog’ – namely the noun to which the oblique inflection attaches. The other argument is ultimately the possessum ‘paw’. 

\ea%25
    \label{ex:manzini:25}
\begin{forest}
    [NP
        [N\\kɑ:ma]
        [($\subseteq$)P
            [D\\ɛ\textsubscript{y}]
            [($\subseteq$)
                [N\\tʃɛni\textsubscript{x}]
                [($\subseteq$)\\t\textsubscript{λx,λy}]
            ]           
        ]
    ]
\end{forest}
    \z 
%%please move the includegraphics inside the {figure} environment
%%\includegraphics[width=\textwidth]{OGSVolumeAug2018ManziniFrancoSavoia-img14.jpg}

What is the status of the \isi{linker} ɛ in (25) and of the agreement it enters into with the head noun? We assign to the \isi{linker} the same D categorization that we adopted for the pre-ad\-jec\-ti\-val context in (24), where we saw that the \isi{linker} provides a partial discharge of the argumental role of the adjectival predicate. We have just proposed that an \is{case!oblique case}oblique case, specifically the \is{case!genitive case}genitive, is an elementary predicate, connecting two arguments (possessor and possessum) via a part\slash whole relation. As already stated, \textit{tʃɛni} ‘dog’ is the internal argument of the ($\subseteq$) case relation (i.e. the possessor or ‘whole’); the \isi{linker} provides a partial saturation of the ($\subseteq$) predication inside the ($\subseteq$)P projection. Recall that the correct (intersective) interpretation of the adjectival modification structure in (24) depends on agreement between the head N and the \isi{linker}, ultimately establishing that there is a single argument satisfying both the N’s and the A’s argumental slot. Similarly, in (25), the head N in virtue of its \is{feature!phi-feature}phi-features acts as a probe for the embedded D \isi{linker}. This allows \textit{kɑ:ma} ‘paw’ to be ultimately interpreted as the external argument of ($\subseteq$).

A considerable number of questions are raised by the account of \is{linker}linkers in (24--25). One which is of particular interest here regards the agreement relation between the head N and the \isi{linker} in (25). The fact is that the configuration in (25) is equally compatible with a different derivation, under which the \isi{linker} probes for the embedded N \textit{tʃɛni} agreeing with it. Full Interpretation at the interface should be achieved anyway, interpreting the \isi{linker} as doubling the \is{case!genitive case}genitive – and the head noun as before as representing the other argument in the ($\subseteq$) relation. Interestingly, we can show that this configuration, though impossible in \ili{Albanian}, is attested in other languages. A case in point is \ili{Aromanian}, which differs in this respect even from its closest cognate, \ili{Romanian}.\footnote{Our data are from varieties of \ili{Aromanian} spoken in South Albania (in the towns of Fier, Diviake and Libofshe).} \ili{Aromanian} has pre-ad\-jec\-ti\-val \is{linker}linkers, which are not present in \ili{Romanian} (\citealt{Campos2008}; \citealt{Cornilescu2013}). On the other hand, in both \ili{Romanian} and \ili{Aromanian} the \isi{linker} \textit{al} is a form of the definite article (cf. Latin demonstrative \textit{ille}) \citep{Giurgea2012}, as in (26--27). While the \isi{linker} agrees with the head noun in \ili{Romanian} (26), it agrees with the embedded \is{case!genitive case}genitive in \ili{Aromanian} (27).

\ea%26
         \ili{Romanian}\label{ex:manzini:26}\\
    \gll două   kămăş-i   ale   băiat-ul-ui     \\
         two   shirts-\textsc{f.pl}  \textsc{lkr}  boy-\textsc{def-obl.m.sg}   \\
    \glt ‘two shirts of the boy’ 
    \z


\ea%27
         \ili{Aromanian} (\citealt{Franco2015}: 324) \label{ex:manzini:27}\\
    \gll libr-a       o   {fit or-u} /  ali   fet-i     \\
         book-\textsc{def.f.sg}   \textsc{lkr}  boy\textsc{{}-m.sg} /   \textsc{lkr}  girl-\textsc{obl.f.sg}   \\
    \glt ‘the boy’s/the girl’s book’
    \z


We can assign to the \ili{Aromanian} \isi{linker} in (27) the same constituent structure assigned to \ili{Albanian} (25), as shown in (28). Interpretively, on the other hand, the \ili{Albanian} \is{case!genitive case}pre-genitival \isi{linker} provides a lower-level satisfaction for an argument slot ultimately bound by the N head of the DP. The \is{case!genitive case}pre-genitival \isi{linker} of \ili{Aromanian}, by contrast, is a copy of the \is{feature!phi-feature}phi-features of the \is{case!genitive case}genitive itself. Recall now that under our proposed formulation of \isi{Agree}, any \is{feature!phi-feature}phi-feature set can act as a probe or as a goal, according simply to the \isi{c-command} configuration. In principle, it is therefore possible that the \is{feature!phi-feature}phi-feature set corresponding to the \isi{linker} head acts as a probe for the embedded \is{feature!phi-feature}phi-feature set. This configuration appears to be realized in \ili{Aromanian} (28).

\ea%28
    \label{ex:manzini:28}
    \begin{forest}
    [NP
        [N\\libra]
        [($\subseteq$)P
            [D\\ali]
            [($\subseteq$)
                [N\\fet]
                [($\subseteq$)\\i]
            ]
        ]
    ]
    \end{forest}
\z
%%please move the includegraphics inside the {figure} environment
%%\includegraphics[width=\textwidth]{OGSVolumeAug2018ManziniFrancoSavoia-img15.jpg}

Suppose we precompile (un)interpretability on lexical and functional heads in the sentence and in the DP. Then it stands to reason that the same element will have interpretable or uninterpretable status cross-linguistically (especially if lexical identity is involved, as in \ili{Romanian} and \ili{Aromanian} \textit{ale\slash ali}). Thus, suppose the \is{case!genitive case}pre-genitival \isi{linker} is uninterpretable, acting as a probe; everything else being equal, we expect its goal to be uniformly the \is{case!genitive case}genitive or the head noun. This is the position argued for by \citet{Philip2012}, for whom \isi{linker} configurations must involve agreement with the head of the DP. According to \citet{Philip2012} there are hardly any known exceptions to the predicted state of affairs – yet \ili{Aromanian} (a relatively familiar language) must be added to her list. 

  On the present view, (un)interpretability is not essential to the working of \isi{Agree}. Therefore, in the absence of pre-encoded features on the \isi{linker}, we allow it to act as a probe for the lower \is{case!genitive case}genitive (\ili{Aromanian}) – as we also allow the more frequently observed configuration where the N head of the DP acts as a probe for the \isi{linker}. The parametric choice ultimately depends on the fact that the ($\subseteq$) elementary predicate has two possible arguments; the \isi{linker} may match the \is{case!genitive case}genitive and agree with it or introduce an instance of the external argument and agree with the head N. We assume, as is generally done, that the phasal organization of grammar prevents the \is{feature!phi-feature}phi-features of the higher N from probing into the lower N; in other words, assuming that a DP is a phase, the two N heads (overtly or covertly closed by a D operator) are in two separate phases, preventing \isi{Agree} from applying.  

  Finally, besides pre-ad\-jec\-ti\-val and \is{case!genitive case}pre-genitival contexts, \is{linker}linkers are found in relative clauses. We exemplified this context with \ili{Kurmanji} \ili{Kurdish} in (11), repeated in (29a) for ease of reference, where the \isi{linker} agrees in \is{feature!phi-feature}phi-features (masculine singular) with the head noun and precedes the relative pronoun \textit{ku.} We construe it in the same manner as both \is{case!genitive case}pre-genitival and pre-ad\-jec\-ti\-val \isi{linker} structures, as in (29b). In fact, adjectival modification bears a particularly close relation to modification by a relative clause. Both contexts involve the conjunction of two predicates, one represented by the predicative content of the head noun and the other represented by the \isi{adjective} or the relative clause. Indeed, wh-relative pronouns are lambda operators turning the embedded sentence into a predicate with an open slot. In the spirit of our proposal concerning pre-ad\-jec\-ti\-val \is{linker}linkers in (24), the \isi{linker} \textit{e} in (29b) introduces a partial saturation of the relative clause predicate. \isi{Agree} applies between the embedded \isi{linker} and the N head of the relative and ensures that the open slot of the relative clause is ultimately bound by the N head.\footnote{\ili{Albanian} has two separate strategies for the formation of relative clauses. One involves the relative pronoun \textit{që}, comparable to the \ili{English} ‘that’ used to relativize direct arguments. A second strategy uses the relative expression \textit{i} \textit{cili} etc. inflected for \is{feature!phi-feature}phi-features and case and introduced by an article, namely \textit{i} in the masculine singular nominative, \textit{e} in the feminine singular nominative and \textit{të} elsewhere. This way of forming relative clauses is disfavoured when relativizing direct arguments, as in (i), but is obligatory when obliques are relativized, as in (ii); note also the obligatory presence of a resumptive clitic. The question is whether the article in (i--ii) is a \isi{linker} in the sense of (29b), or whether it forms part of a complex relative pronoun, on the model of \ili{French} \textit{lequel}, \ili{Italian} \textit{il quale}, etc. There are indications that the latter is correct, for instance the occurrence of the entire phrase under \is{preposition}prepositions: \textit{burrin prë të cilit} ‘the man for whom …’.
  
  \ea   \gll    ?Kam  parë   burrin   të   cilin   e   thirre\\
                I.have   seen   the.man  \textsc{art}   whom  him   you.called\\
  \glt  ‘I saw the man whom you called’
  \z
  
  \ea   \gll    Kam  parë   burrin   të   cilit     i   ke     dhënë   librin\\
                I.have   seen  the.man  \textsc{art}   to.whom    him   you.have   given   the.book\\
        \glt    ‘I saw the man to whom you gave the book’\\
\z}  

\ea%29
    \ea \ili{Kurmanji} \ili{Kurdish} (\citealt{McKenzie1961}: 203)\label{ex:manzini:29}\\
    \gll aw   ḱas-\textbf{e:}     (ku)   {awwil\=\i}   b-e:-t   \\
         \textsc{dem}  person-\textsc{lkr.m}   (that)   first \textsc{subj}-come-\textsc{3sg.prs}\\
    \glt ‘That person who shall come first.’
    \ex
    \begin{forest}
    [NP
        [N\\ḱas]
        [QP
            [D\textsubscript{x}\\e]
            [QP
                [Q\\ku\textsubscript{λx}]
                [,nice empty nodes]
            ]
        ]
    ]
    \end{forest}
    \z
\z
%%please move the includegraphics inside the {figure} environment
%%\includegraphics[width=\textwidth]{OGSVolumeAug2018ManziniFrancoSavoia-img16.jpg}

As indicated at the outset, the empirical focus of the present article is not \is{linker}linkers per se, but rather their unification with case\slash agreement stacking. We now have an analysis of \is{linker}linkers. If we are correct, \isi{Agree} can achieve descriptive adequacy without employing any assumptions about features being interpretable or not interpretable, valued or not valued. Specifically, \isi{Agree} can remain a simple one probe, one goal relation, without having to have access to multiple probing and\slash or multiple goals. Furthermore, there is no reason to modify the simplest \isi{c-command} configuration of probe and goal in order to account for the variation between \ili{Albanian} (25) and \ili{Aromanian} (28).  More importantly, from the point of view of a unification of \is{linker}linkers and stacking, accounting for \is{linker}linkers implied providing a baseline account of \is{case!oblique case}oblique case – or at least of \is{case!genitive case}genitive case. As we will see in §3, the basic descriptive problem of case\slash agreement stacking is that the inner case must always be an oblique. Our account will build on the treatment of obliques as elementary relations developed in this section in relation to \is{linker}linkers.

\section{Case/agreement stacking} %3
\subsection{Punjabi}% 3.1.
In order to understand the \ili{Punjabi} data, it is useful to have a sketch of \ili{Punjabi} morphosyntax at hand \citep{Bhatia2000}.\footnote{{Our \ili{Punjabi} data come} from the Doabi variety spoken in the Indian town of Hoshiarpur. {The \is{case!genitive case}genitival construct illustrated in (32) below for \ili{Punjabi} characterizes several Indo-Aryan and Dardic languages \citep{Payne1995}.} }{} In \ili{Punjabi}, there are two genders, masculine and feminine. A subset of masculine nouns present the inflection \textit{{}-a} in the non-oblique singular form (30a) and \textit{{}-e} in the oblique singular, i.e. when it is followed by a \isi{postposition}, and in the non-oblique plural (30b). The oblique plural masculine, i.e. followed by a \isi{postposition}, is in turn realized as -\textit{ea} (31c). The feminine does not display a specialized oblique form. At least some feminine nouns present the inflection \textit{{}-a} in the plural, as in (31a-a’); another subset of them alternates between a singular form with final \textit{{}-i} and a plural with \textit{{}-ĩa}\~{} , as in (31b-b’). 

\ea%30
    \label{ex:manzini:30}
    \ea muɳɖ-a
        \glt ‘boy-\textsc{m.sg}’
    \ex muɳɖ-e
        \glt ‘boy-\textsc{m.sg.obl}/boy-\textsc{m.pl}’  
    \ex muɳɖ-ea
        \glt ‘boy-\textsc{m.pl.obl}’
\z
\z

\ea%31
    \label{ex:manzini:31}
        \begin{xlista}
        \ex  kita:b        
        \glt ‘book.\textsc{f.sg}’
        \exi{a'.} kitabb-a      
        \glt      ‘book-\textsc{f.pl}’ 
        \ex   kuɾ-i         
        \glt  ‘girl-\textsc{f.sg}’
        \exi{b'.}  kuɾ-ĩã         
        \glt      ‘girl-\textsc{f.pl}’
        \end{xlista}
\z


A \is{case!genitive case}genitive modifying a noun bears its own (oblique) \is{feature!phi-feature}phi-feature inflection, followed by the case \isi{postposition} \textit{d-} and then by a \is{feature!phi-feature}phi-feature inflection agreeing with the modified noun.\footnote{\is{case!genitive case}Genitive in \ili{Punjabi} yields a person split of sorts, since it is realized as \textit{d-} on lexical nouns, but as \textit{r-} on Participant (1\slash 2 person) pronouns, as in (i). In either instance, the \is{case!genitive case}genitive \isi{postposition} is followed by an inflection agreeing with the head noun.

\ea \gll    te-r-i/-ĩã  kəmiddʒ/kəmiddʒ-a\\
            you-\textsc{gen-f.sg/-f.pl}
            shirt.\textsc{f.sg}/shirt-\textsc{f.pl}\\
    \glt    ‘your shirt(s)’
\z

Apart from \is{case!dative case}dative \textit{nu}, \is{case!genitive case}genitive \textit{de\slash re} and \is{case!ergative case}ergative \textit{ne}, other \is{postposition}postpositions in \ili{Punjabi} do not attach directly to the oblique form of the noun, but rather to the noun followed by \is{case!genitive case}genitive morphology, which surfaces in the invariable form \textit{de\slash re}, as in (ii). This ‘case compounding’ phenomenon is consistent with \citet{Svenonius2006}, who brings out the existence in the internal structure of PPs of both case components (here the \is{case!genitive case}genitive \textit{de\slash re}) and of components with lexical\slash interpretive affinity to nouns, namely Axial Parts (here the embedding \isi{preposition}).

\ea \gll    o-de-nal\\
            him-\textsc{gen}-with\\
    \glt    ‘with him’
\z
}
In (32a) \textit{munɖ-} ‘boy’ (in the \is{case!absolutive case}absolutive case, or absolute form; cf. \citealt{Bailey1904}) bears the masculine plural inflection -\textit{ea}, followed by the \is{case!genitive case}genitive -\textit{d}, followed in turn by a masculine singular inflection -\textit{a}, which agrees with \textit{darwaddʒ-a} ‘door’. In (32b-b’) the inflection following -\textit{d} varies according to whether \textit{kita:b} ‘book’ is in the singular or plural. 

\ea%32
    \ili{Punjabi}\label{ex:manzini:32}\\
    \begin{xlista}
    \ex
    \gll munɖ-ea-    d-a     darwaddʒ-a   nam-a   a\\
         boy-\textsc{m.pl.obl-}  \textsc{gen-m.sg}  door-\textsc{m.sg}  new-\textsc{m.sg}    be\\
    \glt ‘The boys’ door is new.’
    \ex  
    \gll munɖ-e-    d-i     kita:b     nam-i     a\\
         boy-\textsc{m.sg.obl-}  \textsc{gen-f.sg}   book(\textsc{f.sg)}   new-\textsc{f.sg}   be\\
    \glt ‘The boy’s book is new.’
    \exi{b'.}  
    \gll munɖ-ea-    d-ia     kitabb-a   nam-ia   a\\
         boy-\textsc{m.pl.obl-}  \textsc{gen-f.pl}   book-\textsc{pl}   new-\textsc{f.pl}   be\\
    \glt ‘The boys’ books are new.’ 
    \end{xlista}
    \z

The structure illustrated in (32) is recursive, as witnessed by the examples in (33). Thus, in the sequence of two \is{case!genitive case}genitives ‘the sister of the friend of the boy’ in (33a), the most embedded \is{case!genitive case}genitive ‘the boy’ bears the -\textit{d} \isi{postposition} followed by a feminine singular -\textit{i} inflection, agreeing with the feminine singular ‘the friend’ – just as ‘the friend’ in turn bears a feminine singular -\textit{i} agreement with ‘the sister’. Recall from the declension schemas in (30--31) that in the feminine, the noun is only inflected for \is{feature!phi-feature}phi-features; in the masculine, however, direct case is differentiated from oblique – i.e. the form of the noun which co-occurs with \is{postposition}postpositions. This case distinction is in fact recorded by the feature set which inflects the \is{case!genitive case}genitive \isi{postposition}. Consider for instance the examples in (33b-b’). The most embedded \is{case!genitive case}genitive, i.e. ‘of the boy’, agrees with the head it modifies, which is in turn a \is{case!genitive case}genitive, i.e. ‘of the brother(s)’. Therefore, the inflection on the \is{case!genitive case}genitive \isi{postposition} \textit{d-} is oblique masculine -\textit{e}. This contrasts with (32a), where the masculine singular head of the construction is in the absolute form (direct case) and the agreement following \textit{d-} is therefore the masculine singular non-oblique -\textit{a}.

\ea%33
    \ili{Punjabi}\label{ex:manzini:33}\\
    \begin{xlista}
    \ex
    \gll     munɖ-e-d-i   dost-d-i   pɛn-ne    kitt-a         a  \\
             boy-\textsc{m.sg.obl-gen-f.sg}  friend\textsc{(f.sg)-gen-f.sg}  sister\textsc{(f.sg)-erg}    done-\textsc{m.sg}   be\\
    \glt     ‘The sister of the friend of the boy did it.’
    \exi{a'.}  
    \gll    munɖ-e-d-i    dost-d-i   pɛn-nu  me  kita:b         ditt-i   a \\
             boy-\textsc{m.sg.obl-gen-f.sg}  friend\textsc{(f.sg)-gen-f.sg}  sister\textsc{(f.sg)-dat}   I          book\textsc{(f.sg)}   given\textsc{{}-f.sg}   be\\
    \glt     ‘I gave the book to the sister of the friend of the boy.’
    \ex      
    \gll    munɖ-e-d-e   pra-d-i   kita:b   nam-i    a \\
             boy-\textsc{m.sg.obl-gen-m.obl}  brother\textsc{(m.sg)-gen-f.sg}  book\textsc{(f.sg)}   new-\textsc{f.sg} be \\
    \glt     ‘The book of the brother of the boy is new.’
    \exi{b'.}
    \gll     munɖ-e-d-e   prama-  d-i   kita:b   nam-i       a\\
             boy-\textsc{m.sg.obl-gen-m.obl}  brother\textsc{(m.pl)-}  \textsc{gen-f.sg}   book\textsc{(f.sg)}   new-\textsc{f.sg} be \\
    \glt     ‘The book of the brothers of the boy is new.’
    \end{xlista}
    \z
    
From a typological point of view, the fact that agreement on \textit{d}{}- is sensitive to direct vs. oblique features establishes the continuity between the phenomena we are describing in \ili{Punjabi} and the prototypical \isi{Suffixaufnahme} of Australian languages, as discussed in §3.1 – as well as with \is{linker}linkers in languages like \ili{Albanian}. Here, however, we are not interested in the functional equivalence between these various phenomena – but rather in whether they share formal properties, including their constituent structure and the rules that apply to it.    

Following the discussion of \ili{Albanian} (25), we take \is{case!genitive case}genitive case in \ili{Punjabi} to correspond to the part-whole elementary predicate, notated ($\subseteq$). The only difference is that, as argued by \citet{Payne1995}, \is{case!oblique case}oblique cases in Indo-Aryan correspond to \is{postposition}postpositions, as opposed to inflections. Thus, in the coordination in (34a), the \textit{d-} \is{case!genitive case}genitive \isi{postposition} takes a coordination of two DPs as its complement. This shows that the nature of \textit{d-} is phrasal, akin to \ili{English} -\textit{’s}, rather than inflectional. A similar argument can be built from \is{case!genitive case}genitive nouns modified by an \isi{adjective}. As can be seen in (34b), the \textit{d-} case \isi{postposition} appears only once in the structure, embedding the whole \is{case!genitive case}genitive NP ‘open door’.

\ea%34
    \ili{Punjabi}\label{ex:manzini:34}\\
    \ea
    \gll    rami   e   ran   d-a   pra\\
              Rami   and  Ran  gen-\textsc{m.sg}  brother\textsc{(m.sg)}  \\
    \glt     ‘Rami and Ran’s brother’
    \ex  
    \gll    kull-e   darwaddʒ-e  d-i   tʃabb-i   lend-i   a  \\
             open-\textsc{m.sg.obl}   door-\textsc{m.sg.obl}  \textsc{gen-f.sg}   key-\textsc{f.sg}  taking-\textsc{f.sg}  be \\
    \glt     ‘I(\textsc{f}) am taking the key of the open door.’ 
    \z
\z

Next, \ili{Albanian} is head-initial, while \ili{Punjabi} is head-final; thus, in structure (35) for example (32a), the N \textit{darwaddʒ-a} ‘door’ follows its \is{case!genitive case}genitive modifier. Second, recall that in \ili{Albanian} (25), we categorize the inflections on N as D, as they carry not only \is{feature!phi-feature}phi-features and case, but also definiteness. In \ili{Punjabi}, the inflections on N are compatible with both a definite and an \isi{indefinite} reading, and do not therefore have D content. Because of this, we assign them the $\Phi $ category in (35). The interpretation of (35) is the same as in \ili{Albanian} (25) – namely that a ($\subseteq$) relation, lexicalized by the \isi{postposition} \textit{d}, holds between the argument to which the \is{case!genitive case}genitive morphology attaches, i.e. \textit{munɖea} ‘the boys’ (the whole or possessor), and the head DP \textit{darwaddʒ}\textit{a} ‘the door’ (the part or possessum).

\ea%35
\label{ex:manzini:35}
\begin{forest}
    [NP
    [($\subseteq$)P
        [($\subseteq$)
            [N\\munɖea]
            [($\subseteq$)\\d]
        ]
        [$\Phi$\\a]
    ]    
    [N\\darwaddʒa]
]
\end{forest}
\z
%%please move the includegraphics inside the {figure} environment
%%\includegraphics[width=\textwidth]{OGSVolumeAug2018ManziniFrancoSavoia-img17.jpg}

The ($\subseteq$)P structural cell in (35) can of course be embedded under another\linebreak oblique, yielding recursive structures of the type illustrated in (36) for example (33b). Recall that in the masculine, what we have called a $\Phi $ inflection in (35) displays sensitivity to direct vs. \is{case!oblique case}oblique case.  Importantly, the oblique inflection of the masculine never appears as a stand-alone form of the noun. In other words, its only occurrences are as a bound form selected by a \isi{postposition}. Based on this observation, we conclude that the oblique masculine inflections -\textit{e} and -\textit{ea} do not bear ($\subseteq$) content, but instantiate $\Phi $ – with the proviso that it is sensitive to selection by a ($\subseteq$) case element, or to agreement with an element selected by ($\subseteq$). Specifically, in (36), the -\textit{e} oblique inflection is triggered by agreement with the oblique inflection of the N, selected by \textit{d-}.

\ea%36
    \label{ex:manzini:36}
    \begin{forest}
    [($\subseteq$)P
        [($\subseteq$)
            [NP
                [($\subseteq$)P
                    [($\subseteq$)
                        [N\\munɖe] [($\subseteq$)\\d]
                    ] [$\Phi$\\e]
                ] [N\\pra]
            ] [($\subseteq$)\\d]
        ] [$\Phi$\\i]
    ]
    \end{forest}
    \z
%%please move the includegraphics inside the {figure} environment
%%\includegraphics[width=\textwidth]{OGSVolumeAug2018ManziniFrancoSavoia-img18.jpg}

In copular sentences, including most of the examples in (33--34), the predicative \isi{adjective} agrees with the DP subject. On the basis of the parallelism observed so far between \is{adjective}adjectives and \is{case!genitive case}genitives, we expect that the postcopular \is{case!genitive case}genitives will present the same agreement structure as \is{case!genitive case}genitives embedded in DPs. This prediction is verified by the data in (37).

\ea%37
    \ili{Punjabi}\label{ex:manzini:37}\\
    \ea
    \gll munɖ-e-d-i   kita:b-d-e   paper   me-r-e  a \\
         boy-\textsc{m.sg.obl-gen-f.sg}   book\textsc{(f.sg)-gen-m.pl}  sheets\textsc{(m.pl)}  me-\textsc{gen-m.pl}   be \\
    \glt ‘The sheets of the boy’s book are mine.’
    \ex
    \gll ghar-d-e   darwaddʒ-ea-d-ia   tʃabb-ia   me-r-ia\\
         house-\textsc{gen-m.pl.obl}  door-\textsc{m.pl.obl-gen-f.pl}  key-\textsc{f.pl}  me-\textsc{gen-f.pl}    \\
    \glt ‘The keys of the house’s door are mine.’
    \z
\z

\citet[295]{Payne1995} reports the existence of reduced relative clauses headed by perfect \is{participle}participles, where the external argument of the perfect \isi{participle} surfaces in the \is{case!genitive case}genitive and agrees with the head noun. In our corpus, this pattern is attested by data like (38). Recall that \ili{Punjabi} is a head-final language. The fact that ‘meat’ in (38a) follows the \isi{participle} ‘done’, of which it is the object, suggests that ‘meat’ heads a DP, modified by the \isi{participle} and by the \is{case!genitive case}genitive that precedes the \isi{participle} – i.e. by a reduced relative. On the other hand, sentences like (38b) are also possible, where ‘the meat’ precedes the \isi{participle} ‘done’ and is in turn preceded by the genitive. In both environments the \is{case!genitive case}genitive alternates with the \is{case!ergative case}ergative.

\ea%38
    \label{ex:manzini:38}
    \ea
    \gll mɛ  kuɾ-i-d-a /  kuɾ-i-ne  bəna-ea      mi:tə      khan-d-i   a\\
         I(\textsc{f})  girl-\textsc{f.sg-gen-m.sg} /  girl-\textsc{f.sg-erg}  done-\textsc{m.sg}  meat.\textsc{m.sg}  eat-\textsc{progr-f.sg}  be\\
    \glt ‘I am eating the meat cooked by the girl.’
    \ex
    \gll mɛ  kuɾ-i-d-a /   kuɾ-i-ne   mi:tə  bəna-ea     khan-d-i   a\\
         I(\textsc{f})  girl-\textsc{f.sg-gen-m.sg} /  girl-\textsc{f.sg-erg}   meat.\textsc{m.sg}  done-\textsc{m.sg}   eat-\textsc{progr-f.sg}   be\\
    \glt ‘I am eating the meat cooked by the girl.’
    \z
\z

Main sentences constructed with a \isi{participle}, an absolute argument and a \is{case!genitive case}genitive argument, as in (39a), yield a meaning that \citet{Stroński2013} characterizes as \isi{resultative} for a range of Indo-Aryan languages. For ease of comparison, (39b) displays an ordinary perfective sentence, with the internal argument in the absolutive form and the external argument in the ergative. As again highlighted by Stronsky, the resultative form requires the presence of the \isi{participle} of ‘be’, \textit{o} in \ili{Punjabi}, which we also see in the \isi{stative} predication in (39c).

\ea%39
    \label{ex:manzini:39}
    \ea
    \gll o-d-i    kəmidʒə  tott-i       o  a\\
         he-\textsc{gen-f.sg}  shirt.\textsc{f.sg}   wash.\textsc{perf-f.sg}  been  be \\
    \glt ‘He has the shirt washed.’
    \ex 
    \gll o-ne/mɛ  kəmidʒə   tott-i       (a/si)\\
         he-\textsc{erg}/I  shirt.\textsc{f.sg}   wash.\textsc{perf-f.sg}   be/be.\textsc{pst}\\
    \glt ‘He has washed the shirt.’
    \ex 
    \gll eval-i     kəmidʒə   tott-i       (o)  a\\
         this-\textsc{f.sg}    shirt.\textsc{f.sg}   wash.\textsc{perf-f.sg}   been  be \\
    \glt ‘This shirt is washed.’
    \z
\z

There is an important stream of literature connecting ergative subjects, as seen in \ili{Punjabi} (39b), with possession. \citet[176–186]{Benveniste1966} concludes that “the Old \ili{Persian} [ergative] structure … is intrinsically possessive in its meaning” (cf. \citealt{Butt2006} on the dative-ergative connection in Indo-Aryan). For \citet{Manzini2015}, the \is{case!ergative case}ergative case in sentences like (39b) has the same ($\subseteq$) content reviewed here for \is{case!genitive case}genitives\slash datives; specifically, it introduces a relation between the DP it embeds ‘he’ and a nominal-like participial predicate, ‘washed the shirt’. In essence, the ($\subseteq$) relation lexicalized by the ergative says that the state\slash event denoted by the VP (the perfect and its internal argument) is included by\slash located at the external argument.\footnote{{Though a bare VP structure for \ili{Punjabi} perfects is proposed by \citet{Manzini2015}, in frameworks which distinguish a} {\textit{v}}{P projection for transitivity from a VoiceP for introducing external arguments \citep{Harley2013}, it is equally possible to characterize the predicate as} {\textit{v}}{P (see \citealt{Nash2014}). More descriptive labels such as PerfP are also possible.}} In other respects, the ergative structure in (40) is characterized by agreement of the perfect \isi{participle} with the internal argument, corresponding in \isi{Minimalist} terms to a probe on \textit{v}/V.

\ea%40
    \label{ex:manzini:40}
    \begin{forest}
    [VP
        [($\subseteq$)P
            [D\\o\textsubscript{x}]
            [($\subseteq$)\\ne\textsubscript{λx,λz}]
        ] [VP\textsubscript{z}
            [DP\\kəmidʒә] [V\\totti]
        ]
    ]
    \end{forest}
\z
%%please move the includegraphics inside the {figure} environment
%%\includegraphics[width=\textwidth]{OGSVolumeAug2018ManziniFrancoSavoia-img19.jpg}

In the generative literature, the existence of a connection between ergative structures and nominalizations – hence between ergative subjects and possessors – is proposed by Johns (\citeyear{Johns1992}; cf. \citealt{Yuan2013} for a \isi{Minimalist} update). In the words of \citet[61]{Johns1992}, the \ili{Inuktitut} sentence in (41) “is constructed syntactically along the lines of ‘The bear is the man’s stabbed one’”. Thus, in Johns’ proposal the verb is a nominalization, which is first merged with the \is{case!genitive case}genitive\slash \is{case!ergative case}ergative possessor; the structure is then completed by the logical object of the verb in the absolutive. The verb agrees with the \is{case!genitive case}genitive\slash \is{case!ergative case}ergative; the morphology of the agreement suffix on the verb is exactly the same found on nouns agreeing with a possessor.

\ea%41
         \ili{Inuktitut} \citep[61]{Johns1992}\label{ex:manzini:41}\\
    \gll anguti-up   nanuq     kapi-ja-nga\\
         man-\textsc{erg}   {polar bear.\textsc{abs}}  stab-\textsc{perf.prt-3sg/3sg}\\
    \glt ‘The man stabbed the bear.’
\z

\ili{Punjabi} (39a) matches quite closely the \ili{Inuktitut} example in (41).\footnote{There is another possible parallel between reduced relatives of the type in (38a) and structures in Japanese (\citealt{Miyagawa2011} and references quoted there), \ili{Turkic languages} \citep{Kornfilt2008}, \ili{Dagur} (Mongolian; \citealt{Hale2002}) and \ili{Polynesian} \citep{Herd2011}, where (reduced) relatives also present a \is{case!genitive case}genitive subject. In several of these languages, though not in all (for instance not in \ili{Japanese} or in standard \ili{Turkish}) the \is{case!genitive case}genitive agrees with the head noun of the relative. \ili{Dagur} in (i) illustrates the agreement between the head of a relative and the embedded \is{case!genitive case}genitive subject.

\ea         Dagur, Mongolian (\citealt[109-110]{Hale2002})\\
    \gll    [[mini   au-sen]         
            mer\textsuperscript{y}{}-min\textsuperscript{y}]   sain\\   \textsc{1s.gen}  buy-\textsc{perf}  horse-\textsc{1s.gen}  good\\
    \glt    ‘The horse I bought (bought by me) is good.’
\z
}  We take our bearings from Johns’ treatment of \ili{Inuktitut} and treat the internal argument-\isi{participle} complex as a nominalization. To be more precise, in the structure in (42) we advance the hypothesis that the noun ‘shirt’ heads the embedded predicate. Following our established practice, we treat the \is{case!genitive case}genitive as an elementary ($\subseteq$) predicate – which implies that the argument it embeds is interpreted as a possessor. The reading is akin to that indicated by Johns for \ili{Inuktitut}, namely a possession predication between ‘he’ and ‘the shirt washed’ – of the type rendered by the possession verb ‘have’ in \ili{English} ‘He has the shirt washed’.  

\ea%42
    \label{ex:manzini:42}
    \begin{forest}
    [NP
        [($\subseteq$)P
            [($\subseteq$)
                [D\\o\textsubscript{x}]
                [($\subseteq$)\\d\textsubscript{λx,λy}]
            ] [$\Phi$\\i]
        ] [NP\textsubscript{y}
            [N\\kəmidʒә]
            [V\\totti]
        ]
    ]
    \end{forest}
    \z
%%please move the includegraphics inside the {figure} environment
%%\includegraphics[width=\textwidth]{OGSVolumeAug2018ManziniFrancoSavoia-img20.jpg}
In (42) the outer $\Phi $ slot of the \is{case!genitive case}genitive registers agreement with the nominal predicate. We take it that what we have described as reduced relatives in (38) involve the embedding of the structure in (40) or in (42), depending on the presence of a \is{case!genitive case}genitive or of an \is{case!ergative case}ergative. 

\subsection{Lardil and the crosslinguistic distribution of stacking}% 3.2. 

The question now arises whether canonical case stacking of the type seen in Pama-Nyungan languages, for instance \ili{Lardil} in (1), can be unified with the \ili{Punjabi} stacking and ultimately with \ili{Albanian} \is{linker}linkers. While the discussions of \ili{Albanian} and \ili{Punjabi} that precede are based on primary data, in the discussion of Pama-Nyungan languages we depend entirely on data and generalizations provided by the literature. As before, our interest is not descriptive, but theoretical – i.e. considering whether, and how, the approach that we have taken to agreement and to \is{case!oblique case}oblique case leads to unified structures and derivations.

  Let us begin with examples of adjectival modification, such as (8b), repeated below for the relevant part in (43). Adjectival structures in \ili{Lardil} do not appear dissimilar from what one would observe in more familiar languages, where \isi{Agree} applies between the \isi{adjective} and the noun, as well as with the \is{determiner}determiners and \is{quantifier}quantifiers of the DP. Thus, the demonstrative acts as a probe for the \isi{adjective} and the noun, ultimately ensuring agreement all the way through. The only notable property of \ili{Lardil} is that \textit{n}, A and D inflections do not appear to have any \is{feature!phi-feature}phi-feature content, but only case content.

\ea%43
         \ili{Lardil}\label{ex:manzini:43}\\
    \gll kiin-i     mutha-n   thungal-i \\
         that-\textsc{acc}   big-\textsc{acc}     tree-\textsc{acc}     \\
    \glt ‘that big tree’
    \z
    
Adnominal modification by a \is{case!genitive case}genitive, as in example (1), is partially reproduced below in (44). The internal structure of the \is{case!genitive case}genitive phrase is the same as proposed for \ili{Albanian} or \ili{Punjabi}, as in (45). Following the parallel with \ili{Albanian} and \ili{Punjabi}, we take it that so-called \is{case!genitive case}genitive case introduces the ($\subseteq$) elementary predicate. \isi{Agree} is responsible for the presence of a partial copy of the possessum, i.e. the external argument of the ($\subseteq$) elementary predicate, within the \is{case!genitive case}genitive phrase ($\subseteq$)P. In this instance, the inflectional properties that copy under \isi{Agree} are \is{case!oblique case}oblique case ones, which we provisionally notate Instr(umental).  

\ea%44
         \ili{Lardil}\label{ex:manzini:44}\\
    \gll marun-ngan-ku   maarn-ku     \\
         boy-\textsc{gen-instr}  spear-\textsc{instr} \\
    \glt ‘with the boy’s spear.’
    \z


\ea%45
    \label{ex:manzini:45}
    \begin{forest}
    [InstrP
        [($\subseteq$)P
            [($\subseteq$)
                [N\\marun]
                [($\subseteq$)\\ngan]
            ] [Instr\\ku]
        ] [Instr
            [N\\maarn]
            [Instr\\ku]
        ]
    ]
    \end{forest}
\z
%%please move the includegraphics inside the {figure} environment
%%\includegraphics[width=\textwidth]{OGSVolumeAug2018ManziniFrancoSavoia-img21.jpg}

For the sake of completeness, we consider relative clause modification as well. The relevant portion of the example in (8c) is reproduced in (46), where the Acc case of the relative clause head is copied on all constituents of the relative clause. We tentatively propose that the state of affairs just observed is due to the fact that the verb is a participial form (here a future \isi{participle}). Evidence for this claim is found in the Ngarna languages (Pama-Nyungan), as discussed by \citet[234–236]{Breen2004}.  

\ea%46
         \ili{Lardil}\label{ex:manzini:46}\\
    \gll thungal-i,   ngithun-i   kirdi-thuru-Ø \\
         tree-\textsc{acc}   I.\textsc{gen-acc}    cut-\textsc{fut-acc} \\
    \glt ‘…tree, which I will cut down (for me to cut down)’
    \z



If the future form in (46) is participial, we expect it to agree with the head of the relative clause. More interestingly, the embedded subject is in the \is{case!genitive case}genitive case; this confirms the construal of the embedded form as a nominalization of sorts (what we have called a \isi{participle}) – and leads to case stacking. Following the discussion of \ili{Punjabi}, the ($\subseteq$) relation is maintained as the content of the \is{case!genitive case}genitive case, where the external argument of ($\subseteq$) in (47) is FutP.  Thus the event ‘cut down (the tree)’ is interpreted as included by\slash located at the speaker ‘I’, which fills the internal argument slot of ($\subseteq$). As for the extra argument slot available in \isi{linker}\slash stacking languages in the projection of ($\subseteq$)P, the closest probe it can enter an agreement relation with is the relative clause. Hence the left edge of the ($\subseteq$)P is taken up by a copy of the inflectional properties of the head of the relative clause, namely the Acc case. 

\ea%47
    \label{ex:manzini:47}
% % %     \begin{forest}
% % %     [InstrP
% % %         [($\subseteq$)P
% % %             [($\subseteq$)
% % %                 [N\\marun]
% % %                 [($\subseteq$)\\ngan]
% % %             ] [Instr\\ku]
% % %         ] [Instr
% % %             [N\\maarn]
% % %             [Instr\\ku]
% % %         ]
% % %     ]
% % %     \end{forest}
    \begin{forest}
    [NP
        [N
            [N\\thungal-,align=center,anchor=south]
            [Acc\\i,align=center,anchor=south]
        ]
        [FutP
            [($\subseteq$)P
                [($\subseteq$)P
                    [D\\ng\textsubscript{x},align=center,anchor=south]
                    [($\subseteq$)\\itun\textsubscript{λx λy},align=center,anchor=south]
                ] [Acc\\-i]
            ] [FutP
                [Fut
                    [Fut [kirdi-thuru,roof]]
                    [Acc]
                ] [N\\\sout{thungali}]
            ]
        ]
    ]
    \end{forest}
    \z
%%please move the includegraphics inside the {figure} environment
%%\includegraphics[width=\textwidth]{OGSVolumeAug2018ManziniFrancoSavoia-img22.jpg}

Let us go back to the simple case of stacking in adnominal modification of the type in (45). So far, we have only seen stacking configurations where a \is{case!genitive case}genitive is involved. The outer case\slash agreement can correspond to any direct or \is{case!oblique case}oblique case, as can be seen in (44), where it is instrumental vs. (46), where it is accusative – but the inner case is \is{case!genitive case}genitive. In fact, the former condition appears to be too restrictive – the inner case can be any oblique, though it cannot be a direct case.\footnote{\citet{Pesetsky2013} draws a parallel between overt case stacking as described by \citet{Richards2013} and case inflections in Russian, which according to him result from the stacking of several cases and deletion of all but the outermost case. Specifically, Pesetsky argues that the innermost case in Russian is always \is{case!genitive case}genitive. Under the present view, however, cases are stacked, recursively, when the outer case lexicalizes agreement with another argument, excluding unification with Pesetsky’s Russian case stacking.}  This generalization can be illustrated in a particularly clear way in Pama-Nyungan languages. For instance, \citet{Dench1988} and \citet{Dench1995} consider in detail the Western Australian language \ili{Martuthunira}. Three typical case stacking configurations where the inner case is not \is{case!genitive case}genitive are provided in (48). In (48b), the inner case on ‘spear’ is proprietive, essentially the equivalent of \ili{English} ‘with’ (comitative\slash instrumental). In (48a), the inner case is privative – i.e. the negation of ‘with’ (‘without’). In (48c) the inner case is \isi{locative}. 

\ea%48
    \ili{Martuthunira}, Pama-Nyungan (\citealt{Dench1988}: 7ff.)\label{ex:manzini:48}\\
    \ea
    \gll … ngurnu-marta   kanyara-marta  tharnta-wirriwa-marta    \\
         … that.\textsc{obl-prop}   man-\textsc{prop}   kangaroo-\textsc{priv-prop}\\
    \glt ‘… the man without a kangaroo’
    \ex  
    \gll  nhawu-layi   ngurnaa  kurryarta-marta-a-rru    \\
         see-\textsc{fut}   that.\textsc{acc}   spear-\textsc{prop-acc}{}-now\\
    \glt ‘I'll see that one with a spear now.’
    \ex  
    \gll ngali   panyu-ngka-a   warra  kalyaran-ta-a   thuur.ta-a   manku-layi\\
         l\textsc{du.incl}   good-\textsc{loc-acc}  \textsc{cont}   tree-\textsc{loc-acc}   fruit-\textsc{acc}   get-\textsc{fut}\\
    \glt ‘We'll get fruit in a better tree.’ 
    \z
\z 

We can find evidence for the same distribution in other language families. Of particular interest here are Indo-Aryan languages. As shown by \citet{Payne1995}, in Kashmiri (Dardic) the benefactive \isi{postposition} \textit{k’ut} ‘for’ has the same agreement behaviour as the \is{case!genitive case}genitive \isi{postposition}, agreeing with ‘house’ in (49a) and with ‘horses’ in (49b). \citet{Payne1995} further reports the existence of agreeing \isi{locative} \is{postposition}postpositions in \ili{Punjabi} such as \textit{vicc} ‘in’ in (50) and \textit{ənd}\textit{ər} ‘inside’. In other words, case\slash agreement stacking is supported by \is{locative}locatives and by high applicatives (benefactives) in the sense of \citet{Pylkkänen2008}.

\ea%49
    \ili{Kashmiri} \citep[293]{Payne1995}\label{ex:manzini:49}\\
    \ea
    \gll\relax [ paranas     k'ut]     gari \\
         {} reading.\textsc{obl}.I   for.\textsc{dir.m.sg}  house.\textsc{dir.m.sg}\\
    \glt ‘house for reading’  
    \ex  
    \gll\relax [ cur'an     k'it'aw]     gur'aw\\
         {} thieves.\textsc{obl}.I   for.\textsc{obl}.II.\textsc{f.pl}   horse.\textsc{obl.II.f.pl}\\
    \glt ‘horses for thieves’
    \z
\z

          

    

\ea%50
         \ili{Punjabi} \citep[289]{Payne1995}\label{ex:manzini:50}\\
    \gll [ pənj\=ab  vicl-\=\i]    h\=alət        \\
         {} Punjab   in-\textsc{f.sg}  situation.\textsc{f.sg}\\
    \glt ‘the situation in the Punjab’ 
    \z

Let us then assume that any case can be stacked on top of an oblique – but no case can be stacked on top of a direct case, as in (51). In morphological terms, the generalization is that direct cases can only be stacked as outermost in a stacking configuration – which is \citegen{Richards2013} formulation: “if a \is{case!structural case}structural case morpheme is to appear, it must be on the periphery of the DP’s inflection”.\footnote{\citet{Richards2013} does not derive the generalization that we are interested in. Rather, he discusses in detail a different restriction on case stacking, illustrated in (i--ii). In (ii), both Instr and Fut surface on ‘spear’ – but in (i) only Instr surfaces on ‘spear’ and not Acc. This is an instance of blocking an outer Acc – i.e. one that is expected to have the same position in (ii) as the outer Fut in (i).

\ea \gll    Ngada latha   liban-i       kurrumbuwa-r.\\
            I   spear   pumpkinhead-\textsc{acc}   multi.pronged.spear-\textsc{instr}\\
    \glt    ‘I speared the pumpkinhead with a multi-pronged spear.’\\
\z

\ea \gll    Ngada     la-thur     liban-kur     kurrumbuwa-ru-r.\\
            I     spear-\textsc{fut}   pumpkinhead-\textsc{fut}   multi.pronged.spear-\textsc{instr-fut}\\
    \glt    ‘I will spear the pumpkinhead with a multi-pronged spear.’\\
\z

These examples also raise the problem of agreement in Fut features. Richards observes that “the future suffix is morphologically identical to the instrumental suffix, with all the same allomorphs”. We in turn note that future\slash irrealis in a language like \ili{English} can be introduced by a \isi{preposition} “morphologically identical” to the high applicative\slash benefactive, namely \textit{for} (e.g. \textit{I desire for John to win}). These are all possible cues towards the conclusion that the content ‘future\slash irrealis’ need not necessarily be conveyed by an exponent of T.}  Nevertheless it is hard to believe that such a strong cross-linguistic generalization reflects some morphological quirk, and not some deeper syntactic property – which the morphology of course externalizes. 

\ea%51
    \label{ex:manzini:51}
    Case/\is{feature!phi-feature}phi-feature stacking (by affixes or \isi{linker} heads) is restricted to oblique DPs (\is{case!genitive case}genitives, \is{case!dative case}datives, \is{case!instrumental case}instrumentals, \is{locative}locatives).  
    \z         

A possible way to conceptualize this state of affairs in (51), suggested by the typological literature, is that case\slash agreement stacking is restricted to structure involving adnominal modifiers. \citet[386]{Dench1995} expresses essentially this generalization by saying that “The NP is defined as a sequence of adjacent nominals over which some nominal suffix is distributed” – in other words, the spreading of nominal suffixes (case stacking) is possible to the extent that a nominal constituent underlies it. The canonical example of adnominal modification is by \is{case!genitive case}genitives – but instrumentals/ comitatives or \is{locative}locatives are equally possible adnominal modifiers. However, this characterization is arguably insufficient. For instance, while the adnominal modification relation is generally clear with \is{case!genitive case}genitives, it is much laxer with other obliques; specifically, it seems from examples like (48c) that the notion of adnominal modification must be stretched to cover environments where the noun and its modifier do not form a constituent. The same point in fact can be made with \is{case!genitive case}genitives in postcopular position, such as \ili{Punjabi} (37) or \ili{Albanian} (12); one would need to argue that the subject of the copula is raised from within a complex DP of which the \is{case!genitive case}genitive is a modifier. This is unlikely, to the extent that the copula is deemed to embed a predicative small clause, not a DP.

More to the point, Dench’s suggested generalization does not really explain (51) in terms of more primitive notions, but rather substitutes for the descriptive notion of ‘oblique’ in (51) the equally descriptive notion of ‘adnominal modification’. In present terms, explaining (51) amounts to providing a theoretical content for the notion ‘oblique’. The approach we have adopted here to \is{case!genitive case}genitive, beginning with the analysis of \ili{Albanian} \is{linker}linkers in §2.1, leads us in the direction of assuming that what sets \is{case!oblique case}oblique cases apart from direct case is their relational nature. Specifically, in their investigation of the case system of \ili{Albanian}, \citet{Manzini2011Grammatical,Manzini2011Reducing} propose a construal of the \is{case!genitive case}genitive\slash \is{case!dative case}dative \isi{syncretism} ‘oblique’ and of the residual ablative (\isi{locative}) in terms of the relation ($\subseteq$) introduced in §2 for the \is{case!genitive case}genitive. \citet{Franco2015} argue in turn that \is{case!instrumental case}instrumentals\slash comitatives instantiate the reverse relation ($\supseteq$), where the DP bearing the DP case is to be construed as ‘included by’/‘possessed by’ the DP head of the complex nominal. 

This conceptualization of \is{case!oblique case}oblique case is easily explained with examples from familiar Western European languages, including \ili{English}, which use Ps as externalizations of the relevant relations. In present notation, in (52a) the \ili{English} \isi{preposition} \textit{of} relates the head and the modifiers of the DP by introducing a ($\subseteq$) part\slash whole relation between them. The \ili{English} \isi{preposition} \textit{to} introduces exactly the same ($\subseteq$) relation in (52b) – holding between the argument embedded by \textit{to} and the theme of the ditransitive predicate (\citealt{Kayne1984}; \citealt{Pesetsky1995}; \citealt{Harley2002Possession}; \citealt{Beck2004}; \citealt{Manzini2016}).  Languages like \ili{English} which have distinct \is{case!genitive case}genitive and \is{case!dative case}dative \is{preposition}prepositions or cases simply have different externalizations for the ($\subseteq$) content when embedded in nominal contexts (52a) or verbal contexts (52b). Languages like \ili{Albanian} (\ili{Aromanian}, etc.) which have the same morphological realization for \is{case!genitive case}genitive and \is{case!dative case}dative simply have a non-context-sensitive externalization for the ($\subseteq$) content.

\ea%52
    \label{ex:manzini:52}
    \ea  the hat [\textsubscript{($\subseteq$)} of [the girl]]
    \ex  I gave [the hat [[\textsubscript{($\subseteq$)} to [the girl]]]]
    \z    
\z

In turn, the comitative\slash instrumental \isi{preposition} \textit{with} may reverse the relation conveyed by the \is{case!genitive case}genitive \citep{Levinson2011}. This is illustrated by the comparison between \ili{English} (52a) and (53a). \ili{English} \textit{with} in (53a) introduces a possessum of the head noun of the DP (the possessor); following \citet{Franco2017}, we notate the relevant relation with ($\supseteq$). They further argue that the comitative and instrumental values also associated with \ili{English} ‘with’ are contextual manifestations of the same ($\supseteq$) relation. Thus in (53b), \textit{with the hat} has the canonical possession interpretation already noted for (53a); \textit{with John} is a comitative, to the extent that ‘John’ carries the same degree of animacy\slash intentionality as ‘the girl’ of which it is predicated. Finally, in (53c) the instrumental reading of \textit{with the hat} depends on it being a concomitant of ‘the girl’ in the role of causer of the event. In this sense, \citet{Franco2017}, following in part \citet{Bruening2012}, propose that the ($\supseteq$) relation applies between the instrumental (the part) and the subevent represented by the VP (‘chasing the fly’).

\ea%53
    \label{ex:manzini:53}
    \ea  the girl [\textsubscript{($\supseteq$)} with [the hat]]
    \ex  The girl left [\textsubscript{($\supseteq$)}with [the hat/John]]
    \ex  The girl chased the fly away [\textsubscript{($\supseteq$)}with [the hat]]
    \z
\z
  
Recall that our goal here is explaining the generalization in (51) restricting case\slash agreement stacking to oblique arguments. What we have now proposed is that there is a common denominator in the oblique system (\is{case!genitive case}genitive-\is{case!dative case}dative-\is{case!instrumental case}instrumental). In terms of this proposal, the generalization in (51) can be restated as in (54). If oblique is seen as the ($\subseteq$)/($\supseteq$) elementary \isi{relator}, then it supports – and in fact it requires (in the languages where the relevant parameter is active) – a lexicalization of both its arguments within its maximal projection. The internal argument is its complement, the external argument is introduced as a \isi{linker} or a stacked affix.

\ea%54
    \label{ex:manzini:54}
    The external argument of the ($\subseteq$)/($\supseteq$) \isi{relator} (part/whole) is instantiated within the \isi{relator}’s maximal projection (phase).
\z

In evaluating the proposal in (54), it is worth keeping in mind what the range of possible alternatives is. The Distributed Morphology literature that endeavours to decompose traditional cases into a system of elementary features recognizes [obl] as a primitive (\citealt{Halle1998}; \citealt{Calabrese2008}). However, in terms of [obl], the best generalization we can reach about \isi{Suffixaufnahme} is indeed (51). The formal syntax literature, in turn, focuses on \citegen{Chomsky1986Knowledge} distinction between \is{case!structural case}structural case (depending on a syntactic configuration) and inherent case (depending on the selection properties of a predicate). Though the canonical \is{case!structural case}structural cases are the direct cases, \is{case!genitive case}genitive is also typically treated as structural by the generative literature. Indeed, in present terms, this means that the ($\subseteq$)/($\supseteq$) content does not necessarily depend on the inherent properties of the predicate head, but rather on a structural configuration. Therefore, the structural\slash inherent distinction has no relevance for the \isi{Suffixaufnahme} generalization. Case\slash agreement stacking examples may involve selected, i.e. inherent, obliques or what would count as structural obliques  – the distinction is irrelevant to the distribution of \isi{Suffixaufnahme}. 

We thus revert to the possibility that there is no syntactic substance to \isi{Suffixaufnahme}. And yet, this makes it extremely difficult to capture the formal (not merely functional) overlapping of affixation and \isi{linker} phenomena – where by common consent linkers involve phrasal syntax. In other words, \isi{Suffixaufnahme} provides an argument in favour of the syntactic relevance of the notion of oblique. The latter in turn suggests that more primitive syntactic notions may underlie the descriptive ‘oblique’. Here we have suggested that the relevant notions are that of elementary \isi{relator} (with the part\slash whole content) and the case\slash agreement stacking corresponds to the presence of complete or partial copies of the arguments satisfying the \isi{relator} within its phrasal projection. 

\section{Conclusions}% 4. 

In this paper, we have shown that stacked suffixes and \isi{linker} heads display the same syntactic distribution, roughly adnominal modification. Furthermore, constituency tests show that \isi{linker} heads, no less than stacked suffixes, are internal to the projection of the modifier phrase (an AP, a \is{case!genitive case}genitive phrase, a relative clause). From a morphological point of view, linkers can display agreement in \is{feature!phi-feature}phi-features or in case or in both, generally with the modified noun; similarly, stacked suffixes may involve agreement in case (\ili{Lardil}) and\slash or in \is{feature!phi-feature}phi-features (\ili{Punjabi}). 

  We have argued that \isi{Agree} can achieve descriptive adequacy without making reference to features being [interpretable] or [valued]. In fact, at least in the DP domain, taking agreement to result from \is{feature!uninterpretable feature}\is{feature!interpretable feature}interpretable-uninterpretable (valued-unvalued) \is{feature!unvalued feature}pairs of features imposes partitions between \is{feature!phi-feature}phi-feature sets that are not evident in the interpretation, where it is hard to determine whether (un)interpretable properties reside on N rather than on Q\slash D – or vice versa. In our conception, each category within the DP acts as a probe for the immediately c-commanded category, all the way down from D to N. \isi{Agree} is necessary for the establishment of equivalence classes between two or more copies of the same \is{feature!phi-feature}phi-feature material – to be interpreted at the interface as individuating a single referent. 

Furthermore, stacking and \is{linker}linkers provide an argument in favour of the syntactic relevance of the notion of oblique. We have argued that more primitive syntactic notions underlie the descriptive label ‘oblique’. We have proposed that there is a common denominator in the oblique system (genitive-dative-instrumental) of natural languages. Specifically, obliques are elementary predicates\slash relators \is{relator}with a part\slash whole content, whose internal argument is the DP\slash AP\slash CP they embed (the modifier) and whose external argument is the modified D\slash NP. Stacked morphemes and \is{linker}linkers introduce partial copies of the external argument (the modifiee) at the edge of the \isi{relator} phrase. 

% % \section{ References}
% 
% Alexiadou, Artemis. 2001. \textit{Functional structure in nominals: Nominalization and ergativity}.\\
% Amsterdam: John Benjamins.
% 
% Baker, Mark. 2008. \textit{The syntax of agreement and concord}. Cambridge: Cambridge University Press.
% 
% Baker, Mark \& Vinokurova, Nadya. 2010. Two modalities of \is{case!case assignment}case assignment in Sakha. \textit{Natural Language \& Linguistic Theory} 28. 593–642.
% 
% Bailey, T. Grahame. 1904. \textit{Panjabi grammar: A brief grammar of Panjabi as spoken in the Wazirabad District}. Lahore: Saaddi Panjabi Academy.
% 
% Beck, Sigrid \& Johnson, Kyle. 2004. Double objects again. \textit{Linguistic Inquiry} 35. 97–124.
% 
% Belvin, Robert \& Dikken, Marcel den. 1997. \textit{There}, happens, \textit{to, be, have}. \textit{Lingua} 101. 151–183.
% 
% Benveniste, Émile. 1966. \textit{Problèmes de linguistique générale 1}. Paris: Gallimard.
% 
% Bhatia, Tej K. 2000. \textit{\ili{Punjabi}: A cognitive-descriptive grammar}. London: Routledge. 
% 
% Breen, Gavan. 2004. Evolution of the verb conjugations in the Ngarna languages. In Bowern, Claire \& Koch, Harold (eds.), \textit{Australian languages: Classification and the comparative method}, 223–240. Amsterdam: John Benjamins.
% 
% Bruening, Benjamin. 2012. \textit{By}{}-phrases in passives and nominals. \textit{Syntax} 16. 1–41.
% 
% Butt, Miriam. 2006. The dative-ergative connection. In Bonami, Olivier \& Cabredo Hofherr, Patricia (eds.), \textit{Empirical issues in syntax and semantics 6}, 69–92. Paris: Colloque de Syntaxe et Sémantique à Paris. 
% 
% Calabrese, Andrea. 2008. On absolute and contextual \isi{syncretism}. In Nevins, Andrew \& Bachrach, Asaf (eds.), \textit{The bases of inflectional identity}, 156–205. Oxford: Oxford University Press.
% 
% Camaj, Martin. 1984. \textit{\ili{Albanian} grammar}. Wiesbaden: Harrassowitz.
% 
% Campos, Héctor. 2008. Some notes on adjectival articles in \ili{Albanian}. \textit{Lingua} 119. 1009–1034.
% 
% Campos, Héctor \& Stavrou, Melita. 2005. Polydefinites in \ili{Greek} and \ili{Aromanian}. In Tomić, Olga M. (ed.), \textit{Balkan syntax and semantics,} 137–173. Amsterdam: John Benjamins.
% 
% Carstens, Vicki. 2001. Multiple agreement and case deletion: Against phi-(in)completeness. \textit{Syntax} 4. 147–163.
% 
% Chomsky, Noam. 1986. \textit{Knowledge of language}. New York: Praeger.
% 
% Chomsky, Noam. 1995. \textit{The \isi{Minimalist} Program}. Cambridge, MA: MIT Press.
% 
% Chomsky, Noam. 2000. \isi{Minimalist} inquiries: The framework. In Martin, Roger \& Michaels, David \& Uriagereka, Juan (eds.), \textit{Step by step: Essays on \isi{Minimalist} syntax in honor of Howard Lasnik}, 89–155. Cambridge, MA: MIT Press. 
% 
% Chomsky, Noam. 2001. Derivation by phase. In Kenstowicz, Michael (ed.), \textit{Ken Hale: A life in language}, 1–52. Cambridge, MA: MIT Press.
% 
% Cornilescu, Alexandra \& Giurgea, Ion. 2013. The \isi{adjective}. In Dobrovie-Sorin, Carmen \& Giurgea, Ion (eds.), \textit{A reference grammar of Romanian, Volume 1}, 97–174. Amsterdam: John Benjamins. 
% 
% Danon, Gaby. 2010. Agreement and DP-internal feature distribution. \textit{Syntax} 14. 297–319. 
% 
% Dench, Alan. 1995. \isi{Suffixaufnahme} and apparent ellipsis in Martuthunira, In Plank, Frans (ed.), \textit{Double case: Agreement by} \textit{Suffixaufnahme}, 380–395. Oxford: Oxford University Press.
% 
% Dench, Alan \& Evans, Nicholas. 1988. Multiple case-marking in Australian languages. \textit{Australian Journal of Linguistics} 8. 1–47.
% 
% Dikken, Marcel den \& Singhapreecha, Pornsiri. 2004. Complex Noun Phrases and linkers. \textit{Syntax} 7. 1–54.
% 
% Dikken, Marcel den. 2006. \textit{Relators and linkers}. Cambridge, MA: MIT Press.
% 
% Dimitrova-Vulchanova, Mila \& Giusti, Giuliana. 1998. Fragments of Balkan nominal structure. In Alexiadou, Artemis \& Wilder, Chris (eds.), \textit{Possessors, predicates and movement in the \isi{determiner} phrase}, 333–360. Amsterdam: John Benjamins. 
% 
% Evans Nick. 1995. Multiple case in Kayardild: Anti-iconic suffix ordering and the Diachronic Filter. In Plank, Frans (ed.), \textit{Double case: Agreement by} \textit{Suffixaufnahme}, 396–430. Oxford: Oxford University Press.
% 
% Fillmore, Charles J. 1968. The case for case. In Bach, Emmon \& Harms, Robert T. (eds.), \textit{Universals in linguistic theory}, 1–88. New York: Holt, Rinehart, and Winston. 
% 
% Franco, Ludovico \& Manzini, M. Rita. 2017. Instrumental prepositions and case: Contexts of occurrence and alternations with datives. \textit{Glossa} 2. 1–37.
% 
% Franco, Ludovico \& Manzini, M. Rita \& Savoia, Leonardo M. 2015. Linkers and agreement. \textit{The Linguistic Review} 32. 277–332.
% 
% Ghomeshi, Jila. 1997. Non-projecting nouns and the Ezafe construction in Persian. \textit{Natural} \textit{Language \& Linguistic Theory} 15. 729–788.
% 
% Giurgea, Ion. 2012. The origin of the Romanian “possessive-genitival article” \textit{al} and the development of the demonstrative system. \textit{Revue Roumaine de Linguistique} LVII. 35–65.
% 
% Giusti, Giuliana. 2008. Agreement and \isi{concord} in nominal expressions In de Cat, Cécile \& Demuth, Katherine (eds.), \textit{The Bantu–\ili{Romance} connection: A comparative investigation of verbal agreement, DPs, and information structure}, 201–237. Amsterdam: John Benjamins.
% 
% Hale, Kenneth. 2002. On the Dagur object relative: Some comparative notes. \textit{Journal of East Asian Linguistics} 11. 109–122.
% 
% Halle, Morris \& Marantz, Alec. 1993. Distributed morphology and the pieces of inflection. In Hale, Kenneth \& Keyser, Samuel Jay (eds.), \textit{The view from Building 20}, 111–176. Cambridge, MA: MIT Press. 
% 
% Halle, Morris \& Vaux, Bert. 1998. Theoretical aspects of Indo-European nominal morphology: The nominal declension of Latin and Armenian. In Jasanoff, Jay \& Melchert, Craig \& Olivier, Lisi (eds.), \textit{Mir Curad: A Festschrift in honor of Calvert Watkins}, 223–240. Innsbruck: Universität Innsbruck.
% 
% Harley, Heidi. 2002. Possession and the double object construction. \textit{Linguistic Variation Yearbook} 2. 29–68. 
% 
% Harley, Heidi. 2013. External arguments and the Mirror Principle: On the distinctness of \isi{Voice} and v. \href{http://www.sciencedirect.com/science/journal/00243841}{Lingua} 125. 34–57.
% 
% Herd, Jonathon \& Macdonald, Catherine \& Massam, Diane. 2011. Genitive subjects in relative constructions in Polynesian languages. \textit{Lingua} 121. 1252–1264.
% 
% Higginbotham, James. 1985. On semantics.~\textit{Linguistic Inquiry~}16. 547–621.
% 
% Holmberg, Anders \& Odden, David. 2008. The noun phrase in Hawrami. In Karimi, Simin \& Samiian, Vida \& Stilo, Donald (eds.), \textit{Aspects of Iranian linguistics}, 129–151. Newcastle: Cambridge Scholars Publishing.
% 
% Johns, Alana. 1992. Deriving ergativity. \textit{Linguistic Inquiry} 23. 57–87. 
% 
% Kayne, Richard S. 1984. \textit{Connectedness and binary branching.} Dordrecht: Foris.
% 
% Kayne, Richard S. 1989. Facets of \ili{Romance} past \isi{participle} agreement. In Benincà, Paola (ed.), \textit{Dialect variation and the theory of grammar}, 85–103. Dordrecht: Foris.
% 
% Kornfilt, Jaklin. 2008. Subject case and Agr in two types of Turkish RCs. In Boeckx, Cedric \& Ulutas, Suleyman (eds.), \textit{Proceedings of the 4th Workshop on Altaic Formal Linguistics}, 145–168. Cambridge, MA: MITWPL. 
% 
% Larson, Richard K. \& Yamakido, Hiroko. 2008. Ezafe and the deep position of nominal modifiers. In McNally, Louise \& Kennedy, Christopher (eds.), \textit{Adjectives and adverbs: Syntax, semantics, and discourse}, 43–70. Oxford: Oxford University Press. 
% 
% Lekakou, Marika \& Szendr\H{o}I, Kriszta. 2012. Polydefinites in \ili{Greek}: Ellipsis, close apposition and expletive determiners. \textit{Journal of Linguistics} 48. 107–149. 
% 
% McKenzie, David N. 1961. \textit{\ili{Kurdish} dialect studies, I}. Oxford: Oxford University Press.
% 
% Manzini, M. Rita \& Franco Ludovico. 2016. \isi{Goal} and DOM datives. \textit{Natural Language \& Linguistic Theory}articlecitationvolume{ 34.}articlecitationpages{ 197}–articlecitationpages{240.}
% 
% Manzini, M. Rita \& Savoia, Leonardo M. 2007. \textit{A unification of morphology and syntax: Studies in \ili{Romance} and \ili{Albanian} varieties}. London: Routledge 
% 
% Manzini, M. Rita \& Savoia, Leonardo M. 2011a. \textit{Grammatical categories.} Cambridge: Cambridge University Press.
% 
% Manzini, M. Rita \& Savoia, Leonardo M. 2011b. Reducing “case” to denotational primitives: Nominal inflections in \ili{Albanian}. \textit{Linguistic Variation Yearbook} 11. 76–120.
% 
% Manzini, M. Rita \& Savoia, Leonardo M. \& Franco, Ludovico. 2015. Ergative case, aspect and person splits: Two case studies. \textit{Acta Linguistica Hungarica} 62. 297–351.
% 
% Marantz, Alec. 2007. Phases and words. In Choe, Sook-Hee (ed.), \textit{Phases in the theory of grammar}, 191–222. Seoul: Dong In.
% 
% Merchant, Jason. 2006. Polyvalent case, geometric hierarchies, and split ergativity. In Bunting, Jackie \& Desai, Sapna (eds.), \textit{Proceedings of the 42nd Annual Meeting of the Chicago Linguistic Society}, 57–76. Chicago: Chicago Linguistic Society.
% 
% Miyagawa, Shigeru. 2011. Genitive subjects in Altaic and specification of phase. \textit{Lingua} 121. 1265–1282.
% 
% Nash, Léa. 2014. The structural source of split ergativity and \is{case!ergative case}ergative case in Georgian. Ms. (Paris: Université de Paris 8.)
% 
% Payne, John, R. 1995. Inflecting postpositions in Indic and Kashmiri. In Plank, Frans (ed.), \textit{Double case: Agreement by} \textit{Suffixaufnahme}, 283–298. Oxford: Oxford University Press. 
% 
% Pesetsky, David. 1995. \textit{Zero syntax}. Cambridge, MA: MIT Press.
% 
% Pesetsky, David. 2013. \textit{Russian case morphology and the syntactic categories}. Cambridge, MA: MIT Press.
% 
% Pesetsky, David \& Torrego, Esther. 2007. The syntax of valuation and the interpretability of features. In Karimi, Simin \& Samiian, Vida \& Wilkins, Wendy K. (eds.), \textit{Phrasal and clausal architecture,} 262–294. Amsterdam: John Benjamins.
% 
% Philip, Joy Naomi. 2012. \textit{Subordinating and coordinating linkers}. London: UCL. (Doctoral dissertation.)
% 
% Plank, Frans. 1995. (Re-)Introducing \isi{Suffixaufnahme}. In Plank, Frans (ed.), \textit{Double case: Agreement by} \textit{Suffixaufnahme}, 3–112. Oxford: Oxford University Press.
% 
% Preminger, Omer. 2014. \textit{Agreement and its failures}. Cambridge, MA: MIT Press.
% 
% Pylkkänen, Liina. 2008. \textit{Introducing arguments.} Cambridge, MA: MIT Press.
% 
% Richards, Norvin. 2010. \textit{Uttering trees}. Cambridge, MA: MIT Press.
% 
% Richards, Norvin. 2013.  \ili{Lardil} “case stacking” and the timing of Case assignment. \textit{Syntax} 16. 42–76
% 
% Samiian, Vida. 1994. The Ezafe construction: Some implications for the theory of X-bar syntax. In Mehdi Marashi (Ed.), \textit{Persian studies in North America}, 17–41. Bethesda, MD: Iranbooks. 
% 
% Samvelian, Pollet. 2007. A (phrasal) affix analysis of the Persian Ezafe. \textit{Journal of Linguistics} 43. 605–645.
% 
% Savoia, Leonardo M. 2008. \textit{Studi sulle varietà albanesi}. Cosenza: Università della Calabria.
% 
% Solano, Francesco. 1972. \textit{Manuale di lingua albanese}. Corigliano Calabro: Arti Grafiche Ioniche.
% 
% Stroński, Krzysztof. 2013. Evolution of \isi{stative} participles in Pahari. \textit{Lingua Posnaniensis} 55. 135–150.
% 
% Svenonius, Peter. 2006. The emergence of axial parts. \textit{Nordlyd} 33. 1–22.
% 
% Turano, Giuseppina, 2002. On modifiers preceded by the article in \ili{Albanian} DPs. \textit{University of Venice Working Papers in Linguistics} 12. 169–215. 
% 
% Turano, Giuseppina. 2004. \textit{Introduzione alla grammatica dell'albanese}. Florence: Alinea.
% 
% Yamakido, Hiroko. 2005. \textit{The nature of adjectival inflection in Japanese}. Stony Brook, NY: Stony Brook University. (Doctoral dissertation.)
% 
% Yuan, Michelle. 2013. \textit{A phasal account of ergativity in Inuktitut}. Toronto: University of Toronto. (Master’s dissertation.)
% 
% Williams, Edwin. 1994. \textit{Thematic structure in syntax.} Cambridge, MA: MIT Press.
% 
% Zeijlstra, Hedde. 2012. There is only one way to agree. \textit{The Linguistic Review} 29. 491–539.
% 
% Zwart, Jan-Wouter. 2006. Complementizer agreement and dependency marking typology. \textit{Leiden Working Papers in Linguistics} 3. 53–72.

\sloppy
\printbibliography[heading=subbibliography,notkeyword=this] 
\end{document}
